\documentclass[12pt,a4paper,openright,twoside]{book}
\usepackage[utf8]{inputenc}
\usepackage{phd-thesis}



\mainlinespacing{1.241} % line spacing in mainmatter, comment to default (1)

\begin{document}
	
\frontmatter
%!TeX root = phd-thesis.tex
\title{Digital Twins Ecosystems for Intelligent Applications in Healthcare}
\author{Samuele Burattini}
\date{\today}

\newgeometry{margin=0.8in}
\begin{titlepage}
	\begin{center}
		% \vspace*{0.2cm}

		\large
		\textbf{\unibo \\ \disi}
		\\
		\noindent\hrulefill
		\vspace{0.4cm}

		\Large
		Dottorato di Ricerca in \\
		Computer Science and Engineering

		\vspace{0.4cm}

		Ciclo XXXVIII

		\vspace{0.4cm}

		Settore Scientifico Disciplinare: ING-INF/05

		Settore Concorsuale: 09/H1

		\Huge
		\vspace{3cm}
		\textbf{
			\thetitle
		}

		{\Large{
		\vspace{3cm}

		\textit{Candidato:\\}
		\centering
		\theauthor}
		\\}
		\large
		\vspace{2.5cm}
		\begin{minipage}[t]{0.49\textwidth}
			\begin{flushleft}
				\textit{Coordinatrice Dottorato:}
				\\
				\textbf{Prof.ssa Ilaria Bartolini}
			\end{flushleft}
		\end{minipage}
		\begin{minipage}[t]{0.5\textwidth}
			\begin{flushright}
				\textit{Supervisore:}
				\\
				\textbf{Prof.} \textbf{Alessandro Ricci}
				\\
				\vspace{0.4cm}
				\textit{Co-supervisore:}
				\\
				\textbf{Dott. Ing.} \textbf{Angelo Croatti}
			\end{flushright}

		\end{minipage}\\

		\vfill
		\noindent\hrulefill
		\vspace{0.3cm}
		\Large

		Esame finale anno 2025
	\end{center}
\end{titlepage}
\restoregeometry

\begin{abstract}	
%Max 2000 characters, strict.

\section*{Italiano}

La digitalizzazione della sanità è vista come una evoluzione necessaria per migliorare efficienza e qualità dei sistemi sanitari, aumentando la tracciabilità e supportando il processo decisionale.
%
% L'applicazione delle tecnologie \acl{IoT} alla sanità porta all'emergere di nuovi paradigmi per il monitoraggio e la gestione in tempo reale delle condizioni dei pazienti.
%
In questo contesto, il concetto di \ac{DT} si è affermato come un'astrazione per modellare e rappresentare asset fisici, costruendone una replica software continuamente aggiornata tramite dati raccolti dal mondo reale.

Tuttavia, domini complessi come quello sanitario sono caratterizzati da asset eterogenei le cui relazioni sono cruciali per fornire una rappresentazione completa e accurata del contesto fisico.
%
Questa tesi esplora le sfide ingegneristiche legate allo sviluppo di \emph{Ecosistemi di Digital Twins (DTE)} che integrano \ac{DT} eterogenei connessi tra loro da relazioni dinamiche.
Il concetto di \acs{DTE} è proposto come layer di astrazione a supporto dello sviluppo di applicazioni intelligenti in grado di sfruttarne le capacità per acquisire una visione globale del mondo fisico e automatizzare processi.
%
Per affrontare tali sfide, il lavoro introduce soluzioni architetturali per lo sviluppo di \ac{DT} focalizzate su modularità e interoperabilità.
%
L’approccio è completato dall'introduzione di descrizioni (semantiche) per \ac{DT}, volte a supportare sia la gestione operativa che l'integrazione nei \acs{DTE}.
%
Ispirandoci alla natura basata su hypermedia del Web e al \acl{WoT}, è stato sviluppato un prototipo di \emph{\ac{HWoDT}} per l'implementazione di \acs{DTE}.
%
Infine, viene esplorato l'uso dei Sistemi Multi-Agente (MAS) come mezzo per realizzare applicazioni intelligenti e affidabili nei \acs{DTE}, chiarendo la relazione con i \ac{DT} e contribuendo all'ingegneria dei \acs{MAS} basati sul modello \acl{BDI}.
%
L'approccio proposto è validato attraverso l'analisi e lo sviluppo di scenari applicativi nel dominio sanitario.



\paragraph{Parole Chiave:} 
Digital Twin, Internet of Things, Web of Things, Cyber-Physical Systems, Artificial Intelligence, Multi-Agent Systems, Software Engineering

\glsresetall

\newpage

\section*{English}

The digitalization of healthcare is envisioned as a necessary development to improve the efficiency and quality of healthcare systems worldwide by improving traceability and supporting decision-making.
%
% The application of \acl{IoT} technologies to healthcare further leads to the emergence of new paradigms for live monitoring and management of patients' conditions.
%
In this context, the concept of \ac{DT} has emerged as a powerful abstraction to model and represent physical assets by creating a software replica which is continuously updated with data from the physical world.

Complex domains such as healthcare, however, are characterized by a variety of heterogeneous assets whose interrelations are crucial to provide a complete and accurate representation of the physical context.
%
This thesis explores the engineering challenges in the development of \emph{\acp{DTE}} integrating heterogeneous \acp{DT} connected by dynamic relationships.
We further propose the \ac{DTE} as a layer of abstraction to support the development of intelligent applications which can exploit its capabilities to acquire a comprehensive view of the physical world and automate processes.
%
To address these challenges, this work proposes architectural and engineering solutions for the development of \acp{DT} focusing on modularity and interoperability.
%
This approach is complemented by the introduction of (semantic) descriptions for \acp{DT} that can support both the operational management and integration challenges in \acp{DTE}.
%
Through an approach inspired by the hypermedia nature of the Web and the \acl{WoT} we develop a prototype for a \emph{\ac{HWoDT}} which can be used to implement \acp{DTE}.
%
Finally, we investigate the use of \acp{MAS} as a way to implement trustworthy intelligent applications in \ac{DTE}, defining their relationship with \acp{DT} and contributing to the engineering of \ac{MAS} based on the \acl{BDI} model. 
%
The proposed approach is validated through the analysis and development of application scenarios in the healthcare domain.


\paragraph{Keywords:} Digital Twin, Internet of Things, Web of Things, Cyber-Physical Systems, Artificial Intelligence, Multi-Agent Systems, Software Engineering

\end{abstract}

\glsresetall

% \begin{dedication} % this is optional
% The most profound technologies are those that disappear.
% \\
% They weave themselves into the fabric of everyday life
% \\
% until they are indistinguishable from it.
% \\
% \vspace{1em}
% -- Mark Weiser
% \end{dedication}


\begin{acknowledgements} % this is optional
Optional. Max 1 page.
\end{acknowledgements}

%%%%%%%%%%%%%%%%%%%%%%%%%%%%%%%%%%%%%%%%%%%%%%%%%%%%%%%%
\tableofcontents   
\listoffigures     % (optional) comment if empty
\lstlistoflistings % (optional) comment if empty
%%%%%%%%%%%%%%%%%%%%%%%%%%%%%%%%%%%%%%%%%%%%%%%%%%%%%%%%

\mainmatter

%%%%%%%%%%%%%%%%%%%%%%%%%%%%%%%%%%%%%%%%%%%%%%%%%%%%%%%%
\chapterOutsidePart{Introduction}
\label{chap:introduction}
%%%%%%%%%%%%%%%%%%%%%%%%%%%%%%%%%%%%%%%%%%%%%%%%%%%%%%%%

\glsresetall

%=======================================================
\section*{Context and Motivations}
%=======================================================

In the last decade, we have assisted to a significant transformation driven by the integration of advanced digital technologies such as the \ac{IoT}, \ac{AI} and \ac{ML} to improve the efficiency and automation of processes across manufacturing and service industries~\missingref{}.
This digital transformation, often referred to as the \emph{Fourth Industrial Revolution} or \emph{Industry 4.0}~\missingref{}, has 
is now becoming a pervasive trend impacting even less technology-driven sectors such as agriculture and healthcare~\missingref{}.

In the healthcare sector, the integration of digital technologies is reshaping the way care is delivered, with a focus on improving patient outcomes, enhancing operational efficiency, and enabling personalized medicine~\missingref{}.
%
Some see this transformation as a way to address the challenges of an aging population, rising healthcare costs, and the need for more efficient and effective care delivery towards accomplishing the United Nations' \ac{SDG} 3 aiming to ensure healthy lives and promote well-being for all at all ages~\cite{un_sdg_report_2025}.

This Ph.D. thesis is the outcome of a collaboration with \auslLong{}, the local public healthcare authority in the southern region of Emilia-Romagna, Italy, aimed at exploring the potential of digital technologies to enhance the internal processes of the local healthcare system. 

In this exploration, we follow the rising trend of digitalizing the core assets of an application domain with \emph{\acp{DT}}: representations of \acp{PA}, connected and synchronized through data streams from sensors and other digital sources~\missingref{}.
%
This concept has emerged as a paradigm shift in the way we model and interact with the physical world when designing \acp{CPS}, focusing on capturing the essential features of each relevant entity, creating personalized models and combining real-time data with modeling techniques, \ac{AI} and simulation~\missingref{}.
%
The focus of \acp{DT} towards individualized modeling is well-paired to the digitalization needs of the healthcare domain supporting a patient-centric approach~\missingref{}.
%
Furthermore, the ability of \acp{DT} to integrate data from various sources
-- spanning from legacy systems to \ac{IoT} devices --
and provide a holistic view of the system is beneficial for the development of \emph{intelligent applications} which can leverage the data and insights provided by \acp{DT} to support -- or partially automate -- healthcare processes.

%=======================================================
\section*{Overview and Contribution}
%=======================================================

Adopting a \ac{DT}-based approach for healthcare digitalization is not without challenges.
%
Healthcare organizations often operate with fragmented processes and legacy systems that hinder the seamless flow of information.
%
Moreover, healthcare processes often involve a variety of very different assets
-- including facilities, medical devices, patients and healthcare professionals --
whose interrelations are crucial for the understanding of workflows.
%
Finally, for an effective monitoring and decision-making support, it is crucial to have an always up-to-date view of all the assets involved.
%
Given the safety critical setting of healthcare, automated decisions must be transparent and reliable, albeit capable of adapting to the dynamic nature of healthcare environments.


These challenges call for an integrated approach towards the development of both \acp{DT} and applications.
%
In this thesis, we propose a framework for the development of \emph{\acp{DTE}} that can incorporate heterogeneous \acp{DT} and expose a holistic, synchronized view of the mirrored physical environment to applications under a unified interoperable interface.
%
To support the development of intelligent applications, we look at \ac{MAS}, which model distributed applications in dynamic environments through sets of autonomous entities~\missingref{}.
We specifically focus on cognitive approaches such as the \ac{BDI} model~\missingref{}, as they merge adaptability and responsiveness with a declarative encoding of domain knowledge and business rules.

%--------------------------------------------------------
\subsection*{Problem Statement}
%--------------------------------------------------------
Digital transformation, can be supported by \aclp{DT} creating personalized models of relevant assets and support their tracking over workflows.
Yet, legacy systems, assets heterogeneity and dynamically evolving relationships hinder the creation of a holistic view that can be leveraged by intelligent, transparent, and adaptive applications to support and automate processes.
Addressing this gap is essential for the successful implementation of \aclp{DTE} making processes more efficient and supporting decision-making.

Towards this goal, we are interested in answering these research questions:

\begin{enumerate}[label=\textbf{RQ\arabic*}]

  \item\label{rq:1}
  \emph{How can we engineer \acp{DT} to integrate heterogeneous data sources and provide an up-to-date uniform representation of the mirrored assets?}
  \newline
  To effectively implement \acp{DTE}, it is necessary to investigate first how to engineer \acp{DT} that can mediate interaction with their physical counterpart and give access to external consumers to the asset's state and capabilities.

  \item\label{rq:2}
  \emph{How can we engineer \acp{DTE} that integrate heterogeneous \acp{DT} and support applications in interacting with them?}
  \newline
  Given the fragmented nature of \acp{DT} we need to ensure that \acp{DTE} can integrate heterogeneous \acp{DT} and offer features that facilitate the interaction of external applications with the set of \acp{DT} within an ecosystem.

  \item\label{rq:3}
  \emph{How can we support the operational management of \acp{DTE}?}
  \newline
  \acp{DT} have a complex development-to-deployment lifecycle that may involve multiple stakeholders. Supporting the operational management of \acp{DTE} requires addressing the challenges of supporting these stakeholders alongside the lifecycle, giving them access to a comprehensive view of the \acp{DT} involved in the ecosystem and possibly deployed to a complex cyber-physical computing infrastructure.
  
  \item\label{rq:4}
  \emph{What are the synergies between \acp{DTE} and \acp{MAS} for the development of intelligent applications?}
  \newline
  Given the dynamic nature of \acp{DTE}, applications interacting with \acp{DT} need to be flexible, and encapsulate strategies and rules to automate decision-making. In this thesis, we are interested to explore how (cognitive) \acp{MAS} can be used to this purpose and in general how can we exploit both autonomous agents and \acp{DT} to engineer intelligent cyber-physical applications.  


  \item\label{rq:5}
  \emph{How can we make the development of \ac{BDI} agents more accessible and easier to integrate with external systems such as \ac{DT}?}
  \newline
  \ac{BDI} agents are a powerful tool for modeling intelligent applications, but their development is often complex as they require a deep understanding of the underlying concepts. We are interested in exploring how to make the development of \ac{BDI} agents more accessible and easier to integrate with external systems looking at the design of new tools and frameworks.
\end{enumerate}


%--------------------------------------------------------
\subsection*{Contributions}
%--------------------------------------------------------

Despite being motivated by the digitalization of healthcare processes through \acp{DTE}, this thesis contributions can be interpreted in the broader scope of engineering \acp{CPS} with \acp{DT}.
%
The area of \ac{DT} engineering is a rapidly evolving field, with ongoing research aimed at improving the design, implementation, and management of \acp{DT} in various application domains where \acp{DT} are being employed~\missingref{}.
%
In this context, this thesis contributions can be applied to different application scenarios sharing similar challenges in terms of integration of heterogeneous data sources, heterogeneous assets connected by dynamic relationships.
We identify the following key contributions to the area of \ac{DT} engineering:
\note{Su ogni punto, citare i miei lavori pubblicati rilevanti}
\begin{itemize}
  \item Proposal of an architectural framework for the development of \acp{DT} focusing on integration of heterogeneous data sources and interoperability with external applications~\missingref{}.
  \item Definition of architectural patterns and best practices for the implementation of \acp{DT} to accurately capture synchronization requirements and augmented features~\missingref{}.
  \item Implementation of a prototype framework for the development of \acp{DT} that demonstrates the proposed architectural concepts and patterns~\missingref{}.
  \item Definition of the concept of \acp{DTE} and its role in the context of \ac{DT}-based systems~\missingref{}.
  \item Proposal for a framework to implement \acp{DTE} that can integrate heterogeneous \acp{DT} and expose a unified interface for external applications~\missingref{}.
  \item Implementation of a prototype platform and toolkit to integrate \acp{DT} into a \ac{DTE}~\missingref{}.
  \item Proposal of an abstraction layer to handle the operational management of \ac{DTE} and their deployment on a cyber-physical computing infrastructure~\missingref{}.
  \item Alignment of the \ac{DT} and \ac{MAS} concepts to establish a synergetic approach for the development of intelligent applications~\missingref{}.
\end{itemize}

Additionally, as this thesis leverages cognitive \ac{MAS} to support the development of intelligent applications that can integrate with \acp{DTE}, it contributes to the area of \ac{BDI} \ac{MAS} engineering by:  
\begin{itemize}
  \item Proposing new tools for the implementation of \ac{BDI} agents in order to facilitate their integration with other programming paradigms~\missingref{}.
  \item Exploring the use of simulation to test and verify \ac{BDI} \ac{MAS} in a controlled environment before deployment~\missingref{}.
  \item Proposing a conceptual framework to derive explanations of \ac{BDI} agents' behavior, enhancing the transparency and trustworthiness of \ac{BDI} applications~\missingref{}.
  \item Investigating the integration of \ac{BDI} agents with hypermedia environments~\missingref{}.
\end{itemize}


%=======================================================
\section*{Thesis Structure}
%=======================================================

This thesis is organized as follows. 
In \Cref{part:background} we provide the necessary context and background on the key concepts that underpin this research, including an overview of the challenges of healthcare digitalization and the reality of \ausl{} in \Cref{chap:back:health4.0}, moving on to introducing the concept of \ac{DT} its technological characterization  in \Cref{chap:back:DT}, and finally giving an overview of the concepts related to \ac{MAS} engineering that are relevant for this thesis in \Cref{chap:back:MAS}.

In \Cref{part:dte} we delve into the contributions towards the definition of the concept of \ac{DTE}. Namely, in \Cref{chap:dte:egineering-dt} we present architectural proposals and patterns for the development of \ac{DT} that can be integrated into a \ac{DTE}, then in \Cref{chap:dte:dte} we introduce the concept of \ac{DTE}, and explore its operational management and prototype implementation through the integration of heterogeneous \acp{DT} in \Cref{chap:dte:dtc} and \Cref{chap:dte:hwodt} respectively.

In \Cref{part:mas} we focus on the role of multi-agent systems in the development of intelligent applications, analyzing how they align with the requirements of healthcare applications in \Cref{chap:mas:requirements} before discussing their alignment for a synergetic exploitation with \acp{DTE} in \Cref{chap:mas:mas-dt} and the contributions to the engineering of \ac{BDI} agents towards the goal of making them more accessible for developers and users in \Cref{chap:mas:engineering}.

Finally, in \Cref{part:validation} we present the application scenarios and validation of the proposed approaches demonstrating their effectiveness in tackling healthcare challenges proposed by our partner \ausl{}
-- namely, pharmaceutical supply chain (\Cref{chap:val:irst}), trauma management (\Cref{chap:val:trauma}), and operating room management (\Cref{chap:val:orm}) --
before concluding by discussing results and presenting open challenges to be addressed in future works.

%****************************************************************************************
%****************************************************************************************
\part{Background and Context}
%****************************************************************************************
%****************************************************************************************

%%%%%%%%%%%%%%%%%%%%%%%%%%%%%%%%%%%%%%%%%%%%%%%%%%%%%%%%
\chapter{Healthcare Digitalization}
\label{chap:back:health4.0}
%%%%%%%%%%%%%%%%%%%%%%%%%%%%%%%%%%%%%%%%%%%%%%%%%%%%%%%%

%=======================================================
\section{Healthcare 4.0}
%=======================================================

%=======================================================
\section{\acl{HIS}}
%=======================================================

%=======================================================
\section{Medical Standards for Interoperability}
%=======================================================

%=======================================================
\section{The \ausl{} Context}
%=======================================================

%%%%%%%%%%%%%%%%%%%%%%%%%%%%%%%%%%%%%%%%%%%%%%%%%%%%%%%%
\chapter{\aclp{DT}}
\label{chap:back:DT}
%%%%%%%%%%%%%%%%%%%%%%%%%%%%%%%%%%%%%%%%%%%%%%%%%%%%%%%%

This chapter provides an overview of the \ac{DT} concept,
its historical development and the more recent models and technologies that have emerged
to support its implementation.
%
The ever-growing literature on the subject reflects the increasing interest in \acp{DT}
and its multi-faceted nature, encompassing various domains and applications.
%
In this thesis we focus on the software engineering perspective of \acp{DT}, 
and the application of \acp{DT} to the engineering of \ac{IoT} systems.
%
Accordingly, this chapter focuses on how the \ac{DT} concept has been interpreted in this context.


%=======================================================
\section{History and Definitions}
%=======================================================

The concept of \ac{DT} was introduced in the early 2000s by Michael Grieves
in the context of Product-Lifecycle Management~\cite{Grieves_2023}
where it was conceived as a virtual representation of a product throughout its lifecycle.
%
At its essence, Grieves defined what later came to be known as a \ac{DT} as a system composed of three main components (\Cref{fig:dt-grieves-original}):
\begin{itemize}
\item a \emph{physical space} and its products;
\item a \emph{virtual space} containing the digital representation of the products;
\item a \emph{connection} between the two, making data flow from the physical to the virtual space and information flow from the virtual to the physical space~\cite{Grieves2017}.
\end{itemize}

\begin{figure}[ht]
    \centering
    \includegraphics[width=0.7\textwidth]{figures/dt-original.png}
    \caption{The original \ac{DT} model by Grieves, from \cite{Grieves_2022}.}
    \label{fig:dt-grieves-original}
\end{figure}

Grieves referred to this concept as the \emph{Mirrored Spaces Model}~\cite{Grieves_2005},
a name which echoes the idea of \emph{Mirror Worlds} introduced by David Gelernter in the 1990s~\cite{gelernter1991mirrorworlds}
who also envisioned the ability to replicate the real world in a completely virtual space~\cite{Singh_Fuenmayor_Hinchy_Qiao_Murray_Devine_2021}.

The idea was later associated with its modern \emph{\acl{DT}} name and popularized by NASA in the 2010s, when it was presented as a key technology
for future development of aircraft and spacecraft in order to virtually simulate extreme conditions,
integrate data from the physical system in operation, 
and provide feedback on the conditions of the \emph{flying twin}
while possibly enacting changes to mitigate damage~\cite{glaessgen2012dtnasa}.
%
NASA's definition hence focused on the ability of the \ac{DT} to integrate multiple simulation models and characterized the \ac{DT} as \emph{ultra-realistic}, as the goal was to virtually replicate the physical system with the highest possible degree of fidelity:

\begin{quote}
    A Digital Twin is an integrated multiphysics, multiscale, probabilistic simulation of an as-built vehicle or system that uses the best available physical models, sensor updates, fleet history, etc., to mirror the life of its corresponding flying twin~\cite{glaessgen2012dtnasa}.
\end{quote}

Given the various phases of a product's lifecycle, the \ac{DT} concept was later
refined to distinguish between the \emph{Digital Twin Prototype} (DTP) and the \emph{Digital Twin Instance} (DTI) and \emph{Digital Twin Aggregate} (DTA)~\cite{Grieves2017}.
The DTP is the virtual representation of the product during its design and development phase,
while the DTI is the virtual representation of the product during its operational phase.
%
The DTA represents instead the aggregation of all DTIs, hence all products that have been built~\cite{Grieves_2022}.
%
Interestingly, the DTP exists before the physical product is even produced, 
and it is rather ``just'' a virtual model of the product. 
%
Grieves argues that requiring that a \ac{DT} exists only when the physical product exists is a \emph{fallacy}~\cite{Grieves_2022}.
%
Nevertheless, it is generally accepted within the community nowadays that, for a proper characterization, what is usually referred as \ac{DT} is hence a DTI, which exists alongside its physical counterpart and is continuously updated with data from the physical world. 

The role of such bidirectional data exchange between the physical and virtual parts
of a \ac{DT} has been, in fact, used to characterize the \ac{DT} concept and distinguish it from other related concepts.
%
A widely accepted taxonomy classifies the concepts of \emph{Digital Model}, \emph{Digital Shadow} and \emph{Digital Twin} based on direction of the automatic data flow between the physical and virtual spaces~\cite{kritzinger2018dtmanufacturing} (\Cref{fig:dt-taxonomy}):
namely, 
a Digital Model (\Cref{fig:dt-taxonomy-digital-model}) is a static model of a physical system (similarly to the DTP) that can be manually updated over time,
the Digital Shadow (\Cref{fig:dt-taxonomy-digital-shadow}) gets automatically updated through an inbound data flow from the physical space,
while the Digital Twin (\Cref{fig:dt-taxonomy-digital-twin}) is the only one that holds a bidirectional data flow which can provide feedback to the physical counterpart.
%
Such connection has been referred to as the \emph{digital thread}~\cite{Singh_Willcox_2018,Grieves_2023} to evoke the idea of a tie that binds the physical and virtual spaces.
%
Other common terms include \emph{twinning}~\cite{JONES202036} or \emph{shadowing}~\cite{Jiang_Yin_Li_Luo_Kaynak_2021,web-of-dt-ricci-2022} process. 
%
This terminology emphasizes the continuous and dynamic nature of the relationship between the physical and virtual entities, 
and highlights the active role of such process in maintaining the \ac{DT} up to date.

\begin{figure}[ht]
    \centering
    \begin{subfigure}[b]{0.3\linewidth}
        \centering
        \includegraphics[width=\linewidth]{figures/kritzinger-digital-model.pdf}
        \caption{Digital Model}
        \label{fig:dt-taxonomy-digital-model}
    \end{subfigure}
    \hfill
    \begin{subfigure}[b]{0.3\linewidth}
        \centering
        \includegraphics[width=\linewidth]{figures/kritzinger-digital-shadow.pdf}
        \caption{Digital Shadow}
        \label{fig:dt-taxonomy-digital-shadow}
    \end{subfigure}
    \hfill
    \begin{subfigure}[b]{0.3\linewidth}
        \centering
        \includegraphics[width=\linewidth]{figures/kritzinger-digital-twin.pdf}
        \caption{Digital Twin}
        \label{fig:dt-taxonomy-digital-twin}
    \end{subfigure}
    \caption{Taxonomy of Digital Twin related concepts based on the data flows between physical and digital objects, adapted from \cite{kritzinger2018dtmanufacturing}.}
    \label{fig:dt-taxonomy}v
\end{figure}

The scope and applicability of the \ac{DT} concept has evolved and expanded
to encompass a wide range of applications and domains.
%
Today, \acp{DT} are used in various fields such as manufacturing, healthcare, smart cities, and more.
%
This is also reflected in the abundance of terminology used to identify the physical counterpart of a \ac{DT}
which is nowadays often referred to as the \emph{physical twin} or
\emph{physical entity}~\cite{Singh_Fuenmayor_Hinchy_Qiao_Murray_Devine_2021,JONES202036,DBLP:journals/jss/DaliborJRSWWW22}.
%
This shift from identifying the physical counterpart as a \emph{product} to a more generic \emph{entity}
highlights the broader applicability of the \ac{DT} concept beyond its original manufacturing context.
%
The \ac{DT} concept has been adopted also to represent people~\cite{Shengli_2021},
processes (e.g., supply chain~\cite{Barykin_Bochkarev_Kalinina_Yadykin_2020}) and organizations~\cite{Parmar_Leiponen_Thomas_2020}, leading to a more abstract interpretation of what can be considered \emph{physical}. 
%
Accordingly, the definitions of \acp{DT} have also diversified, leading to a plethora of interpretations~\cite{DBLP:journals/jss/DaliborJRSWWW22}. 

For the scope of this thesis, 
since the focus is on how \acp{DT} can be used to engineer \ac{IoT} systems, taking as the reference context the healthcare domain,
we adopt the following definition, adapted from \cite{dt-IoT-context-Minerva-2020}:

\begin{quote}
A \acf{DT} is a comprehensive software representation of an individual \acl{PA}.
It includes the properties, conditions, and behavior(s) of the real-life asset through models and data.
A \ac{DT} is a set of realistic models that can simulate an asset's behavior in the deployed environment.
The \ac{DT} represents and reflects its physical twin and remains its virtual counterpart across the asset's entire lifecycle.
\end{quote}

The definition emphasizes the software nature of the \ac{DT} and its ability to model the properties and behavior of its physical counterpart.
%
We deliberately use the term \emph{\ac{PA}} to refer to the physical counterpart of a \ac{DT}
to highlight the fact that an \emph{asset} is something that has a strategic
value in the context of an application domain, and that the \ac{DT} is meant to represent and manage such assets.
%
Compared to the original Grieves model which characterized the \ac{DT} as having three parts (physical, digital and connection), for the reminder of this thesis we will instead consider:
\begin{itemize}
\item the \ac{PA} as the physical counterpart of a \ac{DT};
\item the \ac{DT} as the software implementing the virtual representation of the \ac{PA} through a combination of models and data;
\item the \emph{twinning process} implemented by the \ac{DT} software to keep the \ac{DT} up to date with the \ac{PA} and possibly provide feedback to it.
\end{itemize}

%=======================================================
\section{Reference Architectures and Properties}
%=======================================================

The conceptual definition of \ac{DT} gives little indication on how to implement it,
leaving a degree of freedom in the technical realization of the \ac{DT} software.
%
For this reason, a variety of reference architectures have been proposed, 
each emphasizing different aspects of the \ac{DT} concept~\cite{ferko2022architecting}.

One that has gained significant recognition in the \ac{DT} community is the \emph{five-dimensional model} (5D model) by Tao et al.~\cite{dt-driven-prognostics-tao-2018}.
%
As shown in \Cref{fig:dt-5d-model}, the 5D model characterizes a \ac{DT} as composed of five main components~\cite{qi2021enablingtechdt}:
\begin{itemize}
\item the \emph{physical entity} (PE), which is the physical counterpart of the \ac{DT};
\item the \emph{virtual models} (VM), which is the digital representation of the PE possibly including 3D models, rules and behavioral models;
\item the \emph{data} (DD), which includes all data related to the PE and possibly generated by the VM, such as historical data, real-time data, simulation results;
\item the \emph{service} (Ss), which encompasses all services provided by the \ac{DT} to users, such as monitoring, simulation, diagnostics, and optimization;
\item the \emph{connection} (CN), which represents the communication and data exchange between all other components, there including the bidirectional connection between the PE and the VM that is fundamental for the \ac{DT} operation.
\end{itemize}

\begin{figure}[ht]
    \centering
    \includegraphics[width=0.6\textwidth]{figures/5d-model.pdf}
    \caption{The 5D model of a Digital Twin.}
    \label{fig:dt-5d-model}
\end{figure}

Tao's 5D model extends Grieves' original model by explicitly including the data and services dimensions, better characterizing the \ac{DT} as a data-driven and service-oriented system. 


The service-oriented nature of \acp{DT} has an essential role in their ability to provide value to users.
%
The \ac{DT} is hence not just a passive representation of a \ac{PA}, but an active system that users can interact with to obtain insights and perform actions on the \ac{PA}.
%
This also contrasts the idea of the \ac{PA}-\ac{DT} system as a closed-loop control system, but rather envisions the \ac{DT} as open to interactions with external entities such as human users or other software systems.

Following this perspective, the idea of \ac{DT}-as-a-service has been proposed.
As shown in \Cref{fig:dt-as-a-service}, the proposed reference architecture adds a cyber layer and an application layer on top of the classic physical, digital and communication layers of a \ac{DT}~\cite{aheleroff2021aei}.
%
The reference architecture further highlights different levels of integration following the taxonomy of Digital Model, Shadow, and Twin~\cite{kritzinger2018dtmanufacturing}, with a further Digital Twin predictive level, which includes predictive models and analytics capabilities enabled by the cyber layer grounded on cloud technologies such as Big Data analytics. 
%
The vision additionally integrates \acp{DT} in an incremental and iterative development lifecycle, to accompany and evolve alongside the \acp{PA} they represent. 

\begin{figure}[ht]
    \centering
    \includegraphics[width=\textwidth]{figures/dt-as-a-service.jpg}
    \caption{The DT-as-a-Service reference architecture, from \cite{aheleroff2021aei}.}
    \label{fig:dt-as-a-service}
\end{figure}

Layered and service-oriented architectures for \acp{DT} are the most popular in the literature.
%
These architectures support desired non-functional quality attributes such as performance efficiency, reliability and maintainability, but also compatibility with existing systems and scalability~\cite{ferko2022architecting}

On the functional side, \cite{dt-IoT-context-Minerva-2020} proposes a set of properties \ac{DT} should have, analyzing existing literature and implementations in the \ac{IoT} context.
%
The properties, summarized in \Cref{tab:minerva-properties} detail various aspects that characterize functional behavior of \acp{DT}.
%
Interestingly, such properties go beyond the generic definition of \ac{DT} as a virtual representation of a \ac{PA} and highlights the various capabilities that a \ac{DT} can provide to its users.

First, both the \ac{DT} and the \ac{PA} are required to be \emph{univocally identifiable} in order to establish a clear relationship between the two. This vision supports the possibility of having possibly multiple \acp{DT} representing the same \ac{PA} in different contexts or for different purposes.
%
Accordingly, the second important contribution is the emphasis on \emph{contextualization}. 
This shifts the focus from the ultra-realistic \ac{DT} envisioned by NASA to a more pragmatic approach where the \ac{DT} is representative of the \ac{PA} in a specific context,
hence possibly abstracting away details of the \ac{PA} that are not relevant for the intended use of the \ac{DT} in the target system.
%
Additional properties emphasize the dynamic nature of the \ac{DT} and its ability to reflect changes in the \ac{PA} (\emph{reflection}), in a timely manner (\emph{entanglement}), persisting over time, memorizing past states and offering a reliable representation of the \ac{PA} (\emph{accountability}). 
%
Finally, the properties also highlight the ability of \acp{DT} to provide value to users through services (\emph{servitization}) such as aggregation of multiple assets (\emph{composability}), augmentation of the \ac{PA} capabilities (\emph{augmentation}), management of access control (\emph{ownership}), and anticipation of future states and behaviors of the \ac{PA} (\emph{predictability}).

\begin{table}[ht]
\centering
\caption{Characterizing Properties of \acp{DT}, defined in \cite{dt-IoT-context-Minerva-2020}.}

\begin{tabular}{p{3.5cm}|p{\dimexpr\textwidth-4.5cm\relax}}
\toprule
\midrule
\textbf{Property} & \textbf{Description} \\
\hline 
\hline
{Contextualization} & A \ac{DT} models the \ac{PA} in a way that is representative with regard to the target context. \\
\hline
{Reflection} & A \ac{DT} reflects changes in the \ac{PA} in real-time. \\
\hline
{Replication} & A \ac{DT} replicates an object into different environments. \\
\hline
{Entanglement} & The degree to which a \ac{DT} is interconnected with its \ac{PA}. \\
\hline
{Persistency} & A \ac{DT} persists over time even when the \ac{PA} is not available. \\
\hline
{Memorization} & A \ac{DT} stores and allows retrieving past states and events. \\
\hline
{Composability} & A \ac{DT} can aggregate different assets or \acp{DT}. \\
\hline
{Accountability} & A \ac{DT} recovers from errors and maintains a reliable state.\\
\hline
{Augmentation} & A \ac{DT} can add new capabilities to the \ac{PA}.\\
\hline
{Ownership} & A \ac{DT} manages access control over its data and functionalities.\\
\hline
{Servitization} & A \ac{DT} provides services to its users.\\
\hline
{Predictability} & A \ac{DT} can anticipate future states and behaviors of the \ac{PA}.\\
\hline
\bottomrule
\end{tabular}%
\label{tab:minerva-properties}
\end{table}



%=======================================================
\section{Technologies and Platforms}
%=======================================================

\note{Enabling Technologies}



\note{The shift from silos to platforms}

\note{Azure DT}

\note{Ditto}

%=======================================================
\section{\aclp{DT} and \acl{AI}}
%=======================================================

\note{This is here to serve the purpose of introducing "Cognitive Digital Twins"
and link to MAS}


%=======================================================
\section{Towards \aclp{DTE}}
%=======================================================

\note{From platforms to ecosystems}

\note{Digital Twin interoperability(?)}

\note{National Digital Twin}

\note{Web of Digital Twins}



%%%%%%%%%%%%%%%%%%%%%%%%%%%%%%%%%%%%%%%%%%%%%%%%%%%%%%%%
\chapter{\aclp{MAS}}
\label{chap:back:MAS}
%%%%%%%%%%%%%%%%%%%%%%%%%%%%%%%%%%%%%%%%%%%%%%%%%%%%%%%%

%=======================================================
\section{History and Definitions}
%=======================================================

%=======================================================
\section{Agent-Oriented Software Engineering}
%=======================================================

%=======================================================
\section{\acl{BDI} Agents}
%=======================================================

%=======================================================
\section{Web-based and Hypermedia \acs{MAS}}
%=======================================================

%=======================================================
\section{\aclp{MAS} and \aclp{DT}}
%=======================================================

%======================================================
\subsection{Synergistic Usage of AAs and DTs}
\label{ssec:synergy}
%======================================================


AAs and DTs are not only used separately 
to deliver intelligent functionalities 
in a mutually exclusive way, 
or as alternative solutions to overlapping requirements.
Their synergistic usage is already seen in the available literature, 
especially in industrial deployments. Here, we briefly report recent surveys that highlight the rich intersection of these research areas.

The contamination of AAs in DTs is mostly related to the need to adopt a variety of artificial intelligence techniques in IoT applications, both to enhance the modelling capabilities of complex assets and to implement automatic controls.
Minerva et al.\ present a classification of DTs in relation to the level of intelligent behaviour the DT exhibits~\cite{Minerva2023}.
They identify the ultimate level of intelligent DT as a proactive (or autonomic) DT, capable of enacting autonomous behaviour based on the physical counterpart's current or future context.
This has a clear link with AAs, that are primarily used to model and encapsulate autonomous and intelligent behaviour in software systems.

Authors of \cite{Pretel2024} deliver a systematic literature review considering both MAS to create DTs and MAS to exploit DTs. 
Table 2 therein highlights how manufacturing is the main use case for synergistic AAs and DTs usage. 
This suggests that Industry 4.0 enabled by IoT technologies is a driving force for AAs and DTs applications. 
Furthermore, Table 3 therein emphasises that 73\% of the surveyed papers do not disclose the AAs and DTs development framework 
and that almost 8\% propose their own 
    (second largest percentage behind a ``team'' of 4 agent-oriented development frameworks). 
This suggests a great fragmentation of uncoordinated proposals, 
generating ``reinvention of the wheel'', especially with respect to the integration of AAs and DTs.

The survey in \cite{10.1115/1.4050244}, instead, has its main focus on DTs considering whether and how AAs (especially MAS) are used to complement the functionalities of DTs. 
Of particular relevance for the present article is Figure 5 therein, 
which reports the vision of an ``Intelligent Product'' (IP) from the literature. 
Such IP fosters the synergistic usage of DTs and AAs. 
DTs help create an ``intelligent being'', 
whereas AAs augment it with an ``intelligent agent'' to make it an IP. 
The ARTI reference architecture is inspired by these concepts, 
and it has been among the first architectures proposed to consider 
the combined usage of AAs and DTs (and the most widely adopted). 
Finally, it is relevant that also here manufacturing and IoT emerge as the main application domains for the integrated usage of AAs and DTs.

The last survey we mention is \cite{Kalyani2025}, 
where the focus is on a one-way integration of AAs and DTs: 
AAs supporting DTs implementation. 
A practical example of such an integration is given in \cite{Latsou2021811}, 
where AAs are used to create a complex DT endowed with ``intelligent'' functionalities (e.g.\ prediction). 
An interesting takeaway that the authors of \cite{Kalyani2025} get from the surveyed literature 
is that a main driving force behind this specific integration is exactly endowment of DTs with AA properties. 
In Section~\ref{sec:abstractions-and-principles} we will see how this need is captured by our principles and proposed micro-architectures for AAs and DTs integration.

Providing the complete overview 
of all the available AAs and DTs-based architectures for IoT scenarios 
is out of the scope of this paper---the interested reader is referred to these surveys. 
However, it may be beneficial for the reader
to report some selected examples.
In \cite{Nie2022}, for instance, 
    an AA is used to aggregate information from different DTs in a manufacturing scenario, 
    to predict faults in machinery and (re-)optimise production scheduling.
Another example in manufacturing is in \cite{Xu2024}. 
    There DTs are used mainly as an abstraction layer over the whole manufacturing system, 
    providing a uniform access layer to AAs. 
    AAs are used mainly to support automatic feedback control (digital to physical) 
    and communication and coordination between robot AAs, task AAs, and workstation AAs.
In \cite{DBLP:conf/pads/ClemenALOOSG21}, instead, 
    AAs are used to build the complex DT of a whole city.

It is also worth noting that there may be literature, such as reference \cite{DBLP:journals/jiii/BiZWL22}, 
where the term ``autonomous agent'' does not appear, 
but whose proposal is well aligned with the literature about AAs. 
For instance, in the mentioned paper, 
a Digital Triad is proposed as an advancement beyond DTs, 
where design knowledge and application-specific intelligence is encapsulated 
in a very similar way to AAs. 
Surveying all of this ``submerged'' literature is a complex task given that terminology likely do not match. 
However, we hope that our contribution can promote cross-fertilisation with these related research efforts, 
in a joint effort to avoid reinventing the wheel---our main motivation for sticking to AAs instead of introducing brand new concepts into DTs.
    
%======================================================
\subsection{Intelligent functionalities for AAs and DTs}
\label{ssec:functions}
%======================================================

The literature on distributing intelligence in IoT discussed in previous sections reports on several intelligent functionalities for which AA and DTs are being actively used by system designers. 
%
We summarise and categorise them here regardless of the specific task they accomplish (e.g.\ time series forecasting vs.\ fault prediction) and of the specific technique adopted (e.g.\ regression, SVM, etc.), but focussing on the \textit{kind} of intelligent function they deliver.
%
This categorisation is useful first of all in defining what we can consider, for the scope of this paper, ``intelligence'' in a practical, bottom-up way, stemming from related works in the area without getting trapped in philosophical arguments. 
%
Then, it is also useful to establish the \emph{coverage} of such functionalities by AAs and DTs, as shown in Figure~\ref{fig:radar-aa-dt}, which already suggests that they can be used synergistically to deliver the full spectrum of such intelligence. 
%
\begin{itemize}
    \item \emph{Prediction}, there including time series forecasting, recommendations, namely any form of reasoning meant to ``guess'' new information based on past and current knowledge. 
    In the IoT, common prediction targets are machinery failures, stock levels, resource use, and system states. 
    \item \emph{Simulation}, encompassing any form of reasoning by hypothesising different states, in the past, present, or future. 
    ``What-if''-like analysis falls into this category. 
    In the IoT, it is common to simulate the future states of individual things, for example, to improve the design of a product or a production pipeline. 
    However, complete systems can also be simulated with an added degree of complexity.%usually integrate multiple AAs and DTs in hierarchical architectures. 
    \item \emph{Planning}, that includes any form of synthesis, and \emph{practical reasoning}~\cite{Bratman1988}, which is meant to figure out how to achieve some practical effect (on the target system) by properly sequencing available actions. 
    Planning to achieve a given system configuration is common in the IoT, as well as planning to reconfigure after some disruptions. 
    \item \emph{Inference}, within which we include both \emph{epistemic reasoning}~\cite{Meyer1995}, that is, synthesising novel information from known data, or data-driven inference such as pattern recognition and classification.
    This is perhaps the most common form of intelligence used in IoT, as even simple monitoring and control applications usually infer situations by aggregating different data coming from distributed sources of information. 
    In addition, fault detection and diagnosis can be gathered in this category. 
    %    \item \emph{Diagnosis} \ste{Tutti}{a pensarci bene è una forma di inference imho, che dite?} \samu{Ste}{La definizione di inference è veramente larga quindi ci cade dentro quasi tutto sì.. io ci ho fatto ricadere anche detection poi nella 4}
    \item \emph{Adaptation} instead gathers all the intelligent functionalities aimed at enabling the system to adapt to unforeseen situations that had not been explicitly managed at design time. 
    As in the previous category, this one is quite broad and includes several heterogeneous approaches, ranging from engineered adaptation (e.g.\ the MAPE-K loop~\cite{Oh2022}) to learning-based methods (e.g.\ evolutionary approaches~\cite{Eiben2005}). 
    \item Finally, we highlight the role of \emph{Learning}, which despite not usually being the primary goal for a functionality, is a valuable tool to implement some of the ones highlighted above and still poses important requirements on the architecture of a system that wishes to support any form of learning process in one of its components.
    That includes statistical machine learning, reinforcement learning, and causal structure learning~\cite{Bordini2020-AI,Erduran2023,Mariani2023a,Mariani2023}. 
    In this category, we group the functionalities that aim to make the system, or one of its components, learn to do something. 
    In IoT, the most common form of learning employs statistical machine learning to create prediction models, for example. 
\end{itemize}
%
The next Section refers to these categories of intelligent functionalities to illustrate why and how to use the design principles proposed therein.


%****************************************************************************************
%****************************************************************************************
\part{\aclp{DTE}}
%****************************************************************************************
%****************************************************************************************

%%%%%%%%%%%%%%%%%%%%%%%%%%%%%%%%%%%%%%%%%%%%%%%%%%%%%%%%
\chapter{Engineering \aclp{DT}}
\label{chap:dte:egineering-dt}
%%%%%%%%%%%%%%%%%%%%%%%%%%%%%%%%%%%%%%%%%%%%%%%%%%%%%%%%

In an attempt to answer \ref{rq:1}, 
this chapter presents our proposal for the engineering of \acp{DT}
focusing on the challenges of integrating heterogeneous sources of \ac{PA} data
and exposing \ac{DT} functionalities as services for digital applications.
%
Additionally, the chapter addresses the challenge of managing the \emph{lifecycle} of a \ac{DT}, 
which is crucial to ensure that the \ac{DT} can accurately reflect the state of the \ac{PA}, 
or signal when it may not be able to do so, in order for applications to make informed decisions on how to 
interpret the data provided by the \ac{DT}.
%
Finally, the modelling of \emph{augmented} functionalities of a \ac{DT} is discussed, with a modular approach
in order to decouple them from the primary \ac{DT} responsibility of \emph{shadowing} the \ac{PA} state. 
%
The resulting contribution is the proposal of an architectural framework and patterns for the engineering of \acp{DT}, 
which has been implemented in a supporting open-source framework: the \ac{WLDT} Java Library\footnote{\url{https://wldt.github.io}}.

%=======================================================
\section{An Abstract Architecture for \aclp{DT}}
\label{sec:dte:engineering-dt:abstract-architecture}
%=======================================================
Starting from the principles outlined by the 5D model~\cite{dt-driven-prognostics-tao-2018},
and incorporating insights from both recent academic research~\cite{web-of-dt-ricci-2022,Bellavista_Bicocchi_Fogli_Giannelli_Mamei_Picone_2023} and relevant standards from industry bodies~\cite{etsi-dt-comm-requirements-2024}.
This section presents a proposal aiming to generalize an abstract architecture that captures the core software requirements for \ac{DT} development.

%%%
\begin{figure}[t]
    \centering
    \includegraphics[width=\textwidth]{figures/mapping-Tao-WLDT.pdf}
    \caption{A side-by-side mapping between 5D DT modelling and the abstract \ac{DT} architecture.}
    \label{fig:tao_mapping_dt_modelling}
\end{figure}
%%%

In the 5D model (Figure \ref{fig:tao_mapping_dt_modelling}, left), 
the first dimension, Physical Entities (PE), represents the actual physical object or system being modeled as a \ac{DT}.
The second dimension, Virtual Entity (VE), is the digital representation of the physical entity, replicating its characteristics and behaviors.
The third dimension, Connection (C), ensures linkage between the physical and virtual entities, including communication technologies, data transmission protocols, and synchronization mechanisms for real-time interaction and data exchange.
The fourth dimension, Data (D), covers data management aspects, ensuring accurate and timely data for simulation, prediction, and optimization.
Finally, the fifth dimension, Service (S), involves various services enabled by the \ac{DT}, such as monitoring, simulation, prediction, control, and optimization, leveraging data and insights from the virtual entity to enhance the performance and efficiency of the physical system.

Moving this towards an abstract architectural perspective, a \ac{DT} can be described as the combination of three main high-level components as schematically illustrated in Figure \ref{fig:tao_mapping_dt_modelling} (right): 

\begin{itemize}
    \item \textit{\ac{PI}:} tasked with both the digitalization process and the ongoing synchronization of the \ac{DT} and \ac{PA} throughout their lifecycle based on its characteristic cyber-physical nature and the supported protocols and data formats (e.g., HTTP and JSON, MQTT and binary);
    \item \textit{\ac{DI}:} complementing the \ac{PI}, it manages the routing of \ac{DT}'s internal variations and events directed towards external digital entities and consumers ensuring the \ac{DT}'s interaction, interoperability and observability;
    \item  \textit{\ac{M}:} Defines the \ac{DT}'s behavior through the digitalization process together with augmented functionalities.
    The \textit{shadowing} process is responsible for handling events from both the \ac{PI} and \ac{DI} to model and replicate the state of the asset,
    while \textit{augmentation}~\cite{dt-IoT-context-Minerva-2020} functions extends functionalities of the \ac{PA} through additional features and capabilities supported by the software nature of the twin (e.g., simulation, machine learning inference models).
\end{itemize}


Table~\ref{tab:evaluation} discusses how the proposed architecture supports the core properties of \acp{DT}~\cite{dt-IoT-context-Minerva-2020} compared to both existing research from the scientific literature and features offered by major production-ready platforms, including Microsoft Azure, Eclipse Ditto, Amazon AWS, and Bosch IoT Suite.

These platforms mainly function as data repositories, storing data from physical assets, with pre-processing handled externally.
Data acquisition pipelines, supported protocols and data formats are often defined at the platform level, not on a per-\ac{DT} basis, leading to limitations in the shadowing process and client interactions.

In contrast, the proposed approach is \emph{event-driven}, where both the \ac{PI} and \ac{DI} are responsible for capturing events from their respective domains (physical and digital) and forwarding them to the \ac{M} for processing. This allows \acp{DT} to operate as independent software entities, communicating with the external world through finely configurable interfaces.

Following the proposed architecture, a \ac{DT} can be formally represented as:

\begin{equation} 
\label{eq:dt_basic_model}
        DT = \langle PI, M, DI \rangle\\
\end{equation}

Where the \ac{PI} captures data from the \acf{PA}, represented as a stream of time ordered \emph{physical} events $e_{PA}$, and controls physical actuation on the object through time ordered actions $a_{PA}$, 
%
the model \ac{M} encapsulates the \ac{PA} behavior, processes physical events $e_{PA}$, and generates time ordered \emph{digital} events $e_{DT}$,
%
and, finally, the \ac{DI} consumes digital events $e_{DT}$, exposes them to external consumers, and supports the invocation of \emph{digital} actions $a_{DT}$ to modify the behavior of the \ac{PA} through the \ac{DT}.

To better align with multimodel perspectives of \acp{DT}, the model \ac{M} can be actually interpreted as a set of models:

\begin{equation}\label{eq:multi_model}
        M = \{ m_{1}, m_{2}, \dots, m_{n} \} \\
\end{equation}


\begin{table}
\renewcommand{\arraystretch}{1.2}
\centering
\small
\begin{tabularx}{\textwidth}{>{\arraybackslash}m{2.5cm} >{\arraybackslash}X}
\hline
\textbf{Property} & \textbf{Description} \\ \hline

\multirow{2}{*}{\parbox[t]{2.8cm}{\raggedright\arraybackslash\hspace{0pt}\textbf{Representativeness \& Contextualization}}}
& \textbf{Traditional:} DTs are limited to store properties and relationships received from external entities, and do not implement a model or a behavior themselves. DTs fail to encapsulate the PA, and the responsibility of the contextualization is entirely externalized. \\
& \textbf{Event-Driven:} DTs are active entities aware of the characteristics of the PAs through dedicated adapters able to support a fine-grained representation and contextualization thanks to internal models and the event-driven approach. \\ \hline

\multirow{2}{*}{\textbf{Shadowing}} 
& \textbf{Traditional:} The shadowing is not correctly encapsulated within a DT and is dispersed across different components. The constraint to rely on a fixed set of communication protocols and platform-specific data formats creates a strong vendor lock-in, limiting re-usability. \\
& \textbf{Event-Driven:} DT directly encapsulates the functionalities and responsibilities to communicate with the PA through the adoption of any required protocols and data formats and decouples the responsibilities of interacting with the PA through re-usable and interoperable modules. \\ \hline

\multirow{2}{*}{\textbf{Replication}} 
& \textbf{Traditional:} Adoption of centralized and monolithic solutions (e.g., Cloud-only) represents a critical limitation to cross-domain interactions, scalability, and fault tolerance since DTs are designed as passive singleton instances operating in isolated tenants. \\
& \textbf{Event-Driven:} The event-driven approach natively supports the creation of multiple active, independent, and modular DTs interacting by means of events and operating through multi-layered and cross-domain deployments. \\ \hline

\multirow{2}{*}{\textbf{Composability}} 
& \textbf{Traditional:} The composition logic and its management (regardless of whether it is linking or an effective interconnection) are fully externalized, and the DT is passively subject to them. \\
& \textbf{Event-Driven:} Each DT is active and directly handles the composition process through behavior defined by its model, supported by dynamic adapters and a native, event-based notion of observability. \\ \hline

\multirow{2}{*}{\textbf{Augmentation}} 
& \textbf{Traditional:} DT augmentation is a concept entirely missing from state-of-the-art platforms, and it is outsourced to external computational services. \\
& \textbf{Event-Driven:} Augmentation is implemented through a flexible plug-in mechanism allowing the injection of novel functionalities and relying on existing modules to effectively augment the DT. \\ \hline
\end{tabularx}

\caption{Comparison of DT capabilities: limitations of existing (traditional) approaches versus the benefits of the proposed event-driven solution.}
\label{tab:evaluation}
\end{table}

In the proposed approach, the core functionality of the \ac{DT} is the shadowing process, which handles the transformation of data received from the \ac{PI}, feeding the models that characterize the \ac{DT}.
%
The result of the shadowing process is to compute the state of the \ac{DT} $S_{DT}$, which---following the \ac{WoDT} metamodel~\cite{web-of-dt-ricci-2022}---can be represented through sets of: 
\begin{itemize}
\item \textbf{Properties} ($P$): labeled data that change with the evolution of the \ac{PA} state;
\item \textbf{Relationships} ($R$): capturing the existing and dynamic connections among \acp{PA} within the system, mirroring them between the corresponding \acp{DT}; 
\item \textbf{Events} ($E$): non-persistent signals captured by the information gathered from the associated \ac{PA};
\item \textbf{Actions} ($A$): the possible operations that the \ac{DT} allows to be invoked on behalf of the \ac{PA}, to either send feedback to the physical entity or exploit a service exposed by the \ac{DT}.
\end{itemize}

At any given time $t$, the state of the \ac{DT} can hence be represented as: 

\begin{equation}
    S_{DT} = \langle P, R, E, A, t \rangle \\
\end{equation}

Computed states can be stored to maintain a history $H$ of the \ac{DT}, which can be modeled as a time ordered sequence of states:

\begin{equation}\label{eq:dt_history}
        H = \langle S_{DT_1}, S_{DT_2}, ... , S_{DT_n}\rangle\\
\end{equation}

Additionally, to fulfil the \emph{shadowing} process, 
state updates can then be communicated to applications through the \ac{DI}, fulfilling the objective of the \ac{DT} in providing an up-to-date representation of the \ac{PA}.
%
We can represent the shadowing as a function $Shad$ that given an event from the \ac{PI} $e_{PA}$, the set of models $M$, and the history of previous states $H$, computes the new state of the \ac{DT} and generates an event $e_{DT}$ to be sent through the \ac{DI}:

\begin{equation}
        e_{DT}(S_{DT}) = Shad_{PA \rightarrow DT}(M, e_{PA}, H) \\
\end{equation}

The shadowing is bidirectional:
when actions are invoked, the \ac{DT} receives an action request $a_{DT}$ on its \ac{DI}, validates it through the models $M$ possibly evaluating the history $H$, and then it may propagate the request through its \ac{PI} sending a command $a_{PA}$ to have an effect on the \ac{PA}.

\begin{equation}
    a_{PA} = Shad_{DT \rightarrow PA}(M, a_{DT}, H)
\end{equation}

It is crucial to note that a digital action request does not directly change the state of the \ac{DT}.
Any changes of the state occur only as a result of the propagation of physical changes from the \ac{PA} to the \ac{DT}.
This abstract architectural model addresses the main functional requirements of a \ac{DT}.
It provides a clear separation of concerns between different components focusing on the \emph{shadowing} process as the core responsibility of the \ac{DT} encapsulating the complexity of physical and digital interactions through the \ac{PI} and \ac{DI}. 


Finally, considering these additional specifications, the resulting \ac{DT} model enhances (\ref{eq:dt_basic_model}) to include also the shadowing $Shad$ and the history of states $H$ in the \ac{DT} definition:

\begin{equation}\label{eq:5D_dt_model}
    DT = \langle PI, M, Shad, H, DI \rangle
\end{equation}

In the following sections the abstract architecture will be refined to highlight how it can concretely be implemented to address challenges in \ac{DT} engineering.


%=======================================================
\section{Modular Design of Digital Twin Interfaces}
%=======================================================

As introduced in \Cref{sec:back:dt:interoperability}, interoperability is a key requirement for \ac{DT} development~\cite{Acharya_Khan_Päivärinta_2024,Klar_Arvidsson_Angelakis_2024}.
%
Considering interoperability as the ability to integrate effectively with other system components,
in the \ac{DT} context, such ability can be considered twofold due to their cyber-physical nature.

Figure \ref{fig:dt_application_example} represents a \ac{DT} implemented following the abstract architecture presented in Section \ref{sec:dte:engineering-dt:abstract-architecture} in an exemplary deployment setting.
%
It highlights the role of the \ac{DT} as a bridge between \emph{physical} devices and \emph{digital} applications.

On the one hand, \acp{DT} must effectively integrate with a heterogeneous physical world, interacting with diverse \ac{IoT} devices and systems that operate on different standards, protocols, and architectures.
%
For example machines from different vendors may utilize different communication protocols and interaction patterns.
%
Similarly, in healthcare applications, \acp{DT} must integrate various electronic health records across different data sources, from traditional repositories to sensor networks.
%
This \emph{physical} heterogeneity poses a challenge for the development of \acp{DT}, as, most often, it is unfeasible to modify the existing conditions and the \ac{DT} system must be constructed on top of the legacy ones \emph{adapting} to their requirements.

On the other hand, although \acp{DT} could work in isolation, they are increasingly being envisioned as part of complex software systems.
%
Thus, \acp{DT} needing to integrate with different kinds of services ranging from system automation tools, to data platforms, machine learning pipelines and visualization tools, may expect to face several degrees of \emph{digital} heterogeneity.
%
As for the physical realm, the \ac{DT} might need to \emph{adapt} to the requirements of the digital ecosystem it is part of.
%
Additionally, the modeling of complex \ac{CPS} can be tackled through the idea of \acp{DTE}~\cite{web-of-dt-ricci-2022} which envision multiple connected \acp{DT}, each modeling individual entities of the same domain. 
Such \acp{DTE} can benefit from standardized interactions between \acp{DT} and with higher-level services.

\begin{figure}[t]
    \centering
    \includegraphics[width=\columnwidth]{figures/dt_application_example.pdf}
    \caption{Example of a DT in an application scenario interacting with a physical asset and digital applications through its PI and DI.}
    \label{fig:dt_application_example}
\end{figure}


Tackling such twofold interoperability challenges requires on one side
the adoption of common data models, semantic interoperability frameworks, and compliance with industry standards~\cite{etsi-dt-comm-requirements-2024} such as OPC-UA\footnote{OPC-UA: \url{https://opcfoundation.org/about/opc-technologies/opc-ua/}}, oneM2M\footnote{oneM2M: \url{https://www.onem2m.org/}}, ISO-23247\footnote{ISO-DigitalTwin: \url{https://www.iso.org/standard/81442.html}}, or the W3C Web of Things\footnote{W3C WoT: \url{https://www.w3.org/WoT/}}.
%
On the other side, the \ac{DT} architecture should accommodate physical and digital heterogeneity \emph{by design} to avoid being locked down in those situations where adherence to a single standard can not be enforced.
Without such measures, \acp{DT} risk becoming fragmented, limiting their ability to collaborate and exchange information across different platforms and applications.
%
The proposal of modular \ac{DT} interfaces aims to address these concerns
at the architectural level, to enhances the interoperability and re-usability of \ac{DT} components in cyber-physical applications.

At an appropriate abstraction level, a \ac{PA} may be digitalized by obtaining and sending data through possibly \emph{several} different \texttt{Communication Channels} (\Cref{fig:dt_application_example}), which 
encompass network protocols, data formats, and physical connections.
The \ac{PI} manages interactions with the \ac{PA}, integrating such channels.
%
Similarly, on the digital side, the \ac{DT} supplies data and insights to applications via the \ac{DI}, adapting to different protocols and representations.

Engineering the \ac{PI} and \ac{DI} requires addressing the challenge of effectively integrating such heterogeneous communication channels.
%
To ensure flexibility and adaptability, it is essential that the core of the \ac{DT}, including its models and implemented behaviors, remains decoupled from the heterogeneous physical and digital communication channels.
%
The \ac{DT}'s shadowing process should focus solely on understanding the available physical data, receiving and transmitting telemetry information, and executing action requests, with no concern for the underlying implementation details.

The remainder of this section describes a proposal for the design the \ac{DT}'s \ac{PI} and \ac{DI}, refining and extending the conceptual model originally proposed in \cite{web-of-dt-ricci-2022} with a concrete implementation based on the concept of modular \emph{adapters}.

%..........................................................
\subsection{Physical Interoperability}
\label{ssec:dte:dt-engineering:physical_interoperability}
%..........................................................

\begin{figure}
    \centering
    \includegraphics[width=\columnwidth]{figures/dt-interoperability/dt_interoperability_physical.pdf}
    \caption{PI design with modular physical adapters each producing the corresponding \acl{PAD} that is processed by the \ac{DT} Model.}
    \label{fig:physical_interoperability}
\end{figure}


One of the main challenges in facilitating effective communication through the \ac{PI} is the wide variety of communication protocols and data formats used by \acp{PA}.
%
While it can be the case that one \ac{PA} is directly sending all the data concerning its digitalization through only one channel,
it is far more common to build a \ac{DT} aggregating different sources of information~\cite{qi2018dt-and-bigdata}.
%
\ac{IoT} protocols often differ considerably based on the manufacturer, device type, or application domain, making it difficult for the \ac{DT} software to integrate diverse assets.
%
Arguably, the modularity of the \ac{PI}, which encapsulates these responsibilities, is a crucial factor in the design and implementation of interoperable \acp{DT}.
%
Accordingly, the proposal is to design the \ac{PI} as a composition of multiple \emph{\acp{PhA}}: specialized modules capable of interacting with the \ac{PA} using diverse protocols and data formats.
%
As shown in Figure \ref{fig:physical_interoperability}, each \ac{PhA} is responsible for mediating the bidirectional interaction through a single communication channel, simplifying the design and reusability of the component, and making it configurable to adapt to different application contexts.
%
The responsibility of the \ac{PI} becomes then to manage the different \acp{PhA} and make sure the \ac{DT} model can accurately interpret, process, and leverage the data generated by the physical world to create the digital replica and implement its behaviors.
%
Note that although terminology of \ac{PhA} is borrowed from \cite{web-of-dt-ricci-2022}, where it is originally introduced, the \emph{Physical Asset Adapter} proposed in \cite{web-of-dt-ricci-2022} is in fact a conceptual monolithic component, whereas in this concrete proposal a \ac{PhA} is considered as a single-responsibility module of a potentially more complex \ac{PI}.

\paragraph{Generating \aclp{PAD}}
To facilitate managing different \acp{PhA}, the concept \emph{\ac{PAD}} is introduced: a representation of the capabilities provided by a communication channel in terms of \emph{properties}, \emph{actions}, \emph{events}, and \emph{relationships} that characterize the associated \ac{PA}.
%
Each \ac{PhA}, since it encapsulates the characteristics of a channel, can generate the corresponding \ac{PAD}, effectively decoupling the asset's functionality from the specific communication protocols used.
%
The implementation of the \ac{PAD} generation can be challenging due to the varying capabilities of different communication protocols.
For instance, protocols like \ac{CoAP}~\cite{RFC7252} often come with built-in description and discovery functionalities, which can simplify the process of creating a \ac{PAD} by providing standardized representations of the physical asset's capabilities and behaviors.
On the other hand, protocols like \ac{MQTT}~\cite{MQTTv5} may require developers to add additional information manually to define how information is structured and exchanged, as they do not natively include asset metadata.
In the case of custom or proprietary protocols, the challenge becomes even more pronounced, as these protocols are tailored to specific systems and may lack any standardization or descriptive capabilities.

In all the aforementioned scenarios, the mechanism of \ac{PAD} generation offers a way to encode knowledge about the \ac{PA} bridging the gap by interpreting and extracting relevant information from the protocol used in the associated communication channels.
%
Confining this complexity within the \ac{PI} allows the \ac{DT} model to be agnostic with regard to the complexity of the underlying physical world.

The \ac{PAD} allows the \ac{PI} to discover, extract, and manage asset information and present it to the model that can choose which ones are relevant for the implementation of the \ac{DT} behavior.
%
For example, to create the \ac{DT} of a temperature sensor streaming binary data on MQTT, the \ac{PI} could be composed of a generic MQTT adapter, configured to correctly parse the payload as a decimal number, and generate a \ac{PAD} which advertise the available temperature property to the \ac{DT} model as a Celsius value. 


\subsection{Digital Interoperability}
\label{sec:digital_interoperability}

%%
\begin{figure}[t]
    \centering
    \includegraphics[width=\columnwidth]{figures/dt-interoperability/dt_interoperability_digital.pdf}
    \caption{Digital Interface design with modular adapters and DT description.}
    \label{fig:digital_interoperability}
\end{figure}
%%

As \acp{DT} are meant to bridge between the physical and digital spaces,
interoperability is not only a matter of interfacing with heterogeneous devices, but also other digital applications.
Indeed, despite their initial conception as vertical silos, the concept of \ac{DTE} (\Cref{sec:back:dt:dte}) is emerging to support the the digitalization of complex scenarios through a combination of several \acp{DT}.
%
In this context, \acp{DT} can be used \emph{as-a-service} by other applications implementing the business logic by spanning across different software entities.

To foster interoperability in \acp{DTE}, \acp{DT} must then expose either a standardized general purpose \acf{DI} that can serve different applications or
---following the same design principles adopted to address physical interoperability---
have a modular interface that can satisfy the different needs of different applications
as shown in Figure \ref{fig:digital_interoperability}.

As for \ac{PhA} the terminology is borrowed from \cite{web-of-dt-ricci-2022} and consider to have a \ac{DI} composed of modular \emph{\ac{DiA}}.
The original abstract concept of \ac{DiA} is hence refined to represent a modular component of a more complex \ac{DI}.
Using the concept of \ac{DiA}, the \ac{DI} can expose the \ac{DT} state and services supporting multiple data formats and interaction patterns.
This can be beneficial for integrating it with applications and, especially, legacy systems.
The existence of legacy applications usually implies having little control over the integration requirements, making having more flexible \acp{DT} beneficial to better integrate with the application requirements.

Using several \ac{DiA}, a \ac{DT} could, for instance, support both request-response and publish-subscribe mechanisms to access its current and previous states, support different query languages to access the same data store, expose its current state using different representation formats, etc.
%
This would make the development of the application simpler because adding an application-specific \ac{DiA} won't require intervention in the underlying physical system. 
%
Additionally, the developed application would be more robust and stable since it would depend only on the \ac{DT}, and changes to the physical configuration of the \ac{PA} (e.g., software updates, sensor replacement, network reconfiguration) would have no impact on the application software.
%
Even if the \ac{PA} were to change, (e.g., a software update on a robot changes the telemetry format) the \ac{DI} of the \ac{DT} could stay the same, as the changes would occur within the boundaries of the \ac{PI} and \ac{DT} model.
%
Through this mechanism, \acp{DT} effectively achieve their bridging role, shielding applications from the complexity of physical deployments.

\paragraph{Describing \aclp{DI}}

A further level of interoperability is possible when allowing \acp{DT} to describe their \ac{DI}, advertising capabilities and available communication channels that applications can exploit.

This is especially relevant in contexts where applications are not bound to a specific interface but can instead process machine-readable descriptions of \acp{DT} and choose to interact with specific assets.
%
A \ac{DT} could then use a \emph{\ac{DTD}}, which, similarly to the \ac{PAD}, can represent the features of the \ac{DT} to its observers.
%
The idea of exposing \acp{DTD} is somewhat present in the major platforms supporting the creation of \acp{DTE}, such as 
Microsoft's \azureTwin{}\footnote{\azureTwin{}: \url{https://azure.microsoft.com/en-us/products/digital-twins}} and \ditto{}\footnote{\ditto{} \url{https://eclipse.dev/ditto/}} and is advocated by standardization activities on interoperability in \acp{DTE}~\cite{etsi-dt-comm-requirements-2024}.

The way such descriptions are implemented may differ significantly, but essentially, they should at least allow representing the \ac{DT}  features and APIs to access them.
%
Notably, differently from the \ac{PAD}, the \ac{DTD} is targeted to external consumers, hence, it should preferably adhere to standard formats and representations to be useful in practice in achieving interoperability.
%
To this end, using Semantic Web technologies (see \Cref{sec:back:web:semantic-web-technologies}) could be a possible solution to implement standard machine-readable \acp{DTD}.
\todo{add forward ref to semantic web descriptors sections}

% \begin{figure*}[t]
%     \setlength{\belowcaptionskip}{-13pt}
%     \centering
%     \includegraphics[width=0.93\linewidth]{figures/dt-interoperability/mf_dt_structure.pdf}
%     \caption{The \acl{DT} ecosystem architecture of the microfactory industrial system.}
%     \label{fig:mf_dt_ecosystem}
% \end{figure*}

\subsection{\acl{WoT}: enabling DT Interoperability}
\label{sec:wot-dt-interop}

\ac{WoT}~\cite{wot-arch} standards (\Cref{chap:back:web:WoT}) share the goal of adopting uniform API and data format descriptions to hide low-level complexities and promote interoperability with the proposal of self-descriptive \ac{PhA} and \ac{DiA}.

Despite their similarities, \acp{PAD} and \acp{TD} serve distinct purposes. 
A \ac{TD} provides a structured description of a physical twin's available interactions to external consumers, detailing protocols and interaction possibilities.
%
In contrast, a \ac{PAD} is designed for internal use by the \ac{DT} modules, decoupling the twin's core from the complexities of communication channels.
Its primary role is to facilitate the discovery of available resources and capabilities on the \ac{PA} after establishing a connection with the physical counterpart, providing a structured description of its functionalities and data.
This description is directly interpretable by the \ac{DT} model and independent of the physical characteristics, allowing the reuse of the same model for similar \acp{PA} employing different communication channels.

In contexts in which \ac{WoT} standards are applicable, the devices' \acp{TD} can be automatically mapped to \acp{PAD}, streamlining the \ac{DT} creation process.
Leveraging \ac{WoT} \acp{TD} can significantly reduce the effort required to generate \acp{PAD}, particularly in environments where numerous physical twins already have associated \acp{TD}.
This convergence between \ac{WoT} and \ac{DT} architectures has the potential to accelerate the development and deployment of \ac{DT} solutions by promoting standardization and reusability.
%
Nevertheless, the more general concept of \ac{PAD} can be tailored also to those scenarios in which \ac{WoT} is not applicable. In those cases the responsibility falls back to development (or configuration) of a \ac{PhA} for a specific device to encode the necessary knowledge for the generation of the \ac{PAD}.

\ac{WoT} \acp{TD} can also be used as \acp{DTD}.
The \ac{WoT} architecture actually lists \acp{DT} as one of the possible deployment patterns, with a \ac{DT} mediating the interaction with a \ac{PA} behind a \ac{WoT} interface\footnote{\url{https://www.w3.org/TR/2023/REC-wot-architecture11-20231205/\#digital-twins}}.
%
This is especially useful when devices are not \ac{WoT} compatible or can not be made so due to other constraints, essentially retrofitting the capabilities of the devices with a \ac{WoT} compatible interface.
%
Adhering to \ac{REST} constraints~\cite{fielding2000architectural} and specifically to the HATEOAS and self-descriptive principles, a \ac{TD}-based \ac{DTD} facilitates the discovery of the \ac{DT} model and services.
%
Its machine-readable nature ensures interoperability for applications and consumers.
%
In particular, a \ac{TD}-based \ac{DTD} facilitates the inclusion of \acp{DT} in \ac{WoT} mashups, promoting \acp{DT} as valid \ac{WoT} Things usable by \ac{WoT} Consumers.

Despite its flexibility,
and broad applicability, 
using of \ac{TD} for \acp{DT} presents several limitations in capturing the specific characteristics of \acp{DT} that distinguish them from \emph{Things}.
%
A path to address these limitations could be extending \acp{TD} or supporting additional descriptions specifically for \acp{DT} in \acp{DTE} as explored in \Cref{chap:dte:hwodt}.



%=======================================================
\section{Managing the Digital Twin Lifecycle}
%=======================================================


Recently, there has been an increasing recognition of the importance of the lifecycle of \acp{DT},
particularly in distinguishing the properties that define the relationship between the \ac{DT} and \ac{PA}. 
%
Key concepts such as \emph{reflection} and \emph{entanglement} are critical for accurately representing the \ac{PA}~\cite{dt-IoT-context-Minerva-2020,web-of-dt-ricci-2022}.
%
These properties underscore the necessity for a structured lifecycle for the \ac{DT}, ensuring that its state remains consistently aligned with the \ac{PA} throughout various stages.
%
As highlighted in recent surveys~\cite{ferko2022architecting, 9640612,Hribernik_Cabri_Mandreoli_Mentzas_2021}, while the body of literature on \ac{DT} software architectures is growing, most papers are domain-specific and focus on reference models.
%
However, these models often lack concrete guidance on how to implement the internal components of a \ac{DT}. Many of the existing models envision multiple parallel components or models working together, but fail to address how they communicate and interact to maintain consistency in the \ac{DT} state.
This gap leads to potential inconsistencies between the \ac{DT} and \ac{PA}, as the interaction between models and the management of state changes is not adequately captured~\cite{alam2017access,Malakuti2019fourlayer,Lpez2021}.


% In \cite{Redelinghuys2019}, a six-layer architecture for \acp{DT} is proposed, where the first two layers are dedicated to the \ac{PA} (sensors and controllers), and the third to the fifth layers manage data storage, communication, and cloud integration. However, the architecture lacks a clear process for how these layers communicate with each other or how changes in the \ac{DT} state are captured and updated across the layers. Similarly, in the Generic \ac{DT} Architecture (GDTA)~\cite{app10248903}, a layered approach is used, where the \ac{DT} state is computed at the \emph{information layer} through data processing pipelines. However, this architecture does not incorporate an explicit mechanism for the integration of the different components of the \ac{DT}, leaving gaps in how the evolving state of the \ac{DT} is managed. Other models explicitly reference multiple components that contribute to the definition of the \ac{DT} state. For example, in \cite{alam2017access} and similarly in \cite{Malakuti2019fourlayer}, the authors propose multi-layer \ac{DT} architecture for cyber-physical systems (CPS), where \ac{DT} state changes are driven by multiple functional units. However, in these models, the outputs of each unit are not mediated by a shadowing process, which would ensure consistency and synchronization of the \ac{DT} state and lifecycle management. Our approach addresses these issues by proposing a more structured lifecycle for \acp{DT}, where the communication and interaction between the different components are clearly defined. By introducing a shadowing process that orchestrates the different models, we ensure that the \ac{DT} state remains consistent and accurately reflects the evolving state of the \ac{PA}.


\begin{figure}[t]
    \centering
    \includegraphics[width=\columnwidth]{figures/wldt_lifecycle_simple.pdf}
    \caption{DT Life Cycle with its phases and transitions.}
    \label{fig:dt_life_cycle}
\end{figure}


%START FROM HERE

%-------------------------------------------------------
\subsection{A Digital Twin Synchronization Lifecycle}
%-------------------------------------------------------

The concept of a \ac{DT} lifecycle has recently been introduced and examined, focusing on its core components and the integration of both the software nature of the \ac{DT} and its ability to synchronize with the \ac{PA} over time~\cite{web-of-dt-ricci-2022}.
%
This lifecycle encompasses the various phases that a \ac{DT} undergoes, as shown in Figure~\ref{fig:dt_life_cycle}, from its creation to deactivation.
%
Since a \ac{DT} is fundamentally a software entity, it is vital to track the evolution of such phases considering the different stages of synchronization with its \ac{PA}.
Proper observation and modeling of this lifecycle are essential to ensure that the \ac{DT} representation can be trusted to reflect the physical counterpart's state consistently when the \ac{DT} is \texttt{Synchronized} and can instead be considered to be outdated in the other phases.

The execution lifecycle of a \ac{DT} can be modeled as follows.
Upon creation (from the \texttt{Started} phase),
the \ac{DT} enters the \texttt{Unbound} phase, where all internal modules are initialized and prepared for the binding process with the \ac{PA}.
%
Once the \ac{DT} successfully binds to its \ac{PA},
meaning the \ac{PA} meets the necessary requirements (expected functionalities and properties)
and the \ac{DT} can communicate with it,
the \ac{DT} enters the \texttt{Bound} phase.
%
During this phase, the \ac{DT} is connected to the \ac{PA} and is ready to begin the shadowing process.

The next phase is the \texttt{Synchronized} phase, 
this is different from simply having established a connection with the \ac{PA}, 
but rather indicates that the \ac{DT} is complying with application-specific requirements
and is receiving the necessary and correct amount of information to be fully aligned with its \ac{PA}.
%
During the synchronization phase the \ac{DT} can compute the \ac{DT} state ($S_{DT}$), ensuring that it is able to accurately reflects the status of its physical counterpart.

If any issues arise, such as network failures, and the \ac{DT} synchronization performance falls under the expected requirements, the \ac{DT} enters the \texttt{Out of Sync} phase.
In this state, the \ac{DT}---while still operational---is unable to update its state reliably. 
%
Eventually, the \ac{DT} may return to the \texttt{Synchronized} phase.

Finally, when the \ac{DT} is no longer required or has completed its function, it transitions to the \texttt{Done} phase.
In this phase, the \ac{DT} remains accessible to external consumers and retains its memory, but it is no longer bound to the \ac{PA} or in sync with it. At the end of the lifecycle, the \ac{DT} can be dismissed and moved to the \texttt{Stopped} phase.
Throughout its lifecycle, the \ac{DT} may also return to the \texttt{Unbound} phase if errors are detected during the binding process or if it is rebooted or restarted.


%-------------------------------------------------------
\subsection{Binding the DT with the PA}
%-------------------------------------------------------

The transition from the Unbound to Bound phase in the \ac{DT} lifecycle remains an open research area:
a key challenge is identifying whether all the capabilities of the \ac{PA} relevant to support the \ac{DT}'s behavior are available so that the shadowing process can start.

The modular design of the \ac{PI} presented in the previous section (\Cref{ssec:dte:dt-engineering:physical_interoperability}) can have support this transition.
%
Namely, once a \ac{PhA} successfully connects to the \ac{PA}'s channel and starts receiving data from it, it can send the generated \ac{PAD} to the \ac{DT} model.
The generation of the \ac{PAD} can be used as a synchronization step to signify that the \ac{PhA} is successfully connected to the \ac{PA}.
%
The \ac{DT} model is then responsible for collecting the different \acp{PAD}, assessing whether all the relevant information to start computing the \ac{DT} state is available and consequently moving on to the \texttt{Bound} phase.
%
This mechanism enables managing the \ac{DT} behavior consistently, even with the additional complexity of the modular \ac{PI} design.

Of course the challenge still stands in managing each individual \ac{PhA} connection, but this decoupling through \acp{PAD} allows \ac{DT} developers to establish functional binding requirements that are independent of technical details. 

%--------------------------------------------------------
\subsection{Managing DT and PA Synchronization}
%--------------------------------------------------------


The most significant challenge in current \ac{DT} lifecycle modeling is the broad characterization of the \texttt{Synchronized} phase, which
only generically considers synchronization requirements between the \ac{PA} and the \ac{DT}, without having explicit awareness of the \ac{PA} lifecycle and the potential changes in its operational context.
This general and unstructured approach limits the ability to model the lifecycle in a consistent and uniform manner, failing to properly address the dynamism of the \ac{PA} behavior.

The \texttt{Synchronized} phase of the \ac{DT}'s lifecycle assumes a continuous exchange of information between the \ac{DT} and its associated \ac{PA} usually measured to maintain a frequency within a given threshold, to guarantee that information on the \ac{DT} is \emph{fresh} and can hence be trusted as a valid representation of the \ac{PA} state. 

However, this general assumption may not always hold true due to variations in the \ac{PA}'s internal states.
% The issue arises because \ac{PA} can adjust telemetry frequency and have different value ranges over time, depending on the phases of their lifecycle and their operational context (e.g., from \texttt{Ready} to \texttt{Working}).
% These variations introduce complexity in maintaining the cyber-physical alignment. 
% Without appropriate modeling, these changes might be misinterpreted as failures while they are instead expected variations due to the \ac{PA}'s phase transition.
%
In some cases, \acp{PA} can enter operational states that temporarily inhibit their ability to communicate, even while maintaining an active connection to the \ac{DT}.
These are scenarios in which the absence of messages from the \ac{PA} does not necessarily indicate a failure or misalignment but rather reflects the \ac{PA}'s operational context.
For instance, a \ac{PA} might enter a \texttt{Rebooting} state during which it cannot send telemetry data.
Similarly, resource constraints on the \ac{PA}, such as \textit{CPU overload} or \textit{network bandwidth exhaustion}, may result in disrupted or paused communication that may or may not be tolerable for the \ac{DT} model and application.
%
In other cases, the \ac{PA} may not entirely cease communication but instead alter its update frequency in response to operational changes. For example, an industrial robotic arm might increase its data transmission frequency during a high-precision assembly task to provide real-time feedback, while it could reduce the update rate to conserve energy and network bandwidth when in an idle or maintenance mode.


\begin{figure}
    \centering
    \includegraphics[width=\textwidth]{figures/dt-lifecycle/dt_lifecycle_pt_sync.pdf}
    \caption{Schematic representation of the \ac{DT} lifecycle with a structured example of the Synchronized phase.}
    \label{fig:dt_lifecycle_pt_sync}
\end{figure}


This adaptive behavior necessitates that the \ac{DT} \emph{dynamically adjust its expectations} and processing strategies based on the \ac{PA}'s operational state, avoiding false-positive anomalies caused by \emph{intentional} communication variability.  
By distinguishing between inhibited communication (e.g., \texttt{Rebooting} or \texttt{Overloaded}) and adjusted communication patterns (e.g., frequency scaling in \texttt{Idle} vs. \texttt{Active} states), the \ac{DT} can better align with the \ac{PA}'s lifecycle.
This alignment ensures the \ac{DT} remains a reliable digital counterpart, accurately reflecting the \ac{PA}'s operational context and providing a robust foundation for decision-making.  

To address these situations, the \ac{DT} must incorporate mechanisms to: 
\begin{itemize}
    \item \emph{Detect and Classify Non-Communication States:} Recognize when the lack of communication from the \ac{PA} is due to a valid operational state (e.g., rebooting) rather than a system failure;
    \item \emph{Model \ac{PA} State-Dependent Communication Behavior:} Explicitly account for \ac{PA} states that imply non-communication, such as \textit{idle}, \textit{rebooting}, or \textit{overloaded}, within the lifecycle framework. 
\end{itemize}
This requires extending the \ac{DT} lifecycle model (\Cref{fig:dt_life_cycle}) to represent such scenarios accurately.


The proposal is to refine the \texttt{Synchronized} phase by introducing sub-phases that correspond to the \ac{PA}'s operational states (\Cref{fig:dt_lifecycle_pt_sync}).
For example, in the industrial domain, a \ac{DT} of a machine may enter the \texttt{Synchronized} phase but should then transition through more specific sub-phases such as \texttt{Ready}, \texttt{Working}, or \texttt{Busy}.
Each of these sub-phases would have clearly defined transition criteria, phases, and associated $S_{\ac{DT}}$, providing a more granular and accurate representation of the \ac{DT}'s behavior.

To formalize this approach, the overall \ac{DT} lifecycle phase at a given time $t$ $LP_{DT}(t)$ can be defined as a composition of the \ac{PA} lifecycle phase $LP_{PA}(t)$ and the \ac{DT} software lifecycle phase $LP_{Soft.}(t)$, as shown in Eq.~\ref{eq:lpdt_definition}.

\begin{equation}
LP_{DT}(t) = \langle  LP_{PA}(t), LP_{Soft.}(t)\rangle
\label{eq:lpdt_definition}
\end{equation}

The \ac{DT} representation of the asset $R_{DT}$ can hence be characterized at any given time instant $t_i$ by its lifecycle phase $LP_{DT}(t_i)$, its current state $S_{DT}(t_i)$, and its history $H(t_a, t_b)$ over a specific time interval $(t_a, t_b)$, as shown in Eq.~\ref{eq:dt-definition}.

\begin{equation}
R_{DT}(t_i) = \langle LP_{DT}(t_i), S_{DT}(t_i), H(t_a, t_b)\rangle \quad \text{and} \quad  t_a, t_b \leq t_i
\label{eq:dt-definition}
\end{equation}

The values of the different components of the \ac{DT} at the time instant $t_i$ can then be mapped as a function of the different possible values of $LP_{Soft.}$, as detailed in Eq.~\ref{eq:lpdt_complete_mapping}.

\begin{equation}
\begin{split}
R_{DT}(t_i) = {} &
\begin{cases}
    LP_{PA} = \emptyset, \ S_{DT} = \emptyset, \ H = \emptyset,\\
    \quad \text{if } LP_{Soft.} \in \{\texttt{Started}, \texttt{Stopped}\}, \\[0.4em]
    LP_{PA} = \emptyset, \ S_{DT} = \emptyset, \ H = H(t_f, t_o),\\
    \quad \text{if } LP_{Soft.} \in \{\texttt{UnBound}, \texttt{Bound}, \texttt{OutOfSync}\}, \\[0.4em]
    LP_{PA} = LP_{PA}(t_i), \ S_{DT} = S_{DT}(t_i), \ H = H(t_f, t_i),\\
    \quad \text{if } LP_{Soft.} = \texttt{Synchronized},\\[0.4em]
    LP_{PA} = \emptyset, \ S_{DT} = \emptyset, \ H = H(t_f, t_d),\\
    \quad \text{if } LP_{Soft.} = \texttt{Done}
\end{cases}
\end{split}
\label{eq:lpdt_complete_mapping}
\end{equation}

When $LP_{Soft.}$ is \texttt{Started}, it represents the initialization phase where the \ac{DT} software has been instantiated. At this stage, $LP_{PA}$ is undefined since no communication with the \ac{PA} has been established. 
Similarly, $S_{DT}$ has not been computed, as the \ac{DT} has not acquired any state information. The history $H$ is also empty, as no synchronization or state updates have occurred yet.

When $LP_{Soft.}$ is \texttt{Unbound}, it indicates that the \ac{DT} is operational but not yet synchronized.
In this phase, $LP_{PA}$ remains undefined because the \ac{PA} lifecycle phase cannot be detected.
$S_{\ac{DT}}$ is also undefined since the \ac{DT} can not compute the state yet.
However, the history $H$ retains information about events recorded from the first connection data received from the \ac{PA} $t_{f}$ to the moment the \ac{DT} transitioned out of the synchronized phase, marked by $t_{o}$.
%
The same happens when the \ac{DT} is \texttt{Bound}, as the state has not been computed yet, 
or if the \ac{DT} is \texttt{Out of Sync}, as the state can no longer be updated.

When $LP_{Soft.}$  is \texttt{Synchronized}, the \ac{DT} is fully synchronized with the \ac{PA}.
In this state, $LP_{PA}$ corresponds to the lifecycle phase of the PA (e.g., in the example of Fig.~\ref{fig:dt_lifecycle_pt_sync}, this could be the \texttt{Working} phase) at the time $t_i$.
At this point, $S_{DT}$ contains the current computed state of the \ac{DT}, reflecting the alignment between the \ac{DT} and \ac{PA}.
The history $H$ encompasses all events and states up to the current time $t_{i}$, documenting the \ac{DT}'s evolution in sync with the PA.
If the \ac{DT} operates correctly, it can remain in the \textit{Synchronized} phase for the duration of its lifecycle, continually updating as the PA's lifecycle progresses.
During this phase, multiple records of lifecycle evolution can be captured as $LP_{PA}$ evolves (e.g., moving from \texttt{Working} to \texttt{Busy}), and multiple computations of $S_{DT}$ may occur within the same $LP_{PA}$.
For instance, while in the \texttt{Working} phase, the \ac{DT} could compute multiple states corresponding to variations in physical properties such as accelerometer readings or energy consumption, ensuring the \ac{DT} continuously reflects the PA's behavior in a dynamic and precise manner.

When $LP_{Soft.}$ is \texttt{Done}, the \ac{DT} synchronization has been intentionally paused, and the \ac{DT} remains active for accessing stored information. In this phase, $\text{LP}_{PA}$ is empty, $S_{DT}$ is empty, and $H$ contains the history until the time the \ac{DT} is decommissioned denoted as $t_{d}$.

Finally, when $LP_{Soft.}$ is \texttt{Stopped}, the \ac{DT} lifecycle is terminated. $LP_{PA}$ is empty, $S_{DT}$ is empty, and $H$ is also empty, as the \ac{DT} is offline and data is no longer accessible.

By clearly defining and distinguishing between the various phases, the approach enhances the alignment between \ac{DT}s and their physical counterparts, ensuring that each transition and state change is accurately captured and reflected.
The main benefits of this approach include: 
\begin{itemize}
    \item \textit{Enhanced Cyber-Physical Awareness}: a structured lifecycle model ensures that critical transitions of phases and states of the \ac{PA} are precisely tracked and understood, enhancing the overall awareness and responsiveness of the system;
    \item \textit{Better Decision-Making}: with a clear understanding of each phase and its impact, applications and services can make more informed decisions based on accurate and timely information about the current relationship between \ac{PA} and \ac{DT};
    \item \textit{Adaptability}: this structured approach can be adapted to various applications, making it versatile and applicable across different domains.
\end{itemize}

The approach assumes that the $L_{PA}$ is either provided by the \ac{PA} or can be identified by the \ac{DT} model (e.g., through a classification model).
Modeling these phases can be challenging, especially for complex \acp{DT}.
Despite improved cyber-physical awareness, phases may still be incorrectly mapped due to unforeseen external factors.
Additionally, phase tolerance could impact anomaly detection within the DT, necessitating methodological trade-offs to set this tolerance correctly.

Despite these open challenges, the proposed structured lifecycle model represents an advancement in managing \ac{DT} synchronization with their physical counterparts, providing a robust foundation for future developments.

%=======================================================
\section{Modeling Augmentation Functionalities}
%=======================================================

\acp{PA} are typically limited by the nature of their physical characteristics throughout their lifecycle. In contrast, \acp{DT} can capitalize on software-based flexibility to modify, update, and enhance functionality over time.
These enhancements are accessible through \ac{DT} \acp{API} and can be powered by data-driven models possibly enabling adaptive and intelligent behaviors.
This core property of \acp{DT} is known as \emph{augmentation}~\cite{dt-IoT-context-Minerva-2020}.

Examples of DT-enabled augmentation include adding technical features to retrofit devices through external software, such as enhancing interoperability of \ac{IoT} devices via protocol and data format translation. Another example involves improving security and addressing unresolved bugs by mediating interactions with outdated devices that no longer receive updates. Additionally, \acp{DT} can enhance device services by introducing new high-level actions for operations, or by implementing advanced functionalities powered by machine learning and \ac{AI} algorithms, such as forecasting, anomaly detection, and pattern recognition (\Cref{sec:back:dt:ai}).

Despite the acknowledgment of augmentation as a defining feature of \acp{DT},
this property is often defined only conceptually, with implementations tailored to specific vertical cases.
The lack of a reference model for defining \ac{DT} augmentation results in unclear design patterns and potential performance limitations.
Indeed, without a well-defined and measurable boundary for augmentation features---including resource allocation---the \ac{DT} risks failing to achieve its design goal of accurately representing the \ac{PA} with sufficient \emph{fidelity}~\cite{Bellavista_Bicocchi_Fogli_Giannelli_Mamei_Picone_2023}.

%--------------------------------------------
\subsection{Extending the DT Model with Augmentation}
%--------------------------------------------

Recalling the \ac{DT} model structure introduced in \Cref{sec:dte:engineering-dt:abstract-architecture}. 
We here extend it to explicitly include augmentation functionalities.
%
In practical terms, augmentation serves as a bridge transforming raw data into meaningful, actionable outcomes guided by the underlying model.
\acp{DT} may expose additional services and functionalities that enhance the capabilities of the \ac{PA}.
We can represent such functionalities as a set of \emph{augmentation functions} $F$, where $Augm_{i}$ denotes the individual augmentation functions implemented within the DT. Each augmentation function $Augm_i$ may leverage a specific model $m \in M$, to interpret and respond to changes in the DT.

\begin{equation}\label{eq:augm_set}
    F = \{Augm_{1}, Augm_{2}, \dots, Augm_{n}\}\\
\end{equation}

The model in \Cref{eq:5D_dt_model} can hence be extended to: 

\begin{equation}
DT = \langle PI, M, Shad, F, H, DI \rangle
\tag{\ref{eq:5D_dt_model}}
\end{equation}

We distinguish augmentation functions based on two aspects:


\paragraph{Triggering mechanism} (\( \tau \)): defines how and when the function is activated in response to events within the \ac{DT}.
%
\begin{equation}
\tau \in \{ e_{DT}, e_{Aug}, e_{Time} \}  \sigma \in \{S_{aug}, \varnothing\}
\end{equation}
%
\(\tau\) can be an event of three types:
%
\begin{itemize}
\item \( e_{DT} \) represents the output from the shadowing process indicating a new computation of the \ac{DT} state \( S_{DT} \);
\item \( e_{Aug} \) can be generated either as the output of another augmentation function or as a direct invocation by the shadowing process;
\item \( e_{Time} \) refers to temporal triggers and can be used to model periodic execution or specific scheduled tasks.
\end{itemize}
%
Notably, augmentation functions can not respond to triggers coming directly from the \ac{PI} as such $e_{PA}$.
Physical events need to be processed by the shadowing process first, which may in turn select the activation of a relevant augmentation function firing a specific $e_{Aug}$.



\paragraph{State management} (\( \sigma \)): indicates whether the function maintains an internal state that persists over time or is \emph{stateless}.
%
\begin{equation}
\sigma \in \{S_{aug}, \varnothing\}
\end{equation}
%
A stateful function possesses a state \( S_{Aug} \) that evolves and persists over time in alignment with the evolution of the \ac{DT}.  
A stateless function simply reacts to triggers without maintaining any memory of previous invocations.  
Notably, modeling stateful functions allows for the representation of processes that may even run in parallel with the \ac{DT} main control flow, producing new events \( e_{Aug} \) asynchronously.  
Since the function is stateful, it is triggered to start but can persist over time, as its execution is fully encapsulated within the function context. Additionally, the function may receive new triggers over time that feed into its ongoing process.

The output of an augmentation function \( Augm_{i} \) is determined by its inputs and can be expressed as:

\begin{equation}
    Augm_{i}(\tau, m, \sigma, H) =
    \begin{cases}
    e_{Aug} & \text{\textbf{if} $m$ matches}\\
    \varnothing & \text{\textbf{otherwise}}
    \end{cases}
\end{equation}

The triggering event $\tau$ may carry important information for the function execution.
The parameter $m \in M$ represents the model used within the augmentation function, encapsulating the rules, logic, and conditions that define how the function interprets its inputs and determines the output.
The model is crucial for evaluating whether the function should generate an output based on the current state of the function $\sigma$ and context, which is represented by the \ac{DT} history $H$ which includes the latest state $S_{DT}$.

The output \( e_{Aug} \) represents an event generated by the augmentation function, signaling that the conditions defined by the model \( m \) have been met. Conversely, if the conditions are not satisfied (i.e., if \( m \) does not align with the current state or inputs), the output will be \( \varnothing \), indicating that no new event has been produced.  
This approach captures the idea that the execution of an augmentation function does not always result in the generation of a new event. For example, in the case of an anomaly detection model, an event is only triggered when an anomaly is detected with a certain confidence threshold, meaning that no event is generated if the conditions for an anomaly are not met.
Since $e_{Aug} \in \tau$, augmentation functions can also be chained to model processing pipelines triggered by the outputs of previous functions.

%--------------------------------------------
\subsection{Impact on Shadowing}
%--------------------------------------------

The introduction of augmentation functions requires to adapt the shadowing process to handle augmentation events. Indeed, the shadowing process should be capable of receiving relevant augmentation events (e.g., detected anomalies) and use them to update the state of the DT, as illustrated below:

\begin{equation}
    e_{DT}(S_{DT}) = Shad_{DT \rightarrow DT}(M, e_{Aug}, H)
\end{equation}

The same principle applies when a physical event triggers an augmentation function. There is no direct link between physical world events (\( e_{PA} \)) and augmentation functions; 
instead, the decision to trigger an augmentation function based on the receipt of a physical event is mediated by the shadowing process.
The shadowing process analyzes the \( e_{PA} \) and determines whether it is relevant for an augmentation function in $F$ (or multiple functions simultaneously).  If the event is deemed as relevant, the shadowing process generates a trigger by producing an augmentation event (\( e_{Aug} \)), which then serves as the trigger \( \tau \) for one or more augmentation functions.

\begin{equation}
    e_{Aug} = Shad_{PA \rightarrow DT}(M, e_{PA}, H, F)
\end{equation}


Similarly, actions requests that are received by the \ac{DT} ($a_{DT}$) can also trigger augmentation functions.
Again, there is no direct relationship between the action taken on the DT, as represented by its state, and the execution of the augmentation function. Instead, this process also goes through the shadowing process, which matches the request received on the \ac{DI} with the corresponding augmentation function trigger, thus generating an augmentation event that will serve as the trigger for the augmentation function.
This can be used to model a request-response pattern for the execution of augmentation functions such as requesting the predicted next state of the \ac{DT}. In such cases, since the result of the augmentation does not impact the state of the \ac{DT}, the result can be directly routed to the external consumer of the \ac{DT}.

\begin{equation}
    e_{Aug} = Shad_{DT \rightarrow DT}(M, a_{DT}, H, F)
\end{equation}


Despite the shadowing might not be required to validate the result of an augmentation function triggered by an action request, it is still enforced to handle the incoming requests. 
This design aims to decouple the available implementations of the augmentation functions from the actions presented on \ac{DI} and handle this transparently from the consumer perspective.
For example, there might be several augmentation functions implementing a given functionality and it would be up to the shadowing process to determine which one is best suited to respond to a specific request depending on the current context of the \ac{DT} (e.g. the state prediction functionality could be implemented using different specialized models depending on some conditions in the state of the \ac{DT}).

%----------------------------------------------
\subsection{Engineering Augmentation Functions}
%----------------------------------------------

Engineering augmentation functions following the proposed model requires defining
software architectural patterns that can be practically applied to implement them.
Depending on the implementation, it is possible to define \emph{internal} and \emph{external} augmentation.  
This setup allows for the dynamic addition of new functionalities to the \ac{DT}, either through external or internal processing.

\begin{figure}
    \centering
    \includegraphics[width=\textwidth]{figures/event_driven_augmentation.pdf}
    \caption{Event-Driven architecture of a DT to support and enable effective Augmentation function management and execution.}
    \label{fig:event_driven_augmentation}
\end{figure}

As \Cref{fig:event_driven_augmentation} shows, the core architecture of the \ac{DT} is centered around a shared \texttt{Event Bus}, which serves as the communication backbone, handling events and actions within the \ac{DT} system.
Starting from the \ac{PI}, physical events \( e_{PA} \) are processed by the \texttt{Shadowing Function} ($Shad$) which computes the \ac{DT} State $S_{DT}$ based on the received events and the \texttt{\ac{DT} Models} ($M$), ensuring that the \ac{DT} accurately reflects the current state of the \ac{PA}.
The \texttt{\ac{DT} State} component maintains this state in memory, while the \texttt{History} ($H$) component persists past events and states for reference and analysis.

\texttt{Augmentation Functions} (\( Augm_{i} \)) provide additional capabilities or insights to the \ac{DT}, interacting with \( M \), \( H \), and internal state information \( \sigma \).
The interaction between the \( Shad \) and the \( Augm_{i} \) is entirely event-driven, based on trigger events \( \tau \) and \( e_{Augm} \) flowing out of the augmentation functions with their results.
These results are then sent to the \( Shad \) for further state computation and to the \texttt{Digital Interface} (\( DI \)) to respond to requests for \( Augm_{i} \) from external services and applications.
The \( DI \), mirroring the \ac{PI}, handles outgoing events of the \ac{DT} (e.g., \( e_{DT} \) carrying the new \( S_{DT} \) or its variations) or \( e_{Augm} \) with the results of an augmentation function, and manages incoming actions and requests from digital entities as digital actions \( a_{DT} \).

\begin{figure}
    \centering
    \includegraphics[width=\textwidth]{figures/augmentation_patterns.pdf}
    \caption{Patterns for augmentation functions execution: internally (left), externally (center), or through a hybrid solution (right).}
    \label{fig:augm_function_event_driven_patterns}
\end{figure}


Based on this event-driven model, we identify three approaches for implementing augmentation functions (\Cref{fig:augm_function_event_driven_patterns}):
\begin{itemize}
    \item \textit{internal} augmentation, where functions are executed within the \ac{DT} process;
    \item \textit{external} augmentation, where functions are executed outside the \ac{DT} process by means of an external service;
    \item \textit{hybrid} augmentation, where functions are executed through a combination of internal and external processes.
\end{itemize}

\subsubsection{Internal Augmentation}
Augmentation functions are executed \textit{internally}, i.e. within the same operating system process as the \ac{DT}, thus sharing the same resources allocated to the \ac{DT} instance (e.g., CPU or GPU computation capabilities).
From an external perspective, these functions are perceived as internal functionalities of the \ac{DT} and only the internal event bus of the \ac{DT} is used.

This approach ensures low latency and tight integration with the \ac{DT} core functionalities, as overhead between the \ac{DT} and the augmentation functions is minimized.
%
For instance, in a manufacturing scenario an internal augmentation function could periodically monitor the operational parameters of a machine physical counterpart.
By running within the \ac{DT} process, this function can quickly react to changes in the machine state, such as detecting abnormal vibrations that might indicate a potential failure (e.g., using signal processing approaches).
The internal event bus would handle the real-time data stream from the machine sensors, feeding the computation of the \ac{DT} state while allowing the augmentation function to promptly process this data and update the \ac{DT}.

\subsubsection{External Augmentation}
Augmentation functions run \textit{externally} as processes entirely separate from the \ac{DT}, possibly without any knowledge of the \ac{DT} internal model.
Since the functions are implemented on external processes, they can use inter-process communication to synchronize and hence they can be distributed across different computing nodes and resources, for a more fine-grained allocation.
%
This translates either in multi-process deployment of the \ac{DT} on the same host, or possibly in fully distributed deployment of the \ac{DT} on a computing cluster.

In the latter scenario, multiple event buses are present. The primary event bus will always be the internal one, while additional external event buses, possibly implemented with different technologies, will be integrated with the internal one. Such integration allows the system to provide triggers, events, and information to the augmentation functions and retrieve their results to synchronize them with the \ac{DT} logic in a completely transparent manner, both for the \ac{PA} and external applications.

A \textit{Synchronization Interface (Sync)} within the \ac{DT} will be responsible for synchronizing the internal and external event buses concerning the events and triggers associated with the augmentation functions. This component must be capable of managing the protocols or formats for mapping between the logic of the internal event bus and the external one.

Since augmentation functions can be external, the state (\( \sigma \)) of an augmentation function will be maintained externally to the \ac{DT}. For internal augmentation functions, \( \sigma \) will be internal to the \ac{DT}, but isolated from other \ac{DT} components and solely dedicated to the augmentation function.

Taking as reference a manufacturing use case, we can exemplify the value of such externalization by considering a \ac{DT} monitoring various machines on the production line, collecting real-time data on operational status and performance.
Augmented features for maintenance prediction and production schedule optimizations could be delegated to an external data analytics service which, using the external event bus, may process the \ac{DT} data with advanced machine learning algorithms and send back results to the \ac{DT} without overloading the \ac{DT} core system.

\paragraph{Hybrid Augmentation}
Augmentation functions run both internally and externally, by combining the strengths of both methods.
A synchronization interface ensures communication between the internal and external event buses.
%
Stateful functions can store their state both externally, for those that require external resources, and internally, as a structural element of the \ac{DT} for internal functions.

This approach is the most flexible and offers significant advantages.
For example, in a smart city application, internal augmentation functions can handle real-time traffic monitoring, benefiting from low latency and immediate response times.
Meanwhile, external augmentation functions can process long-term traffic pattern analysis for strategic urban planning, leveraging powerful external computational resources without burdening the \ac{DT} core system.
%
Such hybrid model provides the flexibility to optimize for both real-time and complex, resource-intensive tasks, ensuring that the \ac{DT} remains efficient and scalable.

%=======================================================
\section{The \acl{WLDT} Framework}
%=======================================================


\begin{figure}
    \centering
    \includegraphics[width=\columnwidth]{figures/engineering-wldt/wldt_architecture.pdf}
    \caption{WLDT Framework's Software Architecture.}
    \label{fig:wldt_architecture}
\end{figure}


The WLDT framework has been re-designed and extended with key requirements: \textit{Simplicity}, \textit{Extensibility}, \textit{Portability}, and \textit{Microservice Readiness}.
Simplicity ensures easy creation of DT instances using existing modules or custom behaviors for specific scenarios. 
Extensibility allows easy API extension, enabling customization and addition of new features through concurrent module execution.
Portability and microservice readiness ensure that DTs can run on any platform without modifications.
The core engine is lightweight, offering IoT features for developing DT applications as independent microservices.
The main components of the WLDT architecture are illustrated in Figure~\ref{fig:wldt_architecture}, which outlines the three architectural levels: the core library, the DT model, and the implementation of physical and digital interfaces through adapters.
In the WLDT framework, \textit{adapters} -- modular components that implement the PI and DI -- are essential for facilitating communication between \ac{PA}s and DTs, as well as with external digital entities.
Physical Adapters (PAs) interface with the physical world, while Digital Adapters (DAs) manage interactions within the digital environment.

The \textit{Metrics Manager} provides the functionalities for managing and tracking various metrics within DT instances, combining both internal and custom metrics through a flexible and extensible approach.
The \textit{Logger} is designed to facilitate efficient and customizable logging within implemented and deployed DTs, with configurable log levels and versatile output options.
The \textit{Utils \& Commons} component contains a collection of utility classes and common functionalities that can be readily employed across DT implementations, ranging from handling common data structures to providing helpful tools for string manipulation.

%%%
\begin{figure}[th!]
    \setlength{\belowcaptionskip}{-8pt}
    \centering
    \includegraphics[width=\columnwidth]{figures/engineering-wldt/fischer-layout-first-level-dt.pdf}
    \caption{A schematic view of the target industrial scenario.}
    \label{fig:discher-dt-first-level}
\end{figure}
%%%

The \textit{Event Communication Bus} represents the internal Event Bus, designed to support communication between the different components of the DT instance (PI, DTM and DI). It allows for the definition of customized events to model both physical and digital inputs and outputs. Each WLDT component can publish on the shared Event Bus and define an Event Filter to specify which types of events it is interested in managing, associating a specific callback to process different messages.

%%%
\begin{figure*}[th!]
    \setlength{\belowcaptionskip}{-13pt}
    \centering
    \includegraphics[width=0.95\textwidth]{figures/engineering-wldt/DTs-zoom-in.pdf}
    \caption{Internal structure of machine-level DTs with the details of adapters, states and capabilities}
    \label{fig:dt-internal-structure-overview}
\end{figure*}
%%%

The \textit{Digital Twin Engine} defines the multi-threaded engine of the library, allowing the execution and monitoring of multiple DTs and their core components simultaneously.
It orchestrates the different internal modules of the architecture while keeping track of each one, and it serves as the core of the platform, enabling the execution and control of deployed DTs.
Currently, it supports the execution of twins within the same Java process, but the same engine abstraction could extend the framework to support distributed execution through different processes or microservices.

The DT component consists of a modular structure within WLDT, combining core functionalities and capabilities of physical and digital adapters.
Each DT's core module is its \textit{Model}, which is responsible for implementing the twin's behavior over time. This model controls both the \textit{Shadowing Function} and \textit{Augmentation Function(s)}.
The Shadowing Function manages the digitalization process by integrating data from physical adapters and action requests from digital adapters.
It maintains the DT's \textit{State}, organizing properties, events, actions, and relationships as introduced in Section \ref{sec:dt_modelling}.
Augmentation Functions enhance physical capabilities by introducing new properties and actions through intelligent functionalities or dedicated processing (e.g., anomaly detection or data aggregation).
The DT's core is managed by the library to ensure consistency and lifecycle management, adapting to computational phases and interactions with physical and digital realms via adapters. 
Further sections elaborate on these aspects concerning lifecycle structure and adapters.

%%%
\subsection{Cyber-Physical Awareness \& Life-cycle}
\label{sec:dt-lifecycle}

The modeling of DTs involves defining their life cycle, which we have mapped into five states: \textit{UnBound}, \textit{Bound}, \textit{Synchronized}, \textit{OutofSync}, and \textit{Done}.
These states are directly inspired by recent state-of-the-art definitions in the field, particularly from \cite{web_of_dt} and \cite{bellavista2023requirements}, which provide a comprehensive framework for understanding and managing the dynamic evolution of DTs across different stages.

In the \textit{UnBound} state, all internal modules are active but not yet associated with the corresponding \ac{PA}.
The DT transitions to the \textit{Bound} state once the binding procedure is completed, establishing a connection with the \ac{PA} and enabling bidirectional event flow.
In the \textit{Synchronized} state, the DT’s state is aligned with the \ac{PA}, initiating the shadowing process.
The \textit{OutofSync} state indicates errors in the shadowing process, where state alignment or application-generated events cannot be processed.
Finally, the DT reaches the \textit{Done} state when the shadowing process ends but remains active for external requests (e.g., asset history access after its lifecycle).

To transition from \textit{UnBound} to \textit{Bound}, the DT establishes a relationship with the Physical Layer.
For instance, a \ac{PA} using two protocols, $P_1$ and $P_2$, communicates through dedicated adapters.
A key challenge is discovering the \ac{PA}’s capabilities to begin the digitalization process. WLDT addresses this by describing a \ac{PA}’s characteristics, properties and actions through its physical adapters enabling the DT to synchronize with the \ac{PA}.
The DT transitions to \textit{Bound} only when the required adapters are bound and \ac{PA} descriptions are generated, even when multiple \ac{PA}s are associated with a single DT.

To move from \textit{Bound} to \textit{Synchronized}, the DT identifies \ac{PA} capabilities for digitalization and starts the shadowing process (e.g., monitoring a thermostat's target temperature and humidity sensors).
The DT’s Model begins observing these properties through the adapters, which communicate with the \ac{PA} and generate events for property changes.
The DT’s Shadowing function processes these events to compute the new DT state, maintaining synchronization with the \ac{PA}.
Any change in the DT state generates an event, which is exposed to the digital world through the Digital Interface and its Adapters.
These events may also trigger actions on the DT or propagate requests to the \ac{PA} if needed.

%%%
\subsection{Modular and Reusable Digital \& Physical Adapters}
\label{subsec:adapters}

A Physical Adater (PA) in WLDT is responsible for connecting to a specific \ac{PA}. 
Developers can either leverage existing adapters—customized through configuration for context-specific reuse—or create new ones tailored to their specific requirements.
The core responsibilities of a PA include also the generation of a description of the associated \ac{PA} that details the properties, events, actions, and relationships of the \ac{PA}; producing events to reflect changes in the physical state, such as variations in properties, events, and relationships; and handling action requests from the digital realm via the DT shadowing process and the twin's model.
Each PA has an internal dedicated lifecycle within the DT to ensure it can respond to incoming actions appropriately. Adapters are identified by unique IDs, enabling the library to coordinate multiple adapters, adjust logs, and execute functions upon receiving new events. 

Conversely, Digital Adapters (DAs) in WLDT act as the bridge between the DT's core and the external digital world.
Their primary functions include receiving and exposing events related to changes in properties, events, available actions, and relationships from the DT’s core;
communicating the received information to external systems according to the implementation and supported protocols;
and handling incoming digital actions, forwarding them to the core, and ensuring they are validated and processed by the Shadowing Function.
Each DA has an internal life cycle to support its coordination by the WLDT core and has also direct read access to the current twin's State through callbacks or synchronous access, allowing it to navigate all fields of the current digital state and expose it in the digital domain through various protocols according to the application context.


%========================================================
\section{Benefits of the Proposed Approach}
%========================================================

\subsection{Interoperability in a Manufacturing Use Case}
\label{sec:industrial_use_case}
%%%

%%%
\begin{figure*}[t]
    \setlength{\belowcaptionskip}{-13pt}
    \centering
    \includegraphics[width=\textwidth]{figures/dt-interoperability/dt_interoperability-pad_dt_wot.pdf}
    \caption{The transition from physical to digital representation via the \ac{PAD}, the \ac{DT} State, and the \ac{DTD} for external consumers.}
    \label{fig:pad_dt_wot}
\end{figure*}
%%%

The proposed \ac{DT} architectural approach is showcased using the \emph{Fischertechnik Training Factory Industry 4.0}\footnote{Fischertechnik Industry \& Universities: \url{https://www.fischertechnik.de/en/products/industry-and-universities}} Indexed-Line Station, controlled by a Siemens PLC that publishes real-time data via OPC-UA, integrated with Arduino RP2040 boards\footnote{Arduino RP2040: \url{https://docs.arduino.cc/hardware/nano-rp2040-connect/}} for accelerometer data collection and processing and communicating over MQTT\footnote{MQTT Protocol: \url{https://mqtt.org/}}.

\begin{figure}[t]
  \setlength{\belowcaptionskip}{-5pt}
  \centering
  \begin{subfigure}[t]{0.32\linewidth}
      \centering
      \includegraphics[width=\linewidth]{figures/dt-interoperability/machine_complexity_index.pdf}
      \caption{Machine Digital Twin}
      \label{fig:machine_complexity_index}
  \end{subfigure}
  \begin{subfigure}[t]{0.32\linewidth}
      \centering
      \includegraphics[width=\linewidth]{figures/dt-interoperability/station_complexity_index.pdf}
      \caption{Station (Indexed-Line) Digital Twin}
      \label{fig:station_complexity_index}
  \end{subfigure}
  \begin{subfigure}[t]{0.32\linewidth}
      \centering
      \includegraphics[width=\linewidth]{figures/dt-interoperability/factory_complexity_index.pdf}
      \caption{Factory Digital Twin}
      \label{fig:factory_complexity_index}
  \end{subfigure}
  \caption{Schematic representation of cyber-physical complexity and its impact on various components.}
  \label{fig:dt_complexity_index}
\end{figure}

The scenario emulates a production line, with a \ac{DT} system deployed to monitor production and measuring the overall efficiency.
The \ac{DT} ecosystem architecture is depicted in Figure \ref{fig:mf_dt_ecosystem}.
Four machine-level \acp{DT} are deployed, each with a \ac{PI} supporting the relevant communication protocols to interact with the different machines, namely through MQTT and OPC-UA\footnote{OPC-UA: \url{https://opcfoundation.org/about/opc-technologies/opc-ua/}}.
These \acp{DT} track machine states, monitor performance, and coordinate production by processing data from the physical system and digital action requests.
An additional department-level \ac{DT} aggregates data from machine-level \acp{DT}, extracting system metrics and exposing department-level actions. 

All the \acp{DT} are built using the \ac{WLDT} open-source framework\footnote{WLDT framework: \url{https://wldt.github.io}}: a Java-based multithreaded stack, facilitating the definition of \ac{DT} and supporting the implementation of shadowing and digitalisation processes.
%Within the framework, we implemented our proposal of physical and digital adapters to validate the approach and facilitate interoperability with a broader range of devices and applications.
The frameworks implements our proposal of physical and digital adapters that we exercise in this example.

\ac{DT} of composed physical assets, such as the \textit{Indexed Line Department} \ac{DT} (see Figure \ref{fig:mf_dt_ecosystem}, top) use lower-level \acp{DT} as data sources.
Notably, the physical and digital adapters that implement MQTT, OPC-UA, and HTTP protocols, along with \ac{WoT} over HTTP for serving \acp{TD}, are reusable across different instances.
These adapters, through configurable parameters, enable effective communication and integration within both digital and physical environments.

Machine-level \acp{DT} provide real-time data on their physical counterparts, such as workpiece presence, operational state
(e.g., idle, working, waiting, or broken),
and performance KPIs like \ac{OEE}\footnote{OEE: \url{https://www.oee.com/}}, calculated from production rate, uptime, and downtime.
%
At the department level, \acp{DT} representing the Indexed Departments assess performance by computing Weighted \ac{OEE}~\cite{OEE-manufacturing-cell-Gamberini-2017,Introduction-to-TPM-total-productive-maintenance-Nakajima-1995}, derived from individual machine \acp{DT}, and track throughput, defined as processed pieces per unit time.
Throughput is determined by monitoring entry and exit events within the department, utilizing relationships captured by the associated \acp{DT}.

Upon startup, each \ac{DT} connects to its data source, publishes the \ac{PAD} of the physical asset, and interacts with the PLC or Arduino to collect and send data, enabling digitalization and interoperability among heterogeneous equipment. 

Figure \ref{fig:pad_dt_wot} schematically illustrates the evolution of data and representation from the physical to the digital realms through the adoption of the proposed \ac{DT} architectural approach, \ac{PAD} integration, and \ac{WoT} \acp{TD} used as \acp{DTD} on the digital side, specifically for the output conveyor of the target production line.
On the physical side, data is structured using OPC-UA with a hierarchical organization, containing telemetry data and action methods mapped to specific data types.
Accelerometer data is transmitted over MQTT on a specific topic with a JSON payload containing axis acceleration values. 

The \acp{PAD} act as an intermediary, translating data from the \ac{PA} into a format suitable for the DT core, decoupling it from the complexity of interacting with the \ac{PA} and enabling discovery, understanding, and utilization of available data and methods.
Using the information from the \acp{PAD}, the \ac{DT} model computes the \ac{DT}'s state with target properties, events, and actions, based on either one-to-one matching of physical characteristics or augmented by combining multiple physical properties into computed \ac{DT} properties or events, such as computing the value of the \ac{OEE} property or triggering anomaly detection events based on accelerometer data. 

Finally, the \ac{DiA} of the \ac{DT} leverages \ac{WoT} \ac{TD} to describe the DT's data and functionalities through a standardized, interoperable, and machine-readable approach.
The \ac{TD} specifies how these can be accessed via protocols and data formats.
This approach enables effective interoperability from the physical to the digital, using a modular and decoupled structure through \acp{DT} in the target industrial use case.

\subsection{Mapping Cyber-Physical Complexity}
\label{sec:mappint_cp_complexity}

In our experimental analysis, we also aim to measure the impact of digitization and responsibility decoupling by comparing the system complexity in scenarios with and without \ac{DT} adoption and also with respect to the different \ac{DT}'s architectural layers.
This allows us to assess the benefits of a modular, structured DT-based approach, particularly in terms of interoperability and heterogeneity management.
To quantify how effectively our approach manages cyber-physical heterogeneity, we adopt the \textit{Cyber-Physical Complexity Index (CP-CI)} proposed in \cite{lippi_dt_causality_learning, LOMBARDO2024107203}.
The CP-CI quantifies the complexity of a cyber-physical application based on:
i) \textit{Required Protocols (p)}: communication protocols needed for device interaction;
ii) \textit{Communication Patterns (c)}: interaction models (e.g., Publish/Subscribe, Request/Response);
iii) \textit{Data Formats (d)}: required data representations or serialization methods;
iv) \textit{Interaction Points (n)}: modules or services involved in data exchange; and
v) \textit{Aggregation Points (a)}: levels of abstraction for physical data (e.g., merging data from multiple machines).
Each criterion is weighted with a \textit{Criteria Importance Factor ($CIF$)} from 1 (low) to 3 (high), indicating its impact on development, deployment, and maintenance.

% The CP-CI reflects the perceived complexity of a cyber-physical application, based on the following criteria: i) \textit{Required Protocols (p)}: number of communication protocols needed to interact with physical devices; ii) \textit{Communication Patterns (c)}: number of interaction models (e.g., Publish/Subscribe, Request/Response); iii) \textit{Data Formats (d)}: number of data representations or serialization methods required; iv) \textit{Interaction Points (n)}: number of distinct modules or services involved in data exchange; and v) \textit{Aggregation Points (a)}: number of composition levels needed to abstract physical information (e.g., merging data from multiple machines in a line). Each criterion is weighted using a \textit{Criteria Importance Factor ($CIF$)}, ranging from 1 (low) to 3 (high), to reflect its impact on development, deployment, and system maintenance.

Focusing on \ac{DT} interoperability, we extended the CP-CI to include both inbound and outbound interfaces, essential for bridging the physical and cyber layers with differing requirements.
This extension enables a more precise assessment of the complexity in managing interoperability across system boundaries.
The CP-CI was applied not only to the entire \ac{DT} but also to its internal layers: Physical Interface (PI), Core, and Digital Interface (DI).
These results were compared to those of a monolithic external application that directly handles business logic and interoperability, bearing the full complexity, highlighting the advantages of the modular \ac{DT} design.
High importance ($CIF=3$) was assigned to managing heterogeneous data formats ($d$) and aggregation points ($a$), as they are crucial for creating interoperable data models.
Medium importance ($CIF=2$) was given to protocol diversity ($p$) and interaction with multiple entities ($n$), which become more significant as systems scale.
Low importance ($CIF=1$) was assigned to multiple communication patterns ($c$), as their concurrent use is standard in distributed applications.

The graphs presented in Figure \ref{fig:dt_complexity_index} display the CP-CI values obtained across different levels of deployment: a single machine, a station in the factory (such as the indexed line, where multiple machines and their respective DTs are managed and composed into a single, unified DT), and the entire factory, which includes 9 machines, 2 composed DTs, and one overarching DT for the entire factory, all deployed following the proposed modular and interoperable approach.
For the machines analyzed, the involved protocols ($p$) and communication patterns ($c$) are both set to 2, as we utilize MQTT and OPC-UA with Pub/Sub and Request/Response interactions.
On the digital side, both MQTT and HTTP are employed.
Regarding data formats ($d$), each machine manages 2 distinct formats, accounting for both PLC data and Arduino accelerometer information.
Furthermore, the interaction points ($n$) and aggregation points ($a$) are both 2, reflecting the aggregation and interaction with two different sub-entities for each machine and its corresponding DT.

The results for single machines (Figure \ref{fig:machine_complexity_index}) highlight the critical role of the PI in the \ac{DT} architecture.
As a decoupling layer, the PI isolates physical-world complexities from the DT core, managing priorities like protocols, data formats, interaction points, aggregation strategies, and communication patterns.
This allows the DT core to focus on uniform data, processed through the interface via PAD descriptions and payload transformations, without dealing with heterogeneous physical entities.
Consequently, the core interacts only with its internal communication pattern and consistent interaction points, regardless of the physical environment’s configuration.
A similar approach is applied to the DI, which is decoupled from physical complexities and adapts to external digital systems, ensuring modularity and reuse.

In contrast, systems without a \ac{DT} must embed all heterogeneity management into a single application, burdening it with business logic, protocol diversity, and data format interoperability.
This increases complexity and reduces scalability, as reflected by the complexity index parameters.

% The results for single machines (reported in Figure \ref{fig:machine_complexity_index}) underscore the strategic importance of the PI within the \ac{DT} architecture. Acting as a decoupling layer, the PI isolates the complexities of the physical world from the DT core. It manages key priorities related to the physical domain, including protocols, data formats, interaction points, aggregation strategies, and communication patterns. This separation allows the DT core to focus on uniform data, processed and normalized through the interface using PAD descriptions and payload transformations, rather than managing heterogeneous physical entities. As a result, the DT core is shielded from the diversity of protocols, working only with its internal communication pattern and a consistent set of interaction points, regardless of the physical environment's configuration. A similar approach applies to the DI, which is decoupled from physical-world complexities and its interaction is limited to the DT core, adapting to external digital systems based on the specific application context. This modular approach enhances the separation of concerns and promotes reuse, allowing the same DT core to be deployed with various Physical Interfaces suited to different operational contexts.

% In contrast, systems that lack a \ac{DT} approach, which bridges and manages the cyber-physical layers, must integrate all interoperability and heterogeneity management into a single application. This places the burden of handling business logic, protocol diversity, and data format interoperability on the application itself, leading to increased complexity and diminished scalability, as reflected in the parameters of the complexity index.

As system architecture scales from individual machines to stations, production lines (Figure \ref{fig:station_complexity_index}), or entire factories (Figure \ref{fig:factory_complexity_index}), protocol diversity stabilizes while the number of interaction points and data formats increases due to the need for managing heterogeneous components.
Our complexity measure shows that systems at the station or factory level experience a significant rise in complexity, driven by interaction points, aggregation nodes, and data format diversity.
These results emphasize the benefits of a modular and interoperable approach.
By encapsulating complexity within DT modules and using interoperable representations, developers can simplify application development.
In contrast, monolithic systems struggle with managing interoperability and system evolution as integration requirements grow.


\subsection{Industrial Use Case}
\label{sec:industrial_use_case}
%To demonstrate the applicability of the proposed lifecycle synchronization model, we conducted an experimental evaluation within a real-world industrial scenario. This evaluation highlights how the alignment between the DT and \ac{PA} lifecycles can be effectively achieved through structured data management and phase synchronization. The industrial case involves mapping the lifecycle phases of physical devices.

An industrial scenario has been considered to apply the proposed approach for DT's lifecycle modeling and characterization. Specifically, scaled industrial stations manufactured by Fischertechnik\footnote{Fischertechnik: \url{https://www.fischertechnik.de/en/products/industry-and-universities}} has been adopted as reference industrial prototyping environment with multiple stations and involved machines. The first station, referred to as the \textit{Multiprocess Station with Oven}, features an oven, a robotic arm for handling operations, and a rotary table equipped with multiple processing stages. The second station, named the \textit{Indexed Line}, comprises four conveyor belts, two of which include mechanical processing operations for the workpieces.

\begin{figure}[t]
    \setlength{\belowcaptionskip}{-13pt}
    \centering
    \includegraphics[width=0.9\columnwidth]{figures/dt-lifecycle/multiprocess-station-shcematic-overview.pdf}
    \caption{Multi-process Station schematic overview.}
    \label{fig:multiprocess-station-schematic-overview}
\end{figure}

\begin{figure}
    \setlength{\belowcaptionskip}{-13pt}
    \centering
    \includegraphics[width=\textwidth]{figures/dt-lifecycle/graph_machine_logic.pdf}
    \caption{Machine DT Lifecycle with a focus on the different phases in the Synchronization state.}
    \label{fig:graph-machine-logic}
\end{figure}

\subsection{DT Design \& Development}

The design, development, and deployment of Digital Twins (DTs) for the target use case are motivated by the need for precise monitoring, control, and optimization of physical machinery and stations. By creating digital replicas of the physical systems, we aim to enhance operational efficiency, reduce downtime, and enable predictive maintenance through detailed real-time data analysis and simulation capabilities.

To achieve these objectives, multiple DTs have been designed, developed, and deployed to represent the associated Physical Twins (\ac{PA}s). Specifically, each station was divided into several processing machines, with a DT created for each one. Subsequently, a \textit{composed} DT was developed for each station to aggregate data from the individual machine DTs, enabling a complete digital representation of the station. This allows for the monitoring and management of the combined operations of multiple physical assets working together through the production process. For example, Figure \ref{fig:multiprocess-station-schematic-overview} illustrates the schematic representation of the Multiprocess Station along with the DTs it comprises.

The DTs were implemented using the White Label Digital Twin (WLDT)\footnote{WLDT Library: \url{https://wldt.github.io/}} library~\cite{wldt_picone_2021}. WLDT is a Java implementation of an event-driven DT framework that supports data ingestion through IoT interfaces and message queues. We extended the library to support the proposed lifecycle management approach by implementing both the Physical Twin Lifecycle ($LP_{PA}$) and the Digital Twin Lifecycle ($LP_{DT}$). Each DT features various adapters for external communication, including both physical and digital interfaces. For physical interface management, the DTs controlling individual machines use an OPC-UA\footnote{OPC-UA-Foundation: \url{https://opcfoundation.org/}} adapter, which enables interaction with Siemens S7-1200 PLCs\footnote{Siemens-S7-1200-PLC:\url{https://www.siemens.com/global/en/products/}}. Additionally, each DT includes MQTT\cite{mqtt} Digital and HTTP Digital adapters to expose both its current state and the lifecycle's phases it represents through different protocols and interaction patterns (Pub/Sub and RESTful).

Each machine DT handles the synchronization between the $LP_{PA}$ and the corresponding $LP_{DT}$. Furthermore, an anomaly detection algorithm runs within each DT, enabling it to identify and report issues that the PLC alone might not detect, thus dynamically computing the $LP_{PA}$ when necessary (e.g., when a piece does not reach a reference photocell on the machinery within a predetermined time and an anomaly is consequently detected).

The DTs also calculate the \textit{Overall Equipment Effectiveness} (OEE)\footnote{OEE: \url{https://www.oee.com/}}, a key efficiency metric in the industrial domain that measures the productivity of manufacturing equipment by combining its availability, performance, and quality rates. We compute OEE in our DTs to provide a comprehensive measure of equipment effectiveness, which is critical for identifying areas of improvement and optimizing production processes. This data is exposed for external monitoring.

The composite DT uses an MQTT adapter as its physical interface to collect data and lifecycle phases from the individual machine DTs. It incorporates coordination logic that adjusts machine behavior based on their current phases, ensuring seamless station management. Additionally, the composite DT has its own MQTT Digital and HTTP Digital adapters to expose the station’s overall state externally. The described DT structure is illustrated in Figure \ref{fig:multiprocess-station-dt-structure}.

The DTs of the machines operate based on data produced by their physical counterparts. This data originates from readings of all sensors present on the machinery, along with the states of its actuators. These data are mapped to the DT state, ensuring that at any given moment, the DT reflects the state of the physical object. Leveraging these temporal data, it was possible to model the lifecycle of the physical object within the DT lifecycle.

When the DT lifecycle reaches the Synchronized state, it accurately mirrors the corresponding phase of the physical machinery's operation. The modeled lifecycle, as shown in Figure \ref{fig:graph-machine-logic}, outlines the generic phases defined for each machine. These phases not only reflect the actual operation of the machinery but are also inspired by the reference structure for industrial machine phases defined by ISA-95\footnote{ISA-95:https://www.isa.org/standards-and-publications/isa-standards/isa-standards-committees/isa95}. This alignment ensures that our DTs adhere to recognized industry standards, enhancing their applicability and effectiveness in real-world industrial settings.

To accurately define the phases of the physical machinery, it was essential to specify the transition conditions between these phases by leveraging the raw data and telemetry received from sensors and actuators via OPC-UA from the PLC. The DT processes this data to compute and maintain the synchronized lifecycle, ensuring a precise and real-time reflection of the physical object's operational state. This approach enables the DT to seamlessly transition between phases based on the actual conditions and performance of the machinery, providing an accurate digital representation.

\begin{figure}[t]
    \setlength{\belowcaptionskip}{-13pt}
    \centering
    \includegraphics[width=\columnwidth]{figures/dt-lifecycle/mps_dt_structure_2.pdf}
    \caption{Multi-process Station Digital Twin overview.}
    \label{fig:multiprocess-station-dt-structure}
\end{figure}

\subsection{Lifecycle Awareness \& Coordination}

In an industrial setting, the digital representation of a station is often a hierarchical composition of individual machines, each represented by its own DT. This structure enables intelligent coordination not only between machines within the same station but also across different stations. For example, coordination between the \textit{Multiprocess Station} and the previously mentioned \textit{Indexed Line} becomes more manageable compared to a direct access to low level raw data of the different physical machines. Incorporating the PA lifecycle into the DT lifecycle simplifies coordination by establishing consistent lifecycle phases across machines.
Without this approach, coordination among multiple DTs required using low-level physical information and telemetry, making the process more complex and less intuitive. Now, as illustrated by Algorithm \ref{code:soft-stop-logic}, high-level control logic can be implemented directly leveraging information encapsulated by $LP_{DT}$ mapping in real-time the phase and the associated transitions of the associated physical counterparts $LP_{PA}$. Specifically, the algorithm outlines the steps the composite DT can take to halt production on the multiprocess station. By leveraging this additional layer of abstraction, the composite DT no longer relies on raw sensor data for control operations, enabling a more declarative and effective coordination process, and distributing responsibilities across different DTs.

The composite DT leverages the PA lifecycle phases of individual machines to define the overall lifecycle of the entire multiprocess station. This provides a comprehensive overview of each station's operational behavior, enabling more effective and intuitive system management.

\note{FIX ALGORITHM}
% \begin{algorithm}
% \caption{Multiple DTs Stopping Procedure}
% \begin{algorithmic}
% \State \texttt{invoke(OvenDT, STOP)}
% \While{ \texttt{OvenDT.phase} \textbf{is not} \texttt{STOPPED} }
%     \State \texttt{wait(OvenDT.phase} \textbf{is not} \texttt{WORKING)}
% \EndWhile
% %\vspace{1em}
% %\State \textbf{publish} event ``change-vacuum-gripper-cycle" with value \texttt{false}
% \State \texttt{invoke(GripperDT, STOP)}
% \While{ \texttt{GripperDT.phase} \textbf{is not} \texttt{STOPPED}}
%     %\State \textbf{log} waiting for end of vacuum-gripper \texttt{WORKING} state
%     \State \texttt{wait(GripperDT.phase} \textbf{is not} \texttt{WORKING)}
% \EndWhile
% %\vspace{1em}
% %\State \textbf{publish} event ``change-turntable-cycle" with value \texttt{false}
% \State \texttt{invoke(TurntableDT, STOP)}
% \While{ \texttt{TurntableDT.phase} \textbf{is not} \texttt{STOPPED} }
%     %\State \textbf{log} waiting for end of turntable \texttt{WORKING} state
%     \State \texttt{wait(TurntableDT.phase} \textbf{is not} \texttt{WORKING)}
% \EndWhile
% %\vspace{1em}
% \State \texttt{invoke(OutputConveyorDT, STOP)}
% \end{algorithmic}
% \label{code:soft-stop-logic}
% \end{algorithm}



\subsection{Industrial Digital Twins with WLDT}
\label{sec:industrial-dt-wldt}

The proposed DT architectural approach has been initially applied and validated in a realistic industrial environment: a microfactory which serves as an experimental platform for DTs and industrial ecosystems. The industrial environment has been strategically chosen since it represents one of the most challenging application scenarios for DTs, characterized by stringent requirements in terms of integration with heterogeneous systems, and the management of fragmented data through multiple stakeholders. 

However, it is crucial to emphasize that the proposed approach is inherently flexible and designed to be applicable in a wide spectrum of other  contexts. While the initial validation focused on the industrial domain we foster the application of our framework in other domains and use cases, such as Smart Cities \cite{10.1145/3570361.3614070} and Health \cite{9325551} where the concepts of DTs can bring significant value. The modularity and scalability of our solution facilitate its adoption in various types of applications, and future work may explore in detail its effectiveness in these additional contexts.

Specifically, the validation in the industrial domain was carried out using the \emph{Fischertechnik Training Factory Industry 4.0}\footnote{Fischertechnik Industry \& Universities: \url{https://www.fischertechnik.de/en/products/industry-and-universities}} Indexed-Line Station, featuring the \emph{Multiprocess Station with Oven} module.
This microfactory setup provides a controlled yet dynamic environment for testing and refining DT technologies in a realistic industrial context.

%%%
\begin{figure*}
    \setlength{\belowcaptionskip}{-13pt}
    \centering
    \includegraphics[width=0.92\textwidth]{figures/engineering-wldt/Fischer-relationships.pdf}
    \caption{DTs relationships with both those existing in the physical space as well those belonging to the digital composition.}
    \label{fig:dt-assets-relationships-overview}
\end{figure*}
%%%

As shown in Figure~\ref{fig:discher-dt-first-level}, the module consists of five machines arranged in a flow shop layout: three material handling units (vacuum gripper carrier, turntable, and conveyor) and two transformation stations (oven and saw station).
Each machine is equipped with sensors (light barriers and limit switches) and actuators (with two or three operational states), all managed by a 24V industry-grade digital board.
A Raspberry Pi-based soft-PLC, connected to a 24V Digital I/O expansion board, controls the system and coordinates machine operations via sensors and actuators. Real-time process data is published to an MQTT broker DTs can subscribe to to receive data.

The digitalization goal is to monitor the state, track production performance and power consumption, and regulate the production speed of each machine through a DT.
Additionally, to compute the overall production performance and manage external action requests, and to track the overall power consumption, two department-level DTs are created, matching the two different modeling goals.
Each machine-level DT processes incoming data on properties and events to generate insights such as production and energy KPIs, while also managing action requests to regulate their individual speed.
Relationships between DTs are crucial, as they represent connections between physical assets, which can be used by external applications and other DTs, facilitating \emph{compositions} to form higher-level DTs.

% The digitalization of the department begins with the machine layer, using DT compositions to create higher-level DTs. To capture different aspects of the department (e.g., overall production performance and energy consumption), machine-level DTs can be integrated into two distinct department-level compositions, each specializing in one of the targeted aspects.

% This specialization is achieved by selecting specific information in the DT configuration and using the department DT model to manipulate, augment, and output relevant information. Department-level DTs, structured similarly to machine-level DTs, handle action requests through their internal models, just like the lower-level DTs. Relationships maintain their importance, representing actual connections between the department DT (whether specialized in performance or energy tracking) and other DTs within the system.

%%%

\subsection{Machines Digital Twins}
\label{subsec:dt-machine-level} 

The digitalization process creates five machine-level DTs, one for each machine in the factory (Figure~\ref{fig:discher-dt-first-level}). These DTs each interact via an MQTT broker through the PI, are equipped with an internal model implemented by the Shadowing Function, and have a Digital Interface to share data.
The DT ecosystem digitalizes the physical system without replacing the PLC logic, allowing for the independent operation of the PLC if the DT ecosystem fails.
DTs are deployed on edge nodes for monitoring and control without disrupting core operations.

Each DT connects to the broker through MQTT Physical Adapters, which manage subscriptions to relevant topics and publish the description of PA. The internal model of each DT calculates metrics like OEE (based on production rate, uptime, downtime, and quality) and tracks power consumption. Relationships between machines and DTs (e.g., \emph{previous}, \emph{following-machines}, \emph{is-part-of}) are defined at startup and remain static during the experiment.
Machine DT states are exposed via MQTT and HTTP Digital Adapters, enabling flexible interaction patterns and action requests through HTTP.

%%%
\begin{figure*}
    \setlength{\belowcaptionskip}{-13pt}
    \centering
    \includegraphics[width=\textwidth]{figures/engineering-wldt/experimental_results.pdf}
    \caption{Experimental Results: CPU usage percentage [\%], Memory usage [Mib], Network usage [KBit/sec] and OEE [\%]}
    \label{fig:exp-results}
\end{figure*}
%%%

%%%

\subsection{Department Digital Twins}
\label{subsec:dt-department-level}

The identified department DTs model and monitor KPIs and energy metrics based on the machine DTs structure. The Department KPI DT focuses on tracking performance, including weighted OEE and throughput. Weighted OEE is calculated using OEEs received from each machine DT~\cite{OEE-manufacturing-cell-Gamberini-2017}.
Throughput, defined as the number of processed pieces per time unit in production systems, is tracked by monitoring entry and exit events of pieces processed by machines within the department, facilitated by relationships in the DT ecosystem.

The Department-KPI DT tracks piece processing start and completion, calculates throughput, and monitors total pieces processed.
Relationships between lower and upper-level DTs are shown in Figure~\ref{fig:dt-assets-relationships-overview}.
The Department DT accepts action requests via its HTTP digital adapter, adjusting production rates through its Shadowing Function, which cascades actions to underlying DTs and physical assets.
The Department-Energy DT monitors energy consumption, aggregating data from machine-level DTs, and processes it through an internal model to derive target metrics.
It also uses an HTTP digital adapter for external interaction.
Department-level DTs use MQTT-based physical adapters to ensure consistent communication and enable external applications to coordinate actions across the underlying DT instances.

% The Department-KPI DT can track the start and completion of piece processing, calculate throughput based on time per piece, and total pieces processed. Relationships between lower-level and upper-level DTs are illustrated in Figure~\ref{fig:dt-assets-relationships-overview}. The Department DT accepts action requests via its HTTP digital adapter, supporting operations to adjust production rates. Speed changes are processed through a simple algorithm in the Department DT's Shadowing Function, cascading actions to underlying DTs and associated physical assets. The Department Energy DT monitors overall energy consumption, aggregating data from machine-level DTs regardless of whether it's specific energy consumption or disaggregated voltage and current data. Its internal model handles this heterogeneity to derive target metrics. The Department Energy DT also utilizes an HTTP digital adapter for external interaction, providing access to its internal state information. Department DTs utilize MQTT physical adapters within their physical interface, ensuring consistent implementation of composed DTs. They enable external applications to interact with the department through HTTP digital adapters, managing actions across underlying DTs based on the department's internal model and behavior.


%%%
% \begin{figure*}
%     \centering
%     \setlength{\belowcaptionskip}{-13pt}
%     \includegraphics[width=\textwidth]{images/masa_dt_result.png}
%     \caption{Experimental evaluation of the MASA DT deployment with CPU\%, Memory usage\% and the cumulative Network I/O.}
%     \label{fig:masa_dt_results}
% \end{figure*}
%%%

\subsection{Experimental Evaluation}
\label{ssec:masa_exp_evaluation}

To validate the WLDT framework, we implemented the DTs for the Multiprocess Station with Oven, containerized using Docker, and deployed on an Edge node with an Intel i7 2.4GHz processor and 32 GB of RAM. We monitored three key metrics for each DT: CPU usage (percentage, with each container using one core), memory consumption (in MiB), and network traffic (cumulative inbound and outbound in Kbit/sec).
Additionally, we calculated the OEE computed by each DT to assess the performance of both individual machines and the station. Figure \ref{fig:exp-results} shows the results of this experiment. The total duration was 10 minutes (600 seconds), as shown on the x-axis of each chart. The graphs illustrate CPU usage, memory usage, network traffic, and the computed OEE (from left to right). Analyzing CPU usage, we observe that there were no significant spikes during the production cycle. Even during the most computationally demanding moments, CPU utilization remained around 30\%, indicating a stable and efficient execution.
The memory usage chart shows a similarly stable pattern, with DT instances settling around an average usage of 300 MiB, demonstrating consistent memory behavior over time.
As for network traffic, values remained low—typically around a few tens of Kbit/sec.
Noticeable traffic peaks correspond to active processing phases, where machines and their respective DTs exchange larger volumes of data.
Finally, the OEE graph highlights the performance of industrial machines computed by each DT. The Oven, which is the first machine in the station and requires the longest processing time, shows the highest OEE, as it remains under constant workload.
Conversely, the output conveyor exhibits a lower OEE, as it receives fewer parts to process due to the upstream bottleneck created by the Oven. This experimental phase aimed to demonstrate the effectiveness of WLDT-based DTs in an industrial setting.
The collected metrics confirm that DT instances can operate efficiently on standard commercial Edge nodes, handling CPU, memory, and network constraints while also supporting domain-specific computations such as OEE. These results validate the potential of the WLDT framework for practical DT development and deployment in real-world industrial scenarios.

%%%%%%%%%%%%%%%%%%%%%%%%%%%%%%%%%%%%%%%%%%%%%%%%%%%%%%%%
\chapter{\aclp{DTE}}
\label{chap:dte:dte}
%%%%%%%%%%%%%%%%%%%%%%%%%%%%%%%%%%%%%%%%%%%%%%%%%%%%%%%%

%======================================================
\section{Characterizing \aclp{DTE}}
\label{sec:ecosystems}
%======================================================

As discussed in \Cref{sec:background}, the \ac{DT} concept is shifting from ultra-detailed replicas of individual \acp{PA} to contextualized models tailored to specific applications~\cite{minerva2020dtiot}.
%
However, many solutions either mirror isolated assets -- overlooking their interrelations -- or adopt monolithic models for complex systems (e.g., entire cities~\cite{Deren2021}).
%
In this paper, we focus instead on the idea of modeling such large-scale scenarios using what we informally define as a:

\note{definition}
% \newtheorem*{dte}{Digital Twin Ecosystem}
% \begin{dte}
%     A dynamic set of \aclp{DT}, each representing a \ac{PA}, whose meaningful relationships within a target context make it valuable to consider their collective evolution to accurately reflect the mirrored portion of the physical world.
% \end{dte}

In order to characterize \acp{DTE}, in this section,
we analyze the challenges surfacing from the heterogeneity of complex domains (\ref{rq:dimensions})
and provide an operational definition of \acp{DTE} in terms of functionalities and services that consumers may expect to use (\ref{rq:functionalities}).
Finally, we compare architectural approaches for \acp{DTE} by investigating the state of the practice (\ref{rq:implementations}).


%=======================================================
\section{Operational Characterization}
\label{sec:operational-characterization}
%=======================================================

To better characterize the concept of \ac{DTE}, we analyze the fundamental functionalities to ease the fruition of services that span across several \acp{DT}.
%
We intend this operational definition to be general, regardless of the architectural design adopted to develop the ecosystem.

We consider the perspective of both \emph{managers} of the \ac{DTE} who need to oversee the creation and evolution of the ecosystem, and \emph{consumers} who instead are interested in exploiting the \ac{DTE} abstraction for their application goals.
%
We will then let these functionalities guide our implementation proposal that we describe in \Cref{sec:hwodt-idea}.

\subsubsection{Managing the \ac{DTE}}\label{sssec:operational-managing}

We define \ac{DTE} as a \emph{dynamic} set of \acp{DT} as members could join or leave the ecosystem at any time due for different reasons.
These include the \acp{PA} lifecycle (e.g., decommissioning) or application-specific inclusion criteria adopted when modeling the ecosystem (e.g., entering or exiting a geographical area).
%
To support this dynamism, \acp{DTE} should provide operations to \emph{add} and \emph{remove} \acp{DT}.
These can be public or restricted to administrators, and invocations can be either manual or automatically triggered by monitoring processes (or the \acp{DT} themselves) when detecting relevant changes in the physical environment.

A consequence of this dynamic management is the need to easily \emph{discover} which \acp{DT} are part of the ecosystem.
\acp{DTE} may hold an index of all (currently) registered \acp{DT}, each \emph{uniquely identified} and described with metadata to support fine-grained discovery.
%
A key piece of information to store is the mirrored \ac{PA} identifier, allowing tracking of both the digital and physical components of the ecosystem.
As noted in \Cref{ssec:dimensions}, since multiple \acp{DT} may represent the same \ac{PA} for different purposes, this information becomes especially valuable.
Moreover, to keep track of a \ac{DT} evolution, it should be possible to \emph{update} metadata over time.

\subsubsection{Exploiting the \ac{DTE}}\label{sssec:operational-exploiting}

\emph{Consumers} of a \acp{DTE} represent either human users, applications, or intelligent agents~\cite{burattini2025iot} which may be interested in observing the state of the registered \acp{DT} and exploiting their services.
%
An ecosystem should then offer functionalities to access individual \acp{DT} and ecosystem-wide services. 

This includes the ability to retrieve the current \emph{representation} of the state of all \acp{DT} and their relationship at a given time.
%
As the representation could grow very large, it can be more effectively managed by enabling consumers to \emph{query} the ecosystem, enabling them to select and aggregate relevant data across multiple \acp{DT}.
%
This supports the derivation of insights that would not be easily discovered by accessing each \ac{DT} individually.

Finally, given the dynamic nature of \acp{DT} continuously updating their representation of the corresponding \acp{PA}, \acp{DTE} should support \emph{observation} patterns -- even selectively through continuous queries~\cite{babu2001sigmod} -- to track changes of \acp{DT} and of the whole ecosystem over time.

Although not an operation provided by the ecosystem itself, we include \emph{navigation} by following \ac{DT} relationships as a fundamental feature of \acp{DTE} complementing the other interaction patterns.
%
This allows users to explore and discover information progressively.
Moreover, relationships may cross ecosystem boundaries, offering paths to discover related \acp{DTE}.

%=======================================================
\section{Architecting \aclp{DTE}}
\label{sec:architecting-dte}
%=======================================================

\begin{figure}[ht]
    \centering
    \begin{subfigure}[b]{0.3\linewidth}
        \centering
        \includegraphics[width=\linewidth]{figures/hwodt/ecosystems_types-monolithic.pdf}
        \caption{\textbf{Monolithic} \ac{DTE}}
        \label{fig:ecosystem-monolithic}
    \end{subfigure}
    \hfill
    \begin{subfigure}[b]{0.3\linewidth}
        \centering
        \includegraphics[width=\linewidth]{figures/hwodt/ecosystems_types-homogeneous.pdf}
        \caption{\textbf{Homogeneous} \ac{DTE}}
        \label{fig:ecosystem-homogeneous}
    \end{subfigure}
    \hfill
    \begin{subfigure}[b]{0.3\linewidth}
        \centering
        \includegraphics[width=\linewidth]{figures/hwodt/ecosystems_types-heterogeneous.pdf}
        \caption{\textbf{Heterogeneous} \ac{DTE}}
        \label{fig:ecosystem-heterogeneous}
    \end{subfigure}
    \caption{Three architectural approaches for \acp{DTE}: in \ref{fig:ecosystem-monolithic} a single \ac{DT} models the whole scenario; in \ref{fig:ecosystem-homogeneous} a single platform is used to build and deploy all the \acp{DT} in the ecosystem; in \ref{fig:ecosystem-heterogeneous} maximum flexibility is allowed, possibly reusing existing \acp{DT} under a common interface.}
    \label{fig:ecosystem-types}
\end{figure}

This section analyzes and compares approaches from current practice in applying \acp{DT} to complex scenarios involving multiple \acp{PA}.
We focus on four aspects derived from our \ac{DTE} characterization in terms of heterogeneity and operational behvior:
\begin{inlinelist}
\item the degree to which the ecosystem can evolve to reflect physical-world changes (\emph{evolvability}),
\item whether they support heterogeneous \acp{DT} within the same ecosystem (\emph{heterogeneity}),
and 
\item whether ecosystem-level functionalities are offered to users (\emph{services}).
\end{inlinelist}

We reviewed relevant works from the state of the art, survey papers, and technical documentation of \ac{DT} frameworks and platforms.
Although not a systematic review, our analysis shows three main emerging architectural approaches to implement \acp{DTE}.
%
We name these patterns \emph{monolithic}, \emph{homogeneous}, and \emph{heterogeneous} \acp{DTE}, and we schematically show their different architectures in \Cref{fig:ecosystem-types}.
We summarize our feature comparison in \Cref{tab:comparison-summary} and discuss the main peculiarities of each approach below.


\note{table}
% \noindent
% \begin{table}[ht]
%     \centering
%     \begin{tabular}{l|c|c|c}
%     \toprule
%     \midrule
%     \textbf{} & \textbf{Mon.} & \textbf{Hom.} & \textbf{Het.} \\
%     \hline
%     \hline
%     \textit{Evolvability} & $\times$ & \checkmark & \checkmark
%     \\
%     \hline
%     \textit{Heterogeneity} & $\times$ & $\times$ & \checkmark
%     \\
%     \hline
%     \textit{Services} & \checkmark & \checkmark & \checkmark
%     \\
%     \hline
%     \bottomrule
%     \end{tabular}
%     \caption{Comparison summary of Monolithic (Mon.), Homogeneous (Hom.), and Heterogeneous (Het.) \acp{DTE} that shows supported (\checkmark)
%     %partially supported ($\sim$),
%     and not supported ($\times$) features.}
%     \label{tab:comparison-summary}
% \end{table}

\subsubsection{Monolithic \acl{DTE}}
\label{sssec:monolithic}

The most straightforward approach is to model the entire targeted context as a single, monolithic \ac{DT}.
Due to the fuzzy definition of a \ac{PA}, even complex entities (e.g., an entire city) can serve as valid physical counterparts of a single \ac{DT}, with some models explicitly capturing multiple internal components of such entities.

The main benefit of this approach is its simplicity in deriving the overall system architecture:
all physical-world data can flow into a single system built with a consistent technological stack (\Cref{fig:ecosystem-monolithic}).
%
A monolithic \ac{DT} also offers strong modeling capabilities and facilitates services such as running simulations on a single model of the entire ecosystem.

The drawbacks of the approach are on the management side. Representing with high fidelity dynamic scenarios involving multiple heterogeneous assets can be challenging, and even minor updates -- such as adding new entities or modifying the behavior of existing ones -- may require modifying the entire \ac{DT}.
%
This approach is not suited for rapidly evolving ecosystems or contexts involving multiple stakeholders.
%
Relying on a single \ac{DT} limits technological diversity and prevents the reuse of existing \acp{DT}, forcing the creation of a new, unified model.

Examples of monolithic \acp{DT} include modeling a single \ac{DT} for entire roads~\cite{KUSIC2023101858}, hospital buildings~\cite{dt_hospital}, or smart grids~\cite{9449682}. 

\subsubsection{Homogeneous \acl{DTE}}
\label{sssec:homogeneous}

We refer to \emph{homogeneous} \acp{DTE} when the approach relies on a general-purpose platform that natively supports the definition of multiple \acp{DT} (\Cref{fig:ecosystem-homogeneous}).
%
The \ac{DT} are hence homogeneous in their modeling and implementing technology as they are created within the same platform.
%
The native support for \acp{DTE} enhances adaptability to the evolving physical world as it is simple to add and remove \acp{DT} or update models.
%
Furthermore, the modeling and technological homogeneity make implementing ecosystem services relatively easy, as in the \emph{monolithic} approach.
%
Differently, though, it is possible to highlight the individuality of the involved \acp{DT}, which may independently process the relative data flows, offering better decomposition and separation of concerns.

However, the dependency on a single model and platform may introduce vendor lock-in, modeling and organizational constraints, limiting support for heterogeneous \acp{PA}.

An example of this approach is the Azure Digital Twins platform, which introduced the concept of a \emph{Twin Graph}\footnote{\url{https://learn.microsoft.com/en-us/azure/digital-twins/concepts-twins-graph}} to explicitly represent relationships among \acp{DT} modeled using a uniform \ac{DTDL}.


\subsubsection{Heterogeneous \acl{DTE}}
\label{sssec:heterogeneous}

The third approach is based on introducing a common interoperability layer on top of heterogeneous \acp{DT}.
%
By embracing the inherent heterogeneity of \acp{DTE}, it enables the reuse of existing \acp{DT} while ensuring uniform access and navigation for consumers (\Cref{fig:ecosystem-heterogeneous}).
%
\acp{DT} are implemented as standalone software components, each with specific models and technologies, and each exposing interfaces that can be used by applications directly or through intermediary aggregators.

This approach is well-suited for dynamic, open \acp{DTE} where \acp{DT} are developed by various stakeholders since the individual components are loosely coupled.
%
Furthermore, \acp{DT} are not limited to being part of only one ecosystem, granting an additional degree of flexibility.

To maintain interoperability, though, each \ac{DT} must be mapped to a shared metamodel to represent it within the ecosystem and expose a shared interface to enable seamless interaction.
This can lead to a potential loss of information and expressivity, as with any model conversion.
However, direct access to the original \ac{DT} remains possible when needing to use specialized services not easily mapped in the shared model.

Compared to previous approaches, where services are typically provided by the implementing technology/platform and benefit from modeling uniformity, implementing ecosystem-wide services in heterogeneous \acp{DTE} is inherently more complex.
%
This can be achieved through either fully distributed techniques
-- such as polystores~\cite{dggan2015polystore} or federated and link-traversal queries~\cite{schwarte2011semweb,quilitz2008querydistributed,bogaerts2021linktraversalquery} --
or by introducing a middleware (as shown in \Cref{fig:ecosystem-heterogeneous}) aggregating \acp{DT} and implementing the required functionalities.
%
This doubles as a way to set a clear boundary for operations, allowing a clear definition of which \acp{DT} belong to an ecosystem.

In this paper, we demonstrate how such an interoperability layer and ecosystem services can be implemented using Web technologies and standards, paving the way for the realization of open heterogeneous \acp{DTE}.

To make a parallel with open Web-based ecosystems, this would be similar to a web service exposing a well-documented public HTTP API, which does not prevent specialized consumers from using remote method invocation if they have specialized knowledge on how to interact with the server.

For the approach to be successful and encourage \ac{DT} developers to join heterogeneous ecosystems, the entry barrier needs to be set at a relatively low level to ensure take-up, while still maintaining a sufficient level of complexity to meet the overall objectives~\cite{kendall2021ndt}.



%%%%%%%%%%%%%%%%%%%%%%%%%%%%%%%%%%%%%%%%%%%%%%%%%%%%%%%%
\chapter{Operational Management of \aclp{DTE}}
\label{chap:dte:dtc}
%%%%%%%%%%%%%%%%%%%%%%%%%%%%%%%%%%%%%%%%%%%%%%%%%%%%%%%%

This chapter investigates \ref{rq:3} by exploring the challenges in the operational management of \acp{DTE} that involve the management of multiple \acp{DT} from a deployment and runtime perspective.
%
Several \acp{DT} need to be orchestrated on a cyber-physical computing infrastructure, ensuring that \acp{DT} are able to interact with the \acp{PA} they represent and maintain the target level of performance and synchronization. 
%
This problem is further exacerbated when considering that \ac{DTE} can involve \acp{DT} developed for different platforms, and computing infrastructures that span the whole \emph{edge-cloud continuum}. 

Motivated by these challenges, this chapter introduces the concept of a \emph{\ac{DTC}}, a middleware that aims to facilitate the operational management of \acp{DTE} by providing a unified interface to manage the deployment and runtime of \acp{DT} across heterogeneous \ac{DT} platforms and computing infrastructures.
%
This proposal complements the \ac{HWoDT} approach presented in Chapter~\ref{chap:dte:hwodt}, which focuses on the interaction with \ac{DTE}, rather than their operational management. 

The chapter analyzes the challenges in the operational management of \acp{DTE}, and presents a proposal for an architecture of the \ac{DTC} middleware. The functionalities are supported by a set of \emph{descriptions} that can inform the \ac{DTC} about the characteristics of the \acp{DT} and \acp{PA} involved in the \ac{DTE} and support stakeholders in having a clear understanding of the system state at any stage.


%=======================================================
\section{Operational Challenges}
%=======================================================

An orthogonal aspect of complexity when developing a \ac{DT} is understanding its non-functional requirements that allow for considering the \ac{DT} effectively synchronized.
%
The concept of \emph{entanglement}~\cite{dt-IoT-context-Minerva-2020} has been used to define the level of synchronization of the \ac{DT} with the \ac{PA} and can be measured with metrics that include network latency and computation time on both the physical device and the computing node that is running the \ac{DT}~\cite{bellavista2024odte}.

In recent years, with the increased availability of computing resources at the edge of the network, the concept of edge-cloud continuum has emerged as a paradigm to distribute computation across a variety of computing nodes, obtaining benefits in terms of latency and optimizing the overall workload of an \ac{IoT} system~\cite{Rosendo_Costan_Valduriez_Antoniu_2022}.
%
Like other \ac{IoT} system components, \acp{DT} can also be deployed within the continuum~\cite{Bellavista_Bicocchi_Fogli_Giannelli_Mamei_Picone_2024} with trade-offs in terms of entanglement, cost and resource availability.
%
The advantages of this approach, though, are offset by additional complexity in the operational management of the software system.
When it comes to \acp{DT}, this means choosing the suitable computing node to deploy the \ac{DT} software, to ensure the \ac{QoS} requirements are matched at runtime.
This can be challenging due to the cyber-physical nature of \acp{DT}.
Additionally, when several \acp{DT} are employed as parts of an ecosystem sharing the same computing infrastructure, it should be possible for operators to easily navigate the continuum and observe performance metrics to understand the potential impact of deploying new \acp{DT} in the system and eventually reconfigure it to balance the overall workload.

On the development side, the heterogeneity of \ac{DT} platforms poses additional challenges as they may constrain the ability to deploy a \ac{DT} across the edge-cloud continuum.
%
Indeed, while open-source platforms can be adapted to run on a generic infrastructure, proprietary ones are usually Platform as a Service solutions, bound to the cloud infrastructure. 
%
Developers tasked with implementing a \ac{DT} must hence evaluate the suitability of a given \ac{DT} platform, taking into consideration both the functional (i.e., meta-model, features, services) and non-functional (i.e., latency, computing power, privacy) requirements.

Since these requirements may vary significantly depending on the \ac{PA} being modeled it is realistic to assume that a complex \ac{DT}-based system may include \acp{DT} developed on different platforms.
%
This has surfaced challenges in the management of the development and deployment of \ac{DT}-based systems which may need to guarantee the \ac{QoS} requirements of \acp{DT} developed with heterogeneous technologies and platforms while managing a shared pool of heterogeneous computing resources.

The proposal of the \ac{DTC} aims to address the following goals:
\begin{itemize}
    \item \textbf{reducing deployment complexity} of a \ac{DT}-based system integrating \acp{DT} developed for different platforms sharing the same computing infrastructure.
    \item \textbf{facilitating the interaction of different stakeholders} with a \ac{DT}-based system enabling easy access of relevant information over the development-to-deployment lifecycle of \acp{DT}.
\end{itemize}

%========================================================
\section{Phases and Stakeholders}
%========================================================


The process of developing and managing a \ac{DTE} involves several critical phases, each essential for the effective development, discovery, selection, deployment, and operation of \acp{DT}.
%
These phases abstract the development-to-deployment lifecycle of \acp{DT} in a structured approach which is independent of the target application domain or specific technological choices.

Identifying these phases helps shape the requirements for the \ac{DTC} and understand the roles of the stakeholders that are involved in each phase.
%
\Cref{fig:dtc-phases} summarizes the main phases and the stakeholders involved in the development and management of \acp{DT} in a \ac{DTC} that are described in the following.

\begin{figure}[tb]
    \centering
    \includegraphics[width=\textwidth]{figures/dtc/dt-phases-stakeholders-v2.pdf}
    \caption{Stakeholders involved in different phases of the development and management of \acp{DT} in an ecosystem.}
    \label{fig:dtc-phases}
\end{figure}


\paragraph{DT Software Capabilities Discovery \& Selection}
As a starting point, every DT system is originating from the needs of a cyber-physical context.
%
This usually involves one or several \emph{\ac{PA} Owner} who are the stakeholders that possess the knowledge about the \acp{PA} and the goal that requires \acp{PA} to be modeled and digitalized~\cite{michael2024software}.
The \emph{\ac{PA} Owner} is also in charge of granting access to the \ac{PA} data and sensors. 
%
The first phase hence focuses on identifying DT capabilities that align with specific application and use case requirements.
%
This phase involves defining the functionalities of the DT, associating it with the corresponding physical asset category, specifying communication protocols, and detailing its properties, events, relationships, and available actions. 
%
Notably, an asset might have multiple owners.
For instance, an industrial machine is produced by a manufacturer and acquired by a company to employ it in one of its facilities.
The same machine is then owned simultaneously by the manufacturer who may have control of telemetry data and be interested in monitoring the machine to offer predictive maintenance services, by the company who might want to keep track of all the machines across facilities and by the facility manager who might monitor the state of operational of the machine in the specific production line it is employed. 
All the owners may have access to different data and model the same asset with different target goals.


\paragraph{DT Platform Selection and Development}

A \emph{\ac{PA} Owner} may commission a \emph{DT Developer} to implement the DT for the PA. The developer will implement a DT using a target technological stack, that depends on their expertise and the availability of the computing infrastructure that is available for the deployment of the DT.
%
In this phase it is necessary to ensure that the chosen DT requirements can be effectively implemented on a specific platform.
%
This phase involves defining platform-specific configurations, including implementation details, required configuration parameters, and technical requirements.
%
Notably, the \emph{\ac{PA} Owner} is usually the (virtual) owner of the computing infrastructure on which the DT will be eventually deployed. When the DT is developed as commissioned by the \emph{\ac{PA} Owner} this may influence the technological choices of the \emph{\ac{DT} Developer}.
%
A possible future may envision \acp{DT} implementations for specific kinds of assets made available to owners of such assets either publicly or commercially.
These would be implemented either by manufacturers or a community of \emph{\ac{DT} Developers}.
Such reusable \acp{DT} would then be implemented with a given technological stack and need to be deployed on the available resources of the \emph{\ac{PA} Owners}, in order to be configured and connected to the locally available \acp{PA}. 
Similarly, such reusable \acp{DT} could be offered as-a-service and made available only as instances deployed on commercial platforms. This adds a further layer of complexity to the management of the computing infrastructure running the possibly different \acp{DT} in an organization.


\paragraph{Digital Twin Deployment}

Once a DT is developed (from scratch or reusing existing implementations) the next goal is to run it on a computing infrastructure.
This step involves a \emph{DT Operator} who have knowledge about the \ac{DT} requirements and can configure the computing infrastructure to make sure that the DT is deployed correctly and gets access to the \ac{PA} data streams (this can be the same person as the \emph{DT Developer} in a typical Dev-Ops fashion).
%
Deploying a DT can be a challenging task, involving several steps that depend on the complexity of the deployment infrastructure and the requirements of the DT.
%
This phase is crucial to ensuring that all deployment conditions are met, including  availability of deployable artifacts, and correct setup of communication configurations, enabling interaction with both physical and digital entities.
%
In a computing continuum, a DT may be deployed at different levels either on the edge, fog or cloud depending on the admissible trade-offs between network latency and computing power requirements. Notably, \acp{DT} might move across this continuum for a variety of reasons, either being connectivity requirements with the \ac{PA} (and thus move horizontally in the continuum), or drops in the quality of service that require either scaling to the cloud or moving closer to the edge (and thus moving vertically in the continuum). Finally, the same \ac{PA} could be connected to different \acp{DT} replicas, serving different use cases~\cite{dt-IoT-context-Minerva-2020}.
%
Ideally, the DT Operator can delegate a request to a \emph{\ac{DTC} Manager} who has access to data about \emph{other} \acp{DT} running on the same infrastructure and hence deploy the \ac{DT} in order to use resources effectively.
%
While a \emph{\ac{DT} Operator} is fundamentally interested in monitoring the activity and managing one or more \acp{DT}, the \ac{DTC} Manager is instead interested in monitoring the activity and managing the computing infrastructure on which \acp{DT} are being deployed.

\paragraph{Running Digital Twin}

The last phase starts once the \ac{DT} is up and running. In this phase, the DT must be identified, described, and reachable by \emph{consumers}.
%
This phase is critical for maintaining an accurate and up-to-date representation of the DT instance, managing its interaction protocols, handling lifecycle transitions, and enabling real-time state monitoring and interaction with the DT.
%
The ability to monitor information about running \acp{DT} is essential for ensuring synchronization with physical assets, supporting advanced digitalization processes, and facilitating interoperability within complex cyber-physical systems.
%
In this phase, generic \emph{\ac{DT} Consumers} may interact with the deployed \ac{DT} instance. 
%
Such consumers can either be human users or other systems and applications, leveraging the DT features for their purposes.
%
The \emph{\ac{DT} Consumers} are inherently also \ac{DTC} consumers, as they may interact with the \ac{DTC} to discover which \acp{DT} of which \acp{PA} are available, where to find them and how to interact with them.


\begin{figure}[tb]
    \centering
    \includegraphics[width=0.9\textwidth]{figures/dtc/dtc-requirements_v2.pdf}
    \caption{Requirements associated to each phase of the \ac{DT} development-to-deployment process.}
    \label{fig:dtc-requirements}
\end{figure}

\Cref{fig:dtc-requirements} summarizes the \emph{information} requirements for managing \acp{DT} across these phases, namely what kind of information is necessary to support stakeholders in each phase, and ensure that the phase can run successfully.
%
These requirements guide the design of \emph{descriptions} that can be used to inform the \ac{DTC} about the characteristics of the \acp{DT} to deploy and manage, in a platform-agnostic way.


%=======================================================
\section{Key Elements and Descriptions}
%=======================================================

To address the challenges in the operational management of \acp{DTE}, the \ac{DTC} relies on a set of key elements and their structured descriptions that provide a structured way to represent and manage the various components involved in a \ac{DTE}.

\begin{table}
    \centering
    \small
    \begin{tabular}{p{2cm} p{8.5cm} p{2cm}}
    \toprule
    \textbf{Entity} & \textbf{Description} & \textbf{Req.} \\
    \hline
    \textbf{Physical Asset Schema (PAS)} & Defines the abstract capabilities and interaction patterns of a specific type of industrial asset (e.g., robotic arm, conveyor system). Provides essential metadata for communication protocols, API definitions, and interface configurations without specifying instance-specific values like IP addresses or credentials. & R1, R2, R3 \\ \hline
    \textbf{Physical Asset Instance (PAI)} & Represents a deployed instance of an asset as defined by the PAS. Includes instance-specific configurations like network settings, authentication credentials, and protocol-specific settings. Facilitates the runtime communication between the DT and the physical asset. & R1, R2, R3 \\ \hline
    \textbf{Digital Twin Schema (DTS)} & Defines the capabilities, behaviors, and structural components of a Digital Twin, linking the digital representation to its corresponding physical asset. Specifies properties, events, relationships, actions, and fidelity between the physical and digital counterpart. & R1, R2, R3 \\ \hline
    \textbf{Digital Twin Package (DTP)} & Defines the platform-specific implementation of the DTS, including code, configuration parameters, and dependencies required for deployment. Ensures the fidelity and communication requirements of the DTS are met during deployment. & R6, R7, R8 \\ \hline
    \textbf{Digital Twin \-  Instance (DTI)} & Represents a running instance of the Digital Twin, including metadata for orchestration, monitoring, and lifecycle management. Tracks software lifecycle states and exposes relevant runtime metrics. Ensures the connection between the DTC and the deployed DT instance. & R9, R10, R11, R12 \\
    \hline
    \textbf{Runtime Platform (RP)} & Describes the underlying computing infrastructure and environment where the Digital Twin is deployed. This includes details about the hardware, software, and network configurations that support the DT's operation. & R4, R5 \\
    \bottomrule
    \end{tabular}
    \caption{Overview of DTC Entities: Characteristics, Responsibilities, and Associated Requirements}
    \label{table:dtc_descriptions_requirements}
\end{table}


At the core of this descriptive framework is the \textit{Physical Asset Schema} (PAS), which serves to uniquely define a specific \textit{type} of industrial asset---such as a robotic arm, conveyor system, or CNC machine.
%
The PAS encapsulates essential metadata that describes how assets of this category are expected to interact within a digital environment. This includes supported communication protocols (e.g., MQTT, OPC UA, HTTP), standardized API definitions, and more broadly, the configuration of interfaces and interaction models required for monitoring, control, and integration across systems. It is important to note that the PAS does not contain instance-specific values such as IP addresses, port numbers, or credentials. Instead, it provides a structured description of the expected capabilities and modes of interaction for devices of a given type.
%
This is similar to the concept of \emph{Thing Model} in the context of the \ac{WoT}~\cite{wot-td} which only focuses on capabilities and abstracts completely from protocols and implementation details. The PAS, instead, includes this information, as it is more closely related to the physical devices and not on their virtual \emph{Thing} representation. 

This schema is essential for the \ac{DTC} to understand how different \ac{PA} conform to specific interaction patterns and how to configure communication with them. 

Building upon the PAS, the \textit{Physical Asset Instance} (PAI) represents a deployed instance of an asset as defined by the PAS.
%
While the PAS defines the abstract capabilities and expected interaction patterns of a category of assets, the PAI provides the concrete, instance-specific configuration required for operational integration. This includes network information (such as IP addresses and ports), authentication credentials, and protocol-specific settings like MQTT topics or OPC UA node identifiers.
%
These configurations are essential to enable runtime communication between the DT and the physical asset, supporting dynamic discovery, secure interfacing. As such, the PAI forms the operational bridge that links the abstract asset description to real-world deployment contexts, ensuring that the DT can monitor and interact with its physical counterpart in a precise and reliable manner.
%
The \ac{WoT} \ac{TD} can be used for this scope as it includes the necessary information to describe how to connect to a specific device instance.

Following the same principles and responsibilities of the PAS, the \textit{Digital Twin Schema} (DTS) defines the capabilities and behaviors of a DT independently of any concrete implementation.
%
It establishes a reference to the corresponding \ac{PA} category, ensuring semantic linkage between the physical and digital representations. The schema outlines the structural components of the DT, such as its properties, observable events, defined relationships, and available actions. It also specifies the target \emph{fidelity} of the DT, capturing the mirrored functionalities of the physical counterpart within a given context and under specific application goals.

The \textit{Digital Twin Package} (DTP) represents instead the software implementation of a given schema on a target \ac{DT} platform.
%
The package includes all platform-specific implementation artifacts required for deployment and execution.
%
It incorporates the executable code, configuration parameters, and platform dependencies, ensuring compliance with the DTS's fidelity and communication requirements.
%
It also provides detailed bindings for the declared communication protocols---e.g., how MQTT is used to interface with physical assets and how HTTP APIs enable interaction with external digital services.

When a package is deployed within the DTC, it gives rise to a \textit{Digital Twin Instance} (DTI).
%
This instance represents the active, running version of the DT within the operational infrastructure and includes all metadata necessary for orchestration, monitoring, and software lifecycle management.
The DTI maintains references to its originating schema and the platform on which it is deployed, along with essential access information such as IP addresses, ports, and authentication tokens. It also exposes relevant runtime metrics—including CPU, memory, and network utilization—as well as any endpoints required to access its services and functionalities.
%
In addition, the DTI tracks and exposes the \ac{DT} connectivity information tracking whether the \ac{DT} is connected to the \ac{PA}. These lifecycle states pertain strictly to the operational status of the deployed software component and are independent of the internal logic or functional behavior of the DT itself (differently from the synchronization lifecycle discussed in \Cref{sec:dte:engineering-dt:dt-lifecycle}).
%
Functional state and domain-specific behavior remain under the responsibility of the DT and are communicated through its defined interfaces in the digital space (e.g., through the \ac{DTKG} if considering a \ac{HWoDT} ecosystem \Cref{chap:dte:hwodt})

With reference to the requirements identified in \Cref{fig:dtc-requirements}, \Cref{table:dtc_descriptions_requirements} summarizes the key elements and their associated responsibilities in supporting the operational management of \acp{DTE}.
%
The integration of the PAS, PAI, and DTS effectively addresses the core requirements \textbf{R1.Capabilities}, \textbf{R2.Asset Linkage}, and \textbf{R3.Communication}, which are focused on the representation of features of the \acp{PA} and \acp{DT}, the relationship that links a \ac{DT} to a specific \ac{PA}, and the necessary communication protocols.
%
These descriptions support the \emph{DT Software Capabilities Discovery \& Selection} phase, enabling stakeholders to explore available \acp{PA} and their corresponding DT types based on defined capabilities and interaction patterns.
%
Additionally, the DTP supports requirements \textbf{R6.Computing Node Requirements}, \textbf{R7.Artifacts}, and \textbf{R8.Connection Configuration}, which pertain to platform-specific aspects to support the creation of DT instances within the DTC and can hence support the \emph{Digital Twin Deployment} phase.

The DTI is responsible for addressing the requirements \textbf{R9.Identification}, \textbf{R10.Instance\- Description}, \textbf{R11.Interaction\- Information}, and \textbf{R12.Life\-cycle}, which in the \emph{Running Digital Twin} phase support stakeholders in the identification of the DT instance, the discovery of its status and interaction descriptions, as well as its connectivity with the \ac{PA} during operation.


\textbf{R4.Platform Description} and \textbf{R5.Configuration} are supported by the \textit{Runtime Platform} (RP) description.
%
Differently from the other descriptions, these are tied to the computing infrastructure and the target platforms on which the \ac{DT} can be deployed. 
%
\textbf{R4}, includes platform-dependent attributes such as the platform type, supported communication protocols, and data formats, enabling the DTC to match DTPs to the correct platform.
%
Additionally, \textbf{R5} plays a crucial role in establishing the necessary parameters and operational requirements specific to the platform and implementation. Proper configuration ensures that the DT functions optimally by aligning its settings with platform constraints, communication protocols, and performance expectations.  

Together, these components support stakeholders to seamlessly identify and discover managed \acp{PA}, understand their capabilities, and determine the availability of corresponding \ac{DT} packages for deployment. This structured approach ensures that the DTC facilitates efficient asset discovery, aligns with system interoperability needs, and supports dynamic integration across diverse environments. 
%
Figure \Cref{fig:dtc-descriptions} visually summarizes the key elements and their descriptions within the context of the DTC, illustrating how they interrelate to support the operational management of \acp{DTE}.

\begin{figure}[tb]
    \centering
    \includegraphics[width=0.9\textwidth]{figures/dtc/dt_entities_platform_interactions.pdf}
    \caption{Key elements and their descriptions depicted within a schema of the \ac{DTC}. \acp{PA} instances (bottom) are described by schemas and each have a \ac{DT} instance mirroring them, running on possibly different platforms depending on which DT package among the ones implementing the correct DT Schema (top) is used to deploy the \ac{DT}.}
    \label{fig:dtc-descriptions}
\end{figure}

%=======================================================
\section{Architecture of the \acl{DTC}}
%=======================================================

\begin{figure}[tb]
    \centering
    \includegraphics[width=\textwidth]{figures/dtc/architecture_new_v3.pdf}
    \caption{Functional Architecture of the \ac{DTC} with the main components and the interactions with stakeholders. On the left, DTC nodes abstracting the interaction with physical computing resources. On top of them, DT packages belonging to a given schema are deployed as DT instances. The DTC Manager (right) is composed by a middleware and an orchestrator. The first is responsible for managing the registry of DTs and PAs, while the second is responsible for orchestrating the deployment of DTs across the DTC nodes.}
    \label{fig:dtc-architecture}
\end{figure}

The aim of the DTC is to create a new open-system perspective of connected, interoperable and pervasive DTs modeled and engineered to create an effective cyber-physical abstraction layer.
%
This requires, first, facilitating the execution of DTs on a set of DT platforms.
Second, by exploiting the advantages of a broad compute continuum of cloud and edge resources, to deploy the DTs on the \textit{best} Compute Node (CN) available to guarantee application requirements.

%----------------------------------------
\subsection{Functional Overview}
%----------------------------------------

The architecture of the DTC is depicted in \Cref{fig:dtc-architecture} and includes DTC functional blocks (grayed boxes), infrastructure modules, DT artifacts, and DTC stakeholders. From a high-level perspective, the DTC allows DT operators to expose a collection of \textit{M} DTs for use by DT consumers. Each DT is described by a schema, which in turn is associated with a number of DT packages (up to \textit{N}), enabling deployment on different platforms. New DT schemas and packages can be dynamically onboarded by DT developers.

In the DTC, we distinguish between \emph{compute nodes (CNs)} and \emph{DTC nodes}. 
A compute node represents an individual unit of computation (e.g., a VM, a bare-metal server, or a container host) where a DT instance is executed. 
A DTC node instead abstracts a collection of compute nodes orchestrated by the same Virtual Infrastructure Manager (VIM), similarly to how a MEC node in ETSI MEC represents a set of compute hosts under a common VIM. Typical examples include Kubernetes clusters or OpenStack-managed virtualized clusters.

Within each DTC node, the DTC Agent is responsible for managing and monitoring the execution of DTs on the underlying compute nodes. 
The VIM handles low-level resource management, while the DTC Agent exposes the control and monitoring capabilities needed by the DTC Manager to orchestrate DT execution and placement across the continuum. 
Above these components, the DTC Manager coordinates all DTC Agents to provide a unified operational interface over the entire infrastructure.

DT packages are used to create and execute DT instances on top of DT platforms. 
Each instance runs on a specific compute node inside a DTC node. 
\acp{DT} are deployed across the compute continuum by first selecting the appropriate DTC node (i.e., the cluster orchestrated by a VIM) and then, within it, the compute node that will host the instance. 
This separation enables the DTC Orchestrator to move and replicate DTs not only across compute nodes but also across different DTC nodes, depending on fidelity requirements, resource availability, and performance considerations.

More in detail, a DT operator requesting the creation of a DT corresponding to a given schema contacts the DTC Manager, which is responsible for identifying suitable platforms and corresponding DT packages, based on fidelity and connectivity requirements, and for configuring communication and computing resources accordingly. 
The DTC Manager comprises two logical components: the DTC Middleware and the DTC Orchestrator. 
The Middleware acts as the entry point for all stakeholder requests and configures DTs directly when possible. 
The Orchestrator is responsible for instantiating DTs by interacting with DTC Agents to configure the necessary resources through the underlying VIMs. 
This separation enables the DTC to support heterogeneous underlying infrastructures while providing a unified management layer.

%----------------------------------------
\subsection{Interaction Example}
%----------------------------------------


\begin{figure}[tb]
    \centering
    \includegraphics[width=\textwidth]{figures/dtc/sequence_diagram_v2.pdf}
    \caption{The interaction flow example between DTC architectural components to enable DT registration and deployment.}
    \label{fig:dtc-interaction}
\end{figure}

\Cref{fig:dtc-interaction} presents an illustrative example depicting a typical interaction scenario among the key components of the DTC architecture.
%
Assuming that the PA has been already registered in the DTC providing the corresponding PAS and PAI descriptions, the figure illustrates the sequence of interactions that occur when a generic user aims to register and deploy a DT for a specific PA.

The initial step to deploy a DT on a target platform involves registering a new DT Schema.
%
All the interactions are facilitated by the DTC Manager, which implements the external-facing API and receives, handles and forwards incoming requests to the appropriate internal component.
%
The request to register a DT Schema is forwarded from the DTC Manager to the DT Middleware. Here, the incoming request is validated, and a new schema is created for the target DT associated with its description, capabilities and fidelity together with the type of physical asset. Subsequently, the DTC manager responds to the user with positive feedback.

The subsequent step entails registering or creating a DT Package linked to the specific schema. The interaction flow follows is similar to the one above, with the DTC Manager receiving the request and the DT Middleware registering the package and associating it with the corresponding SchemaID. 

Moving forward, the registration process for the DTC User involves identifying suitable compute nodes for deploying the target DT. This necessitates a node discovery request initiated by the user, which is forwarded to the DTC Manager.
%
Internally, the DT Orchestrator engages in the discovery process, returning a list of available nodes that match the user's specified criteria, such as the support for package deployment and the support for target hardware requirements (e.g., a GPU for the execution of specific DT's functionalities).
%
The selection of nodes could be automated based on predefined policies. For simplicity, in this example, the user chooses the preferred node from the list of available options.

Upon reviewing the list of available compute nodes matching the target package and schema, the user can proceed by sending a start request to the DTC Manager.
%
This request includes the specific package ID associated with the target schema, along with the starting configuration of the twin which include the identifier of the PAI to which the \ac{DT} will be linked together with the node ID indicating where the DT should be deployed. 
The DTC Manager first verifies compliance of the \ac{DT} package with the platform and the selected PAI, then it checks for the availability of the designated DTC Agent managing the specified node.

Subsequently, it forwards a start request to the target DTC Agent managing the selected platform and node.
%
Here, the DTC Agent checks for the presence of the target DT Package and downloads it if necessary. Once the package is available, the Agent initiates and starts the target DT on the designated platform.
%
Finally, the Agent receives the instance ID of the running DT from the platform and communicates it back to the DTC manager. Subsequently, the DTC manager replies with the instance ID to the user, along with the current description of the running DT instance.

This interaction flow exemplifies the collaborative efforts of the DTC components in facilitating the registration and deployment of DTs, ensuring a seamless experience for users while abstracting the underlying complexities of the infrastructure.
%
By leveraging the descriptions introduced in \Cref{table:dtc_descriptions_requirements}, the DTC effectively manages the lifecycle of DTs, from registration to deployment, while ensuring compatibility with the underlying platforms and compute nodes.

%=======================================================
\section{Proof of Concept Implementation}
%=======================================================

A preliminary analysis, implementation, and experimental evaluation of the core functionalities of the DTC have been conducted using a prototype based on Eclipse Ditto and WLDT as reference platforms.
%
The primary objective of this implementation is to demonstrate the feasibility of the DTC and obtain initial insights by leveraging the microservice capabilities of the White Label Digital Twin (WLDT) library.
%
The DTC's functional modules (\Cref{fig:dtc-architecture}) are implemented in Python, featuring a \emph{master API} that orchestrates overall behavior and a set of \emph{DTC agents} that manage specific computing nodes and interface with underlying DT platforms: namely, WLDT and Eclipse Ditto for this experimental setting. 
%
These components expose structured RESTful APIs, enabling standardized interaction with the DTC core services. Monitoring and performance tracking across distributed nodes are facilitated through Prometheus, serving as a centralized time-series database, while local agents collect metrics on individual machines.
%
Observability, visualization, and analysis are supported via Grafana dashboards. For the experimental evaluation, the supporting infrastructure -- including Virtual Machines (VMs), networking, and container runtimes (Docker and Kubernetes) -- was assumed to be preconfigured.

The Edge and Cloud infrastructures are associated with two real Proxmox clusters and deployments.
The Edge node is realized as a server within the same laboratory as the microfactory used as the source of \ac{PA} data, hosting the industrial Edge nodes.
%
The microfactory setup is the same used in \Cref{ssec:dte:dt-engineering:scenario}.
Conversely, the Cloud component is realized using a second Proxmox cluster external to the laboratory network and connected via the Internet, ensuring that delays and latencies are aligned with non-local communication scenarios.
On the respective Proxmox clusters, Linux Ubuntu Server 20.04 VMs were deployed, upon which the containerization environments were installed for the execution of all required applications, including Ditto, WLDT, and the monitoring and logging layers. 
Specifically, the Edge infrastructure utilized a VM provisioned with 2 CPUs, 2 GB of RAM, and 20 GB of storage, while the Cloud infrastructure was realized on a VM featuring 8 CPUs, 8 GB of RAM, and 500 GB of storage.

The \textbf{first experiment} examines the behavior of two \acp{DT} deployed across two platforms: WLDT at the edge and Ditto in the cloud. Rather than focusing on communication delays, this experiment aims to demonstrate how the DTC orchestrates state computation and ensures synchronization across multiple DT instances operating on heterogeneous platforms. Two observers, one on the edge and one in the cloud, monitored the propagation of \ac{PA} and DT states.
%
In the first configuration (\Cref{fig:exp_single_dt:same_model}), both DTs run the same model version. The results confirm that the DTC effectively deploys the two instances which are getting data from the same physical asset and compute the state, with the expected delay difference between edge and cloud platforms.
%
To each \ac{PA} state update (green), the edge DT (WLDT) computes the new state (blue). Similarly the cloud DT (Ditto) receives the updates and computes its state (orange). The DTC ensures that both DTs are correctly deployed and synchronized, demonstrating its capability to manage multiple DT instances across heterogeneous platforms.

In the second configuration (\Cref{fig:exp_single_dt:coarse_model}), the cloud-based DT is switched to a different, coarser, DT schema (V2) that employs a longer time window for state aggregation (receiving updates from the PA every seconds and computing the state every 5 samples), while the edge DT continues operating with the original high-granularity schema.
%
The DTC seamlessly manages this update, ensuring the cloud DT is correctly deployed and integrated without disrupting the ongoing edge operations. This illustrates the DTC's capability to orchestrate heterogeneous DT versions.

\begin{figure*}[ht]
    \centering
    \begin{subfigure}[t]{0.49\textwidth}\label{fig:exp_single_dt:same_model}
        \includegraphics[width=\textwidth]{figures/dtc/wldt_ditto_end_to_end_delay_same_schema_state_comp.pdf}
        \caption{}
    \end{subfigure}\hfill
    \begin{subfigure}[t]{0.49\textwidth}\label{fig:exp_single_dt:coarse_model}
        \includegraphics[width=\textwidth]{figures/dtc/wldt_ditto_end_to_end_delay_diff_schema_state_comp.pdf}
        \caption{}
    \end{subfigure}
    \caption{Comparison of end-to-end DT state computation delays across edge (WLDT) and cloud (Ditto) platforms.(a) shows the same DT model being deployed on both edge and cloud, while (b) shows a coarser DT model being deployed on the cloud. Below each figure, the number of DT states computed in the last 10 seconds, to highlight the difference in the update frequency of the two DTs.}
    \label{fig:exp_single_dt}
\end{figure*}

The first experiment was engineered to address a realistic operational and architectural challenge facing stakeholders in industrial DT deployments:
the management of asymmetric data granularity and synchronization requirements across heterogeneous Edge and Cloud environments.
This asymmetry reflects a core business trade-off: high data frequency is strategic at the Edge to support time-critical applications, such as real-time predictive maintenance and rapid feedback loops, demanding maximum responsiveness.
Conversely, a lower synchronization rate is often sufficient for the Cloud layer, typically supporting historical analysis, aggregate reporting, or non-critical dashboards.
This selective reduction in frequency -- a direct response to the stakeholder need to save bandwidth, storage costs, and processing load in the Cloud -- is a key driver.
The DTC is validated in this context by demonstrating its capability to deploy different DT schemas across distinct platforms.

The \textbf{second experiment} evaluates the scalability of the DTC prototype in a realistic DT ecosystem deployment involving multiple DTs that work together to digitalize the microfactory.
%
This scenario directly addresses a key stakeholder challenge:
the need for a phased and controlled deployment where visibility into the factory floor is progressively increased.

The deployment evolved incrementally over the experimental timeline,
as shown in Figure \ref{fig:dt_resource_usage_dt_count}, 
which depicts the number of active DTs on the edge and cloud platforms.
%
Specifically, eight DTs were first launched on the edge, followed by the activation of four additional DTs in the cloud, achieving full deployment of all twelve DT types present in the Microfactory.
%
Later, edge DTs are also replicated in the cloud and vice-versa, to simulate a scenario, where replicas are needed to support different use cases and applications. Bringing the total to 24 DTs running across the two platforms.

In this configuration, the edge DTs exhibited extended functional behavior compared to the first experiment.

They implemented more advanced DT capabilities, specifically augmenting the PA digitalization by performing complex tasks such as industrial machine state computation and anomaly detection for blocked products within the production line.
Critically, the cloud DTs received their state input indirectly from their corresponding edge DTs, rather than retrieving data directly from the physical assets in order to give to the edge DTs the responsibilities of the first digitalization points and controlling data flows and granularity.
This setup highlights the DTC's ability to coordinate multiple DT instances across heterogeneous platforms, maintaining synchronization, ensuring consistent communication, and optimizing resource utilization as the system scales.

\begin{figure}[htb]
    \centering
    \includegraphics[width=\textwidth]{figures/dtc/edge_cloud_experiment_dt_count.pdf}
    \caption{Number of active DT instances on the Edge and Cloud platforms during the incremental deployment process. The graph illustrates the gradual increase in the number of DTs over time, reflecting the phased deployment strategy adopted in the experiment.}
    \label{fig:dt_resource_usage_dt_count}
\end{figure}

Performance metrics—including CPU usage, memory consumption, and network traffic—were continuously monitored during this scaling phase. The primary objective is to demonstrate the flexibility and robustness of the proposed DTC system in seamlessly coordinating and orchestrating these evolutionary and incremental deployments in a realistic, heterogeneous environment.
%
It is important to emphasize that Ditto manages multiple DTs within a single centralized platform composed of several core microservices (e.g., connectivity, storage, proxy, query, gateway), whereas WLDT deploys each DT as an independent microservice at the Edge.
Consequently, in Ditto, the deployment of multiple DTs affects the overall platform performance, while in WLDT, each DT instance can be monitored independently.
This fundamental difference in platform architecture introduces additional complexity, which is transparently handled by the DTC.
By abstracting these platform-specific details, the DTC allows application designers to focus on defining schemas, packages, and the desired behaviors to be implemented, without needing to manage the underlying deployment differences.
%
During the experiment, detailed measurements were collected for CPU and memory usage, along with inbound and outbound network traffic, for each active microservice---including the core components of Eclipse Ditto and the WLDT-based DT instances. 
These metrics provide insights into how system resources are utilized as the number of active DTs increases across Edge and Cloud deployments. 
The results, illustrated in Figure \ref{fig:dt_resource_usage}, reveal the distinct operational phases of the experiment and demonstrate the DTC's ability to coordinate and manage multiple heterogeneous platforms concurrently.

%%%
\begin{figure}[tb]
    \centering
    \includegraphics[width=\textwidth]{figures/dtc/edge_cloud_experiment.pdf}
    \caption{Evaluation of the DTC prototype enabling multi-platform DT deployment in the Microfactory. The figure compares resource usage (CPU, Memory, Network Sent and Received) for Ditto in the Cloud (top row) hosting multiple DTs within a single Ditto instance composed of several microservices (Conntectivity, UI, Gateway, Nginx, Policies, Things, Things-Search) and WLDT at the Edge (bottom row) deploying individual DTs as microservices (DT1,..., DT12) as the number of active DTs increases.}
    \label{fig:dt_resource_usage}
\end{figure}
%%%

The obtained results underscore the feasibility of dynamically, modularly, and flexibly evolving an Edge-Cloud Digital Twin deployment within a realistic industrial context, thanks to the DTC's structured and integrated approach.
This successful evolution demonstrates the system's capacity to accommodate new DTs over time, 
effectively matching emerging stakeholder requirements and catering to different observers and users of the DTs across local and remote Cloud environments.
%
Crucially, the proposed DTC solution effectively masks the complexities of the various underlying platforms (WLDT and Ditto). 
Without such an integrated orchestration layer, the deployment, synchronization, and lifecycle management of these heterogeneous, cross-platform DT instances would entail a significantly greater level of complexity and manual effort for developers and system operators. 
This abstraction layer allows users to seamlessly select the DTs of interest based on the architectural layer where they need to be deployed, the specific features and functionalities required, and the data sources to be utilized.

Overall, the experiment confirms both the feasibility and robustness of the DTC architecture.
The results show that the DTC can seamlessly encapsulate the heterogeneity of underlying Digital Twin platforms,
offering a unified and coherent system view across deployment layers.
Furthermore, it maintains scalability, observability, and consistent system performance as operational load and deployment complexity increase—demonstrating its suitability for large-scale, distributed industrial scenarios


%=======================================================
\section{Final Remarks}
%=======================================================

This chapter presents the concept of \acl{DTC} as a middleware platform to support the operational management of \ac{DTE} on a compute continuum. 
%
The \ac{DTC} addresses key challenges that emerge when dealing with the deployment and management of systems that involve multiple \acp{DT}, characterized by heterogeneous physical assets, diverse DT platforms, and varying application requirements.

The proposal in this chapter contributes to answering the research question:

\paragraph{\ref{rq:3} How to support the operational management of DTEs?}

By using structured descriptions that capture the essential characteristics of physical assets, digital twins, and runtime platforms, the DTC enables stakeholders to effectively discover, deploy, and manage DT instances across a distributed infrastructure.
%
The descriptions are designed to be platform and application-agnostic, allowing for seamless integration and interoperability among different DT platforms and use cases.
%
The DTC architecture incorporates key functional components, including a middleware layer and orchestrator, which facilitate the interaction with distributed computing nodes through the abstraction of DTC agents. 
%
This approach supports different stakeholders in their respective roles, offering a coherent holistic view of the DT ecosystem alongside the development-to-deployment lifecycle of its components. 




%%%%%%%%%%%%%%%%%%%%%%%%%%%%%%%%%%%%%%%%%%%%%%%%%%%%%%%%
\chapter{Interoperable \aclp{DTE}}
\label{chap:dte:hwodt}
%%%%%%%%%%%%%%%%%%%%%%%%%%%%%%%%%%%%%%%%%%%%%%%%%%%%%%%%

Answering \ref{rq:2}, this chapter presents a proposal for the engineering of \emph{Heterogeneous} \acp{DTE} (\Cref{chap:dte:dte}).
%
Motivated by the design of the Web and its success in creating interoperability among digital services---more recently, in \ac{IoT} systems through the \ac{WoT} (\Cref{chap:back:Web})---the chapter explores whether and to what extent hypermedia principles and Web standards can be applied to design and implement \acp{DTE}. 
%
The result of this investigation, refines and extend the ideas originally presented in \cite{web-of-dt-ricci-2022}, combining the \ac{WoDT} conceptual model with the design rationale of the Web to build a \emph{\ac{HWoDT}}.

This approach offers a practical implementation strategy for \acp{DTE}, enabling interoperability at both the \ac{DT} and ecosystem levels, and facilitating uniform interaction across \acp{DT} regardless of the underlying technological stack.

The chapter presents the \ac{HWoDT} conceptual integration of hypermedia principles and Web standards in \ac{DTE}, 
and the prototype implementation of a supporting set of tools that demonstrate the feasibility of the approach.


%======================================================
\section{A \acl{HWoDT}}
\label{sec:hwodt-idea}
%======================================================

The main driver of the proposed approach is to support seamless integration of existing \acp{DT}, regardless of their underlying technologies and with minimal overhead.
%
Hence, the core idea is to hide the heterogeneity of \acp{DT} behind a \emph{uniform interface} built with Web protocols and standards, and supported by an explicit semantic layer that allows \acp{DT} to provide a uniform description of both their state and their features and services. 
%
Thus, the name \emph{\acf{HWoDT}}, aims to emphasize the hypermedia-driven nature of the approach, following the \ac{HATEOAS} principle of the \ac{REST} architectural style~\cite{fielding2000architectural}.

Such an interoperability layer enables navigation and seamless interaction with heterogeneous \acp{DT}.
%
To complement it with additional ecosystem-level functionalities, the idea of a \emph{WoDT Platform} is introduced as both a scope boundary defining ecosystem membership and an aggregation layer enabling consumers to query, observe, and exploit services across distributed \acp{DT}.

As illustrated in \Cref{fig:hwodt}, the resulting architecture is composed of \acp{DT}:
\begin{inlinelist}
    \item created using heterogeneous technologies,
    \item implementing the uniform interface through \emph{adapters},
    \item connected by relationships that reflect the physical ones,
    \item and that are aggregated into \acp{DTE} by being registered to one or multiple WoDT Platforms.
\end{inlinelist}    

\begin{figure}[htb]
  \centering
  \includegraphics[width=0.8\columnwidth]{figures/hwodt/hwodt.pdf}
  \caption{The Hypermedia WoDT scheme, in the real world PAs (squares) belonging to different organizations and domains are connected by relationships. In the digital world, cross-domain ecosystems of DTs (circles) mirror different portions of reality, supported by the WoDT Platform. Consumers (top) can interact with the ecosystem and its DTs directly, or through the platform.}
  \label{fig:hwodt}
\end{figure}

%-----------------------------------------------------------
\subsection{A Uniform Interface for Heterogeneous DTs}
\label{ssec:uniform-interface}
%-----------------------------------------------------------

The first step towards integrating heterogeneous \acp{DT} is to look into how each individual \ac{DT} can expose a uniform interface, compliant with the requirements of the \ac{HWoDT}.
Achieving standard semantic representations of \acp{DT} is an open challenge for the \ac{DT} community.
%
In this work, adopting a pragmatic approach, the benefits of uniform semantic representations in building interoperability are demonstrated through a proposal grounded on the \ac{HATEOAS} principle of the Web architecture:
each \ac{DT} must be able to provide consumers with both \emph{data} and \emph{affordances}---i.e., action possibilities---through hypermedia representations.
This leads to distinguish two logically different representations:
\begin{itemize}
    \item a \emph{\acf{DTD}}, inspired by the \ac{WoT} \acl{TD}, to hold metadata and affordances, enabling consumers to understand the \ac{DT} model, as well as which services the \ac{DT} exposes and how to access them;
    \item a \emph{\acf{DTKG}} representing the live state of the \ac{DT}, semantically encoding domain knowledge and supporting state observation, querying, and discovering relationships between \acp{DT}.
\end{itemize}

Each \ac{DT} is further required to comply with a set of standard \emph{interaction patterns} to allow consumers to uniformly obtain and manipulate such representations for all \acp{DT} in the \ac{HWoDT}.

%.................................
\subsubsection{The \acl{DTD}}
%.................................

The \ac{DTD} serves the primary purpose of providing management metadata about a \ac{DT}, registering it within the ecosystem and describing the exposed interactions\footnote{Documentation on the structure of the \ac{DTD} schema is available on GitHub \url{https://github.com/Web-of-Digital-Twins/dtd-conceptual-model}}.

A primary concern is \emph{identification} of the \ac{DT} and the associated \ac{PA} to unambiguously distinguish them from other elements of the ecosystem.
While \acp{PA} may use domain-specific identifiers (e.g., a serial number), each \ac{DT} in a \ac{HWoDT} is identified with a persistent, globally unique \ac{URI}.
This ensures identification and accessibility of \acp{DT} as Web resources throughout their whole lifecycle.

Furthermore, to support navigation from a \ac{DT} to the ecosystem it is part of, the \ac{DTD} also links to the \ac{URI} of the ecosystem---in the prototype implementation this is a URI served by the \ac{WoDT} platform to identify the \ac{DTE}.

Other relevant metadata in the \ac{DTD} concerns the model used to represent the \ac{PA} at the digital level.
%
The \ac{WoDT} metamodel (\Cref{sec:back:dt:dte})---closely aligned with the  \ac{WoT} \ac{TD}---can describe the \ac{DT} in terms of which properties, relationships, events, and actions are available to \ac{DT} consumers. 

Finally, the \ac{DTD} presents a description of the \ac{API} that consumers can use to interact with the \ac{DT}, using hypermedia controls to describe protocol bindings for the exposed interaction patterns.
%
The \ac{DTD}, is thus aligned with \ac{REST} principles as it presents both data and controls to consumers using self-descripting messages.

Even if the \ac{DTD} may evolve with new properties or interactions when a \ac{DT} gets updated, it remains mostly static as it describes the \ac{DT}'s identity metadata and software interface, not its real-time state.

%.................................
\subsubsection{The \acl{DTKG}}
%.................................

As a \ac{DT} main responsibility is to provide an up-to-date representation of the \ac{PA} state, the \ac{DTKG} complements the \ac{DTD} by representing the live state of the \ac{DT} with a \ac{KG}.

The \ac{DTKG} by means of \ac{RDF} triples, represents:
\begin{itemize}
    \item current property values;
    \item current relationships with other \acp{DT};
    \item context-dependent available actions.
\end{itemize}
Events, generated by the \ac{DT} are excluded from this representation as they are non-persistent information handled via subscriptions.

By using \ac{RDF}, the \ac{DTKG} can represent knowledge about the \ac{DT}'s state with explicit semantics using domain-specific ontologies to support a common interpretation of \ac{DT} data.
Additionally, following the Linked Data principles~\cite{Bizer_Heath_Berners-Lee_2023}, \acp{DT} linking to other \acp{DT} through relationships supports navigating across the ecosystem in a distributed \ac{KG}.

%.............................................
\subsubsection{\acl{DT} Interaction Patterns}
%.............................................

Other than generating the necessary \ac{DTD} and \ac{DTKG} each \ac{DT} in a \ac{HWoDT} ecosystem is required to implement a Web \ac{API} for consumers. This enables direct use by any Web client.

First, the \ac{API} should expose methods to retrieve the two \ac{DT} representations presented above.
%
As the main function of a \ac{DT} is to represent the \ac{PA} state over time, when dereferencing the \ac{URI} which identifies the \ac{DT} with an HTTP GET request the \ac{DT} must return the current \emph{snapshot} of the \ac{DTKG}.

\begin{figure}[tb]
    \centering
    \begin{subfigure}[t]{0.45\columnwidth}
        \centering
        \includegraphics[width=\textwidth]{figures/hwodt/dtddtkg.pdf}
        \caption{}
        \label{fig:sequence-dtddtkg}
    \end{subfigure}
    \hfill
    \begin{subfigure}[t]{0.46\columnwidth}
        \centering
        \includegraphics[width=\textwidth]{figures/hwodt/dtdactioncrop.pdf}
        \caption{}
        \label{fig:sequence-action}
    \end{subfigure}
    \caption{Interaction patterns to (a) access the \ac{DTD} and \ac{DTKG} and (b) invoke available actions on a \ac{DT} through the uniform interface and standard Web patterns.}
    \label{fig:sequence-interactions}
\end{figure}



Technically, since a \ac{DT} qualifies as a \emph{non-information resource}\footnote{For a discussion on information vs. non-information resources see \url{https://www.w3.org/TR/cooluris/} and \url{https://lists.w3.org/Archives/Public/www-tag/2005Jun/0039.html}}
the standard Web practice is returning a 303 (See Other) status code with the \textit{Location} HTTP header set to the \ac{URL} of the \ac{DTKG} information resource.
%
The \ac{URL} of the \ac{DTD} is linked in an HTTP Link Header with a custom relation type \texttt{dtd} in the response of the \ac{DTKG} GET request.
%
In this way, consumers can follow links between resources and access both \ac{DT} representations with standard Web interactions as shown in \Cref{fig:sequence-dtddtkg}.

Since the \ac{DTKG} evolves over time to reflect the \ac{PA} state, \acp{DT} are required to support a subscription \ac{API} to observe such changes (e.g., via WebSockets, long-polling, or WebSub~\cite{websub}).
Modeling the \ac{DTKG} as a Web resource also enables basic historicization using protocols like Memento~\cite{rfc7089}, if the \ac{DT} acts as a Memento TimeGate.

Finally, the \ac{API} may expose \ac{DT} actions that can be invoked by external consumers.
The \ac{DTKG} shows which actions are currently valid, while the \ac{DTD} describes how to request their execution to the \ac{DT}.
Consumers rely on both to interact and invalid actions should fail (e.g., with HTTP 403), ensuring consistency with the \ac{DT} model.
%
Figure \ref{fig:sequence-action} illustrates the interaction pattern to invoke an action on a \ac{DT}, by first checking its availability in the \ac{DTKG} and then following the affordance described in the \ac{DTD} to execute it.

%---------------------------------------------
\subsection{Ecosystem Services in the HWoDT}
\label{ssec:ecosystem-services}
%---------------------------------------------

In the proposal for realizing the \ac{HWoDT} ecosystem functionalities described in \Cref{chap:dte:dte} are implemented through a middleware:
%
the \emph{\ac{WoDT} Platform} aggregates data from registered \acp{DT} and exposes an \ac{API} that consumers can use to manage and interact with the ecosystem as a whole.

%........................................
\subsubsection{Managing the \acl{HWoDT}}
%........................................

The \ac{HWoDT} proposal assumes that \acp{DT} are existing, independently running software entities which implement the uniform interface presented in \Cref{ssec:uniform-interface}.
%
Hence, the \ac{DT} creation is handled by developers with their technology of choice, and adding it to an ecosystem is simply a matter of registering it to a \ac{WoDT} Platform. 
%
The platform exposes an \ac{API} that either managers, an external service or a \ac{DT} itself, can use to submit a \ac{DTD} to register a new \ac{DT} to the ecosystem.
The \ac{DTD} is then stored in a registry, which further allows it to be updated or removed if the \ac{DT} leaves the ecosystem.
%
Consumers can also use the registry \ac{API} to discover which \acp{DT} are in the ecosystem and to filter them based on metadata in the \ac{DTD}---e.g., they can discover if multiple \acp{DT} in the ecosystem model the same \ac{PA} filtering on the \ac{PA} identifier.

These simple operations are sufficient to manage the ecosystem. 
%
The \ac{DTD} is used in the registration process as it provides the necessary metadata to assess membership and compliance requirements as well as the \ac{API} description that the platform can use to interact with the \ac{DT} upon registration.
% 
Once registered, the platform uses the \ac{API} description within the \ac{DTD} to access and observe the \ac{DTKG}.  
It can also notify successful registration, enabling the \ac{DT} to record the platform \ac{URI} in its list of ecosystems updating its \ac{DTD} accordingly.

%..........................................
\subsubsection{Exploiting the \acl{HWoDT}}
%..........................................

By observing each individual \ac{DTKG}, the platform can cache the latest update and aggregate them to create a global \ac{DTE} \ac{KG} that can be exploited to implement ecosystem-level services for consumers.
%
The \ac{DTD} and \ac{DTKG} of each \ac{DT} are stored within the \ac{KG}, allowing queries to easily retrieve all available information about a \ac{DT}.
%
Each update of a \ac{DTKG} is processed and merged in the global \ac{KG} so that current representation of the whole ecosystem is always available to consumers.



The \ac{DTE} is a non-information resource and has a \ac{URI} which, when dereferenced, returns the global \ac{DTE} \ac{KG}\footnote{As the ecosystem \ac{KG} can be very large, returning the whole thing can be impractical. Alternatives include returning a partial representation with links to the \acp{DT} caches}.
%
Consumers can also \emph{query} the \ac{DTE} \ac{KG} through the \ac{WoDT} platform SPARQL endpoint.
%
This enables the extraction of selected information from the current state of all \acp{DT} in the ecosystem with a standard query language.

Finally, consumers may want to observe the evolution of the ecosystem \ac{KG}. The platform must then support an \ac{API} to receive subscription requests and send updates to observers.
%
Given the large nature of the \ac{DTE} \ac{KG}, consumers might not be interested in receiving \emph{all} updates. Hence, techniques for \ac{RDF} stream processing~\cite{barbieri2009www} might be more effective to support selective observation.

%======================================================
\section{A Prototype Framework for the HWoDT}
\label{sec:hwodt-impl}
%======================================================

To support the prototyping of heterogeneous \acp{DTE} based on the \ac{HWoDT} proposal, a set of tools is implemented aiming at the integration of existing \ac{DT} technologies.
%
The HWoDT framework is open source and available on GitHub\footnote{\url{https://github.com/Web-of-Digital-Twins}}. 

The software distribution includes a prototype \emph{\ac{WoDT} Platform} implementation and \emph{adapters} to implement the \ac{HWoDT} uniform interface for \acp{DT} developed with \azureTwin{}, \ditto{}, and the \ac{WLDT} framework (\Cref{sec:dte:engineering-dt:wldt-framework}).
%
The technology choice is meant to be representative of the state of the art as they differ in terms of functionalities, namely \azureTwin{} is a cloud platform, Ditto is an open-source platform, and \ac{WLDT} allows developing and deploying \acp{DT} as standalone software processes.
This section presents the overall architecture (\Cref{fig:abstract_arch}) and its implementation.


\begin{figure}[tb]
  \centering
  \includegraphics[width=0.8\columnwidth]{figures/hwodt/abstract_arch.pdf}
  \caption{Overall abstract architecture of the \ac{HWoDT} framework, showing the main functional modules and their main interactions. 
  On the top, the \ac{WoDT} platform manages the ecosystem of \acp{DT} by registering them through their \acp{DTD}, observing their \acp{DTKG}, and exposing the \ac{DTE} \ac{KG} to consumers.
  On the bottom, existing \acp{DT} can implement the \ac{HWoDT} uniform interface through the \ac{HWoDT} adapter and be integrated with the platform.}
  \label{fig:abstract_arch}
\end{figure}



%----------------------------------------------
\subsection{A WoT-compatible \acl{DTD}}
%----------------------------------------------

The \ac{DTD} is a central component in the design of the uniform interface of the \ac{HWoDT}.
%
Its design is inspired to the \ac{WoT} \ac{TD}, as an \ac{API} description of the functionalities offered by a \ac{DT}.

Although the \ac{WoT} explicitly reference \acp{DT} in its architectural patterns, the \ac{TD} alone does not support all the metadata that are needed in the proposed \ac{DTD}.
Hence, using \ac{WoT}-compliant mechanisms, \acp{TD} are extended to include the relevant metadata in order to qualify as a functional \ac{DTD} for a prototype implementation of the \ac{HWoDT}.
%
Namely, the \ac{TD} is extended using a custom \emph{\ac{WoDT} vocabulary}\footnote{\url{https://github.com/Web-of-Digital-Twins/wodt-vocabulary}}
which defines concepts to implement the \ac{DTD} and the \ac{DTKG}, as well as HTTP Link Headers relation types used in the interactions of the \ac{HWoDT} (see \Cref{ssec:uniform-interface}).
%
\Cref{lst:dtd-thing-model} shows a generic \ac{TM}~\cite{wot-td} that can be used to implement a valid \ac{DTD}. The \ac{DTD} must include:
\begin{itemize}
    \item the \ac{PA} identifier, 
    \item a link to the \ac{DTKG},
    \item an \texttt{observeallproperties} affordance to subscribe to updates of the \ac{DTKG}.
\end{itemize}

\lstinputlisting[
    label={lst:dtd-thing-model},
    caption={The Thing Model that Thing Descriptions must
    implement to be recognized as a valid \acl{DTD}.},
]{listings/hwodt/dtd.jsonld}


% \begin{figure}
%   \centering
%   \includegraphics[width=\columnwidth]{figures/hwodt/wot-dt-mashups.pdf}
%   \caption{Keeping Digital Twin representations aligned with the \ac{WoT} Thing Description enables the creation of mixed mashups and interoperability with existing \ac{WoT} consumers.}
%   \label{fig:wot-dt-mashups}
% \end{figure}

The alignment with the \ac{WoT} is strategic for the \ac{DT} community as reusing the existing standard ensures compatibility and favors the development of application mashups.
%
The \ac{DTD} can be served by the \ac{DI} of a \ac{DT} to be discovered and consumed by existing \ac{WoT} clients (see \Cref{sec:dte:engineering-dt:physical-digital-adapters}).
%
Clients that have the capability to interpret the \ac{WoDT} vocabulary can then exploit the additional metadata to interact with the \ac{DT} as part of a \ac{HWoDT} ecosystem.% as shown in \Cref{fig:wot-dt-mashups}.

%-----------------------------
\subsection{HWoDT Adapters}
\label{ssec:adapters}
%------------------------------

\acp{DT} in a \ac{HWoDT} ecosystem must all adhere to the uniform interface, regardless of the technology used to implement them.
%
Adapting existing \ac{DT} technologies to the \ac{HWoDT} interface requires aligning data and interaction patterns.
%
To show that this alignment is achievable with configurable reusable components a set of \emph{adapters} is developed to integrate some mainstream technologies within a \ac{HWoDT} ecosystem.
%
All adapters support HTTP interactions and WebSocket-based \ac{DTKG} observation.
%
The abstract architecture followed by all adapters is shown in \Cref{fig:abstract_arch}, bottom.

%..............................................
\subsubsection{\azureTwin{} Adapter}
%..............................................

\azureTwin{} is Microsoft's domain-independent \ac{PaaS} \ac{DT} solution, supporting the management of multiple \acp{DT} connected within a \emph{twin graph}.

Creating \acp{DT} in \azureTwin{} requires defining their model using the \emph{\acf{DTDL}}, a custom JSON-LD format to specify properties, relationships, commands (actions) and telemetry (events) that describe a \emph{type} of \ac{DT}.
%
\ac{DTDL} models can then be used to create \ac{DT} instances.
%
Although the model closely aligns with the \ac{WoDT} model, the support of events and actions is partial\footnote{As of June 2025, commands can be defined but not invoked automatically, while telemetries are not used within \azureTwin{} \url{https://learn.microsoft.com/en-us/azure/digital-twins/concepts-models}} and relationships are limited to linking \acp{DT} within the same \azureTwin{} instance, which hence defines a closed homogeneous ecosystem.

The \azureTwin{} adapter is implemented as a middleware, connecting to an \azureTwin{} instance and mapping \acp{DT} to the \ac{HWoDT} uniform interface.
%
The adapter can be configured to select which \acp{DT} needs to be mirrored and specify the mapping from \ac{DTD} properties and relationships to produce the \ac{DTKG}. 
%
\Cref{fig:azure-adapter-c&c} shows the adapter architecture and the necessary components to connect to an \azureTwin{} instance.
%
The following Azure service pipeline must be set up to link the \azureTwin{} instance to the adapter:
\begin{itemize}
    \item \textit{Azure Event Grid} to capture and forward \azureTwin{} events;
    \item \textit{Azure Function} (\acl{FaaS}) to fetch the current \ac{DT} state and send it to the adapter via SignalR on every new event;
    \item \textit{Azure SignalR} to deliver \acp{DT} state updates to the adapter.
\end{itemize}

The adapter can be deployed either within the Azure cloud or is designed to automatically retrieve the \ac{DTDL} models, stored on \azureTwin{}, and convert them in valid \acp{DTD} for each \ac{DT} instance.
The adapter waits for SignalR events to receive \ac{DT} state updates generated by the Azure Function and convert them into \acp{DTKG} updates.

\begin{figure}[tb]
    \centering
    \includegraphics[width=\textwidth]{figures/hwodt/adtadapter-c&c.pdf}
    \caption{Main components of the \azureTwin{} adapter, implemented as a standalone middleware that maps specific instances of \acp{DT} hosted on an \azureTwin{} instance.}
    \label{fig:azure-adapter-c&c}
\end{figure}

%..............................................
\subsubsection{Eclipse Ditto Adapter}
%..............................................

\emph{Eclipse Ditto}, is an open-source \ac{DT} platform from the Eclipse Foundation. 
It abstracts each \ac{PA} as a \emph{Ditto Thing} and offers a Web-based layer for interacting with \ac{IoT} devices through their \acp{DT}.
%
A Ditto instance can manage multiple \acp{DT}, using a metamodel with:
\begin{itemize}
    \item \emph{Attributes} for static metadata,
    \item \emph{Features} which can group data (properties) and functionalities (messages).
\end{itemize}
Although lacking native support for relationships, actions, and events, these can be modeled using attributes (for links), consumer-to-\ac{DT} messages (for actions), and \ac{DT}-to-consumer messages (for events). Models can be defined via a custom JSON format or a (set of) \ac{WoT} \ac{TM}.

Eclipse Ditto is implemented with a microservice architecture. 
Despite Ditto being open-source and allowing the development of custom extensions, the prototype adapter is implemented as a custom external middleware that can be deployed to map one Ditto Thing to the \ac{HWoDT} uniform interface.
%
The middleware leverages Eclipse Ditto's native WebSocket interface to retrieve data and performs two key transformations: converting the \ac{WoT} \ac{TD} exposed by Ditto into a \ac{DTD}, and serializing \ac{DT} data into a \ac{DTKG}.
This process relies on a configurable mapping between Ditto features and their corresponding \ac{RDF} representations.

\begin{figure}[tb]
  \centering
  \includegraphics[width=0.8\columnwidth]{figures/hwodt/ditto-adapter-c&c.pdf}
  \caption{Main components of the Eclipse Ditto adapter, implemented as a standalone middleware that maps a specific instance of a \ac{DT} hosted on the Ditto platform.}
  \label{fig:ditto-adapter-c&c}
\end{figure}

%..............................................
\subsubsection{\acl{WLDT} Adapter}
%..............................................

The \emph{\acf{WLDT} framework} (\Cref{sec:dte:engineering-dt:wldt-framework}) enables the development of \acp{DT} as standalone software components deployable across cloud or edge environments.
%
Specifically, the adapter is implemented as a \ac{DiA}, used in the \ac{WLDT} architecture to represent digital interfaces through which the \ac{DT} can expose data for consumers.
%
The framework internal metamodel is already aligned with the \ac{HWoDT}, so no additional effort is required to map concepts.
The adapter is provided as a reusable Java library, which \ac{WLDT} developers can easily import into their projects and then configure the mapping towards the \ac{HWoDT} uniform interface leveraging the implementation of the necessary \ac{API} to manage the \ac{DTD} and \ac{DTKG}.

\begin{figure}[tb]
  \centering
  \includegraphics[width=\textwidth]{figures/hwodt/wldt-adapter-c&c.pdf}
  \caption{Main components of the \ac{WLDT} adapter, implemented as a \acl{DiA} and illustrating its integration with the other components of the \ac{WLDT} framework.}
  \label{fig:wldt-adapter-c&c}
\end{figure}


%..............................................
\subsubsection{Implementing Custom Adapters}
%..............................................
The description of the adapters implemented in the prototype \ac{HWoDT} framework should be useful for any \ac{DT} developer wishing to implement a custom adapter for their own \ac{DT} or \ac{DT} technology. 
%
The idea of the \ac{HWoDT} is grounded on the fact that implementing such custom adapters is relatively straightforward and that the little effort spent in developing (or even less in configuring) them can bring benefits in the integration with the \ac{HWoDT} ecosystem.

To summarize, the steps to develop a custom adapter are: 
\begin{enumerate}
    \item understand how to map the concept of the original \ac{DT} model into the \ac{WoDT} metamodel;
    \item understand how to represent the state of the \ac{DT} with an evolving semantic representation for the \ac{DTKG};
    \item implement a module that can reactively produce the \ac{DTKG} whenever the \ac{DT} state is updated;
    \item implement a module that can serve the \ac{DTD} over HTTP, alongside the mandatory \ac{API} for \ac{DTKG} observation.
\end{enumerate}

%-----------------------------------
\subsection{The WoDT Platform}
%-----------------------------------

\begin{figure}[ht]
  \centering
  \includegraphics[width=0.85\columnwidth]{figures/hwodt/platform-c&c.pdf}
  \caption{\ac{WoDT} Platform modules and interfaces, represented using components and connectors.}
  \label{fig:platform-c&c}
\end{figure}

A prototype of a \ac{WoDT} Platform is implemented in Kotlin.
The platform implements the modules of the abstract architecture shown in \Cref{fig:abstract_arch}, top as shown in \Cref{fig:platform-c&c}.
%
Namely, the platform offers multiple interfaces for consumers: an HTTP \ac{API} for \ac{DTE} management (Ecosystem Management Interface) and one for \ac{KG} access with a SPARQL endpoint for read-only queries, as well as a WebSocket endpoint to observe ecosystem updates (\ac{WoDT} Platform Interface).

The platform can process the \ac{WoT}-based \ac{DTD} described above to register \acp{DT}.
When a registration requested is submitted, the platform:
\begin{enumerate}[label=\textbf{Step \arabic*}, leftmargin=5.3em]
    \item locates the \ac{DTKG} observation form (\texttt{observeallproperties}) and starts observing the \ac{DT} for updates;\label{step:observe-dtkg}
    \item notifies the \ac{DT} of successful registration to let it update its \ac{DTD} with the platform URI;\label{step:notification}
    \item maps the \ac{DT} \ac{URI} to a local cache URL;
    \item observes \ac{DTKG} updates and merges them into the \ac{DTE} \ac{KG} stored in memory with Apache Jena.
\end{enumerate}

%
The current prototype supports the observation of \ac{DTKG} through WebSockets for \ref{step:observe-dtkg} and implements \ref{step:notification} sending a request to a hardcoded \texttt{/platform} HTTP endpoint on the \ac{DT}.

Consumers can interact with the \acp{DTE} through the \ac{WoDT} platform
either through (possibly repeated) SPARQL queries to retrieve the information they need or by observing the whole \ac{KG} through WebSockets.
%
In the prototype, all observers receive the whole \ac{KG} whenever there is an update so that consumers don't need to have specific update handling logic.
%

%=======================================
\section{Comparison with Related Works}
\label{sec:dte:hwodt:related}
%=======================================

This section compares the \ac{HWoDT} proposal with related approaches that target the integration of \acp{DT} in ecosystems. 
%
The aim of this comparison is to highlight the similarities and differences with existing solutions, discussing advantages and limitations of existing approaches with respect to the \ac{HWoDT}.

The \ac{HWoDT} proposal is not meant to compete with existing standards or framework, but rather to either complement them or provide an alternative approach to \ac{DT} interoperability.
%
The main principle of the \ac{HWoDT} is to promote interoperability through the adoption of \ac{REST} principles and Semantic Web technologies, given their proven effectiveness in enabling integration across heterogeneous systems on the Web.


%----------------------------------------------------
\subsection{HWoDT and Digital Twin Standards}
%----------------------------------------------------

The \ac{HWoDT} tackles the relevant issue of enabling \ac{DT} interoperability targeting the creation of \emph{heterogeneous \acp{DTE}}.
%
This is an open challenge in the \ac{DT} research community (see \Cref{sec:back:dt:interoperability}), and several interoperability frameworks are being proposed to streamline the integration of \acp{DT}~\cite{Barnard_2024}.
%
Among the most relevant standards are \ac{OPC-UA}, \ac{AAS}, \ac{AML}, and the \ac{W3C} \ac{WoT}. \Cref{tab:interop_comparison} summarizes a comparison of these standards across selected aspects.

\begin{table*}
    \centering
    \small
    \setlength{\tabcolsep}{3pt}
    \renewcommand{\arraystretch}{1.3}
   \begin{tabular}{>{\raggedright\arraybackslash}p{2cm}|>{\raggedright\arraybackslash}p{2.2cm}|>{\raggedright\arraybackslash}p{2.2cm}|>{\raggedright\arraybackslash}p{2.2cm}|>{\raggedright\arraybackslash}p{2.2cm}|>{\raggedright\arraybackslash}p{2.2cm}}
    \toprule
    \midrule
    \textbf{} & \textbf{AAS} & \textbf{OPC-UA} & \textbf{AML} & \textbf{WoT} & \textbf{Semantic Web} \\
    \hline
    \hline
    \textbf{Scope} &
    Asset digital representation &
    Device data / \ac{M2M} &
    Tool interoperability &
    Web-scale access &
    Cross-domain knowledge \\
    \hline
    \textbf{Data model} &
    Submodels + semantic IDs &
    Hierarchical node graph (address space) &
    Hierarchies (CAEX, CAD, PLCopen) &
    Thing Description (JSON-LD) &
    Triple Graphs (RDF / OWL) \\
    \hline
    \textbf{Semantics} &
    Structured submodels, semantic references &
    (Sub)Types and Relationships &
    Role-based, structural  &
    JSON-LD semantics, ontology-aligned &
    General-purpose ontologies \\
    \hline
    \textbf{Querying and Data access} &
    REST API &
    Browse address space &
    File access &
    SPARQL (if RDF-mapped) &
    SPARQL / Web browsing \\
    \hline
    \bottomrule
    \end{tabular}
    \caption{Comparison of interoperability frameworks across selected aspects.}
    \label{tab:interop_comparison}
\end{table*}

The comparison highlights how existing interoperability frameworks address complementary layers of \ac{DT} integration. 
\ac{AAS} focuses on structured digital representations of assets through modular submodels,
\ac{OPC-UA} provides a secure communication infrastructure centered on hierarchical address spaces for device-level data exchange,
and \ac{AML} supports interoperability among engineering tools via structured XML-based hierarchies. 
%
The \ac{W3C} \ac{WoT} promotes Web-scale integration by describing devices through \acp{TD},
while Semantic Web technologies provide a general-purpose, ontology-driven approach to cross-domain knowledge representation based on graph models and query languages such as SPARQL. 
Together, these standards span different scopes, data models, semantic mechanisms, and data access paradigms, reflecting the multi-layered nature of \ac{DT} interoperability.

Given their different nature, these standards are better viewed as complementary rather than competing. 
Integration efforts already aim to leverage their respective strengths:
\ac{OPC-UA} has been integrated with Web technologies~\cite{DBLP:journals/csi/CavalieriSS19},
operations in \ac{AAS}~\cite{platform_i40_aas_part1_v2} and affordances in \ac{WoT}\footnote{\url{https://profiles.opcfoundation.org/workinggroup/97}} can have \ac{OPC-UA} protocol bindings,
and \ac{AML} specifications can be linked to \ac{AAS} for a coherent view of assets across engineering and operational layers~\cite{DBLP:conf/etfa/DrathRH19,DBLP:conf/indin/WengerZ018}. 
In this landscape, Semantic Web technologies -- as adopted in the \ac{HWoDT} -- can be layered on top of existing standards to complement their capabilities.

The Semantic Web excels at heterogeneous knowledge integration under a uniform interface (\Cref{sec:back:web:semantic-web-technologies}),
enabling expressive queries and ontological reasoning across domains. 
Alignments between Semantic Web technologies and industrial standards have therefore been investigated,
highlighting similarities between, e.g., \ac{OPC-UA}~\cite{DBLP:conf/etfa/MajumderWD19,DBLP:conf/etfa/PerzyloP0K19}
and \ac{AAS}~\cite{DBLP:conf/icphys/BedenCB21, platform_i40_aas_part1_v2} data models,
in order to benefit from graph-based semantics and reasoning capabilities.

Following this trend, the \ac{HWoDT} proposes a hypermedia layer that integrates existing \acp{DT}, regardless of whether they are implemented using industrial standards. 
Since most interoperability standards emerged from the Industry~4.0 movement and primarily target industrial domains,
they leave a gap when modeling entities and concepts across heterogeneous domains. 
The \ac{HWoDT} addresses this limitation by relying on Web ontologies that span a wide variety of application areas and can be seamlessly combined within a unified \ac{KG}.

Providing a synchronized representation of the asset state is a defining feature of \acp{DT}. 
The \ac{HWoDT}, designed from a \ac{DT}-centric perspective, requires each \ac{DT} to continuously produce and update a representation of its observable state in the \ac{DTKG}. 
This enables consumers to query an up-to-date view of the entire ecosystem and observe state changes. 
Although similar synchronization can be implemented by \ac{AAS} servers (e.g., Eclipse BaSyx\footnote{\url{https://eclipse.dev/basyx/}}) retrieving data from \ac{OPC-UA} endpoints,
such approaches remain limited by the expressiveness of the underlying APIs and the lack of hypermedia-driven navigation.
%
Trends in the \ac{DT} engineering community show that service-oriented and micro-service architectures are becoming increasingly common in \ac{DT} implementations~\cite{ferko2022architecting},
making the \ac{HWoDT} a natural fit for these modern architectural patterns.
%
Web-based approaches describing \acp{API} with ontologies are emerging,
but they are often limited to specific domains and use cases (e.g. \cite{Liu_Jiang_Jiang_2020}),
and not yet adopted as a general solution for \ac{DT} interoperability as proposed in this thesis.

Leveraging the \ac{HWoDT}, developers can focus on application logic while abstracting away the complexity of heterogeneous \ac{DT} integration. 
This yields advantages similar to those promoted by the \ac{WoT} approach:
\begin{inlinelist}
    \item the application layer remains stable and unaffected by the introduction of new \acp{DT};
    and
    \item the \ac{DT} layer can evolve or replace underlying implementations without impacting upper layers, as long as the exposed interface is preserved.
\end{inlinelist}

The \ac{WoT} is a source of inspiration for the \ac{HWoDT} and the proposal in this thesis aims to extend the \ac{WoT} principles to the specific requirements of \acp{DT} and their ecosystems.
%
Related approaches also exploit the \ac{WoT} in the context of \acp{DT} with similar goals. 
%
A semantic extension of the \ac{WoT} ontology is presented in \cite{González-Gerpe_Cimmino_Bernardos_Poveda-Villalón_García-Castro} proposing to use \acp{TD} to integrate \acp{DT} in the construction domain,
extending it to capture the five-dimensional model of \acp{DT}~\cite{qi2021enablingtechdt}.
%
The \ac{WoT} is also the foundation of WoTwins~\cite{SciulloWoTwins2022} a framework to generate automatically \acp{DT} starting from existing \ac{WoT} things, modeling their behavior and enabling prediction features.
%
These works highlight the relevance of the \ac{WoT} as a standard for the future of \ac{DT} applications, despite its limitations in capturing fully the nature of a \ac{DT} in its entirety which may require extensions and development of custom ontologies aligned with the \ac{WoT} model.

A fundamental difference with the (semantic) \ac{WoT} compared to \ac{DTE} that this thesis exposes, however, lies in the seamless browsing of the \ac{DTE} through a \ac{KG} that explicitly captures the observable state of \acp{DT}.
%
In a W3C \ac{WoT} application, even if \acp{TD} were semantically annotated and stored in a \ac{KG},
selecting \emph{things} based on their current state would require issuing requests to retrieve the value of each property of each thing. 
The \ac{WoT} does not natively address the \ac{DT}-specific requirement of continuously shadowing the asset state,
nor does it specify how to manage dynamic relationships among things,
limiting the expressiveness of cross-asset queries.

The \ac{HWoDT} leverages the intrinsic synchronization capability of \acp{DT} through the \ac{DTKG},
simplifying interaction and enabling direct, state-aware navigation. 
Without such a unifying abstraction, consumers interacting with heterogeneous \acp{DT} would need to individually retrieve, interpret, and merge their state representations to resolve a query (\Cref{fig:comparison-custom-vs-hwodt}). 

\begin{figure}[tb]
  \centering
  \includegraphics[width=\columnwidth]{figures/hwodt/comparison_custom_hwodt.pdf}
  \caption{Comparison of the steps required to perform a ``query'' across heterogeneous \acp{DT} versus with the \ac{HWoDT} approach, which abstracts heterogeneity through a uniform interface.}
  \label{fig:comparison-custom-vs-hwodt}
\end{figure}

Overall, the \ac{HWoDT} is compatible with and complementary to existing \ac{DT} interoperability efforts, positioning itself as an additional semantic and hypermedia layer that integrates \acp{DT} implemented with—or without—industrial standards, supporting:
\begin{itemize}
    \item integration with \acp{DT} beyond the industrial domain,
    \item advanced semantic querying with reasoning,
    \item Web-based navigation and direct access to up-to-date asset representations,
    and
    \item subscription to changes in the \ac{DT} state.
\end{itemize}



%----------------------------------------------------
\subsection{HWoDT in the Landscape of DT Platforms}
%----------------------------------------------------

The \ac{HWoDT} approach offers a solution to implement open \acp{DTE} on top of heterogeneous \ac{DT} technologies.
This contrasts with other \ac{DT} platforms, which typically support homogeneous \acp{DTE} and allow multiple \acp{DT} to be defined within the same instance, but constrain all of them to specific technological choices.

To frame the positioning of \ac{HWoDT} within the broader \ac{DT} ecosystem, \Cref{tab:dt-platform-comparison} compares a selection of representative platforms among refernece \ac{DT} technologies (\Cref{sec:back:dt:technology}) that support the construction of \acp{DTE} along key dimensions.


%\begin{landscape}
%\begin{table}[bp]
\begin{sidewaystable}[p]
\centering
\caption{Comparison of Digital Twin Ecosystem Platforms Along Key Architectural Dimensions}
\label{tab:dt-platform-comparison}
\renewcommand{\arraystretch}{1.3}
\footnotesize
\begin{tabularx}{\linewidth}{
    >{\raggedright\arraybackslash}p{3cm}|
    >{\raggedright\arraybackslash}X|
    >{\raggedright\arraybackslash}X|
    >{\raggedright\arraybackslash}X|
    >{\raggedright\arraybackslash}X|
    >{\raggedright\arraybackslash}X||
    >{\raggedright\arraybackslash}X
    }
\toprule
\midrule
\textbf{Comparison Dimension} &
\textbf{Azure Digital Twins} &
\textbf{Eclipse Ditto} &
\textbf{WLDT} \cite{PICONE2021100661}&
\textbf{DT-as-a-Service} \cite{Talasila_Gomes_Mikkelsen_Arboleda_Kamburjan_Larsen_2023}&
\textbf{TwinBase} \cite{Autiosalo_Siegel_Tammi_2021} &
\textbf{HWoDT Platform} \\

\hline
\hline

\textbf{Primary Role} &
Registry &
Registry &
Runtime &
Runtime &
Registry &
Registry \\
\hline

\textbf{Deployment} &
Cloud (PaaS) &
Microservices &
Containerized &
Microservices &
Git-based &
Containerized\\
\hline

\textbf{Governance} &
Centralized &
Centralized &
Centralized &
Centralized &
Decentralized &
Both \\
\hline

\textbf{DT Distribution} &
Distributed &
Distributed &
Instance &
Instance &
Distributed &
Distributed \\
\hline

\textbf{Platform Coupling} &
Aware &
Aware &
Bound &
Bound &
Aware &
Independent \\
\hline

\textbf{DT Representation} &
Custom (DTDL) &
Standard (WoT) &
Any  &
Any (Service-defined)&
Custom (DT Document) &
RDF (DTD + DTKG) \\
\hline

\textbf{Synchronization Management} &
External (Azure Services)&
External (IoT middleware)&
Internal (Shadowing) &
Internal (Service Orchestration)&
External (Git) &
Internal (DTKG Observation) \\
\hline

\textbf{DT Relationships} &
Explicit  &
Implicit (Attributes)&
Explicit &
Explicit &
Explicit &
Explicit \\
\hline

\textbf{Data Retrieval} &
Query (Custom Language) &
Query (Resource Query Language) &
Any (Adapters) &
Any (Service-defined)&
API &
Query (SPARQL) \\
\hline

\bottomrule
\end{tabularx}
\end{sidewaystable}
%\end{table}
%\end{landscape}


These comparison dimensions adopted are chosen to highlight differences between the different approaches, both in terms of design purpose, structural and operational characteristics, which are critical for analyzing how platforms support the construction of \acp{DTE}. The main comparison dimensions are:
%
\begin{itemize}
    \item \textbf{Primary Role}: indicates whether the platform functions primarily as a \emph{registry} or a \emph{runtime} for \acp{DT}.
    \item \textbf{Deployment}: the technological and operational model used to deploy the platform itself.
    \item \textbf{Governance}: whether the platform adopts a centralized or decentralized management of \acp{DT}. Centralized governance typically implies that \acp{DT} are created and managed within the platform, while decentralized governance allows for independently created \acp{DT} to be registered and integrated.
    \item \textbf{DT Distribution}: the approach used to deploy \ac{DT} instances across the system. Namely, this dimension distinguishes between platforms that require \ac{DT} instances to be deployed within the platform itself (instance-based) and platforms that allow \ac{DT} instances to be distributed and managed independently, with the platform serving as an integration layer.
    \item \textbf{Platform Coupling}: whether \acp{DT} exist only within the scope of the platform (bound), must be aware of being part of the platform (aware) or can be completely platform independent, regardless of the deployment model. This dimension is critical to understand the level of openness and flexibility in integrating \acp{DT} from different sources.
    \item \textbf{DT Representation}: the supported representation format used to describe and catalog \acp{DT} within the scope of a platform.
    \item \textbf{Synchronization Management}: whether and how \ac{DT} state management is handled by the platform or must be implemented externally. 
    \item \textbf{DT Relationships}: whether relationships between \acp{DT} are explicitly modeled or implicit.
    \item \textbf{Data Retrieval}: the mechanisms provided for querying or accessing \ac{DT} data.
\end{itemize}



The selection is not meant to be an exhaustive collection of \ac{DT} development platforms for which we refer to other surveys~\cite{gil2024spe,9640612,Mihai_survey_enabling_2022},
but rather a representative sample of the state of the art,
focusing on platforms that may support the construction and management of multiple \acp{DT}.
%
Accordingly, the comparison includes the following platforms analyzed as follows: 

\paragraph{Azure Digital Twins\protect\footnote{\url{https://azure.microsoft.com/en-us/services/digital-twins/}}} is a cloud-based PaaS solution that supports the definition of multiple interconnected \acp{DT} instances with an explicit notion of relationships between them. \azureTwin{} is selected as a representative of commercial cloud platforms. A similar general-purpose platform is AWS IoT TwinMaker\footnote{\url{https://aws.amazon.com/iot-twinmaker/}}, or other industry-specific platforms such as Siemens Insights Hub\footnote{\url{https://plm.sw.siemens.com/en-US/insights-hub/}}, Bentley iTwin\footnote{\url{https://www.bentley.com/en/products/brands/itwin}}, and PTC ThingWorx\footnote{\url{https://www.ptc.com/en/products/thingworx}}. 
%
Azure Digital Twins functions primarily as a registry of \acp{DT}, which explains its strong support for cataloging multiple \acp{DT} and managing explicit relationships.
Its cloud PaaS deployment ensures distributed availability,
while centralized governance guarantees full control over \ac{DT} creation and lifecycle. 
\ac{DT} instances are platform-aware, meaning they rely on Azure-specific constructs,
and Azure's custom \acf{DTDL} is used for representation to support complex modeling, though at the cost of cross-platform interoperability.
Synchronization generally relies on external Azure services, and data retrieval uses a dedicated query language optimized for the platform.

\paragraph{Eclipse Ditto\protect\footnote{\url{https://www.eclipse.org/ditto/}}} is an open-source microservices-based registry for \ac{IoT} \acp{DT}. Ditto is selected as a representative of open-source platforms that integrate with existing \ac{IoT} ecosystems. FIWARE~\cite{8950963} is another similar platform that has been used in the \ac{DT} context~\cite{9346030}, although more focused on data management and integration rather than \ac{DT} modeling and lifecycle management.
%
Ditto's role as a registry reflects its focus on managing \ac{DT} metadata rather than executing their behavior.
Ditto is implemented as a microservice application that can be deployed in various environments, allowing integration into diverse \ac{IoT} infrastructures.
Governance is centralized within the Ditto instance, while \acp{DT} can be deployed as distributed components implementing the  platform-aware synchronization process through \ac{IoT} middlewares.
Representations use the custom  \emph{Ditto thing} model, but can also expose \ac{WoT} \acp{TD} for standard compliance, \ac{DT} relationships are implicit, inferred from attribute structures.
Data retrieval is supported through the Resource Query Language~\footnote{Resource Query Language Project: \url{https://github.com/persvr/rql}}.


\paragraph{\acl{WLDT}}~\cite{PICONE2021100661} is a research-oriented framework for developing \acp{DT} as standalone software components. \ac{WLDT} is selected as a representative of research prototypes which explores the ability to deploy \acp{DT} in a variety of infrastructures and here is considered in its ability to run multiple \acp{DT} instances within its multi-threaded runtime (see \Cref{sec:dte:engineering-dt:wldt-framework}). Here it is compared to the \ac{HWoDT} to highlight the complementarity of two contributions of this thesis, with \ac{WLDT} providing a flexible runtime for \ac{DT} execution and \ac{HWoDT} offering a unifying layer for heterogeneous \ac{DT} integration and ecosystem management.
%
WLDT serves primarily as a runtime for executing \acp{DT},
deployed in containerized environments for flexible infrastructure support.
Governance is centralized, as \ac{DT} instances are bound to the runtime, unable to operate independently.
The framework allows any representation format, facilitating experimental setups through the modular digital adapters introduced in \Cref{sec:dte:engineering-dt:physical-digital-adapters}.
Synchronization is handled internally through the shadowing process,
relationships are explicitly defined although manually handled by developers,
and data access is adaptable via custom adapters, reflecting its research-oriented flexibility.

\paragraph{DT-as-a-Service Platform (DTaaS)}~\cite{Talasila_Gomes_Mikkelsen_Arboleda_Kamburjan_Larsen_2023} is a research-oriented platform with service-level orchestration to manage the lifecycle of multiple \acp{DT} instances. DTaaS is selected as a representative of research prototypes which explores the ability to orchestrate and combine services to implement and run \acp{DT} instances. A similar approach is taken by the OpenTwins platform~\cite{robles2023cii} which extends Ditto with a composition of additional services.
%
DTaaS operates as a runtime platform orchestrating \acp{DT} as services and supporting both the development and execution of \ac{DT} models e.g., through co-simulation services~\cite{7496424}.
Its microservice deployment supports flexible service composition, while governance remains centralized to control service-based DT lifecycles.
\ac{DT} instances are bound to the service runtime, and representation is service-defined, allowing custom but platform-specific models.
Synchronization occurs internally through orchestration, relationships are explicitly modeled in service definitions, and data retrieval is determined by implemented service APIs.

\paragraph{TwinBase}~\cite{Autiosalo_Siegel_Tammi_2021} is a Git-based registry promoting Web standards and interoperability. TwinBase is selected as a representative of research efforts that support the management of multiple \acp{DT} with a decentralized approach to governance following the idea of a \emph{Digital Twin Web} that is close to the vision pursued in this thesis.
%
TwinBase's design emphasizes registry functionality with decentralized governance,
allowing independently created \acp{DT} to be registered and integrated.
Its Git-based deployment supports distributed \acp{DT} that must be aware of the platform to update their representation.
\ac{DT} representation uses custom \ac{DT} documents (currently as YAML documents), synchronization relies on Git versioning, and relationships can be explicitly defined linking \acp{DT} within the platform.
Data access is provided via APIs, reflecting the platform's focus on openness, interoperability, and adherence to web standards, but limiting complex interactions and query capabilities, reflecting the more static nature of the platform compared to the other approaches.

\paragraph{HWoDT Platform} 
The HWoDT platform is the approach proposed in this thesis to implement hypermedia-based \acp{DTE} on top of heterogeneous \ac{DT} technologies.
%
The HWoDT platform functions primarily as a registry of \acp{DT}, supporting the management of multiple \acp{DT} instances and their relationships within a unified ecosystem.
It is deployed in containerized environments, which allows flexible distribution across heterogeneous infrastructures.
Governance can be managed as centralized to limit the ability to register \ac{DT} instances or decentralized to allow more flexible participation of \acp{DT} in the ecosystem.
Regardless of the governance model, \acp{DT} themselves are fully independent, meaning they can exist and operate outside the platform's direct control as long as they adhere to the uniform interface, which can be implemented as an external adapter as demonstrated in \Cref{sec:hwodt-impl}.
The only platform-aware behavior required from \acp{DT} is to expose the \ac{URI} of the platform in their \ac{DTD} to enable discoverability of ecosystems the \ac{DT} is part of. 
HWoDT uses RDF-based representations combining DTD and DTKG, supporting semantic interoperability and explicit modeling of relationships between \acp{DT}.
%
Synchronization is managed internally through the DTKG observation mechanism which ensures that the platform maintains an up-to-date representation of the state of all registered \acp{DT}.
This inversion of control compared to other platforms, where the platform actively retrieves updates from \acp{DT}, allows for the integration of independently developed \acp{DT} and simplifies the synchronization process.
%
Data retrieval is performed via SPARQL queries, enabling rich semantic access.

Compared to the other platforms, this positions HWoDT as a bridge between centralized commercial clouds like Azure Digital Twins, open-source registries like Ditto, research runtimes like WLDT and DTaaS, and decentralized approaches like TwinBase. It combines their strengths -- support for multiple DTs, cross-platform relationships, semantic expressiveness, and operational flexibility -- while overcoming typical limitations such as platform lock-in, constrained representation formats, and lack of independent DT deployment. In the broader DT ecosystem, HWoDT thus occupies a unique position as a heterogeneous, open, and interoperable platform, capable of supporting both experimental research and operational deployments across diverse infrastructures.


%==========================================
\section{Prototype Performance Evaluation}
%==========================================

A performance evaluation of the prototype is carried out to quantify the performance implications introduced by the additional layer introduced by the \ac{HWoDT} to implement \acp{DTE}.
%
\Cref{fig:latency-hwodt} schematically shows the communication between components in a \ac{HWoDT}-based system. \circled{A} is the latency between \ac{PA} and \ac{DT}, \circled{B} is the latency between the \ac{DT} and the \ac{HWoDT} adapter, \circled{C} is the latency between the adapter and the platform, while \circled{D} is the latency between the platform and the client. In a scenario where the DT is used without the platform \circled{E} would be the latency between the DT and the client. This shows that the additional potential latency loss in a \ac{HWoDT} system is given by \circled{B}+\circled{C}.

\begin{figure}[tb]
  \centering
  \includegraphics[width=0.8\columnwidth]{figures/hwodt/HWoDT-latency.pdf}
  \caption{Latency in an HWoDT-based system. From left, to right, the \ac{PA}, the \ac{DT}, the \ac{HWoDT} adapter, the \ac{WoDT} platform and the client. Each connection between components (labelled with letters) may introduce latency.}
  \label{fig:latency-hwodt}
\end{figure}

Accordingly, the following evaluation assesses the performance of the \ac{HWoDT} platform prototype, thereby characterizing the overhead incurred when integrated atop existing interoperability standards.
Even though performance optimization was not a primary focus in the development of the \ac{HWoDT} platform prototype, it can still be useful to make some considerations on the performance of the \ac{HWoDT} approach.
%
To do so an experimental setup is used to measure the latency \circled{B}+\circled{C}+\circled{D} when interacting with the platform and compare it with \circled{E}.

%-----------------------------------
\subsection{Experimental Setup}
%-----------------------------------

The experiment is designed to stress the performance of the platform in a resource-intensive scenario, and measure the delay perceived by consumers wishing to interact with \acp{DT} and the ecosystem as a whole.
%
A synthetic workload is used because---differently from a real-world \ac{IoT} dataset which may include data spikes over time---the scenario allows us to have a fully controllable environment to scale the traffic and evaluate the response of the platform under different loads.

A \ac{DTE} of a network of temperature sensors is implemented, with \acp{DT} reporting an update every second using \acp{DT} implemented with the \ac{WLDT} framework, running within the same process.
%
Each \ac{DT} generates a random temperature value between 0 and 100 every second, with a random initial offset of up to one second to avoid perfect synchronization of the \acp{DT}' data streams. The \ac{DT} state includes a timestamp marking when the state was computed by the \acp{DT}.
%
The different interaction patterns of the \ac{DTE} are tested:
\begin{itemize}
    \item with clients \emph{observing} all the \acp{DT} receiving all the \acp{DT} state notifications;
    \item with clients periodically \emph{querying} the \ac{DTE} to retrieve all sensors that have a reported temperature greater than 50. 
\end{itemize}
%
The key metrics collected are the \emph{freshness} of the information, which is computed as the difference between the time a state notification is received and the timestamp of the message (or the greatest timestamp when observing the \ac{DTE} \ac{KG}).
%
Additionally, the time taken to perform a query on the ecosystem retrieving the state of all \acp{DT} matching a condition is measured.

The experiments have been carried out running the \acp{DT} the \ac{WoDT} platform and the client applications on the same machine equipped with a 13th Gen Intel(R) Core(TM) i7-13700H CPU and 32 GB of RAM, using the OpenJDK 21.0.7 runtime.
%
Data is collected by analyzing logs of the client applications.

All \acp{DT} have been implemented with the \ac{WLDT} framework. 
The framework supports defining \ac{DT} implementations and uses a plugin-based architecture to expose different interfaces. 
This has been useful to implement different interaction patterns for the same \ac{DT}.

Measurements are repeated with 50, 100 and 200 \acp{DT}, and performing operations with 1, 50, 100 clients each observing all \acp{DT} and querying the entire \ac{DTE}.


%-----------------------------------
\subsection{Results}
%-----------------------------------

This section briefly reports results of the performance evaluation focusing on the different interaction patterns of the \ac{WoDT} platform prototype and comparing them with baseline direct interactions with individual \acp{DT}.
%
The results are summarized in \Cref{fig:results-performance}, showing two bar plots reporting the average values of the two metrics collected in the experiments.
%
Results confirm that the overhead introduced by the \ac{HWoDT} platform is not negligible, but still acceptable for many \ac{DT} applications, especially when considering the benefits of having a uniform interface to interact with heterogeneous \acp{DT}, especially when considering the ability to query the whole \ac{DTE} \ac{KG}.

\begin{figure}[tb]
    \centering
    \begin{subfigure}[t]{0.49\textwidth}
        \centering
        \includegraphics[width=\textwidth]{figures/hwodt/mean_latency_barplot.pdf}
        \caption{}
        \label{fig:observation-delay}
    \end{subfigure}
    \hfill
    \begin{subfigure}[t]{0.49\textwidth}
        \centering
        \includegraphics[width=\textwidth]{figures/hwodt/mean_request_time_barplot.pdf}
        \caption{}
        \label{fig:query-time}
    \end{subfigure}

    \caption{Performance indicators with different loads of \acp{DT} and clients working on the ecosystem: (a) average time difference when observing \acp{DT} either directly or through the \ac{WoDT} platform, and (b) average time per operation when querying the DT ecosystem either through the \ac{WoDT} SPARQL endpoint or directly retrieving the state from each \ac{DT} API.}
    \label{fig:results-performance}
\end{figure}

%...................................
\subsubsection{\aclp{DT} Observation}
%...................................

The \ac{HWoDT} adapter sends updates of the \ac{DTKG} on a WebSocket connection to the platform. Clients are set up to connect to the platform and observe each \ac{DT} through a WebSocket connection.
This aims to measure the delay introduced by the communication between the adapter and the platform, before getting to a final consumer. 
%
As the platform also supports observing the full \ac{DTE} \ac{KG}, a client also observes directly the \ac{KG} updates over a WebSocket connection (\circled{B}+\circled{C}+\circled{D}).
%
This is compared against direct observation of the \ac{DT}, which is implemented by sending each state update to an MQTT broker. Clients connect to all topics \texttt{dt/+/state} to observe all \acp{DT} (\circled{E}).

Whenever a message is received, the \emph{freshness} metric is computed as the difference between the timestamp of the message and the time the message is received using the system clock (which is synchronized as all processes run on the same machine). 
%
For the \ac{DTE} \ac{KG} as a new state update is generated for each change of each \ac{DT}, the greater timestamp among all \acp{DT} is taken to compute the freshness.

\Cref{fig:observation-delay} shows the average time difference with the different observation techniques. 
As expected, directly observing individual \acp{DT} is more efficient, as there is virtually no overhead. Users of the \ac{WoDT} platform should prefer this whenever possible, following the affordances of \acp{DT} in the \ac{DTD}.
%
In all scenarios, the delay grows with the number of \acp{DT}. This is especially evident when considering the observation of the whole \ac{DTE} \ac{KG}, which is the least efficient. Analyzing the performance of the prototype it was possible to deduce that this is caused by the serialization of the whole \ac{KG} to stream it at every update, which is computationally intensive. 
%
This is also the reason why the figure avoids reporting the observation of the ecosystem \ac{KG} with more than 1 client, as the results are significantly worse than the other options, even with just 1 client.
Performances are expected to be less drastically impacted when using smarter \ac{KG} update notification strategies---such as sending incremental changes (e.g., as in \cite{roffia2018fi})---at the cost of requiring more complex clients, capable of applying the changes to reconstruct the full state.

%...................................
\subsubsection{Repeated Querying}
%...................................

Clients can query the \ac{DTE} through the SPARQL endpoint of the \ac{WoDT} platform.
%
This is compared with performing an interrogation directly on all \acp{DT} to retrieve the current state and then filtering it on the client side which is implemented by having all \acp{DT} expose a simple HTTP endpoint \texttt{GET /state} to retrieve the current state of the \ac{DT} (similarly to what is shown in \Cref{fig:comparison-custom-vs-hwodt})
%
For each client, requests are sent every second, ideally, since \acp{DT} send updates at the same frequency, this would be a realistic way to monitor the sensor network in our scenario.

The overall time the operation takes is measured by computing the time of sending the request and the response for the SPARQL query, and the time of performing all HTTP requests and then filtering for the direct interaction.
The time of performing SPARQL queries (\circled{B}+\circled{C}+\circled{D}) is compared against the time taken by a client to perform an HTTP request to retrieve the state of each \ac{DT} and filter them on the client side \circled{E}.
%
\Cref{fig:query-time} shows that the average time taken to perform a query is significantly less than requesting the state on each \ac{DT}, especially when the number of \acp{DT} in the ecosystem grows. 
%
Notably, when running only 1 client, the overhead of the computation of the platform \ac{KG} makes using queries less effective. 

%...................................
\subsubsection{Comparison of Interaction Patterns}
%...................................

Results indicate that, with large numbers of \acp{DT}, using SPARQL queries becomes a more scalable and effective approach for observing the evolution of the ecosystem over time, comparable with direct observation of all \acp{DT}.
%
While this approach may not capture every individual update from each \ac{DT}, it reliably provides the latest available state at the time of each query.
In this experimental setting, for instance, querying every second is realistically sufficient to capture almost all updates, as the client would match the update frequency of the sensors.
This trade-off favors scalability and responsiveness, especially in dynamic settings where maintaining subscriptions for all \acp{DT} would be resource-intensive or impractical.

The centralized implementation of the platform makes it a potential bottleneck. Specifically, interacting with the \ac{DTE} KG is costly because the graph gets updated by all \acp{DT} updates.
%
This is a reasonable trade-off as interacting with the \ac{DTE} provides advanced functionalities that would be impossible or very difficult to replicate simply by interacting with individual \acp{DT}.
%
Nevertheless, it will be interesting to explore optimization techniques for future improvements of the platform, as well as decentralized approaches to provide ecosystem services that would make the system less reliant on the platform itself.

% %=================================================
% \section{Discussion, Limitations and Future Works}
% %=================================================


%==============================
\section{Final Remarks}
%==============================

This chapter presents a proposal for the engineering of heterogeneous \acp{DTE} grounded on principles and technologies from the Web and the Semantic Web.
%
The chapter addresses the research question:

\paragraph{\ref{rq:2} How to engineer DTEs which integrate heterogeneous DTs and support
applications in interacting with them?}
%%
The proposal of the \ac{HWoDT} addresses the challenges of integrating heterogeneous \acp{DT} by means of a uniform self-described interface that allows the representation of \ac{DT} interaction possibilities and metadata (through the \ac{DTD}) as well as the up-to-date state of the \ac{DT} (through the \ac{DTKG}).
%
The services and interfaces offered by the \ac{HWoDT} platform allow consumers to seamlessly interact with the \ac{DTE}, abstracting away the heterogeneity of the underlying \acp{DT}.
%
Consumers can either observe the evolution of the \ac{DTE} \ac{KG} or perform expressive queries to retrieve information through SPARQL. 
%
The prototype implementation of the \ac{HWoDT} platform and the set of adapters for representative \ac{DT} technologies demonstrate the feasibility of the approach in integrating \acp{DT} developed with different technologies through a shared semantic layer that can complement existing interoperability standards.


%****************************************************************************************
%****************************************************************************************
\part{Intelligent Applications with \acl{MAS}}
%****************************************************************************************
%****************************************************************************************

%%%%%%%%%%%%%%%%%%%%%%%%%%%%%%%%%%%%%%%%%%%%%%%%%%%%%%%%
\chapter{Intelligent Applications in Healthcare}
\label{chap:mas:requirements}
%%%%%%%%%%%%%%%%%%%%%%%%%%%%%%%%%%%%%%%%%%%%%%%%%%%%%%%%

%=======================================================
\section{Autonomy and Intelligence}
%=======================================================

%=======================================================
\section{Explainability and Transparency}
%=======================================================

%%%%%%%%%%%%%%%%%%%%%%%%%%%%%%%%%%%%%%%%%%%%%%%%%%%%%%%%
\chapter{\aclp{MAS} and \aclp{DT}}
\label{chap:mas:mas-dt}
%%%%%%%%%%%%%%%%%%%%%%%%%%%%%%%%%%%%%%%%%%%%%%%%%%%%%%%%

%-------------------------------------------------------
\section{Distributing Intelligence}
%-------------------------------------------------------

Given the current interpretation and use of the AA and DT abstractions in the literature, it is apparent that both AAs and DTs can be used to encapsulate some form of intelligence. 
%(the former have been conceived to do this), and both have been used to model and engineer IoT systems and applications (the latter in particular).
Thus, an open engineering question involving both research communities is: \textbf{are AAs and DTs different ways to achieve essentially the same thing? 
Are they (possibly partially) overlapping? 
Or are they complementary?} 

In this section, we argue for the latter. 
We object that existing approaches often blur the boundaries between the responsibilities of AAs and DTs creating confusion by trying to fit everything within the same abstraction (be it an AA, a DT, or any other unprincipled \emph{ad-hoc} combination of the two abstractions). 
%
The direct cause for this is that the AA and DT abstractions are not associated with clear roles within the architecture of an IoT system, hence they do not help produce \emph{modular} and \emph{reusable} designs that can then be implemented across application domains. 
%
This generates several competing architectural models all trying to address similar problems and thus overlapping with existing solutions, leading to fragmentation and confusion for researchers and practitioners seeking the ``best'' design for their specific problem at hand. 

In an attempt to remedy this, 
we first identify a set of principles to assist system designers in the analysis of the functional requirements of their intelligent IoT system at hand (Subsection~\ref{ssec:principles}), 
then frame AAs and DTs as the two abstractions perfectly adhering to such principles, complementarily (Subsection~\ref{ssec:abs-def}). 
Finally, we propose AAs and DTs as the basic building blocks that designers can use and compose to create their application-specific software architectures (Subsection~\ref{ssec:multi-layer}).


%======================================================
\subsection{Design Principles}
\label{ssec:principles}
%======================================================

%Recent literature tried to shed light on the distinguishing traits of the two abstractions/paradigms of AAs and DTs, to motivate their synergistic exploitation in dealing with the engineering of IoT systems and applications.
%
We take our move from the quite abstract software engineering criterion of \emph{separation of concerns}given in~\cite{DBLP:conf/atal/MarianiPR22}, expanding and refining it along three dimensions to make it more practically useful: \emph{specificity}, \emph{scope}, and \emph{time}. 
%
These are meant to be used to analyse the requirements posed by the intelligent functionalities summarised in Subsection~\ref{ssec:functions}. 
%
%Subsection~\ref{ssec:abs-def} discusses how they define a spectrum of design choices leaning towards a DT, an AA, or some integration of both.
We anticipate that it is likely that neither of these dimensions \emph{alone} is sufficient to determine \emph{univocally} which abstraction is best suited for a given functionality. 
Instead, we argue that they all need to be taken into consideration together when approaching the design of an intelligent IoT system.
%
%Even though we are now going to present them separately, it is important to keep in mind that they are %not meant to be a strict set of rules to be followed blindly, but 
%a set of principles that can guide designers in the analysis of a DI system when used together.


The reason to ground our principles in a \emph{software engineering} standpoint, 
that is, from the perspective of the software designer of an IoT system or application, 
is that IoT systems are increasingly no longer ``mostly hardware'' ones, with interconnected devices that simply exchange data. 
Instead, they are increasingly becoming complex cyberphysical systems
where software plays a prominent role, 
especially regarding endowing intelligence into devices (hence into the system).
There, thus, proper engineering of the software layer of the overall IoT deployment 
is crucial to guarantee both functional and non-functional properties. 
Not by chance, the publications discussed in Section~\ref{sec:back} deal mostly with \emph{software} engineering, architectures, and design. 


%--------------------------------------
\subsubsection{Specificity} 
%--------------------------------------

The first principle we propose is what we call \emph{specificity}, whose spectrum is depicted in Figure~\ref{fig:specificity}, and is defined informally as 
\begin{quote}
    ``how much a given (intelligent) functionality is \emph{specifically} serving a given application goal (at one end of a spectrum), rather than being \emph{generally} exploitable by multiple goals and potentially across application scenarios (at the other end)''.
\end{quote}
%
With this, we want to help IoT systems and application designers to distinguish between 
\emph{(i)} those features that are strictly tied to the achievement of a specific \emph{application goal}, and 
\emph{(ii)} those that may instead be \emph{generically} useful to pursue multiple goals and build other intelligent functionalities and/or applications on top. 
%
This is especially important for designing highly modular systems. 
%
%Separating these functionalities effectively is crucial to identify components that can more easily evolve over time and at the same time can support the evolution of the whole system allowing new functionalities to be added on top.

\begin{figure}[!b]
    \centering
    \includegraphics[width=.6\columnwidth]{figures/dt-mas/specificity-spectrum.pdf}
    \caption{The spectrum of \emph{specificity}: from intelligent functionalities specific to an application goal to general-purpose services.}
    \label{fig:specificity}
    %\Description{the spectrum of specificity principle, with a gradient from specific to general}
\end{figure}

In IoT systems, for instance, a widespread general-purpose functionality concerns the digitalization of Physical Assets (PAs) to enable intelligent functionalities such as prediction of that assets' future states, and simulation of alternate states. 
%
%This is an essential step as it allows the collection of data from the devices and the exposure of the operations that can be invoked by other components. 
%
Functionalities like these can be considered to be not tied to the application goals, but to the general services offered by the IoT platform, in an open systems, ``as-a-service'' perspective~\cite{10.1145/3507909}. 
%
In contrast, functionalities such as fault diagnosis or inference of specific information may be relevant only in the context of the application domain or goal. 
Another example would be the implementation of adaptive control policies and decision-making strategies: these are likely to be specialised for the system at hand, as they encode the requirements and goals set by stakeholders. 

However, each specific instance of these categories of intelligent functionalities devised out in Subsection~\ref{ssec:functions} must be carefully examined by the system designer in light of this (and the other) principle(s), as there can be no ``rule of thumb'' generally applicable to every application domain and goal. 
%
%For instance, in a smart building, there could be several policies introduced to optimize comfort or energy consumption. Those policies will be part of the system due to a specific requirement coming from the stakeholders which means that even if they are reusable across different buildings they have been designed. 
%
%However, we want to stress that deciding whether these functionalities can be considered potentially general, instead, as they can output results that can be used by different components in the system, is up to the system designer. 
%He/she must keep our proposed guiding principle in mind, match it against the functionality to be realised (and its requirements), and then find the appropriate answer to the question of specificity. 
%
%Also, keep in mind that taking a single principle in isolation is unlikely to completely determine a design choice in complex IoT systems and applications. 
%That's why we provide more, complementary principles. 
%
Also, generally speaking, there may be edge cases that may feel ``in violation'' of our proposed principles but are actually not.  
For instance, it may be the case that an application goal coincides with the digitalisation of an asset for prediction purposes. 

This case should not be interpreted as ``wrong'' because it defines a prediction function as application-specific. 
On the contrary, the specificity principle, in this specific case, cannot guide designers' choice \emph{alone}. 
This is one reason to propose more, complementary principles, that better capture the different nuances of complex systems and applications---not by chance the edge case described is quite simple. 

%-----------------------------------------
\subsubsection{Scoping} % Locality, Scoping, Coupling 
%-----------------------------------------

%\ste{Tutti}{Abbiamo convenuto che il nocciolo del principio sta nel capire da dove/cosa viene l'input alla funzione: se locale a una entità di interesse o se globale da più entità opportunamente aggregate}

%\ste{Tutti}{ruolo centrale della modellazione: il DT modella una realtà di interesse (PA) ben specifica e concettualmente delimitata, mentre l'agente modella obiettivi applicativi e criteri decisionali che possono avere legame mutevole con la realtà modellata utile allo scopo. I secondi sfruttano i primi come servizio, per non costringere l'agente a modellare la realtà (meglio dirlo in sez. 4.2 forse). Questo abilita il riuso: il DT può essere usato in più applicazioni se offre servizi generici, l'agente può lavorare su più domini grazie a DT che astraggono (anche questo meglio dirlo in sez. 4.2 forse).}

The second principle we propose, we call \emph{scoping}, is defined as follows:
\begin{quote}
    ``to what extent a given (intelligent) functionality requires data from the whole system (at one end of a spectrum) or a single individual entity of interest (at the other end) to deliver its results''
\end{quote}
With this, we want to help IoT systems and applications designers distinguish between \emph{(i)} the features that require data, inputs, and interactions with multiple entities of interest in the system at hand (up to \emph{global} information, synthesised at the system level), and \emph{(ii)} the features that instead require only considering a single asset (the extreme case of \emph{local} information). 
Figure~\ref{fig:scope} depicts this spectrum. 

\begin{figure}[!b]
    \centering
    \includegraphics[width=.6\columnwidth]{figures/dt-mas/scope-spectrum.pdf}
    \caption{The spectrum of \emph{scoping}: from intelligent functionalities requiring knowledge of the whole system, to those confined to individual assets.}
    \label{fig:scope}
   
\end{figure}

In any IoT system, due to its inherent connection with the physical world, there are software components whose main function 
is coupled with PAs, which set boundaries to the \emph{scope} within which intelligent functionalities operate. 
For example, by collecting local information only (e.g.\ prediction of the future states of specific machinery, not others). 
%
Conversely, many other functionalities may require a broader scope that goes well beyond such local boundaries. 
%
If we take as an example the domain of a smart home, we can have 
software components that mirror, and grant access to, the individual devices (e.g.\ for remote monitoring and control), 
but also adaptive control routines spanning multiple devices or even multiple rooms to orchestrate a coordinated action (e.g.\ preparing for the comeback of an inhabitant by activating the A/C, unlocking the doors, raising the curtains, etc.). 
%
Accordingly, with the scoping principle, we suggest that system designers consider where the knowledge and data required to accomplish the intelligent function come from, as well as which entities will be subject to its effects. 
%


It should be noted that the \emph{individual} and \emph{global} scopes are the two extremes of the spectrum depicted in Figure~\ref{fig:scope}. 
Functionalities may require data from multiple entities, but such entities may be confined in a limited space, for instance geographically, or network-wise. 
Or, it could be the case that data is required from different and distant entities that are anyway somehow easy to access together---think of overlay networks. 
%
In this case, the proposed principle is still useful to guide design choices, as it forces designers to clarify and sort out why and how they deem the scope to be worth defining global or local (or even individual). 


%--------------------------------------
\subsubsection{Timing} 
%--------------------------------------

The third (and last, for the time being) principle we propose is what we call \emph{timing}, which we define informally as 
\begin{quote}
    ``how much a given functionality relies on any one notion of \emph{time} (e.g.\ physical or logical) to be either explicitly (i.e.\ \emph{time-aware}) or implicitly (i.e.\ \emph{time-situated}) available''% or, not at all
\end{quote}
%
By ``time-aware'', we mean that the functionality requires access to time-related information, either to make it directly observable to consumers of such functionality or to work properly. 
For instance, time series forecasting (a form of prediction) needs an explicit representation of time to be available, hence it is time-aware. 
By ``time-situated'', instead, we mean that the functionality has some dependency on any one notion of time, but such dependency need not be explicitly captured and managed. 
For instance, real-time monitoring and control of an asset surely depend on time, but such a dependency need not be modelled to deliver the functionality. 
Simply, its existence suffices for the functionality to work properly.
There are also some other cases where time does not matter from the designer's standpoint, as time is not an issue when implementing the functionality---e.g., sending commands to an actuator device. 
This whole spectrum is depicted in Figure~\ref{fig:timing}. 

\begin{figure}[!t]
    \centering
    \includegraphics[width=.6\columnwidth]{figures/dt-mas/timing-spectrum.pdf}
    \caption{The spectrum of \emph{timing}: from intelligent functionalities requiring explicit modelling of physical or logical time, to those that don't care.}
    \label{fig:timing}
\end{figure}

Of course, placing intelligent functionalities within this spectrum cannot be a precise operation, as with the other principles. 
But some general considerations about the categories of intelligent functionalities devised in Section~\ref{ssec:functions} can and should be made to aid system designers. 
%
For example, prediction, simulation, and diagnosis are likely to require time awareness. 
How far into the future should an event or state of interest be predicted? 
How long into the future should a simulation extend? 
Conversely, when did a chain of faults lead to the present faulty situation? 
%
Other functionalities, such as inference, learning, and adaptation, may be less demanding and simply be time-situated. 
Of course adaptation actions are based on some events that happened, or on some predicted future state, and need to be carried out in the future, but possibly they do not require an explicit representation of time to work properly as simply their sequential ordering is sufficient to eventually react accordingly. 

%======================================================
\subsection{AAs and DTs as first-class abstractions}
\label{ssec:abs-def}
%======================================================

%\ste{Tutti}{da un lato ci sono le funzionalità, dall'altro le due astrazioni, i nostri princpipi dovrebbero stare in mezzo e tracciare il percorso dalla funzionalità all'astrazione (come quei giochini della sala giochi dove metti le monete in 1 foro in alto e quelle si incanalano in tanti fori in basso, ma rovesciata---le funzionalità sono più fori, le astrazioni meno)}

In the previous subsection, we propose our design principles guided by the intelligent functionalities that are described in the related literature discussed in Section~\ref{sec:back}. 
Now we match such principles against AAs and DTs, to conceptually assess which abstraction is best suited for which end of the spectrum described above for a given principle. 
Figure~\ref{fig:principles} shows the exemplary functionalities and how they are better encapsulated by AAs or DTs according to our principles. 

\begin{figure}[!t]
    \centering
    \includegraphics[width=\columnwidth]{figures/dt-mas/principles-aa-dt.pdf}
    \caption{Exemplary intelligent functionalities placed within the spectrum of our design principles. AAs and DTs are clustering those functionalities that they better match with, according to such principles.}
    \label{fig:principles}
    %\Description{A graph with three corresponding to the principles and the main categories of functionalities placed within it, two dotted lines represent an area for which DTs or AAs are more suitable}
\end{figure}

Let us start by considering how AAs and DTs have been used in the \emph{specificity} design principle. 
%\begin{itemize}
%\item 
AAs, being by definition \emph{goal-driven} entities, are usually exploited to encapsulate intelligent functionalities that are strictly related to the problem to solve, hence to the domain and objectives of the application~\cite{DBLP:books/daglib/0077762}. 
%\item 
DTs, instead, are usually exploited with the scope of digitalising assets to provide data-oriented services to consumer applications, \emph{servitisation} being one of their core characteristics~\cite{the-digita-twin-Crespi-2023}. 
%\end{itemize}
The principle of specificity, thus, would encourage designers to encapsulate intelligent functionalities closely matching application goals in AAs, and, complementarily, to model general-purpose services as DTs instead. 

Consider now the \emph{scoping} design principle. 
%\begin{itemize}
%\item 
AAs are not tied to any particular source of information in achieving their goals. 
Although part of the definition of AA includes a relationship with the notion of an environment in which AAs are \emph{situated}~\cite{DBLP:journals/ker/WooldridgeJ95}, AAs are not limited in any particular way to any given ``object'' in such an environment. 
In other words, the application environment (physical or digital) in its entirety is a source of information that AAs can perceive and (possibly) affect---their \emph{global} scope. 
%
%\item 
DTs, instead, are specifically defined as being \emph{coupled} with the individual asset in the physical world that they are digitally representing (their physical twin). 
It is thus natural to give DTs a \emph{local} scope, limited to the information available from this physical asset. 
%\end{itemize}
Note that we deliberately choose the term ``scoping'' instead of, for instance, ``locality'', to avoid the misunderstanding that such design principle is inherently tied with some notion of space, such as in a geographical sense or limited by a physical size boundary under which something can be considered local or not.
%
%We simply mean that there should be a clear boundary in terms of how many physical assets are involved with the intelligent functionality that designers are considering. 
%If an individual asset is the source of information for that functionality, it could be suited to be encapsulated by a DT. 
%Otherwise, if the information needed belongs to multiple sources, an AA might be more conceptually aligned.
%%
%However, the boundary of an individual asset can extend to be a rather large entity.
%%
%In a smart home, for instance, we can envision an asset to be an individual device, a room, or the whole house even. 
%Also, nothing prevents an asset from being a composition of multiple others that have a smaller local boundary. 

Finally, let us position AAs and DTs with respect to the \emph{timing} design principle. 
%\begin{itemize}
%\item 
AAs do not include any explicit notion of time in their definition. 
The situatedness feature already mentioned includes the temporal dimension but does not prescribe agents to capture and model time-related aspects. 
Hence AAs are \emph{time-situated} entities, naturally. 
%\item 
DTs, complementarily, are specifically requested to keep in synch %(shadow)
their related PA to provide an updated digital representation in a timely manner. 
This implies explicitly capturing the time at which events in the PA happened, and modelling the temporal evolution of the PAs. %---such as in threading. 
DTs are thus naturally \emph{time-aware}. 
%\end{itemize}

Summarising the above considerations, AAs and DTs can be described as playing the following roles in the design of distributed intelligent functionalities in IoT systems and applications:
\begin{itemize}
    \item \textbf{AAs} are the components of the IoT system that encapsulate the system/application \textbf{goals}, hence are aware of such specific goals and strive to achieve them by autonomously deciding \textbf{whose other entity} to interact in the whole application domain. 
    AAs are also not particularly tied to any specific notion of time. 
    \item \textbf{DTs}, instead, model and encapsulate properties, behaviours, and functions of specific portions of the IoT system or application domain, therefore are \textbf{inherently bound to specific assets}, to provide \textbf{general-purpose services} to other components or directly to the application. 
    Due to such modelling, DTs must at least account for the specific \textbf{notion of time} relevant to the modelled asset. 
\end{itemize}
%We devise these definitions by observing what are the features that are closer to how Digital Twins and agents have been used about the principles that we identified before and when both have been used for a given dimension we decided to distinguish them to make them complementary rather than keeping them overlapped.
%
We emphasise that these definitions are not meant to set crisp boundaries amongst AAs and DTs in any possible use case and scenario. 
%
Consequently, there is a degree of flexibility in how these definitions can be interpreted to distribute responsibilities across different components.
%
However, having well-defined criteria that dictate how to exploit AAs and DTs as complementary abstractions helps to identify responsibilities in the design of an IoT system willing to fully leverage distributed intelligence.
%
%Furthermore, complementarity is key to leveraging the defining features of each component for the definition of complex behaviour.

To better illustrate this flexibility, the next subsection describes some architectural solutions that can arise from these design principles to deliver different intelligent functionalities in the IoT. 
Furthermore, Section~\ref{sec:case-study} discusses a practical case study, applying the principles to analyse the system requirements.


%======================================================
\subsection{From ``macro'' to ``micro'' architectures}
\label{ssec:multi-layer}
%======================================================

%\ste{Ste}{Besides, what is the architecture of AA? In my view, the architecture of AA is the so-called FCBPSS architecture of a system, see the literature \cite{DBLP:journals/eis/WangWDFKIZ16,DBLP:journals/eis/ZhangW16}. These literatures take a system perspective to a broad manufacturing system, at enterprise level, shopfloor level, etc.
%The authors may want to discuss the modeling of AA with FCBPSS in the future work.}

% \ste{Ste}{1. ora dai principi deriviamo possibili architetture a seconda delle funzionalità attese; 
    % 2. prima 1:1 poi contemplando MAS e multi-DT; 
    % 3. poi chiariamo che ogni ``blocco'' architetturale (lettera) può essere composto a piacere; 
    % 4. da qualche parte ci posizioniamo rispetto a lavori EMAS e WoDT (loro guardano al sistema top-down, noi bottom-up)}

The proposed design principles and their match against AAs and DTs help decide whether an AA or a DT is best suited for a given intelligent functionality. 
However, they tell little about how to combine AAs and DTs \emph{synergistically} when intelligent functionality requires it. 
For instance, because it is positioned within the spectrum of each principle in a way that does not perfectly match all the features of either an AA or a DT, but some of both. 
Accordingly, this section discusses the several ``micro-architectures'' for AAs and DTs integration that may arise during the application of our principles---while the next section provides a practical example for a selection of them. 
We use the term \emph{micro-architecture} ($\mu$-arch, for short) to emphasise an important distinction our architectural perspective has concerning the related literature, discussed below.

The goal of clarifying integration architectures is also witnessed in recent literature, although from different perspectives.
% 
Reference~\cite{DBLP:conf/atal/MarianiPR22} enumerates different alternative architectures to integrate AAs and DTs, depending on the goal of the integration itself. 
%
In~\cite{DBLP:conf/emas/MarianiPR23} the goal is to achieve a technical integration between specific technologies to increase the level of abstraction toward the set of concepts most familiar to the developers of AA. 
%
Another effort, the \emph{Web of Digital Twins} (WoDT) vision~\cite{10.1145/3507909}, 
sees DTs as entities interlinked in a \emph{web of semantic, dynamic relationships}, 
working as the digital substrate that enables structuring a dynamic application domain.
%
According to the WoDT vision, 
a layer of networked DTs works as the interface between applications
(either agent-oriented or not)
and the physical environment they must cope with, 
thus decoupling the two layers and possibly providing augmented and cross-domain functionalities.

Common to these and other architectural proposals in the literature~\cite{iiot_dt_architectural_aspects,guinard2010resource,laghari-2016,Guth2016,Cavalcante2015} is the \emph{top-down}, ``macro-architecture'' perspective adopted: what is proposed is a reference architecture for a whole system, a blueprint to adopt in any given IoT scenario (and beyond).
%
For instance, in the field of enterprise systems architecture, the rise of IoT and the relevance of CPSs in general as the ``backbone'' of many intelligent functionalities within the company (that require asset monitoring and control first and foremost), nurtured research in new designs and methodologies supporting IoT-related goals. 
There, the general trend is to adopt a \emph{System of Systems} (SoS) perspective over the whole enterprise, decomposing the requirements along the same application-agnostic dimensions. 
The FCBPSS architecture is an example \cite{DBLP:journals/eis/WangWDFKIZ16}, where the many systems supporting operations of an enterprise are seen in terms of Functions, Context, Behaviour, Principles (guiding the design), Structure (components realising the function and their relations), and State. 
These very same concepts are applied recursively for the whole enterprise integration system---indeed, a SoS.
%
In contrast, here we propose a bottom-up approach, where the architecture of the overall system is the result of the composition of multiple $\mu$-arch, depicted in Figure~\ref{fig:architecture} by making use of the following components: 
%
%\begin{itemize}
%\item 
Physical Assets (``PA'' squares), which represent the entities in the real world (e.g. devices, objects, people, processes, organisations) that the IoT system needs to model; 
%\item 
DTs (``DT'' circles); 
%\item 
AAs (``A'' circles); 
%\item 
and the intelligent functionalities (the 3D boxes), which may be entire applications or simply one of the many functions needed for the application at hand. 
%\end{itemize}
\begin{figure}[!t]
    \centering
    \includegraphics[width=\columnwidth]{figures/dt-mas/2024-toit-si-architecture-aa-dt.pdf}
    \caption{Architectural perspective for the synergistic combination of AAs and DTs in distributing intelligence in IoT systems. Each letter denotes a specific \emph{micro-architecture} that may arise from the application of our principles. Such architectures can be combined to give shape to the overall system architecture, in a bottom-up way.}
    \label{fig:architecture}
\end{figure}
%
The possible $\mu$-archs are: 
%
\begin{itemize}
    \item $\mu$-arch \textbf{(A)} is the most common in the related literature already mentioned~\cite{DBLP:conf/atal/MarianiPR22,DBLP:conf/emas/MarianiPR23,10.1145/3507909}, and it is widely used as a reference architecture for the integration of AAs and DTs~\cite{DBLP:conf/atal/MarianiPR22}. 
    There, the intelligent functionality to realise neither perfectly matches an AA nor a DT. 
    For instance, it could have high specificity, require time-awareness, and a scope locally extended to a set of related PAs. 
    Hence, PAs are digitalised by DTs, that offer services to AAs encapsulating the intelligent functionalities directly serving the applications' goals. 
    %
    This is the typical $\mu$-arch that arises when a given intelligent functionality can be decomposed into an application-specific part and a PA-specific part~\cite{DBLP:conf/atal/MarianiPR22}. 
    The former can request to complement the PA-related information with external data, and directly serve the application goals---hence is better encapsulated by an AA. 
    The latter may have temporal constraints on the PA and can be reused across domains---hence is better modelled by a DT. 
    \item $\mu$-arch \textbf{(B)} encompasses instead an edge case in which the intelligent functionality can be directly mapped onto a DT: it has low specificity, its scope is limited to a given PA (or cohesive set thereof), and time-awareness is required. 
    In this case, the AA abstraction is unnecessary as the nature of DTs makes them suitable to deliver the functionality alone. 
    \item $\mu$-arch \textbf{(C)}
    is quite common in AAs literature, and follows the ``agentification'' paradigm~\cite{PicoValencia2018,Savaglio2020}: wrapping any relevant object as an agent, to model the domain as a Multi-Agent System (MAS)---physical devices and objects included. 
    Again this is an edge case that needs to be treated with care as it can lead to undesired coupling. We argue that this is legit when the goal of the agent is specific to the asset and the scope of its decision-making is strictly localized. Such a goal should not be to serve other application components with a digital representation of the asset of course -- that would be the true nature of a DT instead -- but can, for instance, represent a closed feedback loop on the asset itself. Similarly, whenever the scope of the functionality becomes ``larger'' such as in the case of coordination with other agents, we argue that $\mu$-arch \textbf{(A)} is more effective in decoupling the interaction with other agents from the control on the asset through the corresponding DT. We observe then, that with the rising complexity of a system this approach usually often degenerates in either \textbf{(A)} or \textbf{(E)}.
    %We argue that such a micro-architecture is only actually legit if the PA is already digitalised: for instance, a legacy sub-system. 
    %In that case, however, we emphasise that probably the entity of interest does not need a DT because it is not a real-world object, but only exists in the computational world (indeed, a legacy software system, a database, etc.). 
    \item $\mu$-arch \textbf{(D)} is one we actually argue against. 
    Such an architecture is common, for instance, in the literature about agent-based DTs~\cite{AlelaimatGD20,s21041096}. 
    Also, MAS-based DTs~\cite{WAN2021880,Pretel2024} fall in this case, if more AAs are used. 
    The issue here is that the mirroring, or shadowing, of the PA by the DT seemingly has to go through an agent. 
    This is likely to cause issues, such as delays, in the process itself~\cite{calvaresi2017challenge}. 
    Moreover, as stated before, it should not be the responsibility of an AA to digitalise a PA. 
    We argue that a better replacement for a combined solution considering AAs and DTs is depicted in $\mu$-arch \textbf{(E)}, described below. 
    \item $\mu$-arch \textbf{(E)} simply expresses in a slightly, but impactful, different way the real intent behind $\mu$-arch \textbf{(D)}: \emph{augmenting}, enhancing the capabilities of a DT. 
    Thus, we argue that a better depiction of this need is the one where the DT still is responsible for digitalising the PA, but interacts with an agent in the process to implement or expose augmented capabilities to the consumer components.
    Differently from \textbf{(A)} the agent in this pattern \textit{disappears} within the DT and the other components of the system can not interact with it directly. This is in line with the definition of a DT that externally is seen as the component that mirrors a PA while adding specific functionalities to it, but internally keeps separation of concerns between the digitalisation process and the other functionalities that are better captured with agents (e.g., decision-making, planning, simulation).
    
    %This could correspond, for instance, to an intelligent functionality whose scope is bound to a given PA (or a limited set thereof), that may be generally useful for many application goals, but that may need information from third-party sources to fully provide its services. 
\end{itemize}
%
Most of these $\mu$-archs can be trivially extended to the multi-AAs or the multi-DTs case, where not one but multiple AAs or DTs are used. 
These cases are exemplified at the bottom of Figure~\ref{fig:architecture}. 
Essentially, every time a single AA or DT appears in a $\mu$-arch, that AA or DT can be extended to be a multi-AAs / multi-DTs subsystem. 
An example already used is that of the DT of a complex structured PA (a room, a whole building), which can be obtained by suitably composing multiple DTs in a hierarchy~\cite{Jia2022}.

Finally, we emphasise that these $\mu$-archs can be arbitrarily composed in ``recursive structures'' to give shape to complex IoT systems~\cite{WAN2021880}. 
Let us assume, for instance, that a complex DT is set up to digitalise a whole hospital ward. 
The DT is the composition of multiple DTs, each shadowing a specific room, each in turn shadowing every single equipment. 
Some of these DTs can provide general purpose services and can be assisted by AAs according to $\mu$-arch \textbf{(E)} in doing this. 
Then, these DTs may be exploited by a multitude of AAs as per $\mu$-arch \textbf{(A)} extended to the multi-agent case, each exploiting their general purpose services while providing their application-specific functions. 
%
By iterating this recursive composition of $\mu$-archs, the overall IoT system architecture emerges in a bottom-up way as dictated by the nature of the intelligent functionalities themselves, not as imposed by a reference architecture---which may be unable to capture the specifics of each different domain and application. 

The next section illustrates how both our proposed design principles and the resulting $\mu$-archs can be applied in a practical case study in the domain of smart manufacturing.


%-------------------------------------------------------
\section{Agents in \aclp{DTE}}
%-------------------------------------------------------

%-------------------------------------------------------
\section{Towards Cognitive \aclp{DT}}
%-------------------------------------------------------



%%%%%%%%%%%%%%%%%%%%%%%%%%%%%%%%%%%%%%%%%%%%%%%%%%%%%%%
\section{Exemplary Application of the Design Principles}
\label{sec:case-study}
%%%%%%%%%%%%%%%%%%%%%%%%%%%%%%%%%%%%%%%%%%%%%%%%%%%%%%%

%\ste{Marco, Samu}{bisognerebbe usare le classi di funzionalità definite nella 2.4 in qualche modo, altrimenti le sezioni sono slegate}

In this section, we exemplify the application of our proposed design principles and the resulting micro- and macro-architectures, in the case of a smart manufacturing scenario.
For the sake of exemplifying the design process
we can imagine having a goal of implementing IoT systems capable of dynamically optimising production about the overall energy consumption of a manufacturing plant.
%
This goal can be achieved by designing a distributed intelligence system based on a combination of AAs and DTs that implement a set of high-level functionalities (which we recall from Section \ref{ssec:functions}) that include monitoring, prediction, and planning for dynamic task allocation, and adaptation to external factors.

We unfold the design process by first analysing the functionalities from a domain-related perspective, decomposing the main goal into more fine-grained functionalities that can be directly implemented. 
Then we apply the principles described in Section \ref{sec:abstractions-and-principles}, to identify the main characteristics of such functionalities and finally decide on the combination of AAs and DTs accordingly.
%
In doing so, we provide a guiding interpretation of those principles when confronted with real-world problems and requirements.
%
We believe that this further validates how having such common abstractions in the form of AAs and DTs can contribute to the design of complex distributed intelligence systems.

\begin{figure*}
    \centering
    \includegraphics[width=\columnwidth]{figures/dt-mas/smart_manufacturing_scenario.pdf}
    \caption{A reference Smart Manufacturing scenario with multiple physical entities and a set of applications interested in implementing intelligent functionalities on top of the physical deployment.}
    \label{fig:smart_manufacturing_scenario}
\end{figure*}

%======================================================
\subsection{A Smart Manufacturing Energy Optimisation Scenario}
%======================================================

A typical production system distributes a diverse array of \textit{machinery}, \textit{operators}, and \textit{processes} throughout the shop-floor environment (as schematically represented in Figure \ref{fig:smart_manufacturing_scenario}). 
%These elements can be classified into various categories based on their nature and usage. For instance, there are feeding-material systems responsible for managing material input, transformation systems such as production machines or specialized equipment for processing, output material systems for managing material output, and various material handling equipment including robotic arms and industrial manipulators interacting with industrial operators during different phases.
%
These individual components are often organised into \textit{production nodes}, which serve as logical groupings of related machines.
Several nodes make up the overall production plant.

Industrial energy monitoring \cite{Ageed2021ASO, Cai2022ARO} and optimisation \cite{Hussain2021SmartAI} is a challenging task, especially in a distributed environment. 
There, such tasks might involve several processes, to selectively turn on different nodes when needed, depending on the external information about the energy cost, while keeping the production speed at an acceptable rate.
%
Given the hierarchical organization of the plant and the cyber-physical distribution of the system, this problem calls for a DI approach.

%We here sketch the main high-level requirements for a Smart Manufacturing scenario targeting energy optimization at the plant level.
As summarised in Figure \ref{fig:target_functions}, we decompose the main target of production optimisation into three objectives that could be strategic in the target scenario and that represent a useful reference to analyse and apply the design principles presented and proposed in this article.

We identify three main functionalities that concur to dynamically adjust production based on energy costs, to keep the overall consumption under a threshold. 
The IoT system must be able to collect information about the current energy consumption and needs of each production node through some form of \emph{energy monitoring}. Furthermore, to be able to dynamically adjust the loads in the production nodes, some form of \emph{intelligent coordination} is required to continuously optimise the utilisation of resources. Finally, the overall \emph{production optimisation} techniques are required to make high-level decisions that incorporate data from both individual production nodes and external sources.
These functionalities are an exemplification of decomposition that can be performed in the design process of an IoT system. In the following, we analyse each functionality in more detail and guide this analysis towards the definition of components that can implement the IoT system employing either DTs or AAs or a combination of the two.

\begin{figure*}
    \centering
    \includegraphics[width=\columnwidth]{figures/dt-mas/target_functions.pdf}
    \caption{Three main objectives that concur to the energy optimization of a Smart Manufacturing plant: Energy Monitoring at the machine level, Intelligent Coordination between robots and operators, and overall Production Optimization based on external factors (e.g. energy cost).}
    \label{fig:target_functions}
\end{figure*}

\paragraph{Energy Monitoring}
To make dynamic adjustments to the production process based on the energy consumption of the whole plant it is essential to be able to monitor it in real-time through smart metres embedded in the machines.
Furthermore, it is essential to collect data from each machine or production node to have fine-grained control on what parts of the plant are consuming the most energy.
%
Monitoring can also include some forms of inference to determine whether the observed consumption is considered regular or is an anomaly, and potentially take corrective actions in such cases to improve safety.
%
Finally, prediction or simulation can be used to forecast what the energy consumption is expected to be, given some production tasks, to improve decision-making.

\paragraph{Intelligent Coordination}
This objective involves the use of data collected from the IoT system to introduce coordination capabilities to improve both performance and safety at different levels of deployment (e.g., machine-to-machine or machine-to-operator interactions).
%
For example, machine learning models can be trained to detect patterns and, by continuously monitoring and analysing data streams from machines and operators, the system can automatically understand the context of the currently performed activities.
%
This is essential to understand the best planning policy to schedule the next tasks as soon as machines or operators are free to perform them.
%
%Furthermore, digital interfaces are required to interact with both robots and humans to assign tasks.
%
%For example, if an operator working is stressed or its fatigue level is anomaly increased, the coordination (intelligent) functionality can dynamically defer the assigned task or assign it to other operators.
%
For example, if a portion of the plant is to be shut down for energy saving, the tasks that were queued on those machines should be redistributed across the other active production nodes of the plant.

\paragraph{Production Optimisation}
Digital applications can combine real-time data from the physical world and external sources (e.g., market price and production demand) to dynamically optimise manufacturing processes together with all machines, operators, raw materials, and products.
%
This requires decision-making driven by inference of constraints, as well as means to monitor the current situation to adapt according to the real-time state of the system.
%
For this scenario, we assume that the market price of energy is provided by an external forecasting system. Based on that and according to target business rules, the digital layer can optimise production schedules, resource allocation, and task assignment to maximise efficiency and reduce costs.

\medskip
%The design and implementation of these outlined functionalities, along with other intelligent capabilities, must confront the reality of cyber-physical complexity. Bidirectional interaction with physical entities, their data, and actions poses a critical challenge for applications, necessitating delegation to an intermediate structured and interoperable digital level. This layer is tasked with decoupling these responsibilities from high-level applications.
In the subsequent section, we explore how AAs and DTs can be synergistically exploited to construct a cyber-physical intelligent abstraction layer following the design principles presented in Section \ref{ssec:principles}. %This layer aims to maximise interoperability, simplify interaction with physical entities, and facilitate the adoption of intelligent capabilities.

%======================================================
\subsection{System Design with AAs and DTs}
%======================================================
\label{ssec:system_design_aas_dts}

\note{TODO TABLE}
% \newcolumntype{P}[1]{>{\centering\arraybackslash}p{#1}}
% \newcolumntype{M}[1]{>{\centering\arraybackslash}m{#1}}
% \begin{table*}[!t]
%     \centering
%     \scriptsize
%     \begin{tabular}{ p{1.5cm} | p{2cm} p{1.5cm} P{1cm} P{1cm} P{1cm} P{1cm} P{1cm} }
%         \hline
%         \textbf{Overall Objective} & \textbf{Functionalities} & \textbf{Kind} & \textbf{Specif.} & \textbf{Scoping} & \textbf{Timing} & \textbf{Abst.} & \textbf{$\mu$-arch}\\
%         \hline
%         \multirow{3}{2cm}{Energy\\Monitoring} 
%         & Data collection & n.a. & General & Local & Explicit & DT & \multirow{3}*{E} \\ 
%         & \makecell[l]{Anomaly\\Detection} & \makecell[l]{Inference/\\Prediction} & General & Local & Explicit & DT & \\ 
%         & \makecell[l]{Corrective\\Reaction} & \makecell[l]{Inference/\\Planning} & Specific & Local & Implicit & Agent & \\
%         \hline
%         \multirow{3}{2cm}{Intelligent\\Coordination} 
%         & \makecell[l]{Activity\\Monitoring} & \makecell[l]{Inference/\\Prediction} & General & Local & Explicit & DT & \multirow{3}*{A}\\ 
%         & \makecell[l]{Dynamic\\Planning} & Planning & Specific & Global & Explicit & Agent &\\ 
%         & \makecell[l]{Task\\Delegation} & n.a. & General & Local & Implicit & DT &\\ 
%         \hline
%         \multirow{3}{2cm}{Production\\Optimization} 
%         & Decision-making & Inference & Specific & Global & Explicit & Agent & \multirow{3}*{A}\\ 
%         & \makecell[l]{Process\\Monitoring} & Inference & General & Global & Explicit & DT & \\ 
%         & \makecell[l]{Task\\Delegation} & n.a. & General & Local & Implicit & DT & \\ 
%         \hline
%     \end{tabular}
%     \caption{Mapping of design principles to the target use case and broad categories of intelligent functionalities (\rev{Specif.\ = Specificity, Abst.\ = Abstraction}, n.a.\ = not applicable, pure digitalization function).}
%     \label{tab:usecase_principles_mapping}
% \end{table*}


Table \ref{tab:usecase_principles_mapping} reports the mapping between the intelligence functionalities envisioned, the application of the design principles, and the associated micro-architectural design. % following the analysis with respect to the \textit{Specificity}, \textit{Scoping} and \textit{Timing} principles.
%
We decompose each overall objective into specific functionalities and associate them with the main \textit{ kinds} of intelligent functionalities identified in Section \ref{ssec:functions}.
%
The analysis leads to the adoption of an AA, a DT, or a combination thereof, for functionality, and a different $\mu$-arch for each objective (as shown in Figure \ref{fig:dt_agents_zoom}), choosing from those presented in Section \ref{ssec:multi-layer}.

In the following, we discuss the reasoning behind each choice, providing an illustrative interpretation of the principles for the different functionalities in the scenario presented above. 

\paragraph{Energy Monitoring}

For this objective, we identify the basic functionality of data collection about the energy consumption of a given machine. 
It has low specificity as the very same data required for the monitoring could easily and likely be used for other, future tasks (e.g., prediction of consumption). 
%
Of course, as the data belong (is both generated and consumed by) to a specific PA, the functionality is \textit{local} concerning the Scoping principle (actually, individual), 
%
and we consider an explicit time representation needed for this task to generate a time series. 

We also expect to have an anomaly detection functionality that can analyse time series data and detect unexpected patterns. 
The knowledge required for such a task is still local to the machine, and the results can be useful to general services as they could notify interested parties of a malfunction. 

Finally, we envision an on-the-fly corrective reaction mechanism that is introduced with the specific purpose of shutting down the machine if there is a problem. 
This is a specific behaviour introduced for our main monitoring application goal (hence, with application goal specificity), which still uses only the local data and can act on an event-driven trigger that does not necessarily need an explicit time representation. 

From this analysis, we can reliably identify the need for a machine DT, capable of collecting data and detecting anomalies, and of an AA implementing the emergency shutdown policy. 
This leads to $\mu$-arch \textbf{(E)} described in figure \ref{fig:architecture} as the possible reference implementation for this functionality. 

\begin{figure*}
    \centering
    \includegraphics[width=\columnwidth]{figures/dt-mas/dt_agents_zoom.pdf}
    \caption{Three  illustrative instances showcasing the synergistic usage of AAs and DTs as guided by the principles and $\mu$-archs, within the Smart Manufacturing context.}
    \label{fig:dt_agents_zoom}
\end{figure*}

\paragraph{Intelligent Coordination}

The data necessary for this functionality comes from an activity monitoring task, which has \emph{local scope} on the PA being mirrored (e.g. an operator) to infer the activity currently executed and possibly predict the expected duration (thus, we an explicit representation of time needed, here). 
%
The function would be assigned from dynamic planning performed by another component of the system, after receiving the aggregated global knowledge (hence global scope) from each entity involved in the coordination scenario.

Accordingly, we may expect that each PA would have a DT that mirrors them and encapsulates the activity monitoring and task delegation functionalities, tailoring them to the mirrored entity. 
Planning, instead could be performed by an AA keeping track of the global context (e.g.\ of a whole production node) and applying a specific coordination policy to redirect tasks to different machines (hence, a functionality with goal specificity and global scope, justifying preference for an AA). 
This leads to the adoption of $\mu$-arch \textbf{(A)}, with a single agent using multiple DTs as its data sources and delegating tasks to each.

\paragraph{Production Optimisation}

Optimisation at the plant level will be guided by a decision-making functionality based on two sources of data: external data about the market price of electricity for a specific time and date
and data collected locally from the different production nodes. %that will need to feed into a global process monitoring functionality. 
%
Depending on the cost of electricity, for instance, the policy may decide to shut down the nodes that are not actively involved in the ongoing process, delegating tasks to different nodes.

Accordingly, $\mu$-arch \textbf{(A)} is the best suited in this scenario, involving a DT that mirrors the production process as the main data source, and an AA enacting the policy and acting on the production nodes.
%
It is worth noting that, having already introduced the DTs of the different production nodes, we can also consider linking the DT of the production process to the DTs of the nodes, in a composition pattern that would facilitate the modelling of the production process DT as represented in Figure \ref{fig:dt_agents_zoom}.

%======================================================
\subsection{Discussion of Resulting Architecture}
%======================================================

%In this section, we delve into the analysis of the integrated architecture and design choices concerning the use of DTs and Agents in the target manufacturing scenario, examining their respective advantages and disadvantages with respect also to the application of the envisioned design principles and the distribution of intelligent functionalities in the use case.

\begin{figure}[!b]
    \centering
    \includegraphics[width=\columnwidth]{figures/dt-mas/dt_agents_smart_manufacturing.pdf}
    \caption{A layered architecture with the combination of DTs and Agents for a Smart Manufacturing reference scenario.}
    \label{fig:dt_agents_smart_manufacturing}
\end{figure}

Figure \ref{fig:dt_agents_smart_manufacturing} schematically illustrates the integrated architecture, using both AAs and DTs, resulting from the application of our proposed design principles and the consequential composition of our $\mu$-archs. 
%The integrated architecture is emerging from the micro-patterns identified for each of the main objectives combined and as such leverages the strengths of both abstractions to create a robust, flexible, and intelligent manufacturing system where high-level intelligent functionalities are implemented using DTs and Agents according to their requirements and following principles and analysis presented in Section \ref{ssec:principles} and Section \ref{ssec:system_design_aas_dts}. 

The physical world, composed of multiple heterogeneous entities such as machines, operators, raw inputs, and products, can be effectively digitalised through the use of DTs.
These DTs are responsible for representing the associated PAs in cyberspace through interoperable and homogeneous digital replicas, operating locally to the devices and associated just to their context without the need for a global scope.
This approach can also be applied to industrial processes, such as product transformation from raw inputs to final products, the interaction and collaboration between machines and operators, etc.

DTs excel in abstracting and decoupling PAs and processes into modular digital representations, simplifying interaction and integration. They support standardized communication protocols, enhancing interoperability across various platforms and systems, that is critical for integrated manufacturing operations. DTs can be composed of higher-level aggregates, providing a holistic view of the system, which is crucial for understanding interdependencies and optimising system-wide processes.
%
Moreover, DTs can embed \emph{localized intelligence}, enabling real-time monitoring, anomaly detection, and predictive maintenance, while reducing latency and improving responsiveness.

On the other hand, AAs are designed to operate autonomously, making decisions based on predefined rules or learned behaviours, which allows them to adapt to changing conditions and respond effectively to real-time events. They excel in tasks requiring dynamic coordination and optimization, such as reallocating resources and adjusting schedules based on real-time data and system conditions. Additionally, AAs can be easily scaled and deployed across different parts of the manufacturing system, providing a flexible solution that can grow with the system's needs.
%
In the envisioned and reference smart manufacturing example, AAs can be distributed across the entire deployment spectrum. They support both local inference and decision-making within a DT, allowing it to achieve its internal target goals, and broader spectrum operations enabling dynamic coordination throughout the production line. This includes adapting planning and task allocation in response to variations in the production line context or external market demand. 
%In an intermediate approach, within a composed DT (e.g., operator supervisor), Agents enable the rebalancing of operator schedules and optimize the interaction among machines and operators.

AAs and DTs also benefit significantly from each other, in a perfect synergy. 
AAs benefit of DTs decoupling of complexity when interacting with the physical world and the exploitation of a uniform and interoperable digital representation of it. Conversely, AAs bring intelligence to the system, operating either internally as components of the DT's model (both for single and composed twins), or on a broader scope utilising the different DTs to collect information and act to improve overall performance and behaviour.
%
An additional benefit of such an integrated approach that combines both DTs and AAs is that the components can evolve separately.
%
Policies can change at the AA level (even online, via learning) without having to modify the DTs. 
Similarly, new and improved DT models could be produced as data is collected about the system, improving support given to the AA decision-making, without changing the AA at all. %but without requiring any kind of modification as the entities in the system are all loosely coupled. 

As a final remark, emphasising modularity, it is worth noting that in this example the decision-making is all performed by AAs. This means that in case of need (e.g.\ some critical failure or an expected contingency), to switch the system to a \textit{``manual mode''} it is sufficient to momentarily stop/disconnect the AAs. 
Then, they can be temporarily replaced by human operators to amend the system as needed, while still benefiting from the real-time data collected by DTs that can continue to operate. 

%Interaction mechanisms ensure bidirectional communication between DTs and Agents, enabling informed decision-making and action execution. The integration of DTs and Agents can significantly improve operational efficiency by enabling real-time monitoring, predictive maintenance, and dynamic optimization. Scalability is achieved through the proposed distributed architecture that allows for the addition of new DTs and Agents without disrupting existing operations. Flexibility ensures adaptability to changing conditions and requirements, maintaining resilience and responsiveness. 

All of these desirable non-functional properties do not come without costs: the next Section discuss the challenges that researchers and practitioners need to deal with to effectively adopt AAs and DTs according to our proposed principles and $\mu$-archs. %challenges include system integration, resource scaling, and initial deployment complexity, requiring careful planning and optimization strategies to overcome.
%The next Section discusses some of these.
%In conclusion, the combination of DTs and Agents in smart manufacturing provides a powerful framework for addressing the complexity and dynamic nature of modern production systems.
%Despite challenges, this integrated approach enhances operational efficiency, flexibility, and scalability, paving the way for future advancements in smart manufacturing.

%The integration of intelligent functionalities into smart manufacturing systems empowers organizations to achieve greater agility, responsiveness, and competitiveness in today's dynamic market landscape. These capabilities not only enhance operational efficiency and productivity but also elevate safety, quality, and sustainability throughout the entire manufacturing value chain. However, the design and implementation of these functionalities, along with other intelligent capabilities, must confront the reality of cyber-physical complexity. Bidirectional interaction with physical entities, their data, and actions poses a critical challenge for applications, necessitating delegation to an intermediate structured and interoperable digital level. This layer is tasked with decoupling these responsibilities from high-level applications. DTs and Agents can address these functionalities, constructing a new cyber-physical intelligent abstraction layer following the design principles presented in Section \ref{ssec:principles}. This layer aims to maximize interoperability, simplify interaction with physical entities, and facilitate the adoption of intelligent capabilities.

%In this context, interoperability is also another pivotal aspect of DT technology associated with the target operational scope of the physical asset, facilitating seamless communication and integration with various digital and physical protocols. This interoperability fosters collaboration and data exchange across different platforms, enhancing the efficiency and effectiveness of manufacturing processes. In terms of interaction with applications or services, DTs streamline communication through standardized protocols, enabling seamless integration with external systems. Applications and Agents can interact with DTs by requesting modifications to the state of affairs, pushing configurations and plans, and accessing information about physical assets and their capabilities through the DT structure. This capability results in strategy in combination with composition through all the greenified function categories since it enables a simplified and structured data collection from the DT and an effective actionability enabling agents to return to the physical world to change behaviours, coordinate and optimize processes.

%However, their integration with existing systems and protocols can be complex, requiring careful design and robust communication mechanisms to ensure seamless interaction with DTs and other system components.
%Autonomous agents can also be resource-intensive, particularly if they perform complex computations or require significant data exchange, potentially straining network and computational resources.

%The bottom layer consists of physical assets, while the middle layer is populated with DTs representing these assets, providing a digital abstraction that captures real-time data and historical performance. The top layer consists of Agents that utilize data from DTs to perform intelligent functions such as monitoring, coordination, and optimization.



%%%%%%%%%%%%%%%%%%%%%%%%%%%%%%%%%%%%%%%%%%%%%%%%%%%%%%%%
\chapter{Engineering \acs{BDI} \aclp{MAS}}
\label{chap:mas:engineering}
%%%%%%%%%%%%%%%%%%%%%%%%%%%%%%%%%%%%%%%%%%%%%%%%%%%%%%%%

This chapter presents an overview of contributions in the area of engineering \ac{BDI} \ac{MAS} targeting the goal outlined in \ref{rq:5} of making \ac{BDI} \ac{MAS} engineering more accessible for developers wishing to integrate \ac{BDI} agents into larger software systems, there including \acp{DTE} and more broadly Web-based systems.
%
Additionally, the chapter explores the role of \emph{explainability} in \ac{BDI} agents, to make agent behavior more transparent to users, as this is a key requirement for intelligent applications applied to critical domains such as healthcare (\Cref{sec:back:h40:explainable-ai}).


%=======================================================
\section{\acs{BDI} Agents in \aclp{DTE}}
\label{sec:mas:engineering:bdi-dt}
%=======================================================

The previous chapter presented a general alignment between the abstractions of autonomous agents and \acp{DT},
highlighting the complementary nature of the two paradigms, and the potential benefits of their integration.
%
This section explores how \ac{BDI} agents,
as a specific subset of autonomous agents,
can be relevant in the context of \acp{DT} in general,
and specifically in relationship with the proposal of this thesis of \acp{DTE} and their implementation as hypermedia systems.

The specific application of \ac{BDI} agents is considered in this thesis alongside three main aspects: 
\begin{itemize}
    
    \item \textbf{Controllability of Autonomous Behavior}:
    \ac{BDI} agents belong to the class of \emph{programmed} agents
    and have been historically tied to the design of programming languages and frameworks,
    that allow developers to specify and implement agent behavior which could exhibit a degree of autonomy,
    interpreted as ability to adapt such behavior depending on contextual information.
    This makes \ac{BDI} agents particularly suitable for scenarios where
    developers need to encode business processes and domain logic 
    in a clear and controllable manner. 
    
    \item \textbf{Expressiveness of Cognitive Abstractions}:
    \ac{BDI} agents provide high-level cognitive abstractions to encode autonomous behavior.
    The declarative nature of \ac{BDI} specifications allows developers to focus on defining system-level goals
    and strategies to achieve them explicitly, rather than detailing low-level control flows.
    \ac{BDI} agents further reason on an explicit semantic model in form of beliefs, that can encode knowledge about a specific domain
    and encapsulate contextual information relevant to the agent's operation.
    
    \item \textbf{Explainability of Agent Behavior}:
    \ac{BDI} agents offer inherent explainability features,
    as their internal state and decision-making processes are explicitly represented through beliefs, desires, and intentions
    which can be inspected and queried.
    This transparency is crucial in applications where understanding the rationale behind agent actions is important for user trust and system validation.

\end{itemize}

These features make \ac{BDI} agents a compelling choice for implementing autonomous components within the context of \acp{DTE}.
%
Namely, this thesis studies and contributes to the engineering of \ac{BDI} \ac{MAS} envisioning their application in two main scenarios, outlined in the following.

\paragraph{\ac{BDI} Agents for \emph{Cognitive} \aclp{DT}:}
\Cref{sec:back:dt:ai} introduced the concept of \emph{cognitive} \acp{DT} as a specific class of \acp{DT} that leverage \ac{AI} techniques to enhance their ability to autonomously and proactively adapt to changes in the physical counterpart in a goal-driven fashion.
%
This thesis envisions \ac{BDI} agents as a suitable technology to implement such autonomous components within cognitive \acp{DT}, through the mechanism of \emph{augmentation} outlined in \Cref{sec:dte:engineering-dt:dt-augmentation}.
%
This would effectively implement the micro-architecture \textbf{(E)} presented in \Cref{sec:mas+dt:patterns}.
%
In this context, \ac{BDI} agents can encapsulate the cognitive capabilities of the \ac{DT}, and be encapsulated as an asynchronous augmentation function within a \ac{DT}, that can access the \ac{DT} state, observe its evolution over time, and suggest actions to adapt the \ac{DT} behavior. 
%
To this end, the features of \ac{BDI} agents outlined above, are particularly relevant:
\begin{inlinelist}
\item \emph{controllability} is necessary to ensure that the autonomous behavior is aligned with the domain requirements and constraints;
\item \emph{cognitive abstractions} can be an effective way to interact with the structured and semantic-rich model of the \ac{DT} digital representation;
\item \emph{explainability} is crucial to audit and understand the decisions made by the autonomous components within the \ac{DT}.
\end{inlinelist}

\paragraph{\ac{BDI} Agents as consumers of the \acl{HWoDT}:}
In the original vision of the \ac{WoDT}, software agents were considered as primary consumers of the \ac{WoDT} digital representation, leveraging its structured and semantic-rich model to reason about the physical counterpart and make decisions accordingly~\cite{web-of-dt-ricci-2022}
%
This aligns with the \ac{HWoDT} proposed in this thesis, that can serve as a shared hypermedia environment for \ac{MAS} that follow the vision of \ac{hMAS} introduced in \Cref{sec:back:mas:web}.
%
\acp{DT} can serve the purpose of simplifying the interaction of agents with the physical world, and enhance the reasoning capabilities of agents by providing rich models of the \ac{PA} that can be used also to simulate possible future states hence allowing an agent to perform \emph{what-if} analysis of its actions~\cite{DBLP:conf/eumas/Burattini23}.
%
Micro-architecture \textbf{(A)} presented in \Cref{sec:mas+dt:patterns} envisions agents interacting with \acp{DT} following this pattern.
%
When scaling to ecosystems, the \ac{HWoDT} provides a set of high-level interaction mechanisms that can be exploited by agents to interact with the whole set of \acp{DT} in a uniform manner.
%
\ac{BDI} agents can be particularly suitable for this scenario to encode business logic that spans multiple \acp{DT}, such as coordinating the behavior of different \acp{DT}, or implementing complex workflows that involve interactions among multiple \acp{DT}.
%
In this context, the features of \ac{BDI} agents outlined above are again particularly relevant:
\begin{inlinelist}
\item \emph{controllability} is necessary as the interactions of multiple \acp{DT} must be predictable; 
\item \emph{cognitive abstractions} can help in reasoning about complex interactions among multiple \acp{DT} and express clear system-level goals;
\item \emph{explainability} is crucial to audit and understand the decisions made by agents, especially when they interact with multiple \acp{DT} and potentially affect critical operations.
\end{inlinelist}

\medskip

The integration of \ac{BDI} agents within \acp{DTE} with the goals and strategy outlined above is a long-term research direction of this thesis, especially relevant for the domain of healthcare, in which the automation of complex processes involving autonomous agents is seen as a major opportunity to enhance the quality of care and reduce costs~\cite{Croatti_Gabellini_Montagna_Ricci_2020}.
%
Accordingly, the rest of this chapter focuses on short-term actions aimed at enhancing the engineering of \ac{BDI} \ac{MAS} in general, 
with actions tailored to achieve a more robust engineering process, as well as facilitate their integration with external systems such as \acp{DTE} in the future.
%
The potential integration of \ac{BDI} agents with \acp{DTE} further opens up interesting perspectives on how to better align the knowledge representation of \ac{BDI} agents with the semantic models used in \acp{DT}, how to enhance adaptability in dynamic environments such as the \ac{HWoDT}, including enhancing \ac{BDI} agents with future-oriented reasoning capabilities that may leverage the simulation capabilities of \acp{DT}, and, finally, how to enhance the explainability of \ac{BDI} agents so that the autonomous behavior is transparent to both system developers and end-users.

%=======================================================
\section{Tooling for \acs{BDI} Agent Programming}
\label{sec:mas:engineering:jakta}
%=======================================================

\ac{BDI}~\cite{rao1991modeling} agents are considered
an excellent tool for modelling autonomous or intelligent entities
via high-level \emph{cognitive} abstractions.
%
Therefore, in the early 2000s, the expectation of the community was for \ac{AOP}~\cite{Shoham_1993} to gain its spot among other prominent paradigms,
such as object-oriented (\acs{OOP}), functional (\acs{FP}), imperative (\acs{IP}), and logic (\acs{LP}) programming.

However,
a few decades later,
it can be observed that,
although \ac{BDI} has deeply impacted the research field of \ac{MAS} and \ac{AI},
it still fails at reaching mainstream programming,
even in contexts where the application scenario would make it
a good choice to design autonomous behavior.

Limited adoption of \ac{BDI} in mainstream programming has been widely discussed~\cite{lind2000aose,DBLP:journals/sigsoft/MascardiWR19,DBLP:journals/ijaose/DignumD10,DBLP:books/sp/14/Muller014,DBLP:journals/corr/abs-1209-1428}, with no consensus on the root causes.
%
Some argue that improved tooling is needed~\cite{DBLP:conf/dalt/Hindriks14},
while others suggest that the main barrier is the cost of learning a new paradigm~\cite{DBLP:journals/ijaose/Logan18} without the gain of significant benefits in terms of engineering complex behavior, claiming that new features are needed to make \ac{BDI} more appealing.

These perspectives are not necessarily conflicting, but rather arguably complementary: paradigm adoption depends on both effective abstractions and supportive tools.
%
In \cite{DBLP:journals/ijaose/Logan18} \cpp is mentioned as an example of language that despite little tooling support at its inception, gained popularity thanks to the introduction of new abstractions (e.g., classes) that made it more appealing to mainstream developers.
%
It could be argued that it was the seamless, gradual integration of such new abstractions into existing \ac{IP} practices that made \cpp successful.
%
As seen in the evolution of \cpp and Java, such gradual integration of new abstractions can lower adoption barriers and foster community-driven innovation.
%
Thus, paradigm and tooling evolution should proceed in parallel to facilitate broader adoption, possibly by integrating \ac{BDI} concepts into existing mainstream programming practices, paradigms, software ecosystems, and tools.

%-------------------------------------------------------
\subsection{State of the Art}
%-------------------------------------------------------

Several open-source, publicly available, and actively maintained\footnote{Following the definition in~\cite{Cardoso_Ferrando_2021}.} frameworks exist for \ac{BDI} \ac{AOP} programming.
This section highlights representative frameworks, based on recent surveys of logic-based agent-oriented technologies~\cite{lptech4mas-jaamas35} and agent-based programming for \ac{MAS}~\cite{Cardoso_Ferrando_2021}.

\Cref{tab:frameworks} summarizes key characteristic of selected frameworks.
%
Other notable actively developed frameworks that did not meet the selection criteria but still deserve mention are
\jack{}~\cite{Winikoff2005} (not open-source),
the aforementioned \jacamo{}~\cite{Boissier_Bordini_Hübner_Ricci_Santi_2013} frameworks which extends \jason{} (\Cref{sec:back:mas:aose}),
\sarl{}~\cite{iat2014sarl} (not \ac{BDI}-specific),
and \goal{}~\cite{Hindriks2009} (not \ac{BDI}-adhering).

This short overview highlights how existing frameworks differ in terms of intended target applications, execution platforms and agent programming syntax.
%
A more in-depth analysis of the state-of-the-art frameworks is presented in \cite{DBLP:journals/sncs/BaiardiBCP24}, 
focusing on their features and limitations.
%
From this analysis, it is possible to observe that
most frameworks are build with custom \ac{DSL} syntaxes that require learning new languages, as well as dedicated supporting tools (e.g., IDE plugins) that require additional effort to set up and maintain.
%
This also hinders integration with existing software ecosystems and tools, as well as reuse of existing libraries and components, as the mechanisms to interface with external code are often limited and not always straightforward to use, requiring developers to shift their mindset away from familiar programming practices.

\begin{table}
    \centering
    \small
    \resizebox{\textwidth}{!}{
    \begin{tabular}{r|c|c|c}
        \textbf{Framework } & \textbf{Platform} & \textbf{Syntax} & \textbf{Intended Target} \\\hline\hline
        \textbf{\makecell[r]{\astra~\cite{CollierRL15}} }
        & JVM
        & custom DSL
        & distributed systems
        \\
        \textbf{\makecell[r]{\gwendolen~\cite{dennis2008gwendolen}} } 
        & \makecell[c]{MCAPL~\cite{Dennis2018}}
        & \agentspeak{}-based
        & \ac{BDI} verification
        \\
        \textbf{\makecell[r]{\jadex~\cite{PokahrBL2005}} } 
        & JVM
        & Java + annotations
        & distributed systems
        \\
        \textbf{\makecell[r]{\jason~\cite{Bordini_Hübner_Wooldridge_2007}} } 
        & JVM
        & \agentspeak{}
        & \ac{BDI} research
        \\
        \textbf{\makecell[r]{\jsson~\cite{DBLP:conf/emas/KampikN19}} } 
        & JavaScript
        & JavaScript
        & Web applications
        \\
        \textbf{\makecell[r]{\lightjason~\cite{aschermann2016eumas}} } 
        & JVM
        & \agentspeak{}-based
        & scalable applications
        \\
        \textbf{\makecell[r]{\phidias~\cite{DUrsoLS19}} } 
        & (Micro)Python
        & Python DSL
        & IoT/robotic applications
        \\
        \textbf{\makecell[r]{\spadebdi~\cite{PalancaRCJT22}} } 
        & Python
        & \agentspeak{}
        & distributed systems
        \\
        \hline
        \textbf{\makecell[r]{\jakta~\cite{DBLP:journals/sncs/BaiardiBCP24}} } 
        & JVM*
        & Kotlin DSL
        & general-purpose
        \\
    \end{tabular}
    }
    \normalsize
    \bigskip
    \caption{
        A selection of active \ac{BDI} programming frameworks
        and their respective execution platforms, syntax and intended target.
        \jakta{}, the framework proposed in this section, is included to compare with existing frameworks.
        *\jakta{} is currently implemented on the JVM 
        but could be ported to other platforms thanks to Kotlin's multiplatform capabilities.
    }
    \label{tab:frameworks}
\end{table}


%-------------------------------------------------------
\subsection{Desiderata for Mainstream \acs{BDI} Tooling}
%-------------------------------------------------------

As emerging from the previous section,
the \ac{BDI} \ac{AOP} community has primarily developed toolkits
as libraries, extensions, or entirely custom languages
supported by existing execution platforms.
%
This approach provides a practical environment for prototyping and deployment,
allowing developers to leverage the features of the \emph{host} platform.
%
Hence,
a \ac{BDI} framework typically expects users to work in a top-down manner,
approaching system design as a \ac{MAS} and using \ac{BDI} abstractions to model agents,
while leveraging the host language primarily for implementing vertical, low-level, practical details.

Although this may be effective for \ac{BDI} experts
it may pose a barrier for newcomers.
%
An effective way to learn a new paradigm is to start with a simple yet functional fragment,
then incrementally expand its functionalities,
gradually introducing new concepts and abstractions (example-based learning, see~\cite{vanGog2010}).


Based on these considerations and the analysis of existing frameworks, the following desiderata for mainstream \ac{BDI} tooling are identified:

\paragraph{Interoperability with Mainstream Paradigms:} to have \ac{BDI} work as a \emph{practical} paradigm it is necessary to integrate with external systems that typically are not built using \ac{BDI} concepts, which results in a developer to likely incorporate other paradigms in parts of the developed system.
%
Many modern programming languages (including Kotlin, Python, Scala, etc.) are designed to be \emph{multi-paradigm}, allowing developers to switch among paradigms with relative ease.
%
This approach, known as \emph{paradigm blending},
prioritizes integration of multiple paradigms in one code fragment
over composition of multiple fragments using different paradigms.
%
When paradigms are blended,
the same code fragment may contain \emph{syntactical}
constructs capturing abstractions of diverse paradigms.
%
In this context, \ac{BDI} \ac{AOP} could be envisioned
as yet another paradigm to support,
letting developers free to use a mix of different abstractions in the same codebase.


\paragraph{Code Reuse and Sharing Mechanisms:}

Code reuse is fundamental in software engineering for reducing redundancy and improving maintainability. General-purpose languages support reuse through constructs such as packages, modules, classes, and functions, complemented by import and extension mechanisms.
%
This is important for both \emph{in-project} reuse, but especially for \emph{cross-project} reuse, enabling the creation and sharing of libraries and frameworks. These are usually supported by build automation and dependency management tools that facilitate integration and versioning.
%
In \ac{AOP}, similar mechanisms would enable modularization and reuse of agent specifications. For \ac{BDI} \ac{MAS}, this could involve sharing plans, beliefs, or rules across agents. However, there is no standardized approach for modularity and reuse at the paradigm level.
%
When a framework provides a \ac{FFI}, reuse mechanisms from the host language can be applied to relevant code sections. For example, \jason{} and \spadebdi{} allow internal actions to be defined and reused as Java or Python code, respectively.
%
For code written in the \ac{BDI} language, reuse is often limited to file inclusion (e.g., like in \jason{}). Some like \astra{}, support specification extension, while others, such as \jack{} and \jadex{}, use capabilities~\cite{DBLP:conf/promas/BraubachPL05} to encapsulate reusable components, though interpretations vary.


\paragraph{Development Tools:}
Development tools are critical for effective \ac{BDI} programming.
Modern development environments offer features such as syntax highlighting, code completion, static analysis, and refactoring, which are now expected for a swift development experience.
%
Testing libraries and debugging support are also essential for building reliable systems. However, most \ac{BDI} frameworks lag behind mainstream languages in tool maturity, often due to the high maintenance cost for small communities and limited incentives in tool development in research-driven projects.
%
Developing and maintaining compatibility of plugins for target development environments is challenging.
%
This can be alleviated by developing \emph{internal} \acp{DSL} which can directly leverage existing tools, though sometimes at the expense of language expressiveness.
%
Testing and debugging \acp{MAS} remain complex. \ac{BDI} frameworks offer limited support, and recent efforts~\cite{DBLP:conf/atal/RodriguezTW23,amaral2023atal} are still maturing. Additionally, debugging through host language tools is possible but often exposes low-level details irrelevant to \ac{BDI} abstractions.
%
Simulation is another important tool, especially for distributed systems. Existing \ac{MABS} frameworks are not tailored for \ac{BDI} agents, leading to an abstraction gap and duplicated effort when implementing both production and simulation environments. Ideally, frameworks should facilitate seamless switching between real and simulated deployments as discussed in \Cref{sec:mas:engineering:simulation}

\paragraph{Configurability:}
\ac{BDI} frameworks also manage the runtime environment for agents, including input/output, communication, and concurrency models (\Cref{ssec:mas:engineering:concurrency}). Configurability in these aspects is essential for adapting frameworks to diverse application requirements.
%
The concurrency model, mapping the agent control flow to execution primitives (e.g., threads, processes, event loops), directly impacts efficiency, determinism, and reproducibility and affects system performance. Most frameworks support one or more concurrency models, but few offer user-oriented APIs for easily customizing or selecting the concurrency model.
%
Agent communication is another key aspect, with frameworks differing in their support for local and distributed messaging. Communication typically relies on agent communication languages (ACLs)~\cite{Kone_Shimazu_Nakajima_2000} and underlying network protocols.
%
Pluggable communication mechanisms, using standard protocols, are considered good practice, allowing users to select or extend protocols as needed. However, multi-protocol support and well-documented APIs for communication configuration remain uncommon.
%
Overall, high configurability in concurrency and communication is a desirable property for mainstream \ac{BDI} tooling, enabling broader applicability and easier integration with existing systems.

%-------------------------------------------------------
\subsection{The \jakta{} Framework}
%-------------------------------------------------------

\jakta{} is a new \ac{BDI} programming framework,
aimed at (gradually) addressing the issues discussed in the sections above.
%
\jakta{} is available on GitHub\footnote{\url{https://github.com/jakta-bdi/jakta}}
under a free, open-source, and permissive licence.
%
The main design principles of the tool are \emph{modularity},
\emph{pluggability} and
\emph{paradigm interoperability}.

\jakta{} clearly separates:
\begin{inlinelist}
    \item definition of concepts (e.g., agents, environment, plans, actions),
    \item the logic that rules concepts' behavior (e.g., execution loops, selection functions),
    and
    \item how these are presented to the final user (e.g., syntax, \ac{API}).
\end{inlinelist}
%
Multi-paradigm interoperability is obtained by designing \jakta{} as an internal \ac{DSL} in Kotlin,
thus leveraging Kotlin's multi-paradigm features that already support \ac{OOP}, \ac{FP}, and \ac{IP} styles.

\begin{figure}[tb]
    \centering
    \includegraphics[width=\textwidth]{figures/mas/jakta_modules.pdf}
    \caption{
        The main modules composing the \jakta{} framework and their dependencies.
    }
    \label{fig:jakta-modules}
\end{figure}

The \jakta{} framework is composed of four main modules (\Cref{fig:jakta-modules}), namely:
%
\begin{enumerate}
    \item the \textbf{\ac{BDI} interpreter}, which governs the execution of agents in environments,
    regardless of the particular syntax used to define them;
    \item the \textbf{\ac{DSL}}, which provides a Kotlin syntax to idiomatically define \ac{BDI} \acp{MAS};
    \item the \textbf{concurrency manager}, which regulates runtime, concurrency, and scheduling aspects for
    any system run by the \ac{BDI} interpreter;
    and
    \item the \textbf{\alchemist{} incarnation},
    which bridges the \ac{BDI} interpreter with the \alchemist{}~\cite{PianiniJOS2013} discrete-event simulator.
\end{enumerate}

A detailed description of the \jakta{} syntax is available in \cite{DBLP:journals/sncs/BaiardiBCP24}, for the interested reader.
%
Here some examples of \jakta{} code are reported to showcase the main features of the framework.

\noindent
\begin{minipage}{\linewidth}
\lstinputlisting[
    % float,
    basicstyle=\footnotesize\ttfamily,
    linewidth=\linewidth,
    language=Kotlin,
    caption={Overall structure of a \ac{MAS} specification in \jakta{}.},
    %{Entrypoint of \jakta{}, it contains the description of the Environment and Agents involved in the application.},
    label=lst:mas,
]{listings/jakta-examples/mas_dsl.kt}
\end{minipage}
%
Users can define and launch a \jakta{} \ac{MAS} specification using a \texttt{mas} block (\Cref{lst:mas}).
All elements composing the \ac{MAS} are defined within it, including the (initial) set of agents, the environment model and other \ac{MAS} configuration elements.
%
The \texttt{agent} block in \Cref{lst:mas} defines an individual \ac{BDI} agent,
specifying its initial beliefs and goals, as well as the plan library it will use at runtime.
%
The syntax is valid Kotlin code, which implements at the language level features to support the construction of \acp{DSL} (e.g., lambda with receiver, operator overloading, block-like lambdas, etc.) that can be implemented as \emph{type-safe builders}\footnote{\url{https://kotlinlang.org/docs/type-safe-builders.html}}.
%
This makes the \ac{MAS} specification fully supported by existing Kotlin development tools.

\noindent
\begin{minipage}{\linewidth}
\lstinputlisting[
    % float,
    basicstyle=\scriptsize\ttfamily,
    linewidth=\textwidth,
    language=Kotlin,
    caption={Example of paradigm-blending in \jakta{}. The code snippet uses together \ac{OOP}, \ac{FP}, and \ac{IP} constructs to create a \ac{MAS}.},
    label={lst:blending},
]{listings/jakta-examples/blending.kt}
\end{minipage}
%
% Additionally,
The adoption of a multi-paradigm language such as Kotlin
natively exposes \ac{OOP}, \ac{FP}, and \ac{IP} constructs within the \ac{BDI} \ac{DSL}.
This supports a first level of paradigm blending, which allows composing and parametrizing the \ac{MAS} specification with ease.
%
The blending is exemplified in \Cref{lst:blending},
it describes, using \jakta{} syntax,
a toy example defining a \ac{MAS} with ten agents named as the first ten athletes scraped from a web page.
%
The example showcase the combination of the \ac{OOP} paradigm employed for dealing with regular expressions matching and data extraction,
and the \ac{FP} paradigm used to map URLs to athletes names, and finally to \jakta{} agents.

Further paradigm integration can be leveraged when defining actions.
%
In \jakta{}, a valid action is any instance of the \texttt{Action} interface
(more precisely, of one of its specializations \texttt{InternalAction} or \texttt{ExternalAction}),
and can be written in-place directly inside a \texttt{mas} definition,
possibly using \ac{OOP} or \ac{FP} constructs directly in Kotlin.

\jakta{} has an explicit notion of \emph{environment}. The \jakta{} environment is responsible for:
%
\begin{inlinelist}
    \item tracking the agents in the system;
    \item governing each agent's perception by determining which percepts to deliver;
    \item governing each agent's actuation by making external actions available;
    \item governing communication among agents by deciding how messages are delivered;
    \item enabling stigmergic interaction by making the environment's data accessible to agents.
\end{inlinelist}
%
The basic implementation of the environment acting as a shared data container can be extended to implement more complex functionalities, integrate external systems within the \ac{MAS}, and support distributed communication. 


%=======================================================
\section{Testing \acs{MAS} with Simulation}
\label{sec:mas:engineering:simulation}
%=======================================================

Aside from being useful for in-silico studies,
\emph{simulation} may also aid the \emph{development} and \emph{validation}~\cite{uhrmacher_simulation_2002}
of \acp{MAS} intended for real-world deployment.
%
In fact, simulation enables developers to test the behavior of agents
and their dynamics in complex environments ahead of deployment,
while postponing costs, risks, and efforts associated with real-world execution.

The price for such flexibility, however,
is paid in additional development effort:
first and foremost, to model the deployment context in a simulation environment,
and, additionally,
to maintain alignment between the two versions of the \ac{MAS} codebase,
unless the same \ac{MAS} specification can execute \emph{with no changes}
on both real hardware and a simulator of choice.

This is particularly challenging when dealing with \ac{BDI} \acp{MAS}.
There,
the abstraction gap between the high-level cognitive model of \ac{BDI} agents and most simulators~\cite{singh_integrating_2016},
along with the minimal support of \ac{BDI} technologies
for producing \emph{reproducible} simulations of articulated scenarios~\cite{kehoe2016robust}
lead developers to resort to one of the following approaches~\cite{singh_integrating_2016}:
%
\begin{enumerate}
    \item extension of the \ac{BDI} platform with a dedicated simulation engine,
    which requires additional development effort (e.g., \cite{HubnerB09,ricci_exploiting_2020});
    \item construction and maintenance of two parallel codebases
    one for a simulator extended with \ac{BDI} modeling tools (e.g., \cite{sakellariou_enhancing_2008,TaillandierBCAG16,uhrmacher1998agents}) and one for the actual system,
    leading to consistency issues;
    \item integration of the \ac{BDI} platform with a general-purpose simulation engine,
    typically by synchronizing their execution through some form of middleware (e.g., \cite{singh_integrating_2016,davoust_architecture_2020}).
\end{enumerate}

Despite the idea of using simulation to support \ac{MAS} development being not new
there are not many tools that effectively allow one
\emph{
    ``to execute agents as they are and to switch arbitrarily between execution in the real environment and the virtual test environment''
}~\cite{uhrmacher_simulation_2002}.

This section discusses the challenges of decoupling \ac{BDI} agent concurrency when designing \ac{MAS} frameworks, as a necessary step towards enabling the same specification to run both in real concurrent deployments and in discrete-event simulation environments.
%
The proposed integration of \jakta{} with the \alchemist{} simulator explores the mapping of the \ac{BDI} control loop, showing that different granularities are possible, to address the challenge of providing a seamless transition between real and simulated environments for \ac{BDI} \acp{MAS}.

%-------------------------------------------------------
\subsection{Decoupling Concurrency in BDI MAS}
\label{ssec:mas:engineering:concurrency}
%-------------------------------------------------------

The analysis of a selection of technologies from table \Cref{tab:frameworks} presented in \cite{DBLP:conf/emas/BaiardiBCPRO24} suggests that most \ac{BDI} frameworks tightly couple the \ac{BDI} control loop with the underlying \emph{concurrency model}. 
%
This means that the way in which agents are scheduled and executed is often hardcoded into the framework, limiting flexibility and adaptability to different application requirements.

Several concurrency models are possible for \ac{BDI} \ac{MAS}: 
\begin{itemize}
    \item \emph{\ac{1A1T}}: is the most common where each agent runs in a dedicated thread. This model provides true parallelism but can lead to high resource consumption. 
    \item \emph{\ac{AA1T}}: the degenerate case where all agents run in a single thread, leading to sequential execution (e.g. round-robin scheduling), agents do not run in parallel, but the system is lightweight and easier to debug.
    \item \emph{\ac{AA1EL}}: at the conceptual level is similar to \ac{AA1T}, but instead of using threads, it employs an event loop to manage agent execution. This model is efficient and suitable for handling blocking operations in the agent logic.
    \item \emph{\ac{AA1E}}: similarly to \ac{AA1EL} but there is a shared executor allowing for concurrent execution of agents while still benefiting from the efficiency of event-driven programming. Scaling the executor allows for tuning the level of resource usage and parallelism.
\end{itemize}

Decoupling the specification of the adopted concurrency model from the \ac{BDI} control loop allows to select the most appropriate model for different applications.
%
This requires defining clear interfaces between the \ac{BDI} interpreter and the concurrency manager when designing \ac{BDI} frameworks.
%
Each agent control-loop phase (e.g., perception, deliberation, action) should be treated as a task that can be scheduled on a concurrency abstraction.
%
This decoupling can also serve the purpose of switching between real and simulated execution, as the simulator would act as yet another \emph{executor} of \ac{BDI} tasks.

%-------------------------------------------------------
\subsection{Discrete-Event Simulation for BDI}
%-------------------------------------------------------

\ac{DES} is a powerful technique for modeling complex systems, that can support the scheduling of events over simulated time, and advance the simulation clock to when the next event is due.
%
Given the fact that \ac{BDI} agents can be seen as driven by events (e.g., percepts, messages, internal triggers), \ac{DES} appears as a natural fit for simulating \ac{BDI} \acp{MAS}. 
%
\ac{DES} allows for a fine-grained control over the timing and ordering of events, making it possible to accurately model the asynchronous and concurrent nature of agent interactions.

The integration of a \ac{BDI} \ac{MAS} over a \ac{DES} simulator require mapping the \ac{BDI} control loop iterations onto the simulator's event scheduling.
%
Different granularities of mapping are possible, each with its own trade-offs:
\begin{itemize}
\item \emph{\ac{AMA}}: the entire \ac{MAS} advances in lockstep, with all agents performing one iteration of the control loop in a sequential order at every step. This is the coarsest granularity, usually adopted when simulating \ac{BDI} agents for debugging purposes (e.g., \cite{HubnerB09}).
\item \emph{\ac{ACLI}}: each full iteration of the agent's control loop is treated as a single event. Interleave of agent actions is possible, but the agent execution is treated as atomic steps and cannot model the duration of individual phases.
\item \emph{\ac{ACLP}}: each phase of the control loop (perception, deliberation, action) is mapped to separate events. This allows for more detailed modeling of agent behavior and interactions. All possible interleaving are possible, like in a true concurrent system and hence providing the most accurate simulation of real-world execution.
\item \emph{\ac{ABE}}: each event in the \ac{BDI} agent (e.g., perception, message receipt, internal trigger) is mapped to a discrete event in the simulator. This allows for fine-grained simulation but may increase the number of scheduled events significantly, potentially impacting performance of the simulation, while not providing significant benefits over \ac{ACLP} in most scenarios as the behavior of agents encapsulate its internal events.
\end{itemize}

Other than choosing the appropriate granularity for the mapping, switching between real and simulated execution requires careful handling of time management and \emph{external actions} that are invoked on the environment.

In real execution, time can be perceived from the system clock, while in simulation, time is controlled by the simulator.
%
Therefore, agents must be designed to use a time abstraction that can be adapted to both real and simulated contexts.
%
The same goes for randomness, which should be abstracted to allow for reproducible simulations controlled by the same \emph{seed}. 

For what concerns environment actions, modeling the environment is crucial for accurate simulation. 
%
The agent should be decoupled by the specific implementation of the environment, relying on an interface that can be implemented differently for real and simulated contexts.
%
The simulated environment should mimic the behavior of the real environment as closely as possible, including the generation of perceptions, the effects of actions that modify the environment state and the communication mechanisms among agents which may need to be tested for reliability under different conditions. 

The design of \jakta{} has been influenced by these considerations.
%
This made it possible to integrate \jakta{} with the \alchemist{} \ac{DES} simulator~\cite{PianiniJOS2013}.
%
The integration is achieved by letting \alchemist{} schedule the execution of \jakta{} agents by scheduling their control loop iterations and phases with a time distribution.
%


%-------------------------------------------------------
\subsection{Experimental Demonstrator}
%-------------------------------------------------------

The prototype implementation demonstrates the feasibility of the approach, and showcases how different granularities of mapping can lead to different simulation behaviors with the same \ac{MAS} specification.

To exemplify this, a simple \ac{MAS} is defined to simulate a leader \ac{UAV} moving in a circular path, while other \acp{UAV}
must follow its movement retaining their formation around the leader.

\Cref{fig:code} shows code extracts from the scenario implementation, demonstrating that agent logic is reusable across different execution platforms.

\begin{figure}[tb]
    \begin{minipage}[b]{.6\linewidth}
        \centering
        \lstinputlisting[
            nolol,
            language=Kotlin,
            linewidth=\linewidth,
            basicstyle=\scriptsize\ttfamily,
            label={lst:logic},
    %        caption={Platform-agnostic code}
        ]{listings/jakta+alchemist/agentsLogic.kt}
    \end{minipage}
    \hfill
    \begin{minipage}[b]{.34\linewidth}
        \centering
        \lstinputlisting[
            nolol,
            language=Kotlin,
            linewidth=\linewidth,
            basicstyle=\scriptsize\ttfamily,
%        caption={Entrypoint},
            label={lst:agents}
        ]{listings/jakta+alchemist/agents.kt}
    \end{minipage}
    \caption{
        The agent specifications (left) are completely platform-agnostic and reusable,
        some glue code (right) wires the logics with the underlying execution platform.
    }
    \label{fig:code}
\end{figure}


To investigate the effect of the mapping granularity on the system's behavior, experiments are conducted with the \ac{AMA}, \ac{ACLI}, and \ac{ACLP} granularities.
%
To do so, \ac{AMA} is used as baseline,
letting the entire \ac{MAS} run a full cycle every simulated second.
%
Then the baseline is compared with \ac{ACLI} and \ac{ACLP},
for which each agent is modeled with an execution frequency $f$
following a Weibull distribution with mean $f$ and deviation $f\cdot\tau$ modeling the \emph{relative drift} in the internal clock of different devices.
%
Additionally, for the \ac{ACLP} granularity deliberation and action delays are modeled,
associating each phase with an exponential distribution with rate $\lambda = f$:
faster agents (larger $f$ values) have less delay.
%
Every experiment is repeated $100$ times with a different random seed, changing
the initial positions of the followers
and the distribution in time of the events for the \ac{ACLI} and \ac{ACLP}.

\begin{figure}[tb]
    \centering
    \includegraphics[width=\linewidth]{figures/jakta+alchemist/error_over_time_flattened.pdf}
    \caption{
        Average error with time for different granularities, with \ac{ACLI} and \ac{ACLP} running at $f=1Hz$ and $f=2Hz$.
        Different charts show different values of relative drift $\tau$.
        Coloured shadows represent $\pm 1\sigma$ over multiple runs.
    }
    \label{subfig:err_time}
\end{figure}
%
% \begin{figure}
%     \centering
%     \includegraphics[width=\linewidth]{figures/jakta+alchemist/error_over_variances.pdf}
%     \caption{
%         Mean squared distance error with relative drift ($\tau$), measured for different frequencies for \ac{ACLI} and \ac{ACLP}.
%         Coloured shadows represent $\pm 1\sigma$.
%     }
%     \label{subfig:err_drift}
% \end{figure}
\Cref{subfig:err_time} shows the error on the followers position with respect to the ideal formation over time as the mean square distance of all followers \acp{UAV}.
As expected given the simplistic implementation of the agent's logic, the \ac{AMA} granularity appears to be the most stable, converging quickly to a low error.
%
Differently from \ac{AMA}, both \ac{ACLI} and \ac{ACLP} instead, show a degradation in performance at low frequencies and higher relative drifts.
%
This is due to the fact that the implementation assumed implicit synchronization among agents and virtually no-delay in the perception-action loop (which is the case in \ac{AMA}), but fail to adapt fast enough when the execution is asynchronous and delayed, similarly to real-world deployments.

This showcase demonstrates that selecting fine-grained granularities for mapping the \ac{BDI} control loop into \ac{DES} events is necessary to accurately simulate real-world deployments, thus supporting the argument that the \ac{BDI} framework should expose the ability to finely decouple the execution of individual phases of the control loop from the concurrency model.

%=======================================================
\section{BDI Agents in Hypermedia Environments}
%=======================================================

As discussed in \Cref{sec:back:mas:web}, the Web is increasingly being seen as a \emph{hypermedia environment} where autonomous agents can operate and interact with resources through \ac{REST} interactions, and by consuming semantic descriptions of resources and services.

Even more recently,
with the advent of \ac{LLM}-based \emph{Agentic AI}~\cite{Acharya_Kuppan_Divya_2025}
the creation of \ac{LLM}-driven agents that interact with Web APIs~\cite{10.5555/3692070.3692540}
(or directly with Websites~\cite{10.5555/3692070.3694608})
has gained significant attention.
%
However,
despite being promising,
these approaches
do not make structured knowledge representation obsolete~\cite{pan2024tkde},
and often trade the controllability offered by traditional agent-programming paradigms
for more guarantees in terms of correctness, and completeness.

Arguably,
Web-integrated agents developed with cognitive architectures,
such as \ac{BDI} agents,
would retain controllability and predictability:
they could directly exploit hypermedia to discover new resources and actions,
and use Semantic Web technologies like \ac{RDF} and \ac{OWL}
to act intelligently without compromising on quality.
%
However,
it can be argued that
the integration of \ac{BDI} agents with the Web
is hindered by a conceptual and technological gap,
which limits developers in creating agents that need to interact with (Semantic) hypermedia.
%
More precisely:
\begin{itemize}
  \item\label{gap:logic}
  \ac{BDI} agents are historically tied to \ac{LP}
  %-- e.g., with the popular \agentspeak{} semantic~\cite{DBLP:conf/maamaw/Rao96} --
  whereas the Semantic Web is grounded on description logic,
  and although the two paradigms are theoretically compatible,
  their practical implementations interoperate poorly,
  increasing integration cost significantly;
  % The different technological choices used to concretely implement them add cognitive load to developers working on the integration of the two worlds (e.g., converting \ac{OWL} and \ac{RDF} triples in Prolog-like formulas);

  \item\label{gap:open-world}
  \ac{BDI} agents are typically based on the \ac{PRS} architecture~\cite{georgeff1987reactive}
  which relies on pre-defined plans and does not natively support discovering new actions at runtime,
  a necessary feature to interact proficiently with hypermedia environments.
\end{itemize}

In \cite{burattini2025gap}, a detailed analysis of these gap is presented, reflecting on the nuances of different models and concrete syntaxes used for logic representation, query semantics and inference mechanisms.
%
The analysis results in the definition of a set of requirements that can guide future research on bridging the gap between \ac{BDI} agents and hypermedia environments.
%
For a deeper integration of \ac{BDI} agents with semantic hypermedia environments, the following requirements should be addressed:
\begin{enumerate}[label=\textbf{(R\arabic*)}]
  \item direct manipulation of \ac{RDF} and \ac{OWL} triples in the agent's belief base;
  \label{req:direct}

  \item ontological inference for deliberation (e.g., plan selection and execution);
  \label{req:reasoning}

  \item support for querying the belief base via \acs{SPARQL};
  \label{req:query}

  \item ability to exploit affordances discovered in the environment to dynamically adapt the agents' plans.
  \label{req:actions}
\end{enumerate}

%--------------------------------------------------------------
\subsection{A Generalized BDI Engine}
%--------------------------------------------------------------

Towards addressing \ref{req:direct}, \ref{req:reasoning}, and \ref{req:query}, \cite{burattini2025gap} proposed the development of a generalized \ac{BDI} engine that can be specialized to work with different belief, goal, plan representations and reasoning mechanisms. 
%
A similar approach has been proposed in~\cite{novak2006atal}, 
in which authors discuss the idea of a modular \ac{BDI} architecture,
to make it independent of the logic representation and reasoning mechanisms,
although not in the context of hypermedia.
%
Sharing the core modularity idea, the proposal of a generalized \ac{BDI} interpreter encapsulating agents reasoning cycle,
while decoupling the specific technology used for knowledge representation and manipulation would allow for tailoring the reasoning mechanisms to application-specific needs.
%
In this way,
beliefs
can flexibly support different formats depending on the domain at hand
and be adapted to work directly with e.g., \ac{RDF} triples, to JSON, YAML, and other Web standards.

Similarly, the fundamental operations performed by a \ac{BDI} agent (e.g., plan selection, belief update, goal adoption), could be implemented using customized mechanisms that leverage the specific semantics of the adopted representation.

\begin{figure}[tb]
    \centering
    \includegraphics[width=0.5\textwidth]{figures/generalize_bdi.pdf}
    \caption{
        The generalized \ac{BDI} engine architecture.
        The core \ac{BDI} reasoning cycle is decoupled from the specific implementations of beliefs, goals, plans, and reasoning mechanisms.
        This allows for specializing the engine to work with different representations and logics.
    }
    \label{fig:generalized-bdi-engine}
\end{figure}


\Cref{fig:generalized-bdi-engine} schematically illustrates the architecture of the proposed \ac{BDI} engine showing three layers:
\begin{itemize}
  \item \textbf{generalized \ac{BDI} reasoning layer},
  where the agent's deliberation process is implemented as a reasoning cycle,
  yet agnostic to how beliefs are represented and manipulated.
  %
  This is where
  agents' intentions are maintained
  and updated as agents perceive,
  decide what to do,
  and finally act.
  %
  Any operation involving beliefs, or plan selection, should rely on some \emph{abstract} \acs{API},
  to be implemented by the next layer.
  %
  \item \textbf{concrete specification layer},
  where the aforementioned \acs{API} is implemented to provide concrete operations on the belief base and plan library.
  %
  This layer acts as an adapter between the generalized \ac{BDI} reasoner
  and the specific knowledge representation technologies of choice.
  %
  Lastly,
    \item \textbf{application layer},
  where actual goals, plans, and interactions of a \ac{MAS} are defined
  to tackle some specific use case.
\end{itemize}

This architectural blueprint is a starting point, to which the \jakta{} framework (\Cref{sec:mas:engineering:jakta}) is intended to evolve, to better support the integration of \ac{BDI} agents with other kinds of systems, including hypermedia environments. 

%-------------------------------------------------------
\subsection{Embodiment for Agents on the Web}
%-------------------------------------------------------

When considering open, dynamic, and large-scale systems where different agents may join and leave a shared environment at any time, the ability to \emph{discover information about other agents} becomes a crucial factor for enabling interaction.
%
One way to enable the discovery of agents in such open systems is to equip them with \emph{bodies} that are reified in the shared environment.
Such bodies may also permit agents to communicate indirectly through reciprocal observation of their actions, facilitate contextual interaction, and support accountability mechanisms through the observability of visible agent behavior---all of which is enabled through embodiment.
%
The body itself becomes an indicator of an agent's presence, offering means to perceive its (observable) state and actions---and possibly even to ascribe goals to the agent~\cite{castelfranchi2012abscribingminds}.

Software agents are typically defined as \emph{situated} (\Cref{chap:back:MAS}), but not necessarily \emph{embodied} as the two properties are not dependent on each other when considering virtual environments.
%
On the Web, being \emph{situated} implies agents can browse the Web by discovering and following links from page to page. However, the agents are not \emph{embodied}: Multiple agents browsing the same website are typically unaware of one other, as there is no representation of the other agents currently \emph{on} the same page.
%
This does not preclude \textit{designing} websites that support embodiment---and this can be witnessed regularly, for instance, in collaborative document editors. However, such support must be designed and implemented explicitly into the website.

Following the proposal of \ac{hMAS} to make all significant entities in a \ac{MAS} represented within the hypermedia environment~\cite{Ciortea_Boissier_Ricci_2019}, it is possible to design an agent body as a Web resource that other agents can discover and interact with. 
%
A step in this direction is taken in~\cite{Zimmermann2023}, which motivates the need for agents' situatedness and embodiment in open hypermedia environments and illustrates a potential solution based on the Solid specifications~\cite{Solid_0.9.0:21}. The authors outline several requirements for agent embodiment in decentralized hypermedia systems, but they do not provide a conceptual model for agent bodies as first-class abstractions which can guide the design of digital embodiment in \ac{hMAS} and other kinds of virtual environments.

\cartago{} (\Cref{sec:back:mas:aose}) which is an inspiration for \ac{hMAS} provides an implementation of agent bodies as \emph{artifacts} that are automatically assigned to agents when they join a \emph{workspace}~\cite{Ricci_Piunti_Viroli_Omicini_2009}.
%
The artifact serves as a proxy to the agent in the shared environment, enabling \cartago{} to support heterogeneous agents in the same environment.

Following these ideas, \cite{embodiment2025} proposes a conceptual notion of \emph{agent body} for \ac{hMAS} based on the following set of features:
\begin{itemize}
    \item \textbf{Discoverability}: bodies enable agents to discover one another in an environment and optionally identify the corresponding agent associated to it. 
    \item \textbf{Communication Through Behavior}: bodies enable implicit communication between agents through their observable behavior, by showing the actions being performed by the agent.
    \item \textbf{Accountability}: bodies support accountability mechanisms by making the agent's actions and state observable.
    \item \textbf{Situation-Dependent Interaction}: bodies facilitate interaction by exposing contextually relevant information about the agent as well as affordances to interact with it.
\end{itemize}

To support such desired features, the agent body is hence defined as:
\begin{quote}
\emph{
A body is an artifact reifying and identifying an agent in an environment. The body situates the agent, transparently mediating its actions and allowing the agent to perceive in a timely fashion.
The body allows others to observe the owner agent and its behavior, and to perceive agent-specific affordances to interact with it.
}
\end{quote}

This definition captures the fundamental properties of an agent body in virtual environments, emphasizing its role in reifying the agent (\textbf{concreteness}), identifying it (\textbf{identifiability}), implicitly mediate actions (\textbf{transparency}), routing perceptions from the environment (\textbf{timely perception}) and allowing other agents to observe the agent's behavior through it (\textbf{focusability})~\cite{embodiment2025}.

\begin{code}
\captionof{listing}{A description of agent {\tt alice} and its body artifact.}
\label{lst:agent-description}
\begin{minted}{turtle}
@base <http://localhost:8080/> .
@prefix hmas: <https://purl.org/hmas/> .
@prefix td: <https://www.w3.org/2019/wot/td#> .
@prefix jacamo: <https://purl.org/hmas/jacamo/> .

<workspaces/production/artifacts/alice/#artifact> a td:Thing, hmas:Artifact, jacamo:Body;
  hmas:isContainedIn <workspaces/production/#workspace>;
  jacamo:isBodyOf <workspaces/production/agents/alice/#agent>;
  td:hasActionAffordance [ a td:ActionAffordance, jacamo:Focus;
    td:name "focus";
    td:hasForm [ htv:methodName "POST";
      hctl:hasTarget <workspaces/production/artifacts/alice/focus>;
      hctl:forContentType "application/json";
      hctl:hasOperationType td:invokeAction ];
    td:hasInputSchema [ a js:ObjectSchema;
      js:properties [ a js:StringSchema;
        js:propertyName "callbackURI"]
    ]].

<workspaces/production/artifacts/alice/#agent> a hmas:Agent .
<workspaces/production/#workspace> a hmas:Workspace .
\end{minted}
\end{code}

\Cref{lst:agent-description} shows an example of implementation of an agent body as an artifact within an hypermedia environment exposed by the Yggdrasil platform~\cite{Ciortea_Boissier_Ricci_2019}.
%
Through \ac{RDF} triples, the agent \texttt{alice} is described as an instance of \texttt{hmas:Agent} (with the \ac{hMAS} ontology~\cite{hmas-core}), and its body is shown as currently contained in a workspace, situating the agent in the hypermedia environment.
%
The body artifact is also identified as an instance of \ac{WoT} \emph{Thing}~\cite{wot-td}, and exposes an action affordance \texttt{focus} that other agents can invoke to observe \texttt{alice}'s body (e.g., to monitor its behavior).

The definition of agent bodies for \ac{hMAS} opens up interesting research directions towards better supporting societies of agents operating in hypermedia environments. 
%
\ref{req:actions} assumes particular importance in this context, as agents would not only discover actions exposed by passive \emph{artifacts} in the environment, but possibly also from other agents through their bodies. 

%-------------------------------------------------------
\subsection{Enhancing BDI Agent Adaptability}
%-------------------------------------------------------

This section briefly reports contributions in the area of enhancing the adaptability of \ac{BDI} agents towards addressing \ref{req:actions}.
%
Although not directly focused on hypermedia environments, the proposed approaches can be extended to such contexts to improve agent adaptability in open and dynamic environments.
This follows the ideas presented in Section~5.4 of \cite{Boissier_Ciortea_Harth_Ricci_Vachtsevanou_2023}, which discusses the enhancement of agents through a set of reusable strategies that can support their exploration of open environments such as the Web.

Along this line of research,
\cite{Ciatto_Aguzzi_Battistini_Baiardi_Burattini_Ricci_2025} proposes an approach to leverage the generative capabilities of \acp{LLM} to generate plans on-the-fly, based on the current context and goals of the agent.
%
Although still in preliminary stages, this approach can significantly enhance the adaptability of \ac{BDI} agents in dynamic environments, as they can generate new plans when existing ones are not suitable for the current situation.
%
In hypermedia environments, where agents might be programmed to expect specific structures or resources, the ability to generate plans dynamically can help them adapt to unexpected changes.
%
\ac{LLM} can also be used to interpret the semantics of Web resources and map relevant information to the agent's beliefs and goals, possibly guiding the adaptation process.
%
The exploitation of \ac{GenAI} techniques for the design of agents is an open research direction, to which \ac{BDI} agents can contribute by providing a structured framework of cognitive abstractions that support the design and interpretation of autonomous behavior~\cite{DBLP:conf/atal/Ricci0ZBC24}.


A different approach is taken in \cite{DBLP:journals/aamas/HubnerBRM25}, 
where a proposal for enabling \ac{BDI} agents to forecast their behavior in a simulated environment before acting in the real deployment context is presented as a way to detect future issues and adapt accordingly by refining ordering of options.
%
\cite{wesaac} further explores this idea when considering multiple future-oriented \ac{BDI} agents, noticing how, without the exchange of information, agents may still end up in conflicting situations as they adapt their behavior based on their individual forecasts which may miss the effects of other agents' actions.

This approach can be particularly useful in hypermedia environments,
where the dynamics of the environment and the presence of other agents can lead to unexpected situations.
%
In this context, \acp{DT} can actually help agents to simulate their actions in a controlled environment, letting agents play out their plans on a simulation of the \ac{PA} first to detect potential issues and react accordingly.
%
This opens up interesting future research directions on the interplay of adaptable \ac{BDI} agents and \acp{DTE}~\cite{DBLP:conf/eumas/Burattini23}.


%=======================================================
\section{Explainability in \acs{BDI} Agents}
%=======================================================

Explainability emerged in recent years as an important desired property for artificial intelligence (AI) based systems~\cite{xu2019explainable}.
%
While the term is more prominently employed to describe the process of extracting human-understandable explanations from Machine Learning models to improve the users' level of trust or allow developers to gain insights into the learned parameters~\cite{barredoarrietaExplainable2020}, the field of AI is much broader and includes many other kinds of systems that may benefit from the ability to generate such explanations.

Among these AI-based systems, \ac{MAS} play a prominent role. 
In this context, explainability mainly concerns the ability to motivate the agents' decision-making process and enable users to validate whether the choices conform to expected behavior.

As discussed in \Cref{sec:back:h40:explainable-ai}, this is an essential requirement for the development of intelligent applications in safety-critical domains such as healthcare. 
%
Accordingly, this section explores how explainability mechanisms can be integrated into \ac{BDI} agents, to enhance their transparency.

%-------------------------------------------------------
\subsection{Multi-Level Explainability}
%-------------------------------------------------------

High-level cognitive models and architectures such as \ac{BDI} support explainability \textit{by design}~\cite{Harbers10,Broekens10}, in principle.
%
% Existing approaches
%
This has been clearly recognized in literature by different approaches in the last two decades, including works about exploiting explainability for debugging BDI agents~\cite{10.1007/978-3-642-32729-2_3,Winikoff2017why}, and validating BDI agent's behavior~\cite{winikoff2022badcoffee}. 

The approach proposed in \cite{DBLP:journals/aamas/YanBHR25} follows the intuition of these related works, recognizing that explainability in \ac{BDI} agents can be addressed at different levels of abstraction, each targeting different stakeholders and use cases.
This range from developers working at the implementation level---i.e., using a specific agent programming language---to designers and engineers who may want to abstract from details of the specific technology and focus more on the architectural level, to finally the domain experts and end-users,  who focus on the functional and non-functional requirements of the system as a whole, viewing it as a ``black-box''.

The proposed \emph{multi-level explainability framework} identifies three levels of explainability for \ac{BDI} agents:
\begin{itemize}
    \item \textbf{Implementation Level}: explanations at this level target \textbf{software engineers and developers}, supporting them in the phases of debugging and testing agent behavior. It focuses on the specific implementation details of the agent and is specific to the agent programming language used (in the prototype, \jason{}~\cite{Bordini_Hübner_Wooldridge_2007})
    \item \textbf{Design Level}: explanations at this level target \textbf{designers and architects} as well as developers of the system that want to abstract from the low-level implementation details -- concerning specific e.g. agent programming languages or technologies -- and focus more on the behavior of the agents at the design level.
    %
    This level is a good abstraction for designers and architects who want to comprehend the current system's behavior, interact with other stakeholders and developers or formulate new requirements for improving the system.
    \item \textbf{Domain Level}: the top level, abstracting from how the system has been designed and focusing more on what functionalities the system is meant to provide to stakeholders -- i.e. functional and non-functional requirements -- dealing with domain-specific knowledge and insights.
    %
    This level is meant to be useful both for \textbf{end-users}, to understand and explain the system's behavior as a whole while using it, and for \textbf{software engineers}, to validate the system given the specification and requirements defined during the analysis stage of the software engineering process.
\end{itemize}

The proposed framework relies on a process that converts system logs into natural language \emph{narratives} that can be used to derive explanations for \ac{BDI} agents at the different levels of abstraction, as illustrated in \Cref{fig:explainability-process}.

\begin{figure}[tb]
        \centering
        \includegraphics[width=0.8\textwidth]{figures/multi-explain/process-new.pdf}
        \caption{The explainability process converting system logs into narratives that can be used to derive explanations for \ac{BDI} agents.}
        \label{fig:explainability-process}
\end{figure}

\cite{DBLP:journals/aamas/YanBHR25} details how the process can be implemented in practice, identifying set of events for each level, and a mapping function from one layer to the next one.
%
This allows to generate explanations at one level, depending on the events at the lower level, thus enabling a hierarchical approach to explainability.

The resulting process is illustrated in \Cref{fig:multi-level-explainability}, showing how explanations at each level are derived from the narratives generated at the lower level, essentially reducing the complexity of the explanation as the level of abstraction increases.
%
This is coherent with the idea that users at different levels of abstraction require different kinds of explanations, that progressively abstract from low-level implementation details.
%
The middle level is implemented using high-level cognitive constructs such as goals, plans and intentions.
%
This ideally would make it possible to use it as a common ground for different \ac{BDI} implementations, as these constructs are shared across different \ac{BDI} agent programming languages and frameworks.

\begin{figure}[tb]
    \centering
    \includegraphics[width=0.7\textwidth]{figures/multi-explain/agent-multi-level.pdf}
    \caption{Graphical representation of the multi-level explainability framework for \ac{BDI} agents.}
    \label{fig:multi-level-explainability}
\end{figure}

The framework has been prototypically implemented and evaluated in the context of a \ac{BDI} agent developed with \jason{}~\cite{Bordini_Hübner_Wooldridge_2007}. Logs of the agent's execution are collected by extending the \jason{} interpreter, and then they can be processed by an external system to generate explanations at the implementation and design level. 
%
The domain level remains an open research direction, as it requires the injection of additional knowledge such as functional and non-functional requirements of the system, to be able to generate explanations that are coherent with the domain at hand.


%-------------------------------------------------------
\subsection{Inter-Agent Explainability}
%-------------------------------------------------------

In addition to generating explanations for human stakeholders, \ac{BDI} agents may also benefit from the ability to explain their behavior to other agents in the system.
%
This is particularly relevant in scenarios where agents need to collaborate or coordinate their actions to achieve common goals, or possibly learn from each other. 

\cite{beaumont2025engineering} explores this idea by proposing implementation strategies for building blocks that enable the support of inter-agent explainability in \ac{BDI} agents.
%
The base building block is to extend the agent architecture with a dedicated \emph{explanation store}: a special purpose memory in which agents can store and retrieve \emph{explanatory content} to generate explanations.
%
This is different from the belief base, which is typically used to store \emph{present} knowledge, extending the agent's memory with a dedicated component for storing \emph{past} knowledge and action traces.

The proposed strategies are the following: 
\begin{itemize}
    \item \textbf{Inter-agent explanation protocol:} Define interaction protocols (following the proposal by Ciatto et al.\ \cite{ciatto2023general}) to enable agents to request and exchange explanations via messages, supporting interoperability and structured dialogue.
    \item \textbf{Implicit and explicit addition mechanisms:} Support both automatic (implicit) and developer-driven (explicit) ways to add \emph{explanatory content} to the explanation store, allowing greater control and flexibility.
    \item \textbf{Runtime state identifiers and inspection:} Provide language-level identifiers and inspection mechanisms for agent runtime state (e.g., plans, intentions), enabling agents to extract and relate relevant information for explanations.
    \item \textbf{Configurable retrieval and generation:} Allow developers to configure how explanatory content is retrieved and explanations are generated (e.g., via rules, ML, or logic), adapting to different application needs.
    \item \textbf{Uniform machine-understandable representations:} Use structured, ontology-based formats for explanations to ensure machine interpretability and interoperability among heterogeneous agents.
\end{itemize}

Exploring inter-agent explainability opens up interesting research directions for \ac{BDI} agents, especially in the context of open and dynamic environments such as hypermedia systems, where agents may need to interact with unknown agents and justify their actions to others.



%=======================================================
\section{Final Remarks}
%=======================================================


This chapter presents different contributions in the area of engineering \ac{BDI} agents and \ac{MAS}, 
focusing on addressing challenges that would make agents easier to develop and integrate with external systems, with a specific focus on hypermedia-based systems, linking to the research activities on \ac{hMAS}~\cite{Ciortea_Boissier_Ricci_2019} and the \ac{HWoDT} proposed in \Cref{chap:dte:hwodt}.
%
As a complementary perspective to the development of agents, the chapter also explores the role of testing through simulation and explainability mechanisms to enhance the reliability and transparency of \ac{BDI} agents, towards their adoption in real-world applications such as the automation of healthcare processes.
%
These contributions go towards the long-term vision of combining \ac{BDI} agents with \acp{DTE} to create intelligent, adaptive, and explainable systems that can operate in complex environments, leveraging the rich semantic models and the separation of concerns in the interaction with the physical environment provided by \acp{DT} to support the agents' reasoning and decision-making processes.

\paragraph{\ref{rq:5} How to make the development of \ac{BDI} agents more accessible and easier to integrate with external systems such as \ac{DT}?}

\ac{BDI} agents are arguably not as accessible as other programming paradigms. 
This is not necessarily due to the complexity of the \ac{BDI} model itself---which can be rather considered intuitive due to the cognitive abstractions it provides---but rather to the gap of current \ac{BDI} frameworks that originated in the context of \ac{AI} research rather than mainstream software engineering.
%
\ac{BDI} agent development can be made more accessible through the conception of new frameworks that better align with modern software engineering practices, \emph{blending} abstractions with other mainstream paradigms and decouple the \ac{BDI} reasoning cycle both from concurrency models and knowledge representation technologies, letting developers choose the most suitable ones for their application.
%
Improved reliability of \ac{BDI} agents can be achieved through simulation-based testing. Explainability features can further support the debugging process, and enhance user trust in autonomous systems. 
%
Fundamental research on extending the capabilities of \ac{BDI} agents to adapt to dynamic environments is also necessary to make them more resilient and autonomous in real-world deployment scenarios.  


%****************************************************************************************
%****************************************************************************************
\part{Applications Scenarios and Validation}
%****************************************************************************************
%****************************************************************************************

%%%%%%%%%%%%%%%%%%%%%%%%%%%%%%%%%%%%%%%%%%%%%%%%%%%%%%%%
\chapter{Pharmaceutical Supply Chain}
\label{chap:val:irst}
%%%%%%%%%%%%%%%%%%%%%%%%%%%%%%%%%%%%%%%%%%%%%%%%%%%%%%%%


%=======================================================
\section{Use Case Analysis}
%=======================================================

%=======================================================
\section{\acl{DTE} Proposal}
%=======================================================

%%%%%%%%%%%%%%%%%%%%%%%%%%%%%%%%%%%%%%%%%%%%%%%%%%%%%%%%
\chapter{Trauma Management}
\label{chap:val:trauma}
%%%%%%%%%%%%%%%%%%%%%%%%%%%%%%%%%%%%%%%%%%%%%%%%%%%%%%%%

%=======================================================
\section{Use Case Analysis}
%=======================================================

%=======================================================
\section{\acl{DTE} Proposal}
%=======================================================

%=======================================================
\section{Implemented Solution}
%=======================================================

\subsection{The Trauma Management Use Case}

The healthcare sector is characterized by heterogeneity (see \Cref{ssec:dimensions}) and data integration challenges due to the presence of legacy systems and rich data that often lack adherence to shared standards.
These aspects, united with the need to track complex operations in real-time to enhance coordination among medical teams and improve the patient's journey~\cite{croatti2020jms,ricci2022dthealthcare}, make healthcare use cases perfect candidates for \ac{DT} ecosystems~\cite{10178878}.

Major Trauma Management, originally described in \cite{ricci2022wodt}, involves the coordination of rescue missions to track patient conditions from the initial contact of the medical team over the whole journey in the emergency department.

Typically, when \ac{CEU} operators receive an emergency call, they collect initial information and initiates a new \emph{mission} for each victim of the trauma.
An \emph{ambulance} and a designated \emph{rescuer} are then dispatched to reach the patient, provide first aid and eventually lead them to the hospital.
%
Upon arrival at the trauma location, the rescue crew establishes contact with the \emph{patient}, who may be identified via their health insurance card.
Depending on the patient's condition, a destination trauma center is selected and, during the journey, the patient is continuously monitored.

The trauma management process requires coordination across multiple departments, each relying on diverse systems, which we (safely) assume for the scope of our use case to be developed using different technologies due to the fragmented evolution of electronic health systems.

Specifically, we assume to have two main subsystems: emergency call management, which includes mission planning, and pre-hospital patient monitoring.
%
We further assume these systems to be already \ac{DT} enabled, but using different technologies, namely \ac{WLDT} for the emergency call management, \azureTwin{} for patient monitoring and Eclipse Ditto for the national registry of healthcare users.
%
Figure \ref{fig:use-case-diagram} depicts a possible \ac{HWoDT}-based \ac{DTE} implementation joining \acp{DT} of different systems in a coherent view.
%
The use case implementation and the complete description of the use case processes are available on GitHub\footnote{\url{https://github.com/Web-of-Digital-Twins/major-trauma-management-case-study}}.


\begin{figure}[t]
  \centering
  \includegraphics[width=\columnwidth]{figures/hwodt/major-trauma-management.pdf}
  \caption{The \ac{HWoDT}-based \ac{DTE} for the trauma management case study. \acp{DT} implemented with different technologies (WLDT, Ditto, Azure) become interoperable through the HWoDT Adapters and are integrated through the WoDT Platform. 
  Black dotted arrows represent the interaction with the platform, black dashed arrows represent relationships between \acp{DT} while grey arrows represent the interaction between consumers, the platform and \acp{DT}.
  }
  \label{fig:use-case-diagram}
\end{figure}


\subsection{Implementing a \acs{HWoDT} Ecosystem}

The first step towards creating a \ac{HWoDT} \ac{DTE} is to analyze the current state of the system and identify available \acp{DT}, evaluating their modeling capabilities, and understanding the underlying technologies they rely on.
%
Then, the ecosystem model can be refined by introducing relationships between \acp{DT}, derived from domain knowledge, especially where explicit relationships are not supported by the \ac{DT} technology.

The next step is identifying the membership criteria for the ecosystem. These are usually informed by application-specific requirements. 
%
Through the \ac{HWoDT} applications can compose dedicated \acp{DTE} that support their operations, joining only the \acp{DT} they are interested in.
%
In our example, we create one global ecosystem to provide a unified view of ongoing rescue processes.

After this analysis phase, the first step towards implementation is achieved adapting the \acp{DT} to the \ac{HWoDT} uniform interface through \emph{adapters} (\Cref{ssec:adapters}).
\Cref{fig:use-case-diagram} shows the different \acp{DT} each mapped through the relative technology adapter. 
%
Reusing existing adapters is mostly a matter of configuration (e.g., API keys, connection details). 
In this phase, the semantic mapping to build the \ac{DTKG} is the most crucial step. 
For instance, in the example we represent healthcare resources using the HL7 FHIR\footnote{\url{https://www.hl7.org/fhir/}} standard in the \ac{KG}.

At this stage \acp{DT} are already uniformly reachable by consumers, and can be accessed as-a-service.

Deploying the \ac{WoDT} platform and registering \acp{DT} to it further completes the \ac{HWoDT} providing ecosystem-wide services to inspect and monitor the state of the \ac{DTE}.

Finally, applications can now leverage both the individual \acp{DT}' uniform interface and the \ac{WoDT} platform services to implement their business logic.
For instance, the \emph{Mission Agent} (\Cref{fig:use-case-diagram}, top) in the use case can be developed to observe the \ac{DTE} \ac{KG}, react to new missions being created and find free resources to assign with a SPARQL query like the one in \Cref{lst:ambulance-rescuer-query}.

% \lstinputlisting[
%     label={lst:ambulance-rescuer-query},
%     caption={SPARQL Query performed by the Mission Agent to obtain the available ambulances and rescuers with the appropriate qualification.},
% ]{listings/hwodt/use-case/ambulance-rescuer-query.rq}

\subsection{Benefits and Limitations}
\label{ssec:benefits-limitations}

\note{Maybe move this section somewhere else e.g. conclusion}

The showcase of the implementation and exploitation of an \ac{HWoDT} ecosystem allows us to highlight benefits and discuss limitations of the approach.

\paragraph{Decoupling and Interoperability}
The main benefit is the decoupling from the underlying heterogeneous technologies achieved through the \ac{HWoDT} uniform interface.
%
As a result, ecosystem-aware applications — such as dashboards and automation tools (e.g., the \emph{Mission Agent}) — can be built using standard Web technologies, treating \acp{DT} as Web resources.

\Cref{fig:comparison-custom-vs-hwodt} highlights the difference in interaction with the \ac{DTE} compared to custom, heterogeneous solutions, when querying across multiple \acp{DT}.
%
Without the \ac{HWoDT}, applications must access each \ac{DT} using technology-specific \acp{API}, adapt the data to a common model, and manually merge results to process it for global insights.
In contrast, the \ac{HWoDT} approach enables applications to rely on Web standards interactions and data formats (e.g., SPARQL, \ac{RDF}) directly supporting applications in their interrogations.


A result of the \ac{HWoDT} decoupling is that developers can focus only on application logic, abstracting away the complexity of heterogeneous \ac{DT} integration.  
This yields two important advantages:  
\begin{inlinelist}
    \item the application layer is more stable as it remains unaffected by the introduction of new \acp{DT};
    and  
    \item the \ac{DT} layer can evolve or replace underlying implementations without affecting upper layers as long as the exposed interface is not changed.
\end{inlinelist}

\paragraph{Navigation}
Compatibility with the Linked Data Principles enables consumers to navigate \ac{DT} relationships within the \ac{DTE}, discovering entities and retrieving state information.
%
While this is feasible in homogeneous \acp{DTE} -- such as \azureTwin{} -- where relationships are confined within the same instance, the hypermedia layer of the \ac{HWoDT} addresses this limitation by enabling the definition of cross-platform \ac{DT} relationships using \ac{URI}.   
This enhances the expressiveness of \acp{DTE} and allows developers to model ecosystems that more accurately reflect domain knowledge.
%
As an example, in our trauma management use case, we can link patients with the mission handling their trauma in the \ac{DTE} \ac{KG} despite the respective \acp{DT} being implemented with different technologies as shown in \Cref{lst:mission-dtkg}.
% \lstinputlisting[
%     caption={A fragment of the \ac{DTE} \ac{KG} showing the representation of the Mission \ac{DT} showing the relationship with the patient.},
%     label={lst:mission-dtkg},
% ]{listings/hwodt/use-case/dtkg-missiondt.ttl}

% \lstinputlisting[
%     caption={Portion of the Ambulance \ac{DTD} that shows the relevant data for the \texttt{SetDestinationCommand} action},
%     label={lst:ambulance-dtd},
% ]{listings/hwodt/use-case/ambulance-dtd.json}    

\paragraph{Explicit Semantics and Interoperability}

The explicit semantic representation in the \ac{HWoDT} is both a tool to support easy integration of different \acp{DT} and a deliberate choice to push the importance of establishing shared semantics so that users can get more information on \ac{DT} functionalities.

Furthermore, the \ac{HWoDT} does not enforce the adoption of a single ontology: while \acp{DT} expose a uniform \ac{DTD}, their \ac{DTKG} content may rely on different vocabularies or ontologies.
This is intentional: to support \acp{DT} from diverse application domains, the framework allows varying ontologies within the same \ac{DTE}.  
Semantic uniformity is thus delegated either to \ac{DT} designers -- who may agree on shared representations -- or to different implementations of the \ac{WoDT} platform, which may translate and align models to a common vocabulary. 
%
In the use case, using the HL7 FHIR standard to unify data from different sources greatly improves the quality of the information in the \ac{DT} ecosystem.

Of course, we acknowledge this requires an additional effort when developing (or mapping) \acp{DT}, as the appropriate domain ontologies need to be identified and used.
Nevertheless, we consider this a valuable trade-off aligned with the efforts towards \ac{DT} interoperability~\cite{Klar_Arvidsson_Angelakis_2024} and the support for defining some level of semantics available in many \ac{DT} tools (including, for instance, \azureTwin{} and Ditto).

% \paragraph{Information Loss}
% With any model conversion there is risk of information loss.
% %
% We acknowledge that, despite the metamodel being general enough for most use cases, a potential limitation of the \ac{HWoDT} is that not all \ac{DT} functionalities may be mapped to Web-based interactions with ease (e.g., data-intensive tasks).

% Ideally, though, the existing applications that leverage specific features of a \ac{DT} will still be able to access them, and new applications built for the ecosystem would benefit from the uniform interface instead.

\paragraph{Performance and Scalability}

As with any additional layer on top of existing systems, performance and infrastructural scalability are a possible concern.
%
In general, the \ac{HWoDT} vision does not target hard real-time scenarios, where dedicated techniques and specialized technologies are required to ensure real-time guarantees.
%
Additionally, even though the preliminary performance measurements show interesting results (\Cref{ssec:performance}), performance optimization was not a primary focus in the development of the \ac{HWoDT} Framework. 
%
Despite this, we can still make some considerations on performances of the whole \ac{HWoDT} approach.

The impact of the translation of a \ac{DT} model to the \ac{DTKG} representation within the adapter is usually negligible, the impact of the adapter depends on its implementation as either an internal (such as in \ac{WLDT}) or external component as in the second case care should be taken to scale the adapter effectively.

The centralized implementation of the platform makes it a potential bottleneck. Specifically -- as confirmed with our preliminary performance tests on the prototype -- interacting with the \ac{DTE} KG is costly because the graph gets updated by all \acp{DT} updates.
%
This is a reasonable trade-off as interacting with the \ac{DTE} provides advanced functionalities that would be impossible or very difficult to replicate simply by interacting with individual \acp{DT}.
%
Nevertheless, we are interested in exploring optimization techniques, as well as decentralized approaches to provide ecosystem services.


\paragraph{Security Considerations}
The openness and Web-based nature of the \ac{HWoDT} approach introduces several security challenges making authentication, authorization, and access control critical.
Although we do not implement this in our prototype yet, standard Web security mechanisms (e.g., OAuth 2.0, HTTPS) can be employed to secure interactions.
Fine-grained access control policies must be enforced both at the data layer (e.g., per-property visibility) and at the hypermedia interface level to possibly restrict access to specific resources.
For instance, \ac{WoT} \ac{TD}-based \ac{DTD} already supports the description of security policies~\cite{wotthing} to define access control of properties and actions. 
Furthermore, provenance tracking and data integrity verification (e.g., using digital signatures) can enhance trust. Namely, \acp{DT} should never be able to modify the representation of another \ac{DT}. Now \ac{DTKG} are not validated, but it should not be possible to post triples updating the description of a different \ac{DT} than the one it is sending them.
Finally, the mapping of a \ac{DT} in its \ac{DTKG} can purposefully avoid exposing sensitive information, or a \ac{DT} could support mechanisms to expose different \acp{DTKG} depending on authorization level of the client.
Existing efforts in the Semantic Web community such as the Solid project\footnote{\url{https://solidproject.org/}} or using RDF-star for access control metadata~\cite{rdfstar-security}, may offer interesting solutions to integrate additional security features on the management of the \ac{DTE} \ac{KG}. 




% %%%%%%%%%%%%%%%%%%%%%%%%%%%%%%%%%%%%%%%%%%%%%%%%%%%%%%%%
% \chapter{Personnel Shifts Management}
% \label{chap:val:ER}
% %%%%%%%%%%%%%%%%%%%%%%%%%%%%%%%%%%%%%%%%%%%%%%%%%%%%%%%%

% %=======================================================
% \section{Use Case Analysis}
% %=======================================================

% %=======================================================
% \section{\acl{DTE} Proposal}
% %=======================================================

% %=======================================================
% \section{Implemented Solution}
% %=======================================================

%%%%%%%%%%%%%%%%%%%%%%%%%%%%%%%%%%%%%%%%%%%%%%%%%%%%%%%%
\chapter{\acl{ORM}}
\label{chap:val:orm}
%%%%%%%%%%%%%%%%%%%%%%%%%%%%%%%%%%%%%%%%%%%%%%%%%%%%%%%%


%=======================================================
\section{Use Case Analysis}
%=======================================================

%% come fanno ora la programmazione?
%% NG: oggi la programmazione viene portata avanti, nei singoli ospedali, in maniera autonoma e indipendente. In ogni presidio si riunisce un board chirurgico, più o meno settimanalmente. Durante questo incontro, oltre a linee di indirizzo, si conferma la distribuzione degli slot di sala (purtroppo questo rimane tracciato solo all'interno di file excel). Negli ultimi anni, se si vuole citare, abbiamo implementato un progetto di riorganizzazione (denominato HPR chirurgie 2.0) che era volto a studiare i percorsi nei singoli ambiti territoriali e riportare a tutta l'Azienda elementi di omogeneità, a riorganizzare la configurazione dell'offerta nei singoli ospedali e ad abilitare la lettura delle informazioni. Dopo circa un anno di lavoro si è distribuita una reportistica che - per la prima volta in AUSL Romagna - presenta la situazione quanto più reale possibile della lista di attesa chirurgica (circa 20.000 casi in attesa di chirurgia). Qui, se servono alcuni numeri, posso inserire: consistenza attuale (o media) della lista, numero di sale operatorie e numero di ospedali, produttività media, domanda media. Si potrebbe accennare al fatto che AUSL Romagna assiste un territorio che conta 1,125,0000 assistiti e la complessità deriva anche dalla disponibilità di 90 sale operatorie distribuite in 10 presidi molto differenti distribuiti nel territorio. Infine, per quanto riguarda il monitoraggio dell'efficienza delle sale operatorie in letteratura sono definiti indicatori specifici (utilizzo grezzo, utilizzo a valore, ritardi, anticipi, ...). Dalle cose che ho scritto qui, se ritenete utile alcuni elementi in particolare e mi dite quali, posso cercare dati e fonti (per questo o per l'estensione).

In collaboration with the Local Health Authority of Romagna, Italy (AUSL Romagna) we collected information about the current management of ORs across multiple hospitals and the needs expressed by both the on-site medical staff and the general administration in order to achieve better overall control without adding too much of a burden on the daily processes of all involved parties.

\subsection{Current Situation}
%due righe che incapsulano meglio il commento di Nicola qua sopra?

The current situation is not completely automated and relies on a QR code scanning system, a printed schedule of the planned surgeries for the day and a paper form for each surgery that is filled by hand by the medical staff.

The QR scanning system is designed to allow keeping track of the most relevant phases of a surgical procedure. These steps are the following:
\begin{enumerate}
    \item Patient exits the hospital ward
    \item Patient enters the operating suite
    \item Patient enters the OR\footnote{From now on if, for any unexpected reason, the surgical procedure is interrupted step 8 is the next one}  %% questo credo possa avvenire prima di Anesthesia performed, in generale. Esistono casi per i quali l'anestesia inizia fuori dalla sala operatoria, ma dipende dalla disponibilità o meno, nel blocco, di aree dedicate
    \item Anesthesia is performed\footnote{In some cases this step can happen before step 3 if there are dedicated facilities to support it}
    \item Anesthesia is effective
    \item Surgery is started
    \item Surgery is completed
    \item Patient is moved outside of the OR
    \item Patient leaves the operating suite
    \item Patient enters a hospital ward (which can be different from the one which he came from)
\end{enumerate}
%
After a surgical procedure, OR is cleaned and prepared for the next one. 

Timing of each step execution is also annotated on the paper form, steps that happen between since one of the main current issues is the discrepancy between the time the QR codes are scanned and the time steps are actually performed. This is due to the fact that the medical staff is usually very busy and focused on the patient and can forget to scan at the appropriate time. The paper form allows them to correct the timings annotating afterwards an estimated real-time based on what they remember doing as well as other relevant information about the surgery, which they may find useful to keep track of.

Another critical issue with the current state of the system is that data registered by the QR scanning system is stored in a database that is local to each hospital. This, of course, implies that the coordinators that need to manage the network of ORs are not able to automatically retrieve information about the current state of the ongoing surgical procedures, leading to less efficient management.

The main pain points that emerged in the conversation with the medical staff are here summarised:
\begin{itemize}
    \item Low readability of manually annotated data
    \item Limited space on the paper form to annotate surgery's steps
    \item Discrepancy between the scanned times and the real times
    \item Deletion or delay of planned surgery is currently not tracked since the schedule is printed
    \item Accountability for issues regarding data collection is impossible to define
    \item Sub-optimal management of OR due to the fact that the availability of a facility is not known in real-time
\end{itemize}

\subsection{Requirements}
From the analysis of the current situation and the staff needs the following functional requirements for a new management system are defined:
\begin{itemize}
    \item Real-time monitoring of the availability of ORs: whether they are busy at the moment and what the planned surgeries for each one are
    \item Time monitoring of surgical procedures and individual procedures' steps to understand the overall efficiency
    \item Automatic generation of warnings when there is inconsistency in the transmitted data (e.g. two patients in the same room at the same time, a step that is taking an unexpected amount of time)
    \item Storage of detected warnings for historical data analysis
    \item Easy data access through a dashboard to visualise the current state, review warnings, and annotate steps. The dashboard should also be accessed through an account to solve the problem of accountability.
\end{itemize}

From a more technical point of view, the new system was also required to be integrated with the existing solution, and to support interoperability with other future possible systems that may require access to data.

Note that the objective of this case study was not to completely revolutionise the current approach for OR management, but to improve it by adopting a DT based solution that opened the possibility of future expansion.

%=======================================================
\section{\acl{DTE} Proposal}
%=======================================================



The idea of applying the DT approach in the context of ORM that we put forth in this paper is based on the premises of Section~\ref{sec:background}: the need for efficient tools, possibly based on the most recent technologies, is clear and the DTs seems to provide all the required functionalities for effective scheduling of surgeries, as well as their management and supervision.  
%
They enable the acquisition of diverse data on the real system, the storing of all the historical information describing the evolution of its state, the encapsulation of a model for performing simulation and \textit{what-if} analysis, and finally, the inclusion of AI algorithms for reasoning on knowledge and data acquired, are all ingredients that can make the difference in this context.

In this section, we provide a description of an ecosystem of DTs devised for the operating suite and the surgical procedures performed.
%
Assuming to adopt DTs as a pervasive approach, we can envision having an ecosystem of connected DTs that can map real-world facilities.
%
Each hospital can have its own DT, connected to DTs of each OR which could be even connected to those of the medical equipment available in each room.
%
Similarly, we can mirror people, namely patients and the medical staff, into human DTs that are strongly interlinked with those of the facilities exploited.

The resulting network of DTs could represent the state of the healthcare system at any given time. For example, looking at the DT of a room and analysing the relationships with other DTs one could discover its planned and real-time availability, its equipment, the historical information about its management and so on.

In Fig.~\ref{fig:dt ecosystem} we show an example of how this ecosystem looks like. On top of that, a DT of each perioperative care path can be dynamically added and linked to other DTs in different phases of its lifecycle. 

When a doctor and a patient meet for an appointment and the need for a surgical procedure is identified \emph{(1)} a new \textit{Surgery DT} is created \emph{(2)} and linked to the DT of the patient \emph{(3)}. 
%
Then, whenever the surgery is scheduled in a specific room \emph{(4)} a relationship with the DT of that room is added to keep track of the planned surgeries per each room \emph{(5)}.

The patient is then usually hospitalised and transferred to the OR \emph{(6)}. When the patient actually enters the room the \textit{Surgery DT} creates a new relationship with the OR\footnote{for any reason it could also be a different room from the planned one} DT \emph{(7)} which will report that the room is actually busy.

Finally, when the patient leaves the room \emph{(8)}, the end time is registered -- e.g. by scanning a barcode like in our case study -- \emph{(9)} and the OR can now be set back to an available state after the necessary cleaning operations.

\begin{figure}
    \centering
    \includegraphics[width=\columnwidth]{figures/orm/DTORM.pdf}
    \caption{An example of how a DT ecosystem could support the management of ORs as well as the real-time tracking of elective surgery processes.}
    \label{fig:dt ecosystem}
\end{figure}

%When the process related to surgery starts, \emph{i.e.}, when the patient exits from the hospital ward, a dedicated DT is created to track and monitor the patient within the whole surgery process, collecting operations' periods and exposing information useful for the ORM tasks.
%
%In this perspective, the whole activities occurring within a surgery block -- not only limited to surgeries -- can be tracked and used to improve the ORM process.

%DTs can retrieve data about surgery rooms from the healthcare organisation's software systems. Moreover, the same software systems can provide information about the ongoing state of the surgeries.

The case study proposed in the next section describes a real-world architecture of a DT ecosystem for ORM implementing this overall picture.



%=======================================================
\section{Implemented Solution}
%=======================================================

%%%%%%%%%%%%%%%%%%%%%%%%%%%%%%%%%%%%%%%%%%%%%%%%%%%%%%%%
\chapterOutsidePart{Conclusion}
\label{chap:conclusion}
%%%%%%%%%%%%%%%%%%%%%%%%%%%%%%%%%%%%%%%%%%%%%%%%%%%%%%%%

Motivated by the digital transformation needs of healthcare, 
this thesis presents a collection of contributions towards the definition of \acp{DTE} as an abstraction to engineer systems based on \acp{DT} capable of offering a unified view of several physical entities. 
%
This holistic conceptualization offered by \acp{DTE} is seen as a key enabler to support the design of intelligent applications that can automate processes, benefiting from the decoupling that \ac{DT} offer between the physical and digital worlds.

After an analysis of the state of the art (\Cref{part:background}) in research areas connected to the topics of healthcare digitalization (\Cref{chap:back:health4.0}), \aclp{DT} (\Cref{chap:back:DT}), Web and Semantic Web technologies (\Cref{chap:back:Web}) and \acp{MAS} (\Cref{chap:back:MAS}),
the thesis provides answers to the research questions presented in the introduction through the following contributions:

\begin{itemize}[leftmargin=2.3cm]
    \item[\textbf{\Cref{chap:dte:engineering-dt}}] A reference architecture, architectural patterns, and a supporting implementation framework for the engineering of \acp{DT} with a focus on \ac{IoT} integration, heterogeneity, lifecycle management and design of augmented behavior (\ref{rq:1}).

    \item[\textbf{\Cref{chap:dte:dte}}] A conceptualization of \acp{DTE} as an abstraction to engineer systems based on \acp{DT} capable of offering a unified view of several physical entities, along with an analysis of functionalities and possible architectural patterns to support their design and their trade-offs (\ref{rq:2}).
    
    \item[\textbf{\Cref{chap:dte:hwodt}}] A proposal, and a supporting prototype implementation for the concrete realization of \acp{DTE} based on existing \acp{DT} applying principles from the Web architecture and re-using technologies and Semantic Web standards to address the challenges of interoperability, data integration and knowledge representation (\ref{rq:2}).

    \item[\textbf{\Cref{chap:dte:dtc}}] A proposal and implementation strategies for the \ac{DTC}: a middleware to support the deployment of \acp{DTE} in distributed computing environments, while supporting the interaction of different stakeholders with the \acp{DTE} through different phases of the development-to-deployment lifecycle of \acp{DT} (\ref{rq:3}). 

    \item[\textbf{\Cref{chap:mas:mas-dt}}] An analysis of the complementary nature of \acp{DT} and \ac{MAS} as abstractions to encapsulate intelligent applications, with a proposal of design principles and micro-architectural patterns that can guide the implementation of intelligent \ac{IoT} systems through combinations of \acp{DT} and autonomous agents (\ref{rq:4}).
    
    \item[\textbf{\Cref{chap:mas:engineering}}] A collection of contributions in the area of \acp{MAS} targeting the redesign of supporting tools to lower the entry barrier, favor the use of simulation for stronger engineering guarantees of the developed \ac{MAS} behavior, support a more direct integration with open environments such as the Web and can support explainability features that are essential for real-world deployments and user trust (\ref{rq:5}).
    
\end{itemize}

Finally, \Cref{part:applications} presents the application of the proposed contributions in three healthcare-related scenarios developed thanks to the partnership with \ausl{}: the management the oncologic pharmaceutical supply chain (\Cref{chap:app:irst}), the management of rescue operations when dealing with trauma patients (\Cref{chap:app:trauma}), and the management of operating rooms (\Cref{chap:app:orm}).

%=======================================================
\section*{Discussion and Future Works}
%=======================================================

\todo{Potenzialmente estendere/strutturare in open challenges/future works}


The contribution of this thesis advances the state of the art with respect to the research questions, providing novel insights in the development of \acp{DT} and the conceptualization and practical realization of \acp{DTE} to support intelligent applications in healthcare and other domains.
%
The following sections present a critical discussion of the main findings of this thesis, in light of potential limitations and open challenges that remain to be addressed.

%-----------------------------------------------------------------
\subsection*{Advanced Functionalities in Digital Twins}
%-----------------------------------------------------------------

The contributions in \Cref{chap:dte:engineering-dt} support the engineering of \acp{DT} that can integrate heterogeneous data sources, manage their lifecycle and expose digital services that augment the capabilities of the physical entities they represent. 

The lifecycle management features provide a structured approach to handle the synchronization between \acp{DT} and their corresponding \acp{PA}.
%
Measuring and assessing the quality of this synchronization remains an open challenge, as it depends on several factors such as the nature of the \ac{PA} and the specific application scenario for which the \ac{DT} is used.
%
These requirements could also vary over time, depending on the lifecycle of the \ac{DT}. 

Future works could continue exploring strategies to monitor and adapt the \ac{DT} lifecycle to detect and react to potential issues in the synchronization with the \ac{PA}, possibly leveraging prediction capabilities to compensate for temporary disconnections or data unavailability. 
%
These features should be designed in order to possibly guide practitioners in selecting the appropriate synchronization strategies based on the specific requirements of their \acp{DT}, but also \ac{DT} consumers in understanding the guarantees and \emph{fidelity} of the \ac{DT} representation.

The work on augmented behavior provides initial patterns to design \acp{DT} that can encapsulate intelligent behavior. 
%
This has currently been tested in controlled lab environments, with limited complexity of the \ac{DT} behavior.
%
Embedding more advances form of augmentation in \acp{DT}, possibly integrating prediction and simulation capabilities remains unexplored, and although the model proposed should be general enough to support these features, further validation is needed to assess its effectiveness in supporting more complex \ac{DT} behaviors.

Finally, the exploitation of such advanced \ac{DT} functionalities at an ecosystem scale through \acp{DTE} remains an open challenge.
%
Especially interesting in this setting is the idea of composing simulations or chaining predictions from multiple \acp{DT} to support a forward-looking analysis of the behavior of the ecosystem, based on its current state and graph of relationships among \acp{DT}.
%
This would require further investigation on how to orchestrate and coordinate the execution of such advanced functionalities across multiple \acp{DT} in a coherent manner.

%-----------------------------------------------------------------
\subsection*{Semantic Interoperability in Digital Twin Ecosystems}
%-------------------------------------------------------------------

The findings highlight the feasibility of re-using Web and Semantic Web technologies to address interoperability challenges in \acp{DTE}, following a similar rationale as the ones adopted by the \ac{WoT} and \ac{hMAS} initiatives. 
%
In this sense, the proposal in \Cref{chap:dte:hwodt} of the \emph{\ac{HWoDT}} aims to provide a practical mapping of the \ac{WoDT} vision onto a concrete architecture and implementation based on existing standards from the Web and Semantic Web domains.

In that context, the explicit semantic representation in the \ac{HWoDT} is both a tool to support easy integration of \acp{DT} and a deliberate choice to push shared semantics so that users can get more information on \ac{DT} functionalities.
%
The \ac{HWoDT} does not enforce the adoption of a single ontology: while \acp{DT} expose a uniform \ac{DTD}, their \ac{DTKG} content may rely on different vocabularies from diverse application domains.

Semantic consistency is thus delegated either to \ac{DT} designers -- who may agree on shared representations -- or to different implementations of the \ac{WoDT} platform, which may translate and align models to a common vocabulary.
%
If semantic consistency were a requirement of a specific \ac{DT} ecosystem, the platform could implement validation of \ac{DT} representation through existing Semantic Web mechanisms such as SHACL shapes~\cite{DBLP:conf/rweb/Pareti021}.
%
In general, guaranteeing consistency is an open problem for the Semantic Web community, which introducing semantic representations does not solve on its own~\cite{guizzardi2022ao}.

Of course, it must be acknowledged that this requires an additional effort spent in identifying and using the appropriate domain ontologies when developing (or mapping) \acp{DT}.
Nevertheless, this can be considered a valuable trade-off aligned with the efforts towards \ac{DT} interoperability~\cite{Klar_Arvidsson_Angelakis_2024} and the trend to include some form of semantic representation in many \ac{DT} tools (including, for instance, the aforementioned \acl{ADT} and Ditto).

The preliminary descriptions proposed in \Cref{chap:dte:hwodt} can be further extended and refined to cover additional aspects of \acp{DT}, such as the representation of \ac{DT} behaviors, the modeling of \ac{DT} lifecycle and versioning, or the inclusion of security and privacy-related metadata.
%
Additionally, current descriptions are limited to the representation of the \emph{present} state of a \ac{DT}, while, given the nature of \acp{DT}, it would be interesting to explore how to represent historical data, but also (possible) future \ac{DT} states, using semantic models.
%
Finally, the proposed metamodel allows \acp{DT} to expose their functionalities as \emph{actions}.
%
It might be interesting to explore how to extend this representation to include more complex \ac{DT} \emph{services}, such as requesting predictions or running simulations. 
%
Preliminary work in this direction has been proposed in \cite{DBLP:conf/models/BurattiniZPR24}, but further investigation is needed to define appropriate models and mechanisms to represent temporal aspects in \acp{DT}.

%-----------------------------------------------------------------------
\subsection*{Integration of Intelligent Applications and Digital Twins}
%-----------------------------------------------------------------------
The contribution towards the integration of intelligent applications and \acp{DT} in \ac{DTE} has been focused on the conceptual alignment and technological enablers. 

Hence, despite the contributions in \Cref{part:mas} providing a solid basis to support the integration of \acp{DT} and \acp{MAS}, further work is needed to validate the proposed design principles and micro-architectural patterns with concrete implementations of intelligent applications based on \acp{DTE}. 

This requires the definition and practical realization of more complex scenarios, where autonomous agents can leverage the unified view of multiple \acp{PA} offered by \acp{DTE} to implement more advanced behaviors. 
%
Moving on to full-scale implementations would allow to evaluate the effectiveness of the currently proposed patterns and to identify potential area of improvements.

An interesting perspective for future works in this direction is the integration of autonomous agents towards the implementation of \emph{cognitive \acp{DT}}~\cite{Minerva_Crespi_Farahbakhsh_Awan_2023}.
%
The patterns proposed for augmented behavior in \Cref{chap:dte:engineering-dt} could be adapted to support the integration of autonomous agents as components of \acp{DT}, enabling them to model intelligent behaviors directly at the \ac{DT} level.

A particularly interesting area in this direction is the integration of prediction and what-if analysis capabilities in \acp{DT} with the decision-making capabilities of autonomous agents.
%
Ideally, agents would be able to request predictions from \acp{DT} and use them to inform their actions and decisions, thus creating a feedback loop where \acp{DT} provide insights that guide agent behavior, and agents adapt their strategies based on the evolving state of the physical entities represented by the \acp{DT}, trying to anticipate future states and optimize outcomes.

Finally, the thesis focuses on \ac{MAS} and specifically on \ac{BDI} agents, but other forms of intelligent applications could be explored in relationship to \acp{DTE} to evaluate whether the proposed conceptualization supports other forms of intelligence. 

%--------------------------------------------------
\subsection*{Addressing Physical Distribution}
%--------------------------------------------------

The \ac{DTC} middleware, proposed in \Cref{chap:dte:dtc} provides a first step towards the support of \acp{DTE} in distributed computing environments, addressing challenges related to the deployment, discovery and interaction with \acp{DT} across different networked nodes.
%
The \ac{DTC} is designed to be extensible and adaptable to different deployment scenarios, abstracting the complexity of managing \acp{DTE}. 

However, further work is needed to evaluate the practical implications of deploying \acp{DTE} in real-world distributed infrastructures, assessing performance, guaranteeing reliability of the \ac{DT}-\ac{PA} connection and synchronization and dynamically adapting to changing functional or non-functional requirements. 

The proposal of the \ac{DTC} opens up interesting perspectives on dynamically migrating \acp{DT} across different nodes to optimize resource usage, latency and reliability, which remain unexplored in this thesis.

Likewise, the current design principles to distribute intelligence across \acp{DTE} and autonomous agents presented in \Cref{chap:mas:mas-dt} could be further refined to address challenges related to the physical distribution of intelligence in the edge-cloud continuum. 


%--------------------------------------------------
\subsection*{Evaluation on Real-World Scenarios}
%--------------------------------------------------

Despite the practical applications presented in \Cref{part:applications}
have been analyzed thanks to the interaction with stakeholders and domain experts from \ausl{}, 
they currently remain at a prototype level that serves only to demonstrate the feasibility and potential value of the proposed contributions. 
%
Real-world settings would provide a more robust validation of the proposed concepts, architectures and tools, allowing to assess their effectiveness in addressing real challenges, their usability by end-users and their performance under realistic conditions. 

The preliminary performance evaluation of the \ac{HWoDT} in \Cref{chap:dte:hwodt} provides encouraging results considering the lack of optimization, but further evaluations are needed to assess scalability and robustness in larger deployments with more complex \acp{DT} and higher data volumes.
%
For future works, distributed approaches to the management of the \ac{DTE} \ac{KG}, caching strategies and optimization of data retrieval and update mechanisms could be explored to enhance performance, exploiting existing techniques that are being explored in the context of Web-scale knowledge graphs.

Similarly, the proposed \ac{DTC} in \Cref{chap:dte:dtc} attempts to facilitate the roles of different stakeholders in the lifecycle of \acp{DT}, but further validation is needed to assess its effectiveness in real-world development and deployment processes, considering usability aspects and integration with existing tools and workflows.

Finally, the design principles and micro-architectural patterns proposed in \Cref{chap:mas:mas-dt} for the integration of \acp{DT} and \acp{MAS} need to be validated through more extensive implementations of intelligent applications based on \acp{DTE} in real-world scenarios, assessing their effectiveness in supporting complex behaviors.

%--------------------------------------------------
\subsection*{Security and Governance in Digital Twin Ecosystems}
%--------------------------------------------------

The openness and Web-based nature of the \ac{HWoDT} approach requires to consider security features such as authentication, authorization, and access control which are critical in \ac{DT} settings.
Although this is not implemented in the \ac{HWoDT} prototype yet, standard Web security mechanisms (e.g., OAuth 2.0, HTTPS) can be employed to secure interactions.

Fine-grained access control policies must be enforced both at the data layer (e.g., per-property visibility) and at the hypermedia interface level to possibly restrict access to specific resources.
For instance, \ac{WoT} \ac{TD}-based \ac{DTD} already supports the description of security policies~\cite{wot-td} to define access control of properties and actions. 

Furthermore, provenance tracking and data integrity verification (e.g., using digital signatures) can enhance trust.
%
Guarantees of authenticity of the data needs to be established. Namely, 
\begin{inlinelist}
    \item a \ac{DT} cannot update the representation of another \ac{DT},
    \item \ac{DT} updates need validation so that they conform to the expected structure (e.g. via SHACL).
\end{inlinelist}

Finally, the mapping of a \ac{DT} in its \ac{DTKG} can purposefully avoid exposing sensitive information, or a \ac{DT} could support mechanisms to expose different \acp{DTKG} depending on authorization level of the client.
Existing efforts in the Semantic Web community such as the Solid project\footnote{\url{https://solidproject.org/}} or using access control metadata, may offer interesting solutions to integrate additional security features on the management of the \ac{DTE} \ac{KG}. 





%%%%%%%%%%%%%%%%%%%%%%%%%%%%%%%%%%%%%%%%%%%%%%%%%%%%%%%%
% BIBLIOGRAPHY
%%%%%%%%%%%%%%%%%%%%%%%%%%%%%%%%%%%%%%%%%%%%%%%%%%%%%%%%

\backmatter

\bibliographystyle{alpha}
\bibliography{phd-thesis, bib/iot-dt-mas, bib/hwodt-sosym, bib/edtconf-semantic-rep}

\end{document}