%=======================================================
\section{History and Definitions}
%=======================================================

%=======================================================
\section{Agent-Oriented Software Engineering}
%=======================================================

%=======================================================
\section{\acl{BDI} Agents}
%=======================================================

%=======================================================
\section{Web-based and Hypermedia \acs{MAS}}
%=======================================================

%=======================================================
\section{\aclp{MAS} and \aclp{DT}}
%=======================================================

%======================================================
\subsection{Synergistic Usage of AAs and DTs}
\label{ssec:synergy}
%======================================================


AAs and DTs are not only used separately 
to deliver intelligent functionalities 
in a mutually exclusive way, 
or as alternative solutions to overlapping requirements.
Their synergistic usage is already seen in the available literature, 
especially in industrial deployments. Here, we briefly report recent surveys that highlight the rich intersection of these research areas.

The contamination of AAs in DTs is mostly related to the need to adopt a variety of artificial intelligence techniques in IoT applications, both to enhance the modelling capabilities of complex assets and to implement automatic controls.
Minerva et al.\ present a classification of DTs in relation to the level of intelligent behaviour the DT exhibits~\cite{Minerva2023}.
They identify the ultimate level of intelligent DT as a proactive (or autonomic) DT, capable of enacting autonomous behaviour based on the physical counterpart's current or future context.
This has a clear link with AAs, that are primarily used to model and encapsulate autonomous and intelligent behaviour in software systems.

Authors of \cite{Pretel2024} deliver a systematic literature review considering both MAS to create DTs and MAS to exploit DTs. 
Table 2 therein highlights how manufacturing is the main use case for synergistic AAs and DTs usage. 
This suggests that Industry 4.0 enabled by IoT technologies is a driving force for AAs and DTs applications. 
Furthermore, Table 3 therein emphasises that 73\% of the surveyed papers do not disclose the AAs and DTs development framework 
and that almost 8\% propose their own 
    (second largest percentage behind a ``team'' of 4 agent-oriented development frameworks). 
This suggests a great fragmentation of uncoordinated proposals, 
generating ``reinvention of the wheel'', especially with respect to the integration of AAs and DTs.

The survey in \cite{10.1115/1.4050244}, instead, has its main focus on DTs considering whether and how AAs (especially MAS) are used to complement the functionalities of DTs. 
Of particular relevance for the present article is Figure 5 therein, 
which reports the vision of an ``Intelligent Product'' (IP) from the literature. 
Such IP fosters the synergistic usage of DTs and AAs. 
DTs help create an ``intelligent being'', 
whereas AAs augment it with an ``intelligent agent'' to make it an IP. 
The ARTI reference architecture is inspired by these concepts, 
and it has been among the first architectures proposed to consider 
the combined usage of AAs and DTs (and the most widely adopted). 
Finally, it is relevant that also here manufacturing and IoT emerge as the main application domains for the integrated usage of AAs and DTs.

The last survey we mention is \cite{Kalyani2025}, 
where the focus is on a one-way integration of AAs and DTs: 
AAs supporting DTs implementation. 
A practical example of such an integration is given in \cite{Latsou2021811}, 
where AAs are used to create a complex DT endowed with ``intelligent'' functionalities (e.g.\ prediction). 
An interesting takeaway that the authors of \cite{Kalyani2025} get from the surveyed literature 
is that a main driving force behind this specific integration is exactly endowment of DTs with AA properties. 
In Section~\ref{sec:abstractions-and-principles} we will see how this need is captured by our principles and proposed micro-architectures for AAs and DTs integration.

Providing the complete overview 
of all the available AAs and DTs-based architectures for IoT scenarios 
is out of the scope of this paper---the interested reader is referred to these surveys. 
However, it may be beneficial for the reader
to report some selected examples.
In \cite{Nie2022}, for instance, 
    an AA is used to aggregate information from different DTs in a manufacturing scenario, 
    to predict faults in machinery and (re-)optimise production scheduling.
Another example in manufacturing is in \cite{Xu2024}. 
    There DTs are used mainly as an abstraction layer over the whole manufacturing system, 
    providing a uniform access layer to AAs. 
    AAs are used mainly to support automatic feedback control (digital to physical) 
    and communication and coordination between robot AAs, task AAs, and workstation AAs.
In \cite{DBLP:conf/pads/ClemenALOOSG21}, instead, 
    AAs are used to build the complex DT of a whole city.

It is also worth noting that there may be literature, such as reference \cite{DBLP:journals/jiii/BiZWL22}, 
where the term ``autonomous agent'' does not appear, 
but whose proposal is well aligned with the literature about AAs. 
For instance, in the mentioned paper, 
a Digital Triad is proposed as an advancement beyond DTs, 
where design knowledge and application-specific intelligence is encapsulated 
in a very similar way to AAs. 
Surveying all of this ``submerged'' literature is a complex task given that terminology likely do not match. 
However, we hope that our contribution can promote cross-fertilisation with these related research efforts, 
in a joint effort to avoid reinventing the wheel---our main motivation for sticking to AAs instead of introducing brand new concepts into DTs.
    
%======================================================
\subsection{Intelligent functionalities for AAs and DTs}
\label{ssec:functions}
%======================================================

The literature on distributing intelligence in IoT discussed in previous sections reports on several intelligent functionalities for which AA and DTs are being actively used by system designers. 
%
We summarise and categorise them here regardless of the specific task they accomplish (e.g.\ time series forecasting vs.\ fault prediction) and of the specific technique adopted (e.g.\ regression, SVM, etc.), but focussing on the \textit{kind} of intelligent function they deliver.
%
This categorisation is useful first of all in defining what we can consider, for the scope of this paper, ``intelligence'' in a practical, bottom-up way, stemming from related works in the area without getting trapped in philosophical arguments. 
%
Then, it is also useful to establish the \emph{coverage} of such functionalities by AAs and DTs, as shown in Figure~\ref{fig:radar-aa-dt}, which already suggests that they can be used synergistically to deliver the full spectrum of such intelligence. 
%
\begin{itemize}
    \item \emph{Prediction}, there including time series forecasting, recommendations, namely any form of reasoning meant to ``guess'' new information based on past and current knowledge. 
    In the IoT, common prediction targets are machinery failures, stock levels, resource use, and system states. 
    \item \emph{Simulation}, encompassing any form of reasoning by hypothesising different states, in the past, present, or future. 
    ``What-if''-like analysis falls into this category. 
    In the IoT, it is common to simulate the future states of individual things, for example, to improve the design of a product or a production pipeline. 
    However, complete systems can also be simulated with an added degree of complexity.%usually integrate multiple AAs and DTs in hierarchical architectures. 
    \item \emph{Planning}, that includes any form of synthesis, and \emph{practical reasoning}~\cite{Bratman1988}, which is meant to figure out how to achieve some practical effect (on the target system) by properly sequencing available actions. 
    Planning to achieve a given system configuration is common in the IoT, as well as planning to reconfigure after some disruptions. 
    \item \emph{Inference}, within which we include both \emph{epistemic reasoning}~\cite{Meyer1995}, that is, synthesising novel information from known data, or data-driven inference such as pattern recognition and classification.
    This is perhaps the most common form of intelligence used in IoT, as even simple monitoring and control applications usually infer situations by aggregating different data coming from distributed sources of information. 
    In addition, fault detection and diagnosis can be gathered in this category. 
    %    \item \emph{Diagnosis} \ste{Tutti}{a pensarci bene è una forma di inference imho, che dite?} \samu{Ste}{La definizione di inference è veramente larga quindi ci cade dentro quasi tutto sì.. io ci ho fatto ricadere anche detection poi nella 4}
    \item \emph{Adaptation} instead gathers all the intelligent functionalities aimed at enabling the system to adapt to unforeseen situations that had not been explicitly managed at design time. 
    As in the previous category, this one is quite broad and includes several heterogeneous approaches, ranging from engineered adaptation (e.g.\ the MAPE-K loop~\cite{Oh2022}) to learning-based methods (e.g.\ evolutionary approaches~\cite{Eiben2005}). 
    \item Finally, we highlight the role of \emph{Learning}, which despite not usually being the primary goal for a functionality, is a valuable tool to implement some of the ones highlighted above and still poses important requirements on the architecture of a system that wishes to support any form of learning process in one of its components.
    That includes statistical machine learning, reinforcement learning, and causal structure learning~\cite{Bordini2020-AI,Erduran2023,Mariani2023a,Mariani2023}. 
    In this category, we group the functionalities that aim to make the system, or one of its components, learn to do something. 
    In IoT, the most common form of learning employs statistical machine learning to create prediction models, for example. 
\end{itemize}
%
The next Section refers to these categories of intelligent functionalities to illustrate why and how to use the design principles proposed therein.
