As this thesis work is a partnership with the local public healthcare administration \ausl{},
this chapter focuses on providing context on information systems used in healthcare and the trends in digital transformation that have been shaping the sector in recent years.
These have been driven by \ac{IoT} and \ac{AI} technologies,
enabling the automatic acquisition of large amounts of data, which need to be integrated across different systems and analyzed to extract useful information to support clinical decisions, motivating the need for interoperability and knowledge representation.
%
The chapter provides an overview of the complexity of \ac{HIS},
the main medical standards, and, finally, the context of \ausl{} and its digital transformation journey.

%=======================================================
\section{Healthcare 4.0}
%=======================================================





%=======================================================
\section{\aclp{HIS}}
%=======================================================


%=======================================================
\section{Medical Standards for Interoperability}
%=======================================================

\note{HL7 FHIR, SNOMED-CT, LOINC, OPEN-EHR}


%=======================================================
\section{Intelligent Applications in Healthcare}
%=======================================================

\note{explainability}

\note{robustness}

\note{challenges etc... }

%=======================================================
\section{The \ausl{} Context}
%=======================================================

%=======================================================
\section{Final Remarks}
%=======================================================