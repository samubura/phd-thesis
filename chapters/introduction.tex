\glsresetall

%=======================================================
\section*{Context and Motivations}
%=======================================================

In the last decade, we have assisted to a significant transformation driven by the integration of advanced digital technologies such as the \ac{IoT}, \ac{AI} and \ac{ML} to improve the efficiency and automation of processes across manufacturing and service industries~\missingref{}.
This digital transformation, often referred to as the \emph{Fourth Industrial Revolution} or \emph{Industry 4.0}~\missingref{}, has 
is now becoming a pervasive trend impacting even less technology-driven sectors such as agriculture and healthcare~\missingref{}.

In the healthcare sector, the integration of digital technologies is reshaping the way care is delivered, with a focus on improving patient outcomes, enhancing operational efficiency, and enabling personalized medicine~\missingref{}.
%
Some see this transformation as a way to address the challenges of an aging population, rising healthcare costs, and the need for more efficient and effective care delivery towards accomplishing the United Nations' \ac{SDG} 3 aiming to ensure healthy lives and promote well-being for all at all ages~\cite{un_sdg_report_2025}.

This Ph.D. thesis is the outcome of a collaboration with \auslLong{}, the local public healthcare authority in the southern region of Emilia-Romagna, Italy, aimed at exploring the potential of digital technologies to enhance the internal processes of the local healthcare system. 

In this exploration, we follow the rising trend of digitalizing the core assets of an application domain with \emph{\acp{DT}}: representations of \acp{PA}, connected and synchronized through data streams from sensors and other digital sources~\missingref{}.
%
This concept has emerged as a paradigm shift in the way we model and interact with the physical world when designing \acp{CPS}, focusing on capturing the essential features of each relevant entity, creating personalized models and combining real-time data with modeling techniques, \ac{AI} and simulation~\missingref{}.
%
The focus of \acp{DT} towards individualized modeling is well-paired to the digitalization needs of the healthcare domain supporting a patient-centric approach~\missingref{}.
%
Furthermore, the ability of \acp{DT} to integrate data from various sources
-- spanning from legacy systems to \ac{IoT} devices --
and provide a holistic view of the system is beneficial for the development of \emph{intelligent applications} which can leverage the data and insights provided by \acp{DT} to support -- or partially automate -- healthcare processes.

%=======================================================
\section*{Overview and Contribution}
%=======================================================

Adopting a \ac{DT}-based approach for healthcare digitalization is not without challenges.
%
Healthcare organizations often operate with fragmented processes and legacy systems that hinder the seamless flow of information.
%
Moreover, healthcare processes often involve a variety of very different assets
-- including facilities, medical devices, patients and healthcare professionals --
whose interrelations are crucial for the understanding of workflows.
%
Finally, for an effective monitoring and decision-making support during the execution of workflows, it is crucial to have an always up-to-date view of all the assets involved.
%
Given the safety critical setting of healthcare, automated decisions must be transparent and reliable, albeit capable of adapting to the dynamic nature of healthcare environments.


These challenges call for an integrated approach towards the development of both \acp{DT} and applications supporting and automating healthcare processes.
%
In this thesis, we propose a framework for the development of \emph{\acp{DTE}} that can incorporate heterogeneous \acp{DT} and expose a holistic, synchronized view of the mirrored physical environment to applications under a unified interoperable interface.
%
To support the development of intelligent applications, we look at \ac{MAS}, which model distributed applications in dynamic environments through sets of autonomous entities~\missingref{}.
We specifically focus on cognitive approaches such as the \ac{BDI} model, as they merge the required adaptability and responsiveness with the ability to declaratively encode domain knowledge and business rules.

%--------------------------------------------------------
\subsection*{Problem Statement}
%--------------------------------------------------------
Digital transformation, can be supported by \aclp{DT} creating personalized models of relevant assets and support their tracking over workflows.
Yet, legacy systems, assets heterogeneity and dynamically evolving relationships hinder the creation of a holistic view that can be leveraged by intelligent, transparent, and adaptive applications to support and automate processes.
Addressing this gap is essential for the successful implementation of \aclp{DTE} in healthcare, making processes more efficient and supporting decision-making.

Towards this goal, we are interested in answering these research questions:

\begin{enumerate}[label=\textbf{RQ\arabic*}]
  \item\label{rq:1}
  \emph{Question}
  \newline
  Explanation
  \item\label{rq:2}
  \emph{Question}
  \newline
  Explanation
\end{enumerate}


%--------------------------------------------------------
\subsection*{Contributions}
%--------------------------------------------------------

\note{Something about how despite focusing on healthcare the thesis contributions can be interpreted in the broader scope of engineering \ac{CPS} with \ac{DT}}

This thesis contributes to the area of \ac{DT} engineering with the following contributions:
\begin{itemize}
  \item 
\end{itemize}

Additionally, the thesis contributes to the area of \ac{MAS} engineering by: 
\begin{itemize}
  \item 
\end{itemize}



%=======================================================
\section*{Thesis Structure}
%=======================================================

This thesis is organized as follows. 
