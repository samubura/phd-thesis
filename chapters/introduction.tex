\glsresetall

%=======================================================
\section*{Context and Motivations}
%=======================================================

In the last decade, we have assisted to a significant transformation driven by the integration of advanced digital technologies such as the \ac{IoT}, \ac{AI} and \ac{ML} to improve the efficiency and automation of processes across manufacturing and service industries.
This digital transformation, often referred to as the \emph{Fourth Industrial Revolution} or \emph{Industry 4.0}~\cite{Lasi_Fettke_Kemper_Feld_Hoffmann_2014,kagermann2013industrie}, 
is now becoming a pervasive trend impacting even less technology-driven sectors such as agriculture~\cite{Liu_Ma_Shu_Hancke_Abu-Mahfouz_2021} and healthcare~\cite{Tortorella_Fogliatto_Mac_Cawley_Vergara_Vassolo_Sawhney_2020}.

In the healthcare sector, the integration of digital technologies is reshaping the way care is delivered, with a focus on improving patient outcomes, enhancing operational efficiency, and enabling personalized medicine~\cite{Kraus_Schiavone_Pluzhnikova_Invernizzi_2021}.
%
Some see this transformation as a way to address the challenges of an aging population, rising healthcare costs, and the need for more efficient and effective care delivery towards accomplishing the United Nations' \ac{SDG} 3 aiming to ensure healthy lives and promote well-being for all at all ages~\cite{un_sdg_report_2025}.

This Ph.D. thesis is the outcome of a collaboration with \auslLong{}, the local public healthcare authority in the southern region of Emilia-Romagna, Italy, aimed at exploring the potential of digital technologies to enhance the internal processes of the local healthcare system. 

In this exploration, we follow the rising trend of digitalizing the core assets of an application domain with \emph{\acp{DT}}: representations of \acp{PA}, connected and synchronized through data streams from sensors and other digital sources~\cite{Grieves_2023}.
%
This concept has emerged as a paradigm shift in the way we model and interact with the physical world when designing \acp{CPS}, focusing on capturing the essential features of each relevant entity, creating personalized models and combining real-time data with modeling techniques, \ac{AI} and simulation~\cite{Semeraro_Lezoche_Panetto_Dassisti_2021}.
%
The focus of \acp{DT} towards individualized modeling is well-paired to the digitalization needs of the healthcare domain supporting a patient-centric approach~\cite{Alazab_Khan_Koppu_Ramu_M_Boobalan_Baker_Maddikunta_Gadekallu_Aljuhani_2023}.
%
Furthermore, the ability of \acp{DT} to integrate data from various sources
-- spanning from legacy systems to \ac{IoT} devices --
and provide a holistic view of the system is beneficial for the development of \emph{intelligent applications} which can leverage the data and insights provided by \acp{DT} to support -- or partially automate -- healthcare processes.

%=======================================================
\section*{Overview and Contribution}
%=======================================================

Adopting a \ac{DT}-based approach for healthcare digitalization is not without challenges.
%
Healthcare organizations often operate with fragmented processes and legacy systems that hinder the seamless flow of information.
%
Moreover, healthcare processes often involve a variety of very different assets
-- including facilities, medical devices, patients and healthcare professionals --
whose interrelations are crucial for the understanding of workflows.
%
Finally, for an effective monitoring and decision-making support, it is crucial to have an always up-to-date view of all the assets involved.
%
Given the safety critical setting of healthcare, automated decisions must be transparent and reliable, albeit capable of adapting to the dynamic nature of healthcare environments.


These challenges call for an integrated approach towards the development of both \acp{DT} and applications.
%
In this thesis, we propose a framework for the development of \emph{\acp{DTE}} that can incorporate heterogeneous \acp{DT} and expose a holistic, synchronized view of the mirrored physical environment to applications under a unified interoperable interface.
%
To implement \emph{heterogeneous} \acp{DTE} we explore architectural patterns and best practices taking inspiration from the (semantic) Web, as an example of large-scale decentralized system that successfully integrates heterogeneous resources and supports their interaction through standardized protocols and data formats~\cite{Shadbolt_Berners-Lee_Hall_2006}.
%
Namely, we ground our approach on \emph{hypermedia} representations that can semantically describe the assets and their relationships, and support navigation and interaction through hyperlinks~\cite{fielding2000architectural}.
%
To support the development of intelligent applications, we look at \ac{MAS}, which model distributed applications in dynamic environments through sets of autonomous entities~\cite{2009wooldridge}.
We specifically focus on cognitive approaches such as the \ac{BDI} model~\cite{Bratman1987-BRAIPA,rao1991modeling}, as they merge adaptability and responsiveness with a declarative encoding of domain knowledge and business rules.

%--------------------------------------------------------
\subsection*{Problem Statement}
%--------------------------------------------------------
Digital transformation, can be supported by \aclp{DT} creating personalized models of relevant assets and support their tracking over workflows.
Yet, legacy systems, assets heterogeneity and dynamically evolving relationships hinder the creation of a holistic view that can be leveraged by intelligent, transparent, and adaptive applications to support and automate processes.
Addressing this gap is essential for the successful implementation of \aclp{DTE} making processes more efficient and supporting decision-making.

Towards this goal, we are interested in answering these research questions:

\begin{enumerate}[label=\textbf{RQ\arabic*}]

  \item\label{rq:1}
  \emph{How can we engineer \acp{DT} to integrate heterogeneous data sources and provide an up-to-date uniform representation of the mirrored assets?}
  \newline
  To effectively implement \acp{DTE}, it is necessary to investigate first how to engineer \acp{DT} that can mediate interaction with their physical counterpart and give access to external consumers to the asset's state and capabilities.

  \item\label{rq:2}
  \emph{How can we engineer \acp{DTE} that integrate heterogeneous \acp{DT} and support applications in interacting with them?}
  \newline
  Given the fragmented nature of \acp{DT} we need to ensure that \acp{DTE} can integrate heterogeneous \acp{DT} and offer features that facilitate the interaction of external applications with the set of \acp{DT} within an ecosystem.

  \item\label{rq:3}
  \emph{How can we support the operational management of \acp{DTE}?}
  \newline
  \acp{DT} have a complex development-to-deployment lifecycle that may involve multiple stakeholders. Supporting the operational management of \acp{DTE} requires addressing the challenges of supporting these stakeholders alongside the lifecycle, giving them access to a comprehensive view of the \acp{DT} involved in the ecosystem and support the deployment of \acp{DT} in a complex cyber-physical computing infrastructure.
  
  \item\label{rq:4}
  \emph{What are the synergies between \acp{DTE} and \acp{MAS} for the development of intelligent applications?}
  \newline
  Given the dynamic nature of \acp{DTE}, applications interacting with \acp{DT} need to be flexible, and encapsulate strategies and rules to automate decision-making. In this thesis, we are interested to explore how (cognitive) \acp{MAS} can be used to this purpose and in general how can we exploit both autonomous agents and \acp{DT} to engineer intelligent cyber-physical applications.  


  \item\label{rq:5}
  \emph{How can we make the development of \ac{BDI} agents more accessible and easier to integrate with external systems such as \ac{DT}?}
  \newline
  \ac{BDI} agents are a powerful tool for modeling intelligent applications, but their development is often complex as they require a deep understanding of the underlying concepts. We are interested in exploring how to make the development of \ac{BDI} agents more accessible and easier to integrate with external systems looking at the design of new tools and frameworks.
\end{enumerate}


%--------------------------------------------------------
\subsection*{Contributions}
%--------------------------------------------------------

Despite being motivated by the digitalization of healthcare processes through \acp{DTE}, this thesis contributions can be interpreted in the broader scope of engineering \acp{CPS} with \acp{DT}.
%
The area of \ac{DT} engineering is a rapidly evolving field, with ongoing research aimed at improving the design, implementation, and management of \acp{DT} in various application domains where \acp{DT} are being employed~\cite{Mihai_survey_enabling_2022}.
%
In this context, this thesis contributions can be applied to different application scenarios sharing similar challenges in terms of integration of heterogeneous data sources, heterogeneous assets connected by dynamic relationships.
We identify the following key contributions to the area of \ac{DT} engineering:

\begin{itemize}
  \item Proposal of an architectural framework for the development of \acp{DT} focusing on integration of heterogeneous data sources and interoperability with external applications~\cite{11096178}.
  
  \item Definition of architectural patterns and best practices for the implementation of \acp{DT} to accurately capture synchronization requirements and augmented features~\cite{DBLP:conf/icsa/PiconeMBBFBGR25,11096160}.
  
  \item Implementation of a prototype framework for the development of \acp{DT} that demonstrates the proposed architectural concepts and patterns~\cite{11096247}.
  
  \item Definition of the concept of \acp{DTE} and its role in the context of \ac{DT}-based systems~\cite{giulianelli2025fgcs}.

  \item Proposal for a framework to implement \acp{DTE} that can integrate heterogeneous \acp{DT} and expose a unified interface for external applications using semantic hypermedia representations~\cite{giulianelli2025fgcs,DBLP:conf/models/GiulianelliBCR24,DBLP:conf/models/BurattiniZPR24}.
  
  \item Implementation of a prototype platform and toolkit to integrate \acp{DT} into a \ac{DTE}~\cite{GIULIANELLI2025102275}.
  
  \item Proposal of an abstraction layer to handle the operational management of \ac{DTE} and their deployment on a cyber-physical computing infrastructure~\cite{DBLP:conf/icccn/BarboneBMPRV24}.
  
  \item Alignment of the \ac{DT} and \ac{MAS} concepts to establish a synergetic approach for the development of intelligent applications~\cite{DBLP:journals/iot/BurattiniMMPR25,DBLP:conf/eumas/Burattini23}.
  
  \item Application of the proposed \ac{DTE} approach to real-world healthcare scenarios~\cite{DBLP:conf/percom/BurattiniMGGVZCR24,DBLP:conf/cbms/BurattiniMCGRLPT23}
\end{itemize}

Additionally, as this thesis leverages cognitive \ac{MAS} to support the development of intelligent applications that can integrate with \acp{DTE}, it contributes to the area of \ac{BDI} \ac{MAS} engineering by:  
\begin{itemize}
  \item Proposing tools for the implementation of \ac{BDI} agents in order to facilitate their integration with other programming paradigms~\cite{DBLP:journals/sncs/BaiardiBCP24,DBLP:conf/eumas/BaiardiBCP23,DBLP:conf/acsos/PianiniBBC24}.
  
  \item Exploring the role of concurrency in \ac{BDI} agents and the use of simulation to test and verify \ac{BDI} \ac{MAS} in a controlled environment before deployment~\cite{baiardi2025jaktasim,DBLP:conf/atal/BaiardiBCPOR24,DBLP:conf/emas/BaiardiBCPRO24}.
  
  \item Proposing a conceptual framework to derive explanations of \ac{BDI} agents' behavior, enhancing the transparency and trustworthiness of \ac{BDI} applications and exploring how explainability can support new agent-to-agent interactions~\cite{DBLP:journals/aamas/YanBHR25,DBLP:conf/woa/YanBHR23,beaumont2025explain,beaumont2025engineering}.

  \item Investigating the integration of \ac{BDI} agents with hypermedia environments and with Domain Driven Design to bring development to the \emph{knowledge level}~\cite{burattini2025gap,DBLP:conf/emas/BurattiniCGR23,DBLP:conf/emas/RicciBCC24,ricci2025concpeptual,DBLP:conf/emas/RicciBCC24,DBLP:conf/atal/LemeeBMC23,RAMANATHAN2022217}.

  \item Exploring the ability for agents to anticipate future scenarios and react accordingly modifying their behavior to find alternative courses of action \cite{DBLP:journals/aamas/HubnerBRM25,wesaac}

  \item Exploring the role of \ac{BDI} in the new landscape of \ac{GenAI} agents~\cite{DBLP:conf/atal/Ricci0ZBC24,ciatto2025generative}
\end{itemize}


%=======================================================
\section*{Thesis Structure}
%=======================================================

This thesis is organized as follows. 
In \Cref{part:background} we provide the necessary context and background on the key concepts that underpin this research, including an overview of the challenges of healthcare digitalization and the reality of \ausl{} in \Cref{chap:back:health4.0},
moving on to introducing the concept of \ac{DT} its technological characterization  in \Cref{chap:back:DT},
discussing the architectural principles of the (semantic) Web and how they can build interoperability, presenting the case study of the \ac{WoT} in \Cref{chap:back:Web},
and finally giving an overview of the concepts related to \ac{MAS} engineering that are relevant for this thesis in \Cref{chap:back:MAS}.

In \Cref{part:dte} we delve into the contributions towards the definition of the concept of \ac{DTE}. Namely, in \Cref{chap:dte:engineering-dt} we present architectural proposals and patterns for the development of \ac{DT} that can be integrated into a \ac{DTE}, then in \Cref{chap:dte:dte} we introduce the concept of \ac{DTE}, and explore its operational management and prototype implementation through the integration of heterogeneous \acp{DT} in \Cref{chap:dte:dtc} and \Cref{chap:dte:hwodt} respectively.

In \Cref{part:mas} we focus on the role of multi-agent systems in the development of intelligent applications discussing their alignment for a synergetic exploitation with \acp{DTE} in \Cref{chap:mas:mas-dt} and the contributions to the engineering of \ac{BDI} agents towards the goal of making them more accessible for developers and users in \Cref{chap:mas:engineering}.

Finally, in \Cref{part:applications} we present application scenarios proposed by our partner \ausl{}---namely, pharmaceutical supply chain (\Cref{chap:app:irst}), trauma management (\Cref{chap:app:trauma}), and operating room management (\Cref{chap:app:orm})---to demonstrate the applicability of the proposed approach to \ac{DTE} engineering in real-world healthcare settings,
before concluding by discussing results and presenting open challenges to be addressed in future works.