This chapter provides an overview of the \ac{DT} concept,
its historical development and the more recent models and technologies that have emerged
to support its implementation.
%
The ever-growing literature on the subject reflects the increasing interest in \acp{DT}
and its multi-faceted nature, encompassing various domains and applications.
%
In this thesis we focus on the software engineering perspective of \acp{DT}, 
and the application of \acp{DT} to the engineering of \ac{IoT} systems.
%
Accordingly, this chapter focuses on how the \ac{DT} concept has been interpreted in this context.


%=======================================================
\section{History and Definitions}
%=======================================================

The concept of \ac{DT} was introduced in the early 2000s by Michael Grieves
in the context of Product-Lifecycle Management~\cite{Grieves_2023}
where it was conceived as a virtual representation of a product throughout its lifecycle.
%
At its essence, Grieves defined what later came to be known as a \ac{DT} as a system composed of three main components:
\begin{itemize}
\item a \emph{physical space} and its products;
\item a \emph{virtual space} containing the digital representation of the products;
\item a \emph{connection} between the two, making data flow from the physical to the virtual space and information flow from the virtual to the physical space~\cite{Grieves2017}.
\end{itemize}

\note{TODO: add the classic DT figure might be nice here}

Grieves referred to this concept as the \emph{Mirrored Spaces Model}~\cite{Grieves_2005},
a name which echoes the idea of \emph{Mirror Worlds} introduced by David Gelernter in the 1990s~\cite{gelernter1991mirrorworlds}
who also envisioned the ability to replicate the real world in a completely virtual space~\cite{Singh_Fuenmayor_Hinchy_Qiao_Murray_Devine_2021}.

The idea was later associated with its modern \emph{\acl{DT}} name and popularized by NASA in the 2010s, when it was presented as a key technology
for future development of aircraft and spacecraft in order to virtually simulate extreme conditions,
integrate data from the physical system in operation, 
and provide feedback on the conditions of the \emph{flying twin}
while possibly enacting changes to mitigate damage~\cite{glaessgen2012dtnasa}.
%
NASA's definition hence focused on the ability of the \ac{DT} to integrate multiple simulation models and characterized the \ac{DT} as \emph{ultra-realistic}, as the goal was to virtually replicate the physical system with the highest possible degree of fidelity:

\begin{quote}
    A Digital Twin is an integrated multiphysics, multiscale, probabilistic simulation of an as-built vehicle or system that uses the best available physical models, sensor updates, fleet history, etc., to mirror the life of its corresponding flying twin~\cite{glaessgen2012dtnasa}.
\end{quote}

Given the various phases of a product's lifecycle, the \ac{DT} concept was later
refined to distinguish between the \emph{Digital Twin Prototype} (DTP) and the \emph{Digital Twin Instance} (DTI)~\cite{Grieves2017}.
The DTP is the virtual representation of the product during its design and development phase,
while the DTI is the virtual representation of the product during its operational phase.
%
Interestingly, the DTP exists before the physical product is even produced, 
and it is rather ``just'' a virtual model of the product. 
%
What could be considered a \emph{true} \ac{DT} is hence the DTI, which exists alongside its physical counterpart and is continuously updated with data from the physical world. 

The role of such bidirectional data exchange between the physical and virtual parts
of a \ac{DT} has been used to characterize the \ac{DT} concept and distinguish it from other related concepts.
%
A widely accepted taxonomy classified the concepts of \emph{Digital Model}, \emph{Digital Shadow} and \emph{Digital Twin} based on direction of the automatic data flow between the physical and virtual spaces~\cite{kritzinger2018dtmanufacturing}:
namely, 
a Digital Model is a static model of a physical system (similarly to the DTP) that can be manually updated over time,
the Digital Shadow gets automatically updated through an inbound data flow from the physical space,
while the Digital Twin is the only one that holds a bidirectional data flow which can provide feedback to the physical counterpart.


Such connection has been referred to as the \emph{digital thread}~\cite{Singh_Willcox_2018,Grieves_2023} to evoke the idea of a tie that binds the physical and virtual spaces.
%
Other common terms include \emph{twinning}~\cite{JONES202036} or \emph{shadowing}~\cite{Jiang_Yin_Li_Luo_Kaynak_2021,web-of-dt-ricci-2022} process. 
%
This terminology emphasizes the continuous and dynamic nature of the relationship between the physical and virtual entities, 
and highlights the active role of such process in maintaining the \ac{DT} up to date.

\note{Maybe the figures for Digital Model, Shadow and Twin here?}

The scope and applicability of the \ac{DT} concept has evolved and expanded
to encompass a wide range of applications and domains.
%
Today, \acp{DT} are used in various fields such as manufacturing, healthcare, smart cities, and more.
%
This is also reflected in the abundance of terminology used to identify the physical counterpart of a \ac{DT}
which is nowadays often referred to as the \emph{physical twin} or
\emph{physical entity}~\cite{Singh_Fuenmayor_Hinchy_Qiao_Murray_Devine_2021,JONES202036,DBLP:journals/jss/DaliborJRSWWW22}.
%
This shift from identifying the physical counterpart as a \emph{product} to a more generic \emph{entity}
highlights the broader applicability of the \ac{DT} concept beyond its original manufacturing context.
%
The \ac{DT} concept has been adopted also to represent people~\cite{Shengli_2021},
processes (e.g., supply chain~\cite{Barykin_Bochkarev_Kalinina_Yadykin_2020}) and organizations~\cite{Parmar_Leiponen_Thomas_2020}, leading to a more abstract interpretation of what can be considered \emph{physical}. 
%
Accordingly, the definitions of \acp{DT} have also diversified, leading to a plethora of interpretations~\cite{DBLP:journals/jss/DaliborJRSWWW22}. 

For the scope of this thesis, 
since the focus is on how \acp{DT} can be used to engineer \ac{IoT} systems, taking as the reference context the healthcare domain,
we adopt the following definition, adapted from \cite{dt-IoT-context-Minerva-2020}:

\begin{quote}
A \acl{DT} is a comprehensive software representation of an individual \acl{PA}.
It includes the properties, conditions, and behavior(s) of the real-life object through models and data.
A \acl{DT} is a set of realistic models that can simulate an object's behavior in the deployed environment.
The DT represents and reflects its physical twin and remains its virtual counterpart across the object's entire lifecycle.
\end{quote}

The definition emphasizes the software nature of the \ac{DT} and its ability to model the properties and behavior of its physical counterpart.
%
We deliberately use the term \emph{\ac{PA}} to refer to the physical counterpart of a \ac{DT}
to highlight the fact that an \emph{asset} is something that has a strategic
value in the context of an application domain, and that the \ac{DT} is meant to represent and manage such assets.
%
Compared to the original Grieves model which characterized the \ac{DT} as having three parts (physical, digital and connection), for the reminder of this thesis we will instead consider:
\begin{itemize}
\item the \ac{PA} as the physical counterpart of a \ac{DT};
\item the \ac{DT} as the software implementing the virtual representation of the \ac{PA} through a combination of models and data;
\item the \emph{twinning process} implemented by the \ac{DT} software to keep the \ac{DT} up to date with the \ac{PA} and possibly provide feedback to it.
\end{itemize}

%=======================================================
\section{Conceptual Models and Properties}
%=======================================================

\note{Tao's 5D model}



\note{Minerva properties}



% %=======================================================
% \section{\aclp{DT} in the \acs{IoT} Context}
% %=======================================================

%=======================================================
\section{\aclp{DT} and \acl{AI}}
%=======================================================

\note{This is here to serve the purpose of introducing "Cognitive Digital Twins"
and link to MAS}

%=======================================================
\section{Technologies and Platforms}
%=======================================================

\note{The shift from silos to platforms}



\note{Azure DT}

\note{Ditto}


%=======================================================
\section{Towards \aclp{DTE}}
%=======================================================

\note{From platforms to ecosystems}

\note{National Digital Twin}

\note{Web of Digital Twins}

\note{Digital Twin interoperability(?)}